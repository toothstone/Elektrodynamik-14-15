\chapter{Einleitung}
Gegenstand der Vorlesung ist die (klassische) Theorie der Elektrischen Felder ausgehend von den \textsc{Maxwell}-Gleichungen (1864):

\begin{align*}
\div \vec{B} \ &= \ 0    &\epsilon_0 \ \div \vec{E} \ &= \ \rho\\
\rot \vec{E} \ + \ \pdiff{\vec{B}}{t} \ &= \ 0   &\frac{1}{\mu_0} \ \rot \vec{B} \ - \ \epsilon_0 \ \pdiff{\vec{E}}{t} \ &= \ \vec{j}
\end{align*}

für die Felder $\vec{E}$ und $\vec{B}$ in Abhängigkeit von Ladungs- und Stromverteilung $\rho(\vec{r},t)$ und $\vec{j}(\vec{r},t)$. Von dieser Grundlage aus wollen wir in dieser Vorlesung die physikalischen Erscheinungen für diese Felder schildern und diskutieren.\\
Die Elektrodynamik ist ein Teil des Standardmodells der Teilchenphysik, das einheitlich Teilchen und ihre Wechselwirkungen beschreibt.\\
Klassische Elektrodynamik ist ein Grenzfall der Quantenelektrodynamik (gültig für kleine Impuls- und Energiebeträge, große Brechungszahlen für Photonen).\\
Sie ist im Einklang mit der der speziellen Relativitätstheorie, da $c$ implizit mit in den \textsc{Maxwell}-Gleichungen enthalten ist. Viele interessante Effekte von Materie können jedoch mit der klassischen Theorie nicht beschrieben werden.\\
Zum Beispiel: Wann sind Atome stabil? Wann ist Eisen ferromagnetisch? Warum wird z.B. Blei bei tiefen Temperaturen supraleitend? Für diese Fragen werden Quanteneffekte wichtig.