\section{Energie des stationären Magnetfelds}

\begin{align*}
W_m & \ \quad = \quad  \ \Int{}{}{V}\frac{1}{2\mu_0}\vec{B}^2 \ =  \ \Int{}{}{V} \frac{1}{2\mu_0} \vec{B}\times\rot\vec{A} \ = \ \Int{}{}{V} \frac{1}{2\mu_0} \ \left(\vec{B}\times\nabla\right) \overset{\downarrow}{\vec{A}}\\
& \overset{\text{part. Int.}}{=}  \  \Int{}{}{V} \frac{1}{2\mu_0} \vec{A} \cdot\left(\nabla\times\vec{B}\right) \quad + \quad \text{Oberflächenintegral} \left(\rightarrow 0 \text{ für } V \rightarrow \infty\right)\\
& \;\;\;\;\overset{\dot{\vec{E}}=0}{=} \ \frac{1}{2} \Int{}{}{V} \ \vec{j}\cdot\vec{A}
\end{align*}

Analog zum elektrostatischen Fall ergibt Umschreiben:

\begin{equation*}
W_m = \frac{\mu_0}{8\pi} \ \int\int\d V \d V' \ \frac{\vec{j}(\vec{r}) \ \vec{j}(\vec{r}')}{|\vec{r}-\vec{r}'|}
\end{equation*}
\ \\

Für dünne linienförmige und geschlossene Leiterschleifen $\mathcal{L}_i$ gilt mit $\int\d V\vec{j} \rightarrow \int\d \vec{r} \cdot I$ und unter Anwendung des Satzes von \textsc{Stokes}:

\begin{equation*}
W_m = \frac{1}{2} \cdot \sum_i I_i \Int{\mathcal{L}_i}{}{\vec{r}} \cdot \vec{A} \ = \ \frac{1}{2} \sum_i I_i \Phi_i
\end{equation*}
\ \\
\ \\
Allgemein folgt somit aus $\Phi_i = \sum_k L_{ik} I_k$:

\begin{align*}
W_m \ &= \ \frac{1}{2} \ \sum_{ik}  I_i L_{ik} I_k \ = \ \frac{1}{2} \ \sum_{ik} \Phi_i \tilde{L}_{ik} \Phi_k\\
&= \ \frac{1}{2} \ \sum_{i\neq k} I_i I_k \ \underbrace{\frac{\mu_0}{4\pi} \ \Int{\mathcal{L}_i}{}{\vec{r}} \Int{\mathcal{L}_k}{}{\vec{r}'} \ \frac{1}{|\vec{r}-\vec{r}'|}}_{L_{ik}} \; + \; \text{Selbstenergie für } (i=k)
\end{align*}

Ähnlich wie im elektrostatischen Analogon stößt die klassische Elektrodynamik bei der Berechnung der Selbstenergien für ``dünne'' und somit sonst ideale Leiter an ihre Grenzen. Für eine Leiter schleife endlicher Dicke kann man die Selbstenergie jedoch wieder berechnen, sie beträgt:

\begin{align*}
& \ W_m  \ = \ \frac{1}{2} \ \cdot  \ L \ \cdot \ I^2 \\
\text{mit } \ & \ L \ = \ \frac{\mu_0}{4\pi I^2}\ \Int{}{}{V'} \frac{\vec{j}(\vec{r})\vec{j}(\vec{r}')}{|\vec{r}-\vec{r}'|} \ = \ \frac{1}{I^2} \ \Int{}{}{V} \frac{\vec{B}^2}{\mu_0} \quad
 \left( = \ \frac{\Phi}{I}\right) 
\end{align*}

\section{Beispiele für Energiestromdichten}

\begin{enumerate}[label=\roman*]
\item \underline{Stromdurchflossener gerader Leiter}\\
\ \\
\begin{equation*}
\frac{1}{\mu_0} \rot \vec{B} \ = \ \vec{j} \ + \ \epsilon_0 \dot{\vec{E}}
\end{equation*}

Für $\dot{\vec{E}}=0$ und der integralen Formulierung $\Oint{}{}{\vec{r}} \cdot \vec{B} = \mu_0 I$ ebenjener \textsc{Maxwell}-Gleichung folgt, dass um den geraden Leiter ein tangentiales $\vec{B}$-Feld existiert:

\begin{equation*}
B \ = \ \frac{\mu_0 \ I}{2\pi \ r_{\perp}}
\end{equation*}

Mit dem \textsc{Ohm}'schen Gesetz $\vec{E} = \sigma \cdot\vec{j}$ folgt, dass das $\vec{E}$-Feld entlang des Leiters gerichtet sein muss. Somit gilt für die Energiestromdichte $\vec{S}_P = \frac{1}{\mu_0} \left(\vec{E}\times\vec{B}\right)$, dass sie radial nach innen gerichtet sein muss.\\
\ \\
Bei einem einfachen Stromkreis wird demnach die Energie nicht entlang der Leiter sondern über die erzeugten Feldern von der Spannungsquelle zum Verbraucher transportiert!\\
\ \\
Berechnet man nun außerdem das Flächenintegral über die Energiestromdichte, erhält man für den geraden Leiter:

\begin{equation*}
\Iint{}{}{\vec{A}_F}\cdot\vec{S}_P \ = \ 2\pi r_{\perp} l \ S_P \ = \ 2\pi r_{\perp} l \frac{1}{\mu_0}E\frac{\mu_0 I}{2\pi r_{\perp}} \ = \ l \cdot E \cdot I
\end{equation*}

Der erhaltene Ausdruck $N := l \cdot E \cdot I = U \cdot I$ ist somit anschaulich die abgestrahlte Energie pro Zeiteinheit und ist auch als \textbf{\textsc{Ohm}'scher Verlust} oder \textbf{\textsc{Ohm}'sche Wärme} bekannt.

\ \\

\item \underline{ideale parallele Doppelleiter mit entgegengesetzten Stromrichtungen}
\ \\

Hier betrachten wir gleich zu Beginn das Flächenintegral über der Energiestromdichte und setzen nur die Querschnittsfläche ein:

\begin{align*}
N \ &= \ \Int{}{}{\vec{A}_F} \cdot \vec{S}_P \ = \ \frac{1}{\mu_0} \left(\vec{E}\times\vec{B}\right) \ \overset{\text{stationär}}{=} \ - \frac{1}{\mu_0} \Int{}{}{\vec{A}_F} \cdot \left(\nabla \varphi \times \vec{B}\right)\\
&= \ - \frac{1}{\mu_0} \Int{}{}{\vec{A}_F} \cdot \left(\nabla\times\left(\varphi\vec{B}\right)\right) \; + \; \Int{}{}{\vec{A}_F}\cdot\varphi\cdot\left(\underbrace{\nabla\times\vec{B}}_{=\vec{j}}\right)
\end{align*}



\end{enumerate}