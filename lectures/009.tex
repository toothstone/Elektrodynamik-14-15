\section{Energie des stationären Magnetfelds}

\begin{align*}
W_m & \ \quad = \quad  \ \Int{}{}{V}\frac{1}{2\mu_0}\vec{B}^2 \ =  \ \Int{}{}{V} \frac{1}{2\mu_0} \vec{B}\times\rot\vec{A} \ = \ \Int{}{}{V} \frac{1}{2\mu_0} \ \left(\vec{B}\times\nabla\right) \overset{\downarrow}{\vec{A}}\\
& \overset{\text{part. Int.}}{=}  \  \Int{}{}{V} \frac{1}{2\mu_0} \vec{A} \cdot\left(\nabla\times\vec{B}\right) \quad + \quad \text{Oberflächenintegral} \left(\rightarrow 0 \text{ für } V \rightarrow \infty\right)\\
& \;\;\;\;\overset{\dot{\vec{E}}=0}{=} \ \frac{1}{2} \Int{}{}{V} \ \vec{j}\cdot\vec{A}
\end{align*}

Analog zum elektrostatischen Fall ergibt Umschreiben:

\begin{equation*}
W_m = \frac{\mu_0}{8\pi} \ \int\int\d V \d V' \ \frac{\vec{j}(\vec{r}) \ \vec{j}(\vec{r}')}{|\vec{r}-\vec{r}'|}
\end{equation*}
\ \\

Für dünne linienförmige und geschlossene Leiterschleifen $\mathcal{L}_i$ gilt mit $\int\d V\vec{j} \rightarrow \int\d \vec{r} \cdot I$ und unter Anwendung des Satzes von \textsc{Stokes}:

\begin{equation*}
W_m = \frac{1}{2} \cdot \sum_i I_i \Int{\mathcal{L}_i}{}{\vec{r}} \cdot \vec{A} \ = \ \frac{1}{2} \sum_i I_i \Phi_i
\end{equation*}
\ \\
\ \\
Allgemein folgt somit aus $\Phi_i = \sum_k L_{ik} I_k$:

\begin{align*}
W_m \ &= \ \frac{1}{2} \ \sum_{ik}  I_i L_{ik} I_k \ = \ \frac{1}{2} \ \sum_{ik} \Phi_i \tilde{L}_{ik} \Phi_k\\
&= \ \frac{1}{2} \ \sum_{i\neq k} I_i I_k \ \underbrace{\frac{\mu_0}{4\pi} \ \Int{\mathcal{L}_i}{}{\vec{r}} \Int{\mathcal{L}_k}{}{\vec{r}'} \ \frac{1}{|\vec{r}-\vec{r}'|}}_{\mathcal{L}_{ik}} \; + \; \text{Selbstenergie für } (i=k)
\end{align*}

Ähnlich wie im elektrostatischen Analogon stößt die klassische Elektrodynamik bei der Berechnung der Selbstenergien für ``dünne'' und somit sonst ideale Leiter an ihre Grenzen. Für eine Leiter schleife endlicher Dicke kann man die Selbstenergie jedoch wieder berechnen, sie beträgt:

\begin{align*}
& \ W_m  \ = \ \frac{1}{2} \ \cdot  \ L \ \cdot \ I^2 \\
\text{mit } \ & \ L \ = \ \frac{\mu_0}{4\pi I^2}\ \Int{}{}{V'} \frac{\vec{j}(\vec{r})\vec{j}(\vec{r}')}{|\vec{r}-\vec{r}'|} \ = \ \frac{1}{I^2} \ \Int{}{}{V} \frac{\vec{B}^2}{\mu_0} \quad
 \left( = \ \frac{\Phi}{I}\right) 
\end{align*}

\section{Beispiele für Energiestromdichten}

\begin{enumerate}[label=\roman*]
\item \underline{Stromdurchflossener gerader Leiter}\\
\ \\
\begin{equation*}
\frac{1}{\mu_0} \rot \vec{B} \ = \ \vec{j} \ + \ \epsilon_0 \dot{\vec{E}}
\end{equation*}

Für $\dot{\vec{E}}=0$ und der integralen Formulierung $\Oint{}{}{\vec{r}} \cdot \vec{B} = \mu_0 I$ ebenjener \textsc{Maxwell}-Gleichung folgt, dass um den geraden Leiter ein tangentiales $\vec{B}$-Feld existiert:

\begin{equation*}
B \ = \ \frac{\mu_0 \ I}{2\pi \ r_{\perp}}
\end{equation*}

Mit dem \textsc{Ohm}'schen Gesetz $\vec{E} = \sigma \cdot\vec{j}$ folgt, dass das $\vec{E}$-Feld entlang des Leiters gerichtet sein muss. Somit gilt für die Energiestromdichte $\vec{S}_P = \frac{1}{\mu_0} \left(\vec{E}\times\vec{B}\right)$, dass sie radial nach innen gerichtet sein muss.\\
\ \\
Bei einem einfachen Stromkreis wird demnach die Energie nicht entlang der Leiter sondern über die erzeugten Feldern von der Spannungsquelle zum Verbraucher transportiert!\\
\ \\
Berechnet man nun außerdem das Flächenintegral über die Energiestromdichte, erhält man für den geraden Leiter:

\begin{equation*}
\Iint{}{}{\vec{A}_F}\cdot\vec{S}_P \ = \ 2\pi r_{\perp} l \ S_P \ = \ 2\pi r_{\perp} l \frac{1}{\mu_0}E\frac{\mu_0 I}{2\pi r_{\perp}} \ = \ l \cdot E \cdot I
\end{equation*}

Der erhaltene Ausdruck $N := l \cdot E \cdot I = U \cdot I$ ist somit anschaulich die abgestrahlte Energie pro Zeiteinheit und ist auch als \textbf{\textsc{Ohm}'scher Verlust} oder \textbf{\textsc{Ohm}'sche Wärme} bekannt.

\ \\

\item \underline{ideale parallele Doppelleiter mit entgegengesetzten Stromrichtungen}
\ \\

Hier betrachten wir gleich zu Beginn das Flächenintegral über der Energiestromdichte und setzen nur die Querschnittsfläche ein:

\begin{align*}
N \ &= \ \Int{}{}{\vec{A}_F} \cdot \vec{S}_P \ = \ \frac{1}{\mu_0} \left(\vec{E}\times\vec{B}\right) \ \overset{\text{stationär}}{=} \ - \frac{1}{\mu_0} \Int{}{}{\vec{A}_F} \cdot \left(\nabla \varphi \times \vec{B}\right)\\
&= \ - \frac{1}{\mu_0} \Int{}{}{\vec{A}_F} \cdot \left(\nabla\times\left(\varphi\vec{B}\right)\right) \; + \; \Int{}{}{\vec{A}_F}\cdot\varphi\cdot\underbrace{\left(\nabla\times\vec{B}\right)}_{=\vec{j}}
\end{align*}

Nach Umformen mit \textsc{Stokes} erhält man für den ersten Summanden:

\begin{equation*}
\Oint{\partial A_F}{}{\vec{r}}  \cdot \varphi \vec{B}
\end{equation*}

doch dieser Anteil geht für $\partial A_F \rightarrow \infty$ schnell genug gegen Null. Somit folgt für die Leistung:

\begin{equation*}
\Rightarrow \; N \ = \ \Int{}{}{\vec{A}_F} \cdot \varphi \vec{j} \ = \ I \left(\varphi_1 - \varphi_2 \right) \ = \ I \cdot U_{12}
\end{equation*}
Beim Doppelleiter wird die Leistung entlang der Leiter transportiert.
\end{enumerate}


\section{Energie einer ebenen harmonischen Welle}

Wir betrachten:

\begin{align*}
\vec{E} \ &= \ \vec{E}_0 \cdot \operatorname{Re} e^{i\left(\vec{k}\vec{r}-\omega t\right)} \qquad \text{mit} \qquad \vec{E}\perp\vec{k}, \; \vec{E}_0 \text{ reell}\\
\vec{B} \ &= \ \frac{1}{c}\left(\vec{e}_k\times\vec{E}\right) \qquad \text{mit} \qquad \vec{e}_k = \frac{\vec{k}}{k}
\end{align*}

\underline{Energiedichte:}\\

\begin{align*}
W \ &= \ \frac{\epsilon_0}{2} \vec{E}^2 \; + \; \frac{1}{2\mu_0}\vec{B}^2 = \left[\frac{\epsilon_0}{2}\vec{E}_0^2 + \frac{1}{2\mu_0 c^2} \ \left(\vec{e}_k\times\vec{E}\right)^2\right] \cos^2\left(\vec{k}\vec{r}-\omega t\right)\\
\ \\
W \ &= \ \epsilon_0 \vec{E}_0^2 \ \cos^2\left(\vec{k}\vec{r}-\omega t\right)
\end{align*}

Räumliche oder zeitliche Mittelung: $\qquad\left(\overline{\cos^2(.)} = \frac{1}{2}(.)\right)$

\begin{equation*}
\overline{W} \ = \ \frac{\epsilon_0}{2} \ \vec{E}_0^2
\end{equation*}

\ \\
\underline{Energiestromdichte:}

\begin{align*}
\vec{S}_P \ &= \ \frac{1}{\mu_0} \vec{E}\times\vec{B} \ = \ \frac{1}{\mu_0 c} \vec{E}_0 \times \left(\vec{e}_k\times\vec{E}_0\right) \ \cos^2\left(\vec{k}\vec{r}-\omega t\right)\\
&= \ \epsilon_0 c \; \Bigg(\vec{e}_k\vec{E}_0^2 \ - \ \vec{E}_0\underbrace{\left(\vec{e}_k\cdot\vec{E}_0\right)}_{=0}\Bigg) \ \cos^2\left(\vec{k}\vec{r}-\omega t\right)\\
\vec{S}_p \ &= \ c \ \vec{e}_k \ \epsilon_0 \ \vec{E}_0 \ \cos^2 \left(\vec{k}\vec{r}-\omega t\right) \ = \ c \cdot  \vec{e}_k \cdot w
\end{align*}

nach Mittelung:

\begin{equation*}
\vec{S}_P \ = \ c \cdot \vec{e}_k \cdot \overline{w} \ = \ c \cdot \vec{e}_k \ \frac{\epsilon_0}{2} \ \vec{E}_0 \ \left( = \frac{1}{2\mu_0} \operatorname{Re} \ (\vec{E}\times\vec{B}^*)\right)
\end{equation*}
\ \\

\section{Impulsbilanz des elektromagnetischen Feldes}
\ \\
Impulsänderung = Kraft

\begin{equation*}
\Rightarrow \ \vec{F}_L \ = \ \dot{\vec{p}}_{\text{mech}} \ = \  \dot{\vec{p}}_{\text{elm}} \quad \text{(Impuls im em. Feld)}
\end{equation*}
\ \\
\ \\
\ \\

Bilanzgleichung pro Volumen:

\begin{align*}
\pdiff{\vec{g}}{t} \; + \; \div \tens{T} & \ = \ - \vec{f}_L \\
\ \\
\text{mit} \qquad \vec{g} & \ - \ \text{Impulsdichte}\\
\vec{f}_L & \ - \ \text{Lorentzkraftdichte} \quad \left( = \rho\vec{E} \ + \ \vec{j}\times\vec{B}\right)\\
\tens{T} & \ - \ \text{Impulsstromdichte}
\end{align*}

$ - \vec{f}_L$ ist dementsprechend die elektromagnetische Impulserzeugungsrate pro Volumen $\nu_g$.
$\vec{g}$ und $\tens{T}$ hängen im Allgemeinen von Feldern ab. \textsc{Maxwell} liefert uns:

\begin{align*}
- \vec{f}_L \ &= \ \epsilon_0 ( \nabla \cdot\overset{\downarrow}{\vec{E}} ) \ + \ \frac{1}{\mu_0} \ \vec{B} \times (\nabla \times \vec{B}) \ + \ \epsilon_0 \dot{\vec{E}} \times \vec{B}\\
\overset{\text{Umformen}}{\Longrightarrow} \quad\vec{E}\times\vec{B} \ &= \ \partial_t ( \vec{E} \times \vec{B} ) \ - \ \vec{E}\times\dot{\vec{B}} \ =  \ \partial_t (\vec{E}\times\vec{B}) \ + \ \vec{E}\times (\nabla\times\vec{E})\\
\vec{E}\times ( \nabla\times \overset{\downarrow}{\vec{E}} ) \ &= \ \nabla \ (\overset{\downarrow}{\vec{E}}\cdot\vec{E} ) \ - \ ( \vec{E}\cdot\nabla ) \ \overset{\downarrow}{\vec{E}} \ = \ \frac{1}{2}\nabla\vec{E}^2 \ - \ ( \vec{E}\cdot\nabla) \ \overset{\downarrow}{\vec{E}}
\end{align*}

Für $\vec{B}\times(\nabla\times\vec{B})$ analog, mit $(\nabla\cdot\overset{\downarrow}{\vec{B}})\vec{B}=0$. Insgesamt erhalten wir:

\begin{equation*}
(-\vec{f}_L)_k \ = \ \partial_t \ \epsilon_0 \ (\vec{E}\times\vec{B})_k \ - \ \pdiff{}{x_i} \epsilon_0 \ \left(\frac{\vec{E}^2}{2} \delta_{ik} \ - \ \vec{E}_i \vec{E}_k \right) \; + \; \pdiff{}{x_i} \frac{1}{\mu_0} \ \left(\frac{\vec{B}^2}{2}\delta_{ik} \ - \ \vec{B}_i \vec{B}_k \right) 
\end{equation*}

\ \\
Der Vergleich mit $\dot{\vec{g}} + \div\tens{T} = - \vec{f}_L$ ergibt:

\begin{align*}
\text{Impulsdichte} \qquad & \vec{g} \ = \ \epsilon_0 \ (\vec{E}\times\vec{B})\\
\text{Impulsstromdichte} \qquad & \tens{T} \ = \ \epsilon_0 \ \left(\frac{\vec{E}^2}{2} \mathbb{1} \ - \ \vec{E}\circ\vec{E}\right) \ + \ \frac{1}{\mu_0} \ \left(\frac{\vec{B}^2}{2}\mathbb{1} \ - \ \vec{B}\circ\vec{B}\right)\\
\text{Impulserzeugungsrate} \qquad - & \vec{f}_L \ = \ \nu_g
\end{align*}

\ \\
\underline{Diskussion:}

\begin{enumerate}[label = \roman*]
\item Impulsdichte:\

\begin{equation*}
\vec{g} \ = \ \epsilon_0\mu_0 \vec{S}_P \ = \ \frac{1}{c^2}\vec{S}_P
\end{equation*} 

Allgemeingültig für Feldtheorien bei Ausbreitung mit $c$, vgl. Relativitätstheorie.
\begin{equation*}
\text{für Welle gilt: } |\vec{S}_P| \ = \ c \cdot w \quad \Rightarrow \quad |\vec{g}| \ = \ \frac{w}{c} 
\end{equation*}

Zusammenhang mit Strahlungsdruck (Absorption einer em. Welle): 

\begin{align*}
\text{Impulsübertrag:  } \qquad & \Delta \vec{p} \ = \ \vec{g} \ \Delta V \ = \ \vec{g} \ c \ \Delta t \ \Delta A_F\\
\text{Druck: } \qquad & \frac{|\Delta\vec{p}|}{\Delta t \ \Delta A_F} \ = \ c \ |\vec{g}| \ = \ w
\end{align*}
\ \\

\item Impulsstromdichte: \\

Tensor $\tens{T}$ mit folgenden Eigenschaften:

\begin{itemize}
\item $T_{ik}$ mit $k$ = Impulskomponente, $i$ = Transportrichtung
\item $T_{ik} \ = \ T_{ki}$
\item $ [T_{ik}] \ = \ [\vec{f}] \cdot [l] \ = \ \frac{[\vec{F}]}{[l]^2} \quad \Rightarrow \quad$ Druck, Spannung
\item Stationäre Felder: $\quad \dot{\vec{g}} \ = \ 0 \quad \Rightarrow \quad \div \tens{T} \ = \ -\vec{f}_L$\\
\ \\
Volumenintegral + Satz von \textsc{Gauss} $ \quad \Rightarrow \quad \Oiint{}{}{\vec{A}_F} \cdot \tens{T} \ = \ - \vec{F}_L \quad$\\
\ \\
\textbf{Oberflächenkräfte}, Interpretation des Vorzeichens: Fläche schließt Ströme/Ladungen ein, auf die die Kräfte wirken
\end{itemize}
\end{enumerate}