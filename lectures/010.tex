\chapter{Kraftwirkung auf Ladungen und Ströme}

Erinnerung: \textbf{Lorentzkraftdichte}

\begin{equation*}
\vec{f}_L \ = \ \rho \ \vec{E} \ + \ \vec{j}\times\vec{B}
\end{equation*}

\section{Elektrischer Dipol}

Wir betrachten einen elektrischen Dipol am Ort $\vec{r}$, dessen Ladungen den Abstand $\vec{a}$ voneinander haben. Die Kraft auf ihn beträgt:

\begin{equation*}
\vec{F} \ = \ Q \cdot \vec{E}\left(\vec{r} \ + \ \frac{\vec{a}}{2}\right) \ - \ Q \cdot\vec{E} \left(\vec{r} \ - \ \frac{\vec{a}}{2}\right) \qquad (=0 \text{ für $\vec{E}$ homogen})
\end{equation*}

Wir entwickeln diesen Ausdruck für das Dipollimit $|\vec{a}| \rightarrow 0$:

\begin{align*}
\vec{F}  \ &= \ Q \cdot \left( \vec{E} (\vec{r}) \ + \ \frac{1}{2} \left(\vec{a}\cdot\pdiff{}{\vec{r}}\right) \vec{E}(\vec{r}) \ - \ \vec{E}(\vec{r}) \ + \ \frac{1}{2} \left(\vec{a}\cdot\pdiff{}{\vec{r}}\right)\vec{E}(\vec{r})\right) \\
\vec{F} \ &= \ Q\cdot \left(\vec{a}\cdot\nabla\right) \ \vec{E}  \ = \ \left( \vec{p} \cdot \nabla\right) \ \vec{E}
\end{align*}

Den Ausdruck für das Drehmoment auf einen Dipol im elektrischen Feld erhalten wir analog:

\begin{align*}
\vec{M}  \ &= \ Q \cdot \left(\frac{\vec{a}}{2} \times \vec{E}\left(\vec{r} \ + \ \frac{a}{2}\right) \ + \ \frac{\vec{a}}{2} \times \vec{E}\left(\vec{r} \ - \ \frac{\vec{a}}{2}\right)\right)\\
\vec{M}  \ &= \ Q \cdot \vec{a}\times\vec{E}  \ = \ \vec{p}\times\vec{E} 
\end{align*}

\section{Magnetischer Dipol}

Wir betrachten einen Kreisstrom $I$, dessen Mittelpunkt sich am Ort $\vec{r}$ befindet und welcher die Fläche $\vec{A}_F$ umschließt und somit ein Dipolmoment von $\vec{m} = I \cdot \vec{A}_F$ erzeugt. Die Kraft auf diesen magnetischen Dipol beträgt: ($\vec{r}'$ ist dabei ein Ort auf dem Rand des Kreisstroms)

\begin{align*}
\vec{F} \ &= \ \Int{}{}{V'} \ \vec{j}(\vec{r}') \times \vec{B}  \ = \ I \ \Oint{}{}{\vec{r}'} \times\vec{B} \qquad (= 0 \text{ für homogenes Feld})\\
\vec{F} \ &= \ I \ \Oint{}{}{\vec{r}'} \times \vec{B}(\vec{r'} - \ \vec{r}'') \qquad\qquad \text{mit }\  \vec{r}''  = \ \vec{r}' - \ \vec{r} 
\end{align*}

Wir entwickeln für $|\vec{r}''| \ll |\vec{r}|$: 

\begin{align*}
\vec{F} \ &= \ I \ \Oint{}{}{\vec{r}'} \times \left[ \underbrace{\vec{B}(\vec{r})}_{=0} \ + \ \left(\vec{r}'' \cdot\pdiff{}{\vec{r}}\right)\vec{B}(\vec{r}\right]\\
&= \ I \ \underbrace{\Oint{}{}{\vec{r}'}\times\left(\vec{r}'' \cdot \pdiff{}{\vec{r}}\right)}_{(*)}\vec{B}(\vec{r}) 
\end{align*}

Lösung des Integrals $(*)$:

\begin{align*}
\Oint{}{}{\vec{r}'} (\vec{r}''\cdot \ \vec{a}) & \quad = \quad \ \oint{}{}{\vec{r}'} \cdot (\vec{r'}- \ \underbrace{\vec{r}}_{=0}) \vec{a}\\
& \overset{\text{Kap. 5.3}}{=} \ \frac{1}{2}\oint(\vec{r}'\times\d\vec{r}')\times\vec{a} \ + \ \frac{1}{2}\Oint{}{}{\vec{r}'}(\vec{r}'\cdot\vec{a})\\
& \quad = \quad \ \vec{A}_F \times\vec{a}
\end{align*}

$\Rightarrow\quad$ Einsetzen:

\begin{align*}
\vec{F}  \ &= \ \underbrace{I \ (\vec{A}_F}_{\vec{m}} \times \nabla)\times \vec{B} \ \overset{\text{bac-cab}}{=} \ \nabla(\vec{m}\cdot\overset{\downarrow}{\vec{B}}) \ - \ \vec{m}(\underbrace{\nabla\cdot\overset{\downarrow}{\vec{B}}}_{=0})\\
\vec{F}  \ &= \ \nabla(\vec{m}\cdot\vec{B}) \ \overset{\text{bac-cab}}{=} \ \vec{m} \times (\underbrace{\nabla\times\vec{B}}_{(**)}) \ + \ (\vec{m}\cdot\nabla)\vec{B}
\end{align*}

$(**)=0$, da $\dot{\vec{E}}=0$ und $\mu_0\vec{j}\rightarrow 0$ außerhalb der Quellen.\\
Somit :

\begin{equation*}
\vec{F}  \ = \  (\vec{m}\cdot\nabla) \vec{B} \qquad\qquad (\text{vgl. el. Dipol: } \vec{F}  \ = \ (\vec{p}\cdot\nabla)\vec{E})
\end{equation*}

Für das Drehmoment auf den magnetischen Dipol gilt:

\begin{align*}
\vec{M}  \ &= \ \Int{}{}{V'} \vec{r}\times\left[\vec{j}(\vec{r}') \times \vec{B}(\vec{r}')\right] \ = \  I  \ \Oint{}{}{\vec{r}''} \times \left[\d\vec{r}' \times\vec{B}(\vec{r}')\right]\\
 &= \ I \ \Oint{}{}{\vec{r}'}(\vec{r}''\cdot\vec{B}(\vec{r}')) \ - \ I \ \oint(\d\vec{r}'\cdot\vec{r}'')\vec{B}(\vec{r}')
\end{align*}

Näherung: $\vec{B}$ ist homogen. Auflösung der Integrale:

\begin{align*}
&\Oint{}{}{\vec{r}'} \cdot \vec{r}'' \ = \ \Oint{}{}{\vec{r}'}\cdot\vec{r}'  \ = \ \oint\frac{1}{2}\d(\vec{r}'^2)  \ = \ 0\\
&\Oint{}{}{\vec{r}'} (\vec{r}''\cdot\ \vec{B})  \ = \ \vec{A}_F \times\vec{B} \qquad \text{wie in } (*)
\end{align*}

\begin{equation*}
\Rightarrow \quad \vec{M}  \ = \ I\cdot \vec{A}_F \times \vec{B} \ = \ \vec{m}\times\vec{B} \qquad\qquad (\text{vgl. el. Dipol: } \vec{M}  \ = \ \vec{p}\times\vec{E})
\end{equation*}