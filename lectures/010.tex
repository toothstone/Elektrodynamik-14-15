\chapter{Kraftwirkung auf Ladungen und Ströme}

Erinnerung: \textbf{Lorentzkraftdichte}

\begin{equation*}
\vec{f}_L \ = \ \rho \ \vec{E} \ + \ \vec{j}\times\vec{B}
\end{equation*}

\section{Elektrischer Dipol}

Wir betrachten einen elektrischen Dipol am Ort $\vec{r}$, dessen Ladungen den Abstand $\vec{a}$ voneinander haben. Die Kraft auf ihn beträgt:

\begin{equation*}
\vec{F} \ = \ Q \cdot \vec{E}\left(\vec{r} \ + \ \frac{\vec{a}}{2}\right) \ - \ Q \cdot\vec{E} \left(\vec{r} \ - \ \frac{\vec{a}}{2}\right) \qquad =0 \text{ für $\vec{E}$ homogen}
\end{equation*}

Wir entwickeln diesen Ausdruck für das Dipollimit $|\vec{a}| \rightarrow 0$:

\begin{align*}
\vec{F}  \ &= \ Q \cdot \left( \vec{E} (\vec{r}) \ + \ \frac{1}{2} \left(\vec{a}\cdot\pdiff{}{\vec{r}}\right) \vec{E}(\vec{r}) \ - \ \vec{E}(\vec{r}) \ + \ \frac{1}{2} \left(\vec{a}\cdot\pdiff{}{\vec{r}}\right)\vec{E}(\vec{r})\right) \\
\vec{F} \ &= \ Q\cdot \left(\vec{a}\cdot\nabla\right) \ \vec{E}  \ = \ \left( \vec{p} \cdot \nabla\right) \ \vec{E}
\end{align*}

Den Ausdruck für das Drehmoment auf einen Dipol im elektrischen Feld erhalten wir analog:

\begin{align*}
\vec{M}  \ &= \ Q \cdot \left(\frac{\vec{a}}{2} \times \vec{E}\left(\vec{r} \ + \ \frac{a}{2}\right) \ + \ \frac{\vec{a}}{2} \times \vec{E}\left(\vec{r} \ - \ \frac{\vec{a}}{2}\right)\right)\\
\vec{M}  \ &= \ Q \cdot \vec{a}\times\vec{E}  \ = \ \vec{p}\times\vec{E} 
\end{align*}

\section{Magnetischer Dipol}

Wir betrachten einen Kreisstrom $I$ am Ort $\vec{r}$, welcher die Fläche $\vec{A}_F$ umschließt und somit ein Dipolmoment von $\vec{m} = I \cdot \vec{A}_F$ erzeugt. Die Kraft auf diesen magnetischen Dipol beträgt:

\begin{align*}
blub
\end{align*}