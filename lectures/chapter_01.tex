\chapter{Einleitung}
Gegenstand der Vorlesung ist die (klassische) Theorie der Elektrischen Felder ausgehend von den \textsc{Maxwell}-Gleichungen (1864):

\begin{align*}
\div \vec{B}&=0 &\varepsilon_0\div \vec{E}&=\rho\\
\rot  \vec{E}+\pdiff{\vec{B}}{t}&=0 &\frac{1}{\mu_0}\rot \vec{B}-\varepsilon_0\pdiff{\vec{E}}{t}&=\vec{j}
\end{align*}


für die Felder $\vec{E}$ und $\vec{B}$ in Abhängigkeit von Ladungs- und Stromverteilung $\rho(\vec{r},t)$ und $\vec{j}(\vec{r},t)$  sollen physikalische Erscheinungen geschildert werden.\\
Die Elektrodynamik ist ein Teil des Standardmodells der Teilchenphysik, das einheitlich Teilichen und ihre Wechselwirkungen beschreibt.\\
Klassische Elektrodynamik ist ein Grenzfall der Quantenelektrodynamik (gültig für kleine Impuls- und Energiebeträge, große Brechungszahlen für Photonen).\\
Sie ist im Einklang mit der der speziellen Relativitätstheorie (c ist implizit in den \textsc{Maxwell}-Gleichungen enthalten). Viele interessante Effekte von Materie können mit klassischer Theorie nicht beschrieben werden.\\
Zum Beispiel: Wann sind Atome stabil? Wann ist Eisen ferromagnetisch? Warum wird z.B. Blei bei tiefen Temperaturen supraleitend? Für diese Fragen werden Quanteneffekte wichtig.