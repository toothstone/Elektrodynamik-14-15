\chapter{Elektromagnetische Wellen}

\section{Wellengleichung}

Bisher haben wir in der Elektrostatik und in unserer Betrachtung von Stationären Strömen nur Fälle behandelt, bei denen galt:

\begin{equation*}
\rho(\vec{r})\neq 0, \vec{j}(\vec{r})\neq 0, \dot{\vec{E}}=0, \dot{\vec{B}}=0
\end{equation*}

Jetzt wollen wir elektromagnetische Wellen im Vakuum , also ohne Quellen, betrachten. Daher muss gelten: 

\begin{equation*}
\rho=0, \vec{j}=0,\dot{\vec{E}}\neq 0,\dot{\vec{B}}\neq 0
\end{equation*}

Daraus folgt zunächst für die \textsc{Maxwell}-Gleichungen:

\begin{align*}
\div \vec{E} &=0 \qquad \qquad\div \vec{B} = 0\\
\rot  \vec{E} &= -\dot{\vec{B}} \qquad\quad \ \rot \vec{B} = \epsilon_0\mu_0 \dot{\vec{E}}\\
\end{align*}

\begin{align*}
\Rightarrow & \rot\dot{\vec{B}} = \epsilon_0\mu_0\ddot{\vec{E}} = -\rot\rot\vec{E} = -\nabla\underbrace{(\nabla\cdot\vec{E})}_{\div\vec{E}=0} \ + \ \nabla^2 \vec{E}\\
\ \\
\overset{\epsilon_0\mu_0 = \frac{1}{c^2}}{\Rightarrow} &\qquad \left(\frac{1}{c^2} \ \pddiff{}{t} \ - \ \bigtriangleup\right)\vec{E} =: \Box\vec{E} = 0
\end{align*}

Die erhaltene partielle Differentialgleichung ist die sogenannte \textbf{Wellengleichung}, welche sich auch analog für das $\vec{B}$-Feld herleiten lässt. Das Symbol $\Box$ wird auch als \textbf{Wellen-} oder \textbf{\textsc{D'Alembert}-Operator} bezeichnet.


\section{Lösungen der Wellengleichungen}

$\Box U = 0$

\begin{enumerate}\bfseries
\item \textbf{eindimensionale Lösung} ($\vec{r}\  \rightarrow \ x$)
\begin{equation*}
\left(\frac{1}{c^2} \ \pddiff{}{t} \ - \ - \pddiff{}{x}\right)\ U(x,t) \ = \ 0 \ = \ \left(\frac{1}{c}\ \partial_t \ - \ \partial_x\right)\left(\frac{1}{c} \ \partial_t \ + \ \partial_x\right) \ U(x,t)
\end{equation*}

\end{enumerate}