\documentclass[a4paper,12pt,portrait]{book}
\title{Theoretische Elektrodynamik}
\author{Matthias Vojta\\ \\ \"ubertragen von\\Sebastian Schmidt und Lukas K\"orber}
\date{Wintersemester 2014/2015}

  
\usepackage[ngerman]{babel}
\usepackage[utf8]{inputenc}
\usepackage{amsmath}
\usepackage{mathtools}
\usepackage{amssymb}
\usepackage{calc}
\usepackage{fancyhdr}
\usepackage{xcolor}
\usepackage{fourier}
\usepackage[frak=mma]{mathalfa}
\usepackage{sectsty}
\usepackage{empheq}
\usepackage{bm}
\usepackage{bbm}
\usepackage{enumitem}
\usepackage{eucal}
\usepackage{ulem}
\usepackage{relsize}
\usepackage{nicefrac}
\usepackage{wrapfig}
\usepackage{tikz}
\usetikzlibrary{patterns}
\tikzset{
	partial ellipse/.style args={#1:#2:#3}{
		insert path={+ (#1:#3) arc (#1:#2:#3)}
	}
}


%layout options
\definecolor{dblue}{HTML}{183CE0}
\definecolor{hblue}{HTML}{0E75F7}
\definecolor{hgrey}{HTML}{FAFAFA}
\definecolor{bblue}{HTML}{CBE3FD}
\sectionfont{\color{hblue}}
\chapterfont{\color{dblue}}
\pagestyle{headings}
\setlength\parindent{0pt}


%defined commands
%differential operators
\newcommand{\diff}[2]{\frac{\mathrm{d}{#1}}{\mathrm{d}{#2}}}
\newcommand{\ddiff}[2]{\frac{\mathrm{d}^2{#1}}{\mathrm{d}{#2}^2}}
\newcommand{\dddiff}[2]{\frac{\mathrm{d}^3{#1}}{\mathrm{d}{#2}^3}}
\newcommand{\ndiff}[2]{\frac{\mathrm{d}^n{#1}}{\mathrm{d}{#2}^n}}
\newcommand{\pdiff}[2]{\frac{\partial{#1}}{\partial{#2}}}
\newcommand{\pddiff}[2]{\frac{\partial ^2{#1}}{\partial{#2}^2}}

\newcommand{\tens}[1]{\bm{\hat{#1}}}
\renewcommand{\vec}[1]{\bm{#1}}
\renewcommand{\d}{\mathrm{d}}

\newcommand{\Int}[3]{\int\limits_{{#1}}^{{#2}}\mathrm{d}{#3}}
\newcommand{\Iint}[3]{\iint\limits_{{#1}}^{{#2}}\mathrm{d}{#3}}
\newcommand{\Iiint}[3]{\iiint\limits_{{#1}}^{{#2}}\mathrm{d}{#3}}
\newcommand{\Oint}[3]{\oint\limits_{{#1}}^{{#2}}\mathrm{d}{#3}}
\newcommand{\Oiint}[3]{\oiint\limits_{{#1}}^{{#2}}\mathrm{d}{#3}}

\setlist[enumerate,1]{label =\textbf{\roman*)}}
\setlist[enumerate,2]{label =\alph*)}
\setcounter{chapter}{-1}
	
\begin{document}
\maketitle
\tableofcontents

\chapter{Einleitung}
Gegenstand der Vorlesung ist die (klassische) Theorie der Elektrischen Felder ausgehend von den \textsc{Maxwell}-Gleichungen (1864):

\begin{align*}
\div \vec{B} \ &= \ 0    &\epsilon_0 \ \div \vec{E} \ &= \ \rho\\
\rot \vec{E} \ + \ \pdiff{\vec{B}}{t} \ &= \ 0   &\frac{1}{\mu_0} \ \rot \vec{B} \ - \ \epsilon_0 \ \pdiff{\vec{E}}{t} \ &= \ \vec{j}
\end{align*}

für die Felder $\vec{E}$ und $\vec{B}$ in Abhängigkeit von Ladungs- und Stromverteilung $\rho(\vec{r},t)$ und $\vec{j}(\vec{r},t)$. Von dieser Grundlage aus wollen wir in dieser Vorlesung die physikalischen Erscheinungen für diese Felder schildern und diskutieren.\\
Die Elektrodynamik ist ein Teil des Standardmodells der Teilchenphysik, das einheitlich Teilchen und ihre Wechselwirkungen beschreibt.\\
Klassische Elektrodynamik ist ein Grenzfall der Quantenelektrodynamik (gültig für kleine Impuls- und Energiebeträge, große Brechungszahlen für Photonen).\\
Sie ist im Einklang mit der der speziellen Relativitätstheorie, da $c$ implizit mit in den \textsc{Maxwell}-Gleichungen enthalten ist. Viele interessante Effekte von Materie können jedoch mit der klassischen Theorie nicht beschrieben werden.\\
Zum Beispiel: Wann sind Atome stabil? Wann ist Eisen ferromagnetisch? Warum wird z.B. Blei bei tiefen Temperaturen supraleitend? Für diese Fragen werden Quanteneffekte wichtig.
\chapter{Mathematische Hilfsmittel}

\section{Skalar- und Vektorfelder}

Felder sind Größen, welche an jedem Raumpunkt einen bestimmten Wert haben, der zudem noch zeitabhängig sein kann.\\ \linebreak


\begin{wrapfigure}[]{r}[0cm]{0cm}
	\raisebox{0pt}[\dimexpr\height-1\baselineskip\relax]{
		\colorbox{hgrey}{
			\begin{tikzpicture}
			\draw[->](0,0)-- (4,0) node[right]{$x$};
			\draw[->](0,0)--(0,3) node[above]{$y$};
			\draw (3,3) node[right]{$T(x,y)$};
			\draw plot[smooth, tension=.8] coordinates {(0.5,1.5)(2,0.7)(3.5,0.4)} node[right]{$T=20^{\circ}$} ;
			\draw plot[smooth, tension=.8] coordinates {(0.5,2.2)(2,1.1)(3.5,0.8)} node[right]{$T=40^{\circ}$} ;
			\draw plot[smooth, tension=.8] coordinates {(0.5,3)(2,1.6)(3.5,1.2)} node[right]{$T=60^{\circ}$} ;
			\end{tikzpicture}
		}
	}
	\caption{Isothermen}
\end{wrapfigure}


i)\textbf{ skalare Felder}\ $\phi=\phi(x,y,z,t)$\\ 


Jedem Raumpunkt wird ein Wert in Form einer (reellen) Zahl zugeordnet, wie zum Beispiel Temperatur, Druck, Ladung oder Energie. Flächen oder Linien mit konstantem Wert nennt man Äquipotentialflächen beziehungsweise -linien.\\ \linebreak\linebreak

\begin{wrapfigure}[]{r}[0cm]{0cm}
	\raisebox{0pt}[\dimexpr\height-1\baselineskip\relax]{
		\colorbox{hgrey}{
			\begin{tikzpicture}
			
			\draw[->](0,0)-- (4,0) node[right]{$x$};
			\draw[->](0,0)--(0,3) node[above]{$y$};
			\draw (3,3) node[right]{$\vec{F}(x,y)$};
			\foreach \x/\angle in {0.5/20, 1.5/40} {
				\foreach \y in {0.5, 1.5, 2.5} {
					\fill (\x,\y) circle[radius=1pt];
					\draw[->,thick]  (\x, \y) -- ++(\angle:1);
				}
			}
			\foreach \x/\angle in {2.5/60, 3.5/80} {
				\foreach \y in {0.5, 1.5} {
					\fill (\x,\y) circle[radius=1pt];
					\draw[->,thick]  (\x, \y) -- ++(\angle:1);
				}
			}
			\end{tikzpicture}
		}
	}
	\caption{Kraftfeld}
\end{wrapfigure}


ii)\textbf{ Vektorfelder}\ $\vec{E}=\vec{E}(x,y,z,t)$\\

Jedem Raumpunkt wird ein Vektor zugeordnet, der lokal die Richtung des Feldes beschreibt, wie etwa ein Geschwindigkeits- oder Kraftfeld. Vektorfelder lassen sich durch Feldlinien veranschaulichen, entlang derer sich zum Beispiel ein Teilchen bewegt, das die entsprechende Kraft erfährt.


\section{Integrale auf Feldern}
Integrale über skalare Felder werden wie bekannt gebildet.\\
Integriert man über ein Vektorfeld, spielt die Richtungsinformation eine entscheidende Rolle. Man unterscheidet je nach Dimension des Parameterbereichs von Linien-, Flächen- und Volumenintegralen.\\
\linebreak


i) \textbf{Linienintegrale}

\begin{equation*}
\varphi=\int\limits_{C}\vec{E}(\vec{r})\d\vec{r}
\end{equation*}

\begin{wrapfigure}[]{r}[0cm]{0cm}
	\raisebox{0pt}[\dimexpr\height-1\baselineskip\relax]{
		\colorbox{hgrey}{
			\begin{tikzpicture}
			
			\draw[->](0,0)-- (4,0) node[right]{$x$};
			\draw[->](0,0)--(0,3) node[above]{$y$};
			\draw[->](1.2,0.8)--(1,2) node[above]{$\vec{E}$};
			\draw[->](2,2.3)--(2.4,3) node[right]{$\vec{E}$};
			\draw(2,1.8) node{$C$};
			\draw[thick] plot[smooth, tension=.8] coordinates {(1.2,0.8)(2,2.3)(3.5,3)};
			\filldraw[black] (3.5,3) circle (2pt) node[right]{$\vec{r}(\tau_2)$};
			\filldraw[black] (1.2,0.8) circle (2pt) node[below left]{$\vec{r}(\tau_1)$};
			
			\end{tikzpicture}
		}
	}
	\caption{Linienintegral}
\end{wrapfigure}
Wir parametrisieren die Kurve durch $\vec{r}=\vec{r}(\tau)$ und erhalten somit
\begin{equation*}
\varphi=\int\limits_{\tau_0}^{\tau_1}\vec{E}(\vec{r}(\tau))\diff{\vec{r}}{\tau}\d \tau
\end{equation*}
Ein Speziallfall des Linienintegrals ist das sogenannte \textbf{geschlossene Linienintegral}, welches durch $\oint$ gekennzeichnet wird.\\
\linebreak


\begin{wrapfigure}[10]{r}[0cm]{0cm}
	\raisebox{0pt}[\dimexpr\height-1\baselineskip\relax]{
		\colorbox{hgrey}{
			\begin{tikzpicture}
			
			\draw plot[smooth, tension=.8] coordinates {(3,0) (5,1.4) (1.5,2) (0,1) (0,0) (3,0) };	
			\draw[->] (2,1)--(2,3) node[left]{$\Delta\vec{A}_i$};
			\draw[->] (2,1)--(3,2.8) node[right]{$\vec{E}(\vec{r}_i)$};
			\draw (0.5,0.5) node[right]{$\mathcal{F}$};
			
			
			\end{tikzpicture}
		}
	}
	\caption{Flächenintegral}
\end{wrapfigure}


ii) \textbf{Flächenintegrale}\\
\begin{equation*}
\Phi=\iint\limits_{S}\vec{B}\cdot\d \vec{A} \ \ \ \text{mit } \d\vec{A}=\d A\cdot\vec{n}
\end{equation*}


Ganz analog zu i) kann die Fläche $\vec{r}=\vec{r}(u,v)$ parametrisiert werden. Es ist jedoch beim Bilden der Funktionaldeterminante auf die Richtung des Flächenelements zu achten. Die beiden möglichen Lösungen unterscheiden sich natürlich nur um ein Vorzeichen. Wir erhalten also
\begin{equation*}
\Phi=\int\limits_{v_1}^{v_2}\int\limits_{u_1}^{u_2}\vec{B}(u,v)\cdot\left(\pdiff{\vec{r}}{u}\times\pdiff{\vec{r}}{v}\right)\d u\d v
\end{equation*}

iii) \textbf{Volumenintegrale}\\
\linebreak
\begin{equation*}
Q=\iiint\limits_G\d V\cdot\rho(\vec{r})=\iiint\limits_G\d ^3 r\cdot\rho(\vec{r})=
\end{equation*}
Beim Volumenintegral wird wiederum (nicht wie beim Flächenintegral) das Vorzeichen des Volumenelements vernachlässigt, da physikalisch die \textbf{Richtung} des Volumens nur sehr selten wirklich von Bedeutung ist. Mit entsprechender Parametrisierung $\vec{r}=\vec{r}(u,v,w)$ ergibt sich
\begin{equation*}
q=\int\limits_{w_1}^{w_2}\int\limits_{v_1}^{v_2}\int\limits_{u_1}^{u_2}\rho(u,v,w)\cdot\left|\pdiff{\vec{r}}{u}\cdot\left(\pdiff{\vec{r}}{v}\times\pdiff{\vec{r}}{w}\right)\right|\d u\d v\d w
\end{equation*}

\section{Vektorielle Ableitungen und Integrale}

\begin{enumerate}
\item \textbf{Gradient}\\

Der Gradient $\grad\varphi\ $ eines Skalarfeldes beschreibt dessen Änderung und steht senkrecht auf den Äquipotentialflächen (oder allgemeiner: Niveaumengen). Der Gradient lässt sich durch den \textbf{Nabla-Operator} ausdrücken welcher in kartesischen Koordinaten lautet:
 
\begin{equation*}
\nabla=\pdiff{}{x}\vec{e}_x+\pdiff{}{y}\vec{e}_y+\pdiff{}{z}\vec{e}_z
\end{equation*}

Wichtig ist, dass $\nabla$ ein vektorieller Differenzialoperator ist. Er folgt Ableitungsregeln, wie etwa der Kettenregel, und $\nabla\varphi$ verhält sich unter Koordinatentransformation wie ein Vektor.\\
\linebreak
Andere Schreibweisen: $\pdiff{}{\vec{r}},\ \partial_{\vec{r}},\ \nabla_{\vec{r}}$\\
\linebreak

\underline{Beispiele:}

\begin{align*}
\nabla |\vec{r}|  \ &= \ \frac{\vec{r}}{|\vec{r}|} \ = \ \vec{e}_r\\
\nabla \frac{1}{|\vec{r}|} \ &= \ -\frac{1}{r^2}\vec{e}_r
\end{align*}

\ \\
\item \textbf{Divergenz} (Quellenstärke eines Vektorfeldes)\\

Die Divergenz $\div\vec{E}=\nabla\cdot\vec{E}$ ist ein Skalar unter Koordinatentransformation und kann als \textbf{lokale Quellenstärke} interpretiert werden. Häufig benötigt man auch den \textsc{Laplace}-Operator, der die zweite Ableitung repräsentiert.\\

\begin{equation*}
\div\grad \varphi \ = \ \nabla^2\varphi \ = \ \laplace\varphi
\end{equation*}

\underline{Beispiele:}

\begin{align*}
\div\vec{r}\ &= \ 3 \quad\text{(Anzahl der Dimensionen)}\\
\div(\varphi\vec{A}) \ &= \ \nabla\cdot(\varphi\vec{A}) \ = \ \vec{A}(\nabla\varphi)+\varphi(\nabla\vec{A}) \ = \ \vec{A}\cdot\grad \varphi+\varphi\cdot\div\vec{A}
\end{align*}

\ \\
\item \textbf{Rotation} (Wirbelstärke eines Vektorfeldes)\\

Die Rotation $\rot\vec{B}=\nabla\times\vec{B}$

\begin{equation*}
\nabla\times\vec{B}=\begin{vmatrix}
\vec{e}_x & \vec{e}_y & \vec{e}_z \\
\pdiff{}{x} & \pdiff{}{y} & \pdiff{}{z}\\
B_x & B_y & B_z
\end{vmatrix}
\end{equation*}

kann als \textbf{lokale Wirbelstärke} verstanden werden. Ihre Komponenten lassen sich auch als

\begin{equation*}
(\nabla\times\vec{B})_i=\sum\limits_{j,k}\varepsilon_{ijk}\cdot\pdiff{}{x_j}\cdot B_k
\end{equation*}

darstellen wobei $\varepsilon_{ijk}$ der total antisymmetrische Tensor 3. Stufe ist.\\
\newpage

\underline{Beispiele:}

\begin{align*}
&\vec{v} \ = \ \vec{\omega}\times\vec{r} \quad \Rightarrow \quad \nabla\times\vec{v} \ = \ 2\vec{\omega}\\
&\nabla\times\vec{r} \ = \ 0
\end{align*}

\ \\
\item \textbf{\textsc{Gauss}'scher Satz}\\

\begin{equation*}
\iiint\limits_V\div\vec{E}\cdot\d V=\oiint\limits_{\partial V}\vec{E}\cdot\d \vec{A}
\end{equation*}

Der Satz von \textsc{Gauss} verknüpft Eigenschaften im Inneren eines Volumens mit dem Verhalten auf dem Rand.\\

Über den Satz von \textsc{Gauss} lässt sich auch die partielle Integration in drei Dimensionen umformen zu:

\begin{equation*}
\Int{V}{}{V} \ \pdiff{}{\vec{r}} (u\cdot v) = \Int{V}{}{V} \  \pdiff{u}{\vec{r}}\cdot v \ + \ \Int{V}{}{V} \  u \cdot\pdiff{v}{\vec{r}} \ = \ \Oiint{\partial V}{}{A} \  (u\cdot v)
\end{equation*}

\ \\
\item \textbf{\textsc{Green}'scher Satz}\\

\begin{equation*}
\Int{V}{}{V}(\varphi\laplace\psi-\psi\laplace\varphi) \ = \ \Oint{\partial V}{}{\vec{A}_F}(\varphi\nabla\psi-\psi\nabla\varphi)
\end{equation*}

\ \\
\item \textbf{\textsc{Stokes}'scher Satz}\\

\begin{equation*}
\iint\limits_S\rot\vec{B}\cdot\d\vec{A}=\oint\limits_{\partial A}\vec{B}\cdot\d\vec{r}
\end{equation*}

Analog zum \textsc{Gauss}'schen Satz verknüpft der Satz von \textsc{Stokes} das Verhalten eines Feldes auf einer Fläche mit dem auf dem Rand der Fläche. Für geschlossene Flächen gilt:

\begin{equation*}
\oiint\limits_{S=\partial V}\rot\vec{B}\cdot\d\vec{A}=0
\end{equation*}
\end{enumerate}

\section{Differentialoperatoren in krummlinigen Koordinaten}

Hier sind vor allem Kartesische Koordinaten als auch Kugel- und Zylinderkoordinaten wichtig.\\

\begin{align*}
\text{Kartesische Koordinaten:}\qquad &\nabla\psi \ = \ \pdiff{}{x}\psi\vec{e}_x + \pdiff{}{y}\psi\vec{e}_y + \pdiff{}{z}\psi\vec{e}_z\\
\text{Zylinderkoordinaten:}\qquad &\nabla\psi \ = \ \pdiff{}{\rho}\psi\vec{e}_\rho + \frac{1}{\rho}\pdiff{}{\phi}\psi\vec{e}_\phi + \pdiff{}{z}\psi\vec{e}_z\\
\text{Kugelkoordinaten:} \qquad &\nabla\psi \ = \ \pdiff{}{r}\psi\vec{e}_r+\frac{1}{r}\pdiff{}{\theta}\psi\vec{e}_\theta+\frac{1}{r\sin\theta}\pdiff{}{\phi}\psi\vec{e}_\phi\\
\\
\text{Generell:}\qquad &(\nabla\psi)_u \ \equiv \ (\nabla\psi)\vec{e}_u \ = \ \frac{1}{g_u}\pdiff{\psi}{u} \text{ mit } g_u \ = \ \left|\pdiff{\psi}{u}\right|
\end{align*}


\section{\textsc{Fourier}-Transformation}

\begin{equation*}
\tilde{f}(\omega) \ = \ \frac{1}{\sqrt{2\pi}}\Int{-\infty}{\infty}{t} f(t)e^{-i\omega t}
\end{equation*}

\begin{equation*}
f(t) \ = \ \frac{1}{\sqrt{2\pi}}\Int{-\infty}{\infty}{\omega} \tilde{f}(t)e^{i\omega t}
\end{equation*}

Verallgemeinert auf $n$ Dimensionen ergibt sich:\\

\begin{equation*}
\tilde{f}(\vec{k}) \ = \ \frac{1}{({2\pi})^{\frac{n}{2}}}\Int{-\infty}{\infty}{^n r}f(\vec{r})e^{-i\vec{k}\vec{r}}
\end{equation*}

\ \linebreak

\begin{enumerate}

\item \textbf{Differentiation}\\

\begin{equation*}
\diff{}{t}f(t)=\frac{1}{\sqrt{2\pi}}\Int{-\infty}{\infty}{\omega}i\omega\tilde{f}(\omega)e^{i\omega t}
\end{equation*}

\ \linebreak
\item \textbf{Faltung}\\

\begin{equation*}
(f*g)(t)=\frac{1}{\sqrt{2\pi}}\Int{-\infty}{\infty}{s}f(t-s)G(s)
\end{equation*}

\begin{equation*}
\widetilde{(f*g)}(\omega)=\tilde{f}(\omega)\tilde{g}(\omega)
\end{equation*}

\ \\
\item \textbf{Rechenregeln}\\

\begin{align*}
f'(t) & \leftrightarrow  i\omega\tilde{f}(\omega)\\
-itf(t) & \leftrightarrow \tilde{f}'(\omega)\\
f(t+a) & \leftrightarrow  e^{i\omega a}\tilde{f}(\omega)\\
e^{i\omega t}f(t) &\leftrightarrow  \tilde{f}(\omega-a)\\
f(at) & \leftrightarrow \frac{1}{|a|}\tilde{f}\left(\frac{\omega}{a}\right)\\
f^*(t) & \leftrightarrow  \tilde{f}^*(\omega)\\
\tilde{\tilde{f}}(t) & \leftrightarrow  f(-t)\\
\end{align*}
\end{enumerate}

\section{Delta-Distribution}

Die Delta-Distribution ist über folgende Eigenschaften definiert:

\begin{enumerate}
\item
\begin{equation*}
\delta(\vec{r}) = \begin{cases}
0 & \text{für }\vec{r}\neq\vec{r}_0\\
\infty & \text{für } \vec{r} = \vec{r}_0
\end{cases}
\end{equation*}

\item
\begin{equation*}
\int\limits_{\vec{r}_0\in V}\d V \ \delta({\vec{r}-\vec{r}_0}) = 1
\end{equation*}
\end{enumerate}

Alle Aussagen gelten analog für die Delta-Distribution $\delta(x)$ in einer Dimension.\\
Bei höherdimensionalen Deltadistributionen gilt allerdings nur in kartesischen Koordinaten:

\begin{equation*}
\delta(\vec{r} - \vec{r}_0) = \delta(x-x_0)\cdot\delta(y-y_0)\cdot\delta(z-z_0)
\end{equation*}
\ \\
Faltet man die Delta-Distribution mit einer Funktion $f(\vec{r})$, so ergibt sich aus ihren Eigenschaften:

\begin{equation*}
\int\limits_{\vec{r}_0\in V}\d V \ \delta({\vec{r}-\vec{r}_0}) \ f(\vec{r}) = f(\vec{r}_0)
\end{equation*}\\
\ \\

\section[\textsc{Green}'sche Funktion]{\textsc{Green}'sche Funktion zur Lösung inhomogener linearer DGL}

Wir betrachten die lineare, inhomogene Differentialgleichung

\begin{equation*}
L \ \phi (x_1,\dotsc,x_n) = \rho (x_1,\dotsc,x_n) \; \text{ oder kurz } \; L\phi = \rho
\end{equation*}

wobei $L$ ein linearer Operator und $\rho$ die Inhomogenität sein soll.\
\\
Die \textsc{Green}'sche Funktion $G(x,x)$ zum Operator $L$ ist die Lösung der Differentialgleichung mit $\delta$-förmiger Inhomogenität.

\begin{equation*}
L \ G(x,x') = \delta (x-x') \; [= \delta(x_1-x_1')\cdot\dotsc\cdot\delta(x_n - x_n')]
\end{equation*}
\newpage
Wenn $g$ bekannt ist, dann kann die Lösung für beliebige Inhomogenität durch Superposition gewonnen werden.

\begin{equation*}
\phi (x) = \int\d x' \ G(x,x') \rho(x')
\end{equation*}

Den Beweis hierfür erhält man leicht durch Einsetzen:

\begin{equation*}
L \ \phi(x) = \int\d x' \ L \ G(x,x') \rho(x') = \rho(x)
\end{equation*}

\chapter{Grundbegriffe und \textsc{Maxwell}-Gleichungen}
\section{Kräfte und Punktladungen}

Aus der Erfahrung ergibt sich für eine ruhende Ladung

\begin{equation*}
\vec{F}(\vec{r},t)=Q\cdot\vec{E}(\vec{r},t)
\end{equation*}

Dabei ist die Ladung $Q$ eine Körpereigenschaft und $\vec{E}$ eine Eigenschaft, die die Umwelt charakterisiert.\\
\ \\
Bei bewegten Ladungen beobachten wir etwas anderes. Die Kraft hat hier die Form

\begin{equation*}
\vec{F}=Q(\vec{E}+\vec{v}\times\vec{B})
\end{equation*}
\linebreak

\section{Ladungs- und Stromdichte, Ladungserhaltung}

Über eine Ladung in einem Volumenelement lässt sich der Begriff der \textbf{Ladungsdichte} definieren.

\begin{equation*}
\rho(\vec{r},t)=\diff{Q}{V}
\end{equation*}
\newpage
Eine Ladungsänderung nennen wir schließlich den elektrischen Strom.

\begin{equation*}
-I:=\dot{Q}=\diff{}{t}\Int{V}{}{\rho(\vec{r},t)}{V}=\Int{V}{}{\pdiff{\rho}{t}}{V}
\end{equation*}

Betrachten wir nun den Stromfluss durch ein Oberflächenelement d$\vec{A}$. Die Ladungsträger, welche durch diese Fläche wandern haben die Geschwindigkeit $\vec{v}$, sodass anschaulich ein kleines Volumenelement $\d V = \vec{v}\mathrm{d}t\cdot\mathrm{d}\vec{A}$ aufgespannt wird:

\begin{align*}
\mathrm{d}Q &= \rho(\vec{r},t) \ \vec{v}(\vec{r},t) \ \mathrm{d}t \ \mathrm{d}\vec{A}\\
\diff{Q}{t} =-I &= \rho\vec{v}\cdot\mathrm{d}\vec{A}=:\vec{j}(\vec{r},t)\cdot\mathrm{d}\vec{A}
\end{align*}

Wir nennen $\vec{j} = \rho\vec{v}$ der Anschaulichkeit nach die \textbf{Stromdichte}, denn man sieht leicht:

\begin{equation*}
\iint\limits_A\vec{j}\cdot\mathrm{d}\vec{A}=I
\end{equation*}

Setzen wir nun dies in die Gleichung für die Ladungserhaltung ein:

\begin{equation*}
0 = \dot{Q} + I = \iiint\limits_V\mathrm{d}V\pdiff{\rho}{t} \ + \oiint\limits_{\partial V}\mathrm{d}\vec{A}\cdot\vec{j} = \iiint\limits_V \mathrm{d}V\left(\pdiff{\rho}{t} + \pdiff{\vec{j}}{\vec{r}}\right) 
\end{equation*}

Da dies für für alle möglichen Volumina gelten soll, folgt daraus die \textbf{Kontinuitätsgleichung}:
\begin{empheq}[box=\highlightbox]{equation*}
\dot{\rho} + \div\vec{j} = 0 \vphantom{\bigg|}
\end{empheq}

Für den Grenzfall eines unendlich großen Volumens gilt zunächst $\vec{j}\rightarrow 0$ auf der Oberfläche, woraus man auf die für diesen Grenzfall logische Konsequenz schließen kann, dass

\begin{equation*}
\dot{Q} = -\oiint\vec{j}\mathrm{d}\vec{A} = 0
\end{equation*}

die Ladung im gesamten Raum erhalten ist.\\
Mit der eingeführten Stromdichte $\vec{j}$ kann man nun auch den Ausdruck der \textbf{\textsc{Lorentz}kraft-Dichte} $\vec{f} := \frac{\vec{F}}{V}$ definieren:

\begin{align*}
\mathrm{d}\vec{F} & \ = \ \mathrm{d}Q(\vec{E} + \vec{v}\times\vec{B}) \\
\alignedbox{\hphantom{a}\Rightarrow \vec{f}}{ \ = \ \rho(\vec{r},t)\cdot\left(\vec{E}(\vec{r},t) + \vec{v}(\vec{r},t)\times\vec{B}(\vec{r},t)\right) \ = \ \rho\vec{E}+ \vec{j}\times\vec{B}\hphantom{a}\vphantom{\Bigg|}}
\end{align*}

\section{Die \textsc{Maxwell}-Gleichungen}

Die \textsc{Maxwell}-Gleichungen wurden 1864 vom schottischen Physiker James Clerk \textsc{Maxwell} aufgestellt und bilden ein Differentialgleichungssystem für die Felder  $\vec{B}(\vec{r})$ und $\vec{E}(\vec{r})$. Zusammen mit der Kontinuitätsgleichung beschreiben sie die gesamte (klassische) Elektrodynamik, da $\rho$ und $\vec{j}$ die Quellen und Wirbel des $ \ \vec{B} \ $- und $ \ \vec{E} \ $-Feldes eindeutig bestimmen:

\begin{align*}
\div \vec{B} &= 0 \qquad\qquad\qquad\quad \epsilon_0\div \vec{E} = \rho \\
\rot \vec{E} + \dot{\vec{B}} &= 0 \qquad\qquad \frac{1}{\mu_0}\text{rot} \ \vec{B} - \epsilon_0\dot{\vec{E}} = \vec{j}
\end{align*}


Nun könnte man fragen, ob die  Beschreibung der Elektrodynamik über lokale Felder denn zweckmäßig ist oder ob man sie nicht eliminieren könnte. Das \textsc{Coulomb}-Gesetz wäre ein Beispiel für diese Fernwirkungstheorie. Zwei Gründe sprechen für die lokale Feldtheorie: sie ist zum einen schlichtweg einfacher mathematisch zu beschreiben und zum anderen unabhängig vom Vorhandensein von Materie und demzufolge Ladungsträgern.

\section{Konstruktion der \textsc{Maxwell}-Gleichungen}
Versucht man die Elektrodynamik zu beschreiben, so kann man sich zu Beginn von phänomenologischen Seite diesem Problem nähern und fordern, dass Symmetrien in Zeit und Raum die Gültigkeit der Gleichungen erhalten sollen. Dies ist eine gängige physikalische Vorgehensweise; man verlangt, dass die beschriebene (reale) Physik unabhängig von der Wahl der Koordinaten sein soll.
Wir fordern also zunächst, dass die die Form der Gleichungen unter den Symmetrietransformationen der Rauminversion $(\vec{r}\rightarrow-\vec{r})$ und der zeitlichen Reversibilität ($t\rightarrow-t$) invariant ist. Zudem wollen wir uns als Ziel setzen, die Gesetze möglichst einfach zu formulieren, das heißt, es sollen maximal Differentialgleichungen 1. Ordnung auftauchen.
Betrachten wir nun also zunächst das Transformationsverhalten verschiedener Objekte:\ \\
\ \\


\begin{tabular}{c|c|c|l}
Objekt & $t\rightarrow-t$ & $\vec{r}\rightarrow-\vec{r}$ & Bemerkung\\
\hline $t, \pdiff{}{t}$ & - & + & Definition\\
$\vec{r},\pdiff{}{\vec{r}}$ & + & -& Definition\\
$\dot{\vec{r}}$ & -& - & durch Multiplikation der Vorzeichen erhalten\\
$\ddot{\vec{r}},\vec{F},\vec{f}$ & + & - & Erfahrung aus Mechanik:\ $\ddot{\vec{r}}=\frac{\vec{F}}{m}$\\
$Q,\rho$ & + & + & Annahme\\
$\vec{j}\ (=\rho\cdot\dot{\vec{r}})$ & - & - & \\
$\vec{E}$ & + & - & Vektor, erhalten aus: $\vec{F}=Q(\vec{E}+\vec{v}\times\vec{B})$\\
$\vec{B}$ & - & + & Pseudovektor\\
$\pdiff{}{\vec{r}}\cdot\vec{E}$ & + & + & Skalar\\
$\pdiff{}{\vec{r}}\cdot\vec{B}$ & - & - & Pseudoskalar\\
$\pdiff{}{\vec{r}}\times\vec{E}$ & + & + & Pseudovektor\\
$\pdiff{}{\vec{r}}\times\vec{B}$ & - & - & Vektor\\
$\pdiff{}{t}\vec{E}$ & - & - & Vektor\\
$\pdiff{}{t}\vec{B}$ & + & + & Pseudovektor
\end{tabular}
\ \\
\ \\
\ \\
\ \\
Da wir gefordert hatten, dass unsere gewünschten Gleichungen invariant unter den Transformationen sein sollten, dürfen wir nun nur die Größen mit dem gleichen Transformationsverhalten verknüpfen:\\
\ \\


\begin{enumerate}
\item
\underline{$++$ Skalar} \quad $\rho,\div \vec{E}\\
\\
\Rightarrow \rho = \epsilon_0 \cdot \div\vec{E} \hspace{4cm}  (\epsilon_0$  ist beliebige Konstante)

\item
\underline{$--$ Vektor} \quad $\vec{j}, \rot \vec{B},\dot{\vec{E}}\\
\\
\Rightarrow \vec{j} = \alpha\cdot\dot{\vec{E}} + \frac{1}{\mu_0}\cdot\rot \vec{B} \hspace{2.8cm} (\alpha,\frac{1}{\mu_0}$ sind beliebige Konstanten)

\item
\underline{$--$ Skalar} \quad $\div \vec{B}\\ 
\\
\Rightarrow \ 0 = \div \vec{B}$

\item
\underline{$++$ Vektor} \quad $\rot\vec{E},\dot{\vec{B}}\\
\\
\Rightarrow \ 0 = \rot \vec{E} + \beta\cdot\dot{\vec{B}} \hspace{3.4cm}(\beta$  ist beliebige Konstante)\qquad
\newpage

\item
\underline{$+-$ Vektor}  \quad $\vec{E},(\vec{r},\ddot{\vec{r}})\\
\\
\Rightarrow \ 0 =  \vec{E}$

\item
\underline{$-+$ Vektor} \quad $\vec{B}\\
\\
\Rightarrow \ 0 = \vec{B}$
\end{enumerate}
\ \\

Das System i)-iv) ist ein widerspruchsfreies und vollständiges System von Differentialgleichungen für das $\vec{E}$- und das $\vec{B}$-Feld, da diese durch ihre Quellen und Wirbel jeweils eindeutig bis auf Konstanten bestimmt sind. Diese werden problemabhängig aus den gegebenen Randbedingungen bestimmt. Die Gleichungen v) und vi) werden aus naheliegenden Gründen weggelassen; sie stehen zwar nicht im Widerspruch zu den ersten 4 Gleichungen, doch würde das Differentialgleichungssystem mit ihnen nur noch die Triviallösung ohne physikalisch interessante Bedeutung liefern.\\
\ \\

\underline{\textbf{Konstantendiskussion:}}
\ \\
\begin{enumerate}
\item Die Konstante $\epsilon_0$ ist zunächst frei wählbar, da die Ladung $Q$ nur bis auf einen Faktor genau bestimmt ist. Für die Wahl von $\epsilon_0$ gibt es verschiedene Ansätze:\
\begin{enumerate}
\item $\epsilon_0$ wird als 1 definiert. Diese Defintion wird im cgs-System umgesetzt.\
\\
\item $4\pi\cdot\epsilon_0$ wird 1 gesetzt. Das sich aus dieser Definition ergebende Einheitensystem nennt man das \textsc{Gauss}-System.\
\end{enumerate}
\ \\
Im SI-System wird dagegen $\epsilon_0$ über $\mu_0$ festgelegt, wobei für $\mu_0$ gilt:

\begin{equation*}
[\mu_0] = \frac{[\vec{E}]}{[I]}\frac{[l]^2}{[l]}=\frac{[\vec{f}]}{[\vec{j}]}\frac{[l]}{[I]}=\frac{[\vec{F}]}{[I]^2}=\frac{N}{A^2}
\end{equation*}
\begin{equation*}
\mu_0 = 4\pi\cdot 10^{-7} \frac{N}{A^2}
\end{equation*}

$\epsilon_0$ erhält man nun daraus über die Fundamentalbeziehung im SI-System:
\begin{equation*}
\epsilon_0\mu_0 = \frac{1}{c^2}
\end{equation*}

\item
Die Konstante $\alpha$ erhalten wir, in dem wir von Gleichung (2) die Divergenz bilden und dann div $\vec{j}$ aus der Kontinuitätsgleichung einsetzen:

\begin{equation*}
(\epsilon_0 + \alpha) \ \pdiff{}{t} \ \div \vec{E} \overset{!}{=} 0 \; \Rightarrow \; \alpha = -\epsilon_0 
\end{equation*}

\item
Dass die Konstante $\beta$ im SI-System gleich 1 sein musss, erhält man aus Überlegungen, dass die \textsc{Maxwell}-Gleichungen von Inertialsystem zu Inertialsystem invariant sein müssen.
\end{enumerate}
\ \\
\underline{Bemerkung:}
Im \textsc{Gauss}-System erhält man aufgrund der Wahl der Konstanten für die \textsc{Lorentz}-Kraft:
\begin{equation*}
\vec{F} = Q (\vec{E} + \frac{\vec{v}}{c}\times\vec{B})
\end{equation*}
woraus folgt:
\begin{equation*}
\epsilon_0\mu_0\cdot\beta = \frac{1}{c^2} \; \text{ und } \; \beta = \frac{1}{c}, \mu_0 = \frac{4\pi}{c}
\end{equation*}

\section{Integrale Fromulierung der \textsc{Maxwell}-Gleichungen}
Die integrale Formulierung der \textsc{Maxwell}-Gleichungen ist äquivalent zu der differentiellen und ergibt sich entweder aus Volumen- oder Flächenintegration auf beiden Seiten der entsprechenden Gleichung und dann der Anwendung der Integralsätze von \textsc{Gauss} oder \textsc{Stokes}:
\ \\
\begin{align*}
\text{i)} \quad & \epsilon_0 \ \div \vec{E} = \rho & \Leftrightarrow \qquad\qquad & \epsilon_0\oiint\limits_{\partial V} \mathrm{d}\vec{A} \cdot\vec{E} = Q_{\text{in}} \\
\text{ii)} \quad & \div\vec{B} = 0 & \Leftrightarrow \qquad\qquad & \oiint\limits_{\partial V} \mathrm{d}\vec{A}\cdot\vec{B} = 0 \\
\text{iii)} \quad & \rot \vec{E} + \dot{\vec{B}} = 0 & \Leftrightarrow \qquad\qquad & \oint\limits_{\partial A} \mathrm{d}\vec{r}\cdot\vec{E} \ + \ \iint\limits_A \mathrm{d}\vec{A}\cdot\dot{\vec{B}} = 0 \\
\text{iv)} \quad & \frac{1}{\mu_0} \ \rot \vec{B} - \epsilon_0 \dot{\vec{E}} = \vec{j} & \Leftrightarrow \qquad\qquad & \frac{1}{\mu_0}\oint\limits_{\partial A}\mathrm{d}\vec{r}\cdot\vec{B} \ - \ \epsilon_0\iint\limits_A\mathrm{d}\vec{A}\cdot\dot{\vec{E}} = I_{\text{in}}
\end{align*}

\underline{Bemerkung:}\\

$\vec{r}$ und $t$ sind unabhängige Variablen, das heißt, dass die Felder $\vec{B}$ und $\vec{E}$ jeweils von $\vec{r}$ und $t$ abhängen, nicht aber von $\dot{\vec{r}}$.
Zudem ist es aufgrund unserer Forderungen bei der Konstruktion der \textsc{Maxwell}-Gleichungen verboten, dass eine explizite Abhängigkeit der Grundgleichungen von $\vec{r}$ und $t$ vorliegt, da es sonst außergewöhnliche Zeiten und Orte gäbe, was aber die geforderte Homogenität verletzen würde.

\section{Induktionsgesetz für Leiterschleifen}

\begin{wrapfigure}[]{l}[0cm]{0cm}
	\raisebox{0pt}[\dimexpr\height-1\baselineskip\relax]{
		\colorbox{hgrey}{
			\begin{tikzpicture}
			
			\draw[thick,->] (3.2,0) arc (60:420:1);
			\draw[dotted,->] (3,0.9) arc (80:440:1.8);
			\draw[->] (3.7,-1)--node[right]{$\mathrm{d}\vec{r}$}(3.85,0) ;
			\draw[->] (3.7,-1)--(4.45,-1.15)node[right]{$\vec{v}\cdot\Delta t$} ;
			\draw (2.65,-.9) node{$A$};
			
			
			\end{tikzpicture}
		}
	}
	\caption{Flächenänderung}
\end{wrapfigure}
Zunächst definieren wir den magnetischen Fluss $\Phi$ durch eine Fläche $\vec{A}$ im Raum:

\begin{equation*}
\Phi := \iint\limits_A\mathrm{d}\vec{A}\cdot\vec{B}
\end{equation*}

Man sieht leicht, dass sich der Fluss $\Phi$ bei Flächenänderung und Änderung der magnetischen Flussdichte $\vec{B}$ ändert:\\


\begin{align*}
\Delta\Phi &=  \Delta\left(\iint\mathrm{d}\vec{A}\cdot\vec{B}\right) = \iint\limits_A\mathrm{d}\vec{A}\cdot\Delta\vec{B} \; + \; \iint\limits_{\Delta A}\mathrm{d}\vec{A}\cdot\vec{B}\\
&= \Delta t \iint\limits_A\mathrm{d}\vec{A}\cdot\pdiff{\vec{B}}{t} \; + \; \oint\limits_{\partial A}\left(\vec{v}\Delta t\times\mathrm{d}\vec{r}\right)\cdot\vec{B}\\
&= \Delta t\left(\iint\limits_A\mathrm{d}\vec{V}\cdot\dot{\vec{B}} \; - \; \oint\limits_{\partial A}\mathrm{d}\vec{r}\cdot\left(\vec{v}\times\vec{B}\right)\right)\\
\Rightarrow \dot{\Phi} &= \iint\limits_A\mathrm{d}\vec{A}\cdot\dot{\vec{B}} \; - \; \oint\limits_{\partial A}\mathrm{d}\vec{r} \ \left(\vec{v}\times\vec{B}\right)
\end{align*}

Nach Anwenden der dritten \textsc{Maxwell}-Gleichung erhält man das \textbf{Induktionsgesetz}:

\begin{empheq}[box=\highlightbox]{equation*}
\dot{\Phi} = - \oint\limits_{\partial A}^{\vphantom{a}}\mathrm{d}\vec{r} \ \left(\vec{E} + \vec{v}\times\vec{B}\right) = - U_{\mathrm{induziert}}
\end{empheq}

Das letzte Minuszeichen nennt man auch die \textbf{\textsc{Lenz}`sche Regel}, welche besagt, dass ein induzierter Strom immer ein Magnetfeld erzeugt, welches seiner eigenen Ursache ($U_{\mathrm{induziert}}$) entgegengerichtet ist.
\ \\
Auffällig bei dem Induktionsgesetz ist seine Ähnlichkeit mit der auf eine freie Ladung wirkende Kraft $\vec{F} = Q(\vec{E} + \vec{v}\times\vec{B})$. Darin liegt auch die Begründung für ebenjenes Gesetz:\

Wir stellen uns eine Leiterschleife vor, welche an einer Stelle durchbrochen ist, damit kein Strom durch die Schleife fließen könnte. Auf einen sich in dieser Schleife bewegenden Ladungsträger wirkt die Kraft:

\begin{equation*}
\vec{F} = Q(\vec{E} + \vec{v}\times\vec{B}) =: Q\vec{E'}
\end{equation*} 

Man sieht, dass das $\vec{E}$-Feld abhängig vom Bezugsystem ist, daher haben wir für $\vec{E'}$ ein Bezugssystem konstruiert, welches sich mit der Geschwindigkeit $\vec{v}$ gegenüber dem Laborsystem bewegt. Damit haben wir im mitbewegeten Bezugssystem erreicht, dass $\vec{v'} = 0$ ist. Bilden wir nun das Weginteral für ein Teilchen entlang der Leiterschleife im $\vec{E}$-Feld erhalten wir:

\begin{equation*}
\oint\limits_{\mathrm{Schleife}}\mathrm{d}\vec{r}\cdot\left(\vec{E} + \vec{v}\times\vec{B}\right) = \oint\limits_{\mathrm{Schleife}}\mathrm{d}\vec{r}\cdot\vec{E'} = \int\limits_{\mathrm{Beginn}}^{\mathrm{Ende}}\mathrm{d}\vec{r}\cdot\vec{E'} = U_{\mathrm{induziert}}
\end{equation*}
\chapter{Elektrostatik}

\section{Grundgleichungen und elektrostatisches Potential}

In der Elektrostatik betrachten wir, wie der Name schon andeutet, zeitunabhängige Felder. Dementsprechend kann man als erste Konsequenz daraus folgern, dass $\dot{\vec{E}} = 0$ und $\dot{\vec{B}} = 0$ ist. Fallen nun in den \textsc{Maxwell}-Gleichungen alle Beiträge mit $\dot{\vec{E}}$ und $\dot{\vec{B}}$ weg, kann man die Felder $\vec{E}$ und $\vec{B}$ getrennt voneinander betrachten. Laienhaft gesprochen entkoppeln wir die Phänomene \grqq Elektrizität\grqq und "Magnetismus". Des Weiteren betrachten wir in der Elektrostatik nur ruhende Ladungen, woraus folgt, dass außerdem $\vec{j}=0 \Rightarrow \vec{B}=0$ ist.\

Damit erhalten wir aus der dritten \textsc{Maxwell}-Gleichung, dass rot $\vec{E} = 0$ gilt, wodurch das Einführen eines Potentials für $\vec{E}$ möglich wird:

\begin{equation*}
\vec{E} =: \ -\grad \varphi
\end{equation*}

Mit div $\vec{E} = \frac{\rho}{\epsilon_0}$ erhält man daraus die \textbf{\textsc{Poisson}-Gleichung} der Elektrostatik:

\begin{empheq}[box=\highlightbox]{equation*}
\laplace\varphi = - \frac{\rho\vphantom{\big|}}{\epsilon_0\vphantom{\big|}}
\end{empheq}

Für $\laplace\varphi = 0$ nennt man die \textsc{Poisson}-Gleichung auch \textbf{\textsc{Laplace}-Gleichung}.

\section{Kugelsymmetrische Ladungsverteilung}

Für eine kugelsymmetrische Ladungsverteilung gilt:

\begin{equation*}
\rho(\vec{r}) = \rho(|\vec{r}|) = \rho(r) \; \Rightarrow \; \varphi(\vec{r}) = \varphi(r)
\end{equation*}

Dem kann man entnehmen, dass die Äquipotentialflächen Kugelflächen sein müssen und somit der Gradient von $\varphi$ auch parallel zum Ortsvektor stehen muss.($\vec{E}(\vec{r}) = \vec{E}(r)\vec{e}_r$)\\
Für das $\vec{E}$-Feld gilt weiterhin:

\begin{equation*}
\epsilon_0\oiint\limits_{\partial \text{Kugel}}\mathrm{d}\vec{A}\cdot\vec{E} \; \overset{\vec{A}\parallel\vec{E}}{=} \; \epsilon_0\oiint\limits_{\partial \text{Kugel}}\mathrm{d} A \cdot E = 4\pi\epsilon_0\cdot r^2 \cdot E(r) = Q_{\mathrm{in}}(r)
\end{equation*}

Damit ergibt sich für das $\vec{E}$-Feld und das Potential:

\begin{empheq}[box=\highlightbox]{align*}
\vec{E}(r) &= \frac{Q_{\mathrm{in}}(r)}{4\pi\epsilon_0\cdot r^2} \cdot \vec{e}_r\\
\ \\
\varphi(r) &= \frac{Q_{\mathrm{in}}(r)}{4\pi\epsilon_0\cdot r} + \varphi_0 \quad \text{ mit } \quad \varphi_0 = \varphi(r \rightarrow 0)
\end{empheq}

\section[Feld einer beliebigen Ladungsverteilung]{Feld einer beliebigen räumlich begrenzten Ladungsverteilung}

\begin{enumerate}
\item \textbf{Punktladung bei} $\vec{r}_0$:
\begin{equation*}
\varphi(\vec{r}) = \frac{Q}{4\pi\epsilon_0 \ |\vec{r}-\vec{r}_0|}
\end{equation*}


\item \textbf{Mehrere Punktladungen} (Superpositionsprinzip anwendbar wegen Linearität der \textsc{Maxwell}-Gleichungen):
\begin{equation*}
\varphi(\vec{r}) = \sum\limits_i \ \frac{Q_i}{4\pi\epsilon_0 \ |\vec{r}-\vec{r}_i|}
\end{equation*}

\item \textbf{Kontinuierliche Ladungsverteilung}:
\begin{empheq}[box=\highlightbox]{equation*}
\varphi(\vec{r}) = \Int{\vphantom{a}}{\vphantom{a}}{V'} \ \frac{\rho(\vec{r}')}{4\pi\epsilon_0 \ |\vec{r}-\vec{r}'|}
\end{empheq}
\end{enumerate}

Die allgemeine Gleichung für die kontinuierliche Ladungsverteilung ergibt sich aus der Lösung der \textsc{Poisson}-Gleichung mithilfe der bekannten \textsc{Green}'schen Funktion für eine Punktladung der Größe 1:\quad $G(\vec{r}) = \frac{1}{4\pi\epsilon_0 \cdot |\vec{r}|}$\

\begin{align*}
-\epsilon_0 \cdot \laplace\varphi &= \rho\\
\Rightarrow -\epsilon_0 \cdot \laplace G(\vec{r}) &= \delta(\vec{r})\\
\end{align*}

Dabei gilt: $G(\vec{r},\vec{r}') = G(\vec{r}-\vec{r}')$ aufgrund der Translationsinvarianz der \textsc{Green}-Funktion.

\begin{equation*}
\Rightarrow \varphi(\vec{r}) = \int\mathrm{d}V' \ G(\vec{r}-\vec{r}')\cdot\rho(\vec{r}') = \frac{1}{4\pi\epsilon_0} \ \int\mathrm{d}V' \ \frac{\rho(\vec{r}')}{|\vec{r}-\vec{r}'|}
\end{equation*}
\ \\

Aus dieser allgemeinen Form lässt sich natürlich auch im umgekehrten Falle das $\vec{E}$-Feld einer Punktladung in $\vec{r}_0$ herleiten. Dafür muss nur $\rho(\vec{r}) = Q\cdot\delta(\vec{r}-\vec{r}_0)$ gesetzt werden:

\begin{equation*}
\varphi(\vec{r}) = \int\mathrm{d}V' \ \frac{\rho(\vec{r}')}{4\pi\epsilon_0 \cdot |\vec{r}-\vec{r}'|} = \ \frac{Q}{4\pi\epsilon_0} \ \underbrace{\int\mathrm{d}V' \ \frac{\delta(\vec{r}'-\vec{r}_0)}{|\vec{r}-\vec{r}'|}}_{=\frac{1}{|\vec{r}-\vec{r}_0|}}
\end{equation*}

\section{Feld eines elektrischen Dipols}

\begin{wrapfigure}[17]{r}[0cm]{0cm}
	\raisebox{0pt}[\dimexpr\height-1\baselineskip\relax]{
		\colorbox{hgrey}{
			\begin{tikzpicture}
			
				 \draw[->](0,0)-- node[left]{$\vec{r}$} (0,2.8) ;
				 \draw[->](0,0)--node[right]{$\ \vec{r}+\vec{a}$}(2.85,2.8);
				 \draw[->](0.2,3)-- node[above]{$\vec{a}$}(2.75,3);
				 \filldraw[black] (0,0) circle (2pt) node[below left]{$0$};
				 \filldraw[black] (0,3) circle (2pt) node[above]{$+Q$};
				 \filldraw[black] (3,3) circle (2pt) node[above]{$-Q$};
			
			
			\end{tikzpicture}
		}
	}
	\caption{elektrischer Dipol}
\end{wrapfigure}
Ein Dipol besteht aus zwei gleich großen, entgegengesetzt geladenen Ladungen $\pm Q $, welche  einen festen Abstand $\vec{a}$ voneinander entfernt sind. Daher ergibt es Sinn, als charakteristische Eigenschaft des Dipols das \textbf{Dipolmoment} $\vec{p}$ wie folgt zu definieren:

\begin{empheq}[box=\highlightbox]{equation*}
\vec{p} := Q \cdot \vec{a}\vphantom{\Bigg|}
\end{empheq}

\begin{equation*}
\text{Dipollimit: } \quad |\vec{a}| \ \rightarrow \ 0, \ Q \ \rightarrow \ \infty 
\end{equation*}

\begin{equation*}
\Rightarrow \ |\vec{p}| = \ \text{const.}
\end{equation*}

Für das Potentialfeld eines solchen Dipols gilt offensichtlich:

\begin{equation*}
\varphi(\vec{r}) = \frac{Q}{4\pi\epsilon_0}\cdot\left(\frac{1}{|\vec{r}|} - \frac{1}{|\vec{r}+\vec{a}|}\right)
\end{equation*}

Für große Abstände von diesem Dipol, d.h. $\vec{r}\gg\vec{a}$ wollen wir das Potentialfeld \textsc{Taylor}-entwickeln, um besser mit ihm arbeiten zu können.\
Dazu betrachten wir den Term $\frac{1}{|\vec{r}+\vec{a}|}$ ein wenig genauer:\\

\begin{equation*}
\frac{1}{|\vec{r}+\vec{a}|} \cong \frac{1}{|\vec{r}|} + \left(\vec{a}\cdot\pdiff{}{\vec{r}}\right) \ \frac{1}{|\vec{r}|} = \frac{1}{|\vec{r}|}-\vec{a}\cdot\frac{\vec{r}}{|\vec{r}|^3}
\end{equation*}
\ \\
Damit gilt für das Potential:

\begin{empheq}[box=\highlightbox]{equation*}
\varphi(\vec{r})=\frac{Q\vphantom{\big|}}{4\pi\epsilon_0\vphantom{\big|}}\left(\frac{1}{r}-\frac{1}{r}-\left(\vec{a}\cdot\pdiff{}{\vec{r}}\right)\frac{1}{r}\right) = \frac{\vec{p}\cdot\vec{r}}{4\pi\epsilon_o\cdot r^3}
\end{empheq}

und das $\vec{E}$-Feld:

\begin{align*}
\vec{E}(\vec{r}) &= - \nabla \varphi = \frac{1}{4\pi\epsilon_0}\ \nabla \left(\vec{p}\cdot\nabla\right)\ \frac{1}{r} = \frac{\vec{p}}{4\pi\epsilon_0}\ \underbrace{\left(\nabla \circ \nabla\right)\ \frac{1}{r}}_{(*)}\\
\alignedbox{\hphantom{a}\vec{E}(\vec{r})}{= \frac{1}{4\pi\epsilon_0}\ \frac{3(\vec{p}\cdot\vec{r})\vec{r} - \vec{p}r^2}{r^5}\hphantom{a}}\\
\ \\
\text{mit}\quad (*) &= \left(\pdiff{}{\vec{r}} \circ \pdiff{}{\vec{r}}\right)\frac{1}{|\vec{r}|} = - \pdiff{}{\vec{r}}\circ\frac{\vec{r}}{|\vec{r}|^3} = \frac{3\vec{r}\circ\vec{r}-\mathbbm{1}\cdot\vec{r}^2}{|\vec{r}|^5}
\end{align*}
\ \\

\section[Fernfeld einer Ladungsverteilung]{Fernfeld einer räumlich eingegrenzten Ladungsverteilung}

\begin{wrapfigure}[]{l}[0cm]{0cm}
	\raisebox{0pt}[\dimexpr\height-1\baselineskip\relax]{
		\colorbox{hgrey}{
			\begin{tikzpicture}
			
			\filldraw[pattern=north west lines]  plot[smooth, tension=.8] coordinates {(0,1) (1,0) (3,1.5) (1.5,2) (0,1)};	
			\draw[|-|] (-1,0) -- node[left]{$a$} (-1,2);
			\draw  (2,-0.5)node[below]{$\rho(\vec{r'})$} -- (1.7,0.5);
			\draw[->] (2,1) -- (3,3) node[above]{$\vec{r}$};
			
			
			\end{tikzpicture}
		}
	}
	\caption{Verteilung}
\end{wrapfigure}

Wenn man das Fernfeld einer räumlich begrenzten Ladungsverteilung ermitteln möchte, spricht man in diesem Zusammenhang auch immer von der sogenannten \textbf{Multipolentwicklung}.

Wir betrachten nun eine räumlich eingegrenzte Ladunugsverteilung der Dichte $\rho$, für die zunächst einmal allgemein gilt:

\begin{equation*}
\varphi\left(\vec{r}\right) = \frac{1}{4\pi\epsilon_0}\cdot\int\d V \frac{\rho(\vec{r}')}{|\vec{r}-\vec{r}'|}
\end{equation*}

Unter der Annahme, dass $|\vec{r}| \gg \ a$ gilt (wobei $a$ die größte räumliche Ausdehnungsrichtung der Ladungsverteilung ist), werden wir nun den Term $\frac{1}{|\vec{r}-\vec{r}'|}$ entwickeln:
\newpage
\begin{equation*}
\frac{1}{|\vec{r}-\vec{r}'|} = \sum_{n=0}^{\infty} \; \frac{1}{n!}\left(-\vec{r}'\cdot\pdiff{}{\vec{r}}\right)^n \ \frac{1}{r} = \frac{1}{r} + \frac{\vec{r}'\cdot\vec{r}}{r^3} + \frac{1}{2} \ \frac{3 (\vec{r}'\cdot\vec{r})^2 - \vec{r}'^2 \ \vec{r}^2}{r^5} + \dotsc
\end{equation*}

\begin{align*}
\Rightarrow \quad \varphi(\vec{r}) &= \frac{1}{4\pi\epsilon_0} \left[ \frac{1}{r}\int\d V' \rho(\vec{r}') + \frac{\vec{r}}{r^3} \int\d V' \vec{r}' \rho(\vec{r}') + \sum_{i,j} \ \frac{x_i x_j}{r^5} \ \int\d V' \rho(\vec{r}') \ (3x_i'x_j' - \delta_{ij}\vec{r}'^2) + \dotsc\right] \\
\alignedbox{\hphantom{a}}{= \frac{1\vphantom{\big|}}{4\pi\epsilon_0\vphantom{\big|}} \left[\frac{Q}{r} \quad + \quad \frac{\vec{r}\cdot\vec{p}}{r^3} \quad + \quad \frac{1}{2}\frac{\vec{r}\cdot \tens{D} \cdot\vec{r}}{r^5} \quad + \quad \dotsc \quad \right]\hphantom{a}}\\
& \hspace{1.35cm} {\scriptstyle \sim\frac{1}{r}} \hspace{1.15cm} {\scriptstyle \sim \frac{1}{r^2}} \hspace{2cm} {\scriptstyle\sim\frac{1}{r^3}}
\end{align*}

Die einzelnen Summanden bezeichnet man auch als \textbf{Multipolmomente} einer Ladungsverteilung:

\begin{align*}
\mathrm{Monopol:}& \qquad Q = \int\d V \rho(\vec{r})\\
\mathrm{Dipol:}& \qquad \vec{p} = \int\d V \rho(\vec{r})\vec{r}\\
\mathrm{Quadrupol:}& \qquad \tens{D} = \int\d V \rho(\vec{r}) \ (3\vec{r}\circ\vec{r}-\mathbbm{1}\vec{r}^2)\\
\mathrm{Oktupol:}& \qquad \dotsc\\
\vdots \qquad &
\end{align*}

Im Allgemeinen hängen die Multipolmomente vom Bezugspunkt ab, nur das erste nicht verschwindende Moment ist unabhängig vom selbigen.\\
Der Quadrupol-Tensor $\tens{D}$ hat dabei folgende Eigenschaften:\\

\begin{itemize}
\item $\bm{D}_{ij} = \bm{D}_{ji}$, insbesondere gilt:

\begin{equation*}
\operatorname{Spur}\tens{D} = \sum_j \bm{D}_{jj} = \Int{}{}{V}(3\vec{r}^2 - 3 \vec{r}^2) =0
\end{equation*}

\item $\tens{D}$ hat 5 unabhängige Komponenten
\item $\tens{D}$ kann hauptachsentransformiert werden
\item aus $\operatorname{Spur} \tens{D}=0$ folgt $\tens{D}=0$ für Kugel und Kegel 
\end{itemize}
\ \\

Aufgrund der der charakteristischen Richtungsabhängigkeit ist es sinnvoll, das Potential der Ladungsverteilung mit Kugelflächenfunktionen zu entwickeln. Ausgangspunkt ist hierbei wieder das allgemeine Potential für eine beliebige Ladungsverteilung: 

\begin{equation*}
\varphi(\vec{r}) = \frac{1}{4\pi\epsilon_0}\int\d V' \frac{\rho(\vec{r})}{|\vec{r}-\vec{r}'|} = \int\d V' G(\vec{r}-\vec{r}')\rho(\vec{r}')
\end{equation*}

Wobei $G$ die \textsc{Green}'sche Funktion ist, welche die \textsc{Poisson}-Gleichung mit $\delta$-förmiger Inhomogenität löst:

\begin{equation*}
-\epsilon_0 \ \laplace G(\vec{r}) = \delta(\vec{r})
\end{equation*}

Als nächstes separieren wir die Winkel- und Richtungsabhängigkeit des Differentialoperators $\laplace$, welches sich am besten explizit in Kugelkoordinaten vornehmen lässt.

\begin{equation*}
\laplace = \frac{1}{r} \pddiff{}{r}  \; +\; \underbrace{\frac{1}{r^2 \sin\theta} \pdiff{}{\theta} \sin\theta \; \pdiff{}{\theta} \; + \; \frac{1}{r^2 \sin^2\theta} \ \pddiff{}{\phi}}_{=: \frac{1}{r^2}\Lambda(\theta,\phi)}
\end{equation*}

Nun wenden wir auf die Differentialgleichung

\begin{equation*}
\Lambda \ Y(\theta,\phi) \; = \; -l(l+1) \ Y(\theta,\phi) \qquad l \in \mathbb{N}
\end{equation*}

den Separationsansatz $Y(\theta,\phi) = P(\theta)\cdot Q(\phi)$ an und erhalten zunächst für $Q(\phi)$:

\begin{align*}
\ddiff{}{\phi}Q(\phi) &= - m^2 Q(\phi)\\
\Rightarrow Q &= e^{im\phi} \qquad \text{ mit } \quad m \in [-l,l] \subset \mathbb{Z}
\end{align*}

Substituieren wir nun oben $\cos\theta = x$, so führt dies auf eine verallgemeinerte \textbf{\textsc{Legendre}-Differentialgleichung} für $P(x)$

\begin{equation*}
\left(\diff{}{x} \ \left(1+x^2\right) \ \diff{}{x} \ \left(-\frac{m^2}{1-x^2} + l(l+1)\right)\right) \ P_l^m(x) = 0
\end{equation*}

Es genügt diese für $m=1$ zu lösen, denn:

\begin{equation*}
P_l^m(x) = (-1)^m(1-x)^{\frac{m}{2}} \ \left(\diff{}{x}\right)^{|m|} P_l(x)
\end{equation*}

Somit bleibt nur noch folgende \textsc{Legendre}-Differentialgleichung übrig:

\begin{equation*}
(1-x^2) P_l'' \ - \ 2x \ P_l' \ + \ l(l+1) \ P_l \ = \ 0
\end{equation*}

Deren Lösungen $P_l$ sind sogenannte \textbf{\textsc{Legendre}-Polynome}:

\begin{equation*}
P_l(x) = \frac{1}{2^l l!} \ \left(\diff{}{x}\right)^l \ (x^2+1)^l \qquad l\in \mathbb{N}
\end{equation*}

(Die ersten $P_l$ lauten explizit: $P_0(x) =1, \; P_1(x) = x, \; P_2(x) = \frac{1}{2}(3x-1), \dotsc$)
Nun erhalten wir aus $P$ und $Q$ unsere ursprüngliche, separierte Funktion $Y(\theta,\phi)$:

\begin{equation*}
Y_{lm}(\theta,\phi) = \sqrt{\frac{2l + 1}{4\pi} \ \frac{(l-|m|)!}{(l+|m|)!}} \ P_l(\cos\theta) \ e^{im\phi}
\end{equation*}

(Die ersten $Y_{lm}$ lauten explizit: $Y_{00} = \frac{1}{\sqrt{4\pi}}, \; Y_{10} = \sqrt{\frac{3}{4\pi}}\cos\theta, \; Y_{1,\pm1} = \mp \sqrt{\frac{3}{8\pi}}\sin\theta \ e^{i\phi}$\
\\
\ \\
\underline{Bemerkung zu den $Y_{lm}$:}\\
\ \\
Die $Y_{lm}$ sind sogenannte \textbf{Kugelflächenfunktionen} und Lösungen der Differentialgleichung

\begin{align*}
\left(\frac{1}{\sin\theta}\pdiff{}{\theta} \ + \ \sin\theta \pdiff{}{\theta} \ + \ \frac{1}{\sin^2\theta} \pddiff{}{\phi} \ + \ l(l+1)\right) \ Y_{lm}(\theta,\phi) = 0\\
l \in \mathbb{N}, m \in [-l,l]
\end{align*}

Anschaulich gesprochen sind die Kugelflächenfunktionen die Eigenfunktionen des Winkelanteils des \textsc{Laplace}-Operators und daher sehr gut als Darstellungsbasis für sämtliche Funktionen auf Kugeloberflächen geeignet. Man kann auch zeigen, dass sie nicht nur eine beliebige Basis, sondern sogar eine Orthonormalbasis sind; dazu überprüfen wir zunächst die Orthogonalität der Basiselemente zueinander:

\begin{equation*}
\langle Y_{lm},Y_{l'm'}\rangle =: \int_{-1}^1\d(\cos\theta)\int_{0}^{2\pi}\d\phi \ Y^{*}_{lm}(\theta\phi) Y_{lm}(\theta\phi) = \delta_{ll'}\delta_{mm'}
\end{equation*}

Nach bekannter Vorgehensweise lässt sich nun jede beliebige Funktion $f$ auf einer Kugeloberfläche aus den $Y_{lm}$ darstellen:

\begin{align*}
f(\theta,\phi) &= \sum_{l=0}^{\infty} \sum_{m=-l}^{l} f_{lm} Y_{lm}(\theta,\phi)\\
\text{mit } f_{lm} &= \langle Y_{lm},f\rangle = \int\d(\cos\theta) \int\d\phi \ Y^{*}_{lm}(\theta,\phi)f(\theta,\phi)
\end{align*}

Somit lässt sich auch mit ihnen die allgemeine Lösung der \textsc{Laplace}-Gleichung $\laplace\varphi=0$ darstellen: 

\begin{equation*}
\varphi(r,\theta,\phi) = \sum_{l=0}^{\infty}\sum_{m=-l}^{l} \left(A_l \cdot r^l + B_l \cdot r^{-l-1}\right) Y_{lm}(\theta,\phi)
\end{equation*}

Wir können nun zur Entwicklung von $\frac{1}{|\vec{r} - \vec{r}'|}$ zurückkehren:

\begin{align*}
\frac{1}{|\vec{r} - \vec{r}'|} = \sum_{l=0} \left(A_l \cdot r^l + B_l \cdot r^{-l-1}\right) P_l(\cos\gamma) \quad \text{ mit } \gamma = \sphericalangle\left(\vec{r},\vec{r}'\right)\\
(\gamma\text{ ohne $\phi$-Abhängigkeit wegen axialer Symmetrie)}
\end{align*}

Wähle nun für die $A_l, B_l$, dass $\vec{r}\parallel\vec{r}'$ ist und führe so die Entwicklung fort

\begin{align*}
\frac{1}{|\vec{r}-\vec{r}'|} &= \sum_{l=0}^{\infty} \ \frac{1}{l!} \left(\-r_{<}\diff{}{r_{<}}\right)^l \ \frac{1}{r_{>}} \qquad \text{ mit } r_{<} := \min\{r,r'\}, \ r_{>} \text{ analog}\\
&= \frac{1}{r_{<}}\sum_{l=0}^{\infty} \ \left(\frac{r_{>}}{r_{<}}\right)^l\\
&=  \sum_{l=0}^{\infty} \frac{r_{>}}{r_{<}^{l+1}} \ P_l (\cos\gamma)\\
\\
P_l (\cos\gamma) &= \frac{4\pi}{2l+1}\sum_{m=-l}^{l}  Y^{*}_{lm}(\theta',\phi')Y_{lm}(\theta,\phi)\\
& \quad\left(\cos\gamma = \cos\theta\cos\theta' \ + \ \sin\theta\sin\theta'\cos(\phi-\phi')\right)\\
\\
\Rightarrow \frac{1}{|\vec{r}-\vec{r}'|} &= 4\pi \sum_{l=0}^{\infty}\sum_{m=-l}^{l} \frac{1}{2l+1}\ \frac{r_{>}^l}{r_{<}^{l+1}} Y^{*}_{lm}(\theta',\phi')Y_{lm}(\theta,\phi)
\end{align*}

Wir haben nun $\frac{1}{|\vec{r}-\vec{r}'|}$ vollständig faktorisiert und können nun das Potential einer Ladungsverteilung aufstellen:\\

\begin{empheq}[box=\highlightbox]{equation*}
\varphi (\vec{r}) = \frac{1}{4\pi\epsilon_0} \sum_{l,m} \ \sqrt{\frac{4\pi}{2l +1}} \ \frac{Y_{lm}(\theta,\phi)}{r^{l+1}} \ \underbrace{\Int{}{\vphantom{\Big|}}{V'} \  \rho(\vec{r}) \  Y_{lm}^{*}(\theta',\phi') \ {r'}^l \sqrt{\frac{4\pi}{2l+1}}}_{q_{lm} \hat{=} \text{ Multipolmomente}}
\end{empheq}
\newpage
Aus dem allgemeinen Ausdruck $q_{lm}$ für die Multipolmomente können wir nun auch die uns bereits bekannten Momente ableiten:

\begin{align*}
q_{00} &= \sqrt{4\pi}\int\d V' \ \rho(\vec{r}') \ Y_{00} = Q
\\
q_{10} &= \int\d V' \ \rho(\vec{r}') \ \underbrace{r'\cos\theta'}_{z'} = p_z\\
q_{1,\pm 1} &= \pm \frac{1}{\sqrt{2}} \ \int\d V' \ \rho(\vec{r}')  \ r'\sin\theta' \; e^{i\phi'} = (p_x \mp i p_y) \cdot \frac{1}{\sqrt{2}}\\
\\q_{2m} &\rightarrow \text{ 5 skalare Komponenten } \rightarrow \text{ Quadrupol}
\end{align*}

\section{Randbedingungen}

Die allgemeine Lösung der \textsc{Poisson}-Gleichung $\epsilon_0 \bigtriangleup \varphi = \rho$ hängt von ihren Randbedingungen ab. Die vollständige Lösung erhält man durch Addition der allgemeinen Lösung der zugehörigen homogenen Differentialgleichung und einer partikulären Lösung der inhomogenen Gleichung: $\quad \varphi = \varphi_p + \varphi_h$.\\
Es bietet sich an die Randbedingungen in den homogenen Teil einzubauen (bisher haben wir immer angenommen, dass $\varphi (\infty) = 0$). Mathematisch liefert uns eine einzige Randbedingung auf einem geschlossenen Rand R eine physikalisch eindeutige Lösung für eine Differentialgleichung 2. Ordnung, da es sich durch den geschlossenen Rand effektiv um zwei Randbedingungen handelt. Wir unterscheiden dabei verschiedene gängige Varianten sich dem Problem zu nähern:


\begin{enumerate}
\item $\varphi(R)$ ist gegeben\
\\
Diese Variante nennt man auch die \textsc{Dirichlet}-Randbedingung\

\item $\pdiff{\varphi}{n}(R)$ ist gegeben\
\\
\ \\
Diese Variante nennt man auch die \textsc{von-Neumann}-Randbedingung\
\\
\ \\
($\pdiff{\varphi}{n} := \pdiff{\varphi}{\vec{r}}\cdot\vec{e}_n = - \vec{E}_n$ ist dabei die Normalenableitung)\

\item $\alpha \varphi \ + \ \beta\pdiff{\varphi}{n}$ ist gegeben
\end{enumerate}

\section{Leiter im elektrischen Feld}

Bis jetzt hatten wir in der Elektrostatik nur ruhende Ladungen betrachtet. In Leitern  gibt es allerdings bewegliche Ladungen im Inneren. Diese befinden sich im Gleichgewicht bei $\vec{F}=0 \ \Rightarrow \vec{E}= 0$
Daraus kann man dieser folgern, dass $\varphi =$ \textit{const.} \ im Inneren des Leiters und auf der Leiteroberfläche gilt. Dafür muss gelten, dass $\rho = 0$ im Leiterinneren ist. Außerdem folgt direkt, dass $\vec{E} = -\pdiff{\varphi}{\vec{r}}$ senkrecht zur Oberfläche stehen muss und das es ausschließlich von \textbf{Oberflächenladungen} erzeugt wird. Um diese zu definieren betrachten wir ein Volumen $\Delta V$ auf dem Leiteroberflächenstück $\Delta \vec{A}$, welches die Ladung $\Delta Q$ in sich trägt.

\begin{equation*}
\epsilon_0 \oiint_{\partial\Delta V}\d\vec{A}\cdot\vec{E} = \Delta Q \quad \Rightarrow \quad \epsilon_0 \ \Delta\vec{A}\cdot\vec{E}_n = \Delta Q
\end{equation*}

Darüber können wir uns die \textbf{Flächenladungsdichte} $\sigma$ definieren, um die Oberflächenladungen beschreiben zu können:

\begin{align*}
\alignedbox{\hphantom{a}\sigma} {:= \frac{\Delta Q}{\Delta A} = \epsilon_0 \ E_n\vphantom{\Bigg|}\hphantom{a}}\\
Q  &= \iint\d A \cdot \sigma
\end{align*}

Die Oberflächenladungen werden durch äußere elektrische Felder bestimmt und schirmen das Leiterinnere von diesen Feldern ab.\
\\
\ \\
Betrachten wir nun den Innenraum eines Hohlleiters. Hier gilt genau wie bei einem normalen Leiter, dass auf der Leiteroberfläche das Potential konstant ist. Zudem ist der Innenraum ladungsfrei, woraus folgt, dass auch dort $\varphi$ =\textit{const.} \ gilt uns somit auch $\vec{E} = 0$. Dieses Prinzip ist auch als \textbf{\textsc{Faraday}'scher Käfig} bekannt.\
\\
Die Begründung für dieses Prizip kann man auch direkt aus den \textsc{Maxwell}-Gleichungen herleiten, denn es gilt div $\vec{E} = 0$ und rot $\vec{E} = 0$ im Inneren des Hohlleiters. Jede Feldlinie im Inneren müsste demzufolge auf dem Rand anfangen und enden. Für eine Integration entlang einer Feldlinie $\int\d \vec{r} \cdot \vec{E} = \Delta \varphi$ würde dies jedoch ein endliches $\Delta \varphi$ zwischen Anfangs- und Endpunkt liefern, welches im Widerspruch zu $\varphi$ = const. auf dem Rand stehen würde. Also muss $\vec{E} = 0$ im Inneren des Hohlleiters gelten.

\newpage
\underline{\textbf{Beispiele}}
\begin{enumerate}
\item \textbf{ Punktladung und ebene Leiterfläche}\\
\ \\
Wir betrachten eine Punktladung $Q$, welche sich im Abstand $a$ von einer ebenen Leiteroberfläche befindet. Letztere sei entlang der y-Achse unseres Koordinatensystems ausgerichtet, während sich $Q$ auf der x-Achse befindet.
Demzufolge erhalten wir die \textsc{Poisson}-Gleichung:

\begin{equation*}
-\epsilon_0 \ \laplace\varphi = Q \ \delta(\vec{r}-a\vec{e}_x)
\end{equation*}

mit der Randbedingung $\varphi(x=0) = 0$ auf der Leiteroberfläche.\
Wir wissen, dass die Feldlinien der Punktladung senkrecht auf die Leiteroberfläche aufkommen müssen. Daher können wir uns fragen, wie man eben jenes Feldlinienbild beschreiben könnte. Man erhält es durch das Einbringen einer zweiten, gedachten Ladung $-Q$ bei $-a\vec{e}_x$, sodass die gesamte Anordnung für x$>0$ das gesuchte Feldlinienbild ergibt. Die imaginäre Punktladung bei $-a\vec{e}_x$ nennt man \textbf{Spiegelladung}. Die Begründung für dieses Phänomen ist, das das Einbringen einer Leiteroberfläche in ein gegebenes Potential $\varphi (\vec{r})$ entlang einer Äquipotentialfläche das Feld außerhalb des Leiters nicht ändert. Dort gilt weiterhin $-\epsilon_0 \laplace\varphi = \rho$ unverändert und die Randbedingungen sind effektiv identisch zu der Gleichung, welche das Feldlinienbild mithilfe der Spiegelladung beschreibt. Diese lautet hier:

\begin{equation*}
- \laplace\varphi = Q \left(\delta(\vec{r}-a\vec{e}_x) \ - \ \delta(\vec{r}+a\vec{e}_x)\right)
\end{equation*}

welche für das Potential liefert:

\begin{equation*}
\varphi = \frac{Q}{4\pi\epsilon_0} \left(\frac{1}{|\vec{r}-a\vec{e}_x|} \ - \ \frac{1}{|\vec{r}+a\vec{e}_x|}\right)
\end{equation*}
\newpage

\item \textbf{ Kugeloberfläche in einem asymptotisch homogenen Feld}\\
\ \\
Wir betrachten eine leitende Kugel mit dem Radius $R$, welche sich im Ursprung des Koordinatensystems in einem homogenen elektrischen Feld $\vec{E}_0$ befindet. Für $|\vec{r}|>R$ gilt dementsprechend: $\bigtriangleup\varphi = 0$ mit den Randbedingungen $\varphi(|\vec{r}| = R) = \varphi_0 := 0$ und $\varphi(|\vec{r}|\rightarrow\infty) = -\vec{E}_0\cdot\vec{r}$ (Homogenität des Feldes). Aus Symmetrieüberlegungen erhalten wir außerdem, dass $\varphi(\vec{r},R,\vec{E}_0)$ linear in $\vec{E}_0$ sein muss.\
Dementsprechend wählen wir den Ansatz $\varphi = -\vec{E}_0 \cdot\vec{r} \ G(r,R)$ mit den resultierenden Randbedingungen $G(r=R)=0$ und $G(r\rightarrow\infty
) =1$, welcher nach Einsetzen in die \textsc{Laplace}-Gleichung folgende homogene DGL für $G$ liefert:

\begin{equation*}
\laplace\varphi = -\vec{E}_0\cdot\vec{r} \ \left(\frac{4}{r} \ \diff{G}{r} \ + \ \ddiff{G}{r}\right) = 0
\end{equation*}
\ \\
Diese lösen wir mit dem Ansatz $G \sim r^n$:
\begin{align*}
& 4n + n(n+1) = 0\\
\Rightarrow & n_1 = 0, n_2=3\\
\Rightarrow & G = C_1 + \frac{C_2}{r^3}\\
\ \\
\text{Rb.: } & G(r\rightarrow\infty) = 1 \quad \Rightarrow \quad C_1 = 1\\
& G(r=R) = 0 \quad \Rightarrow \quad C_2 = -R^3 
\end{align*}

Also ergibt sich für $\varphi$:

\begin{equation*}
\varphi = \vec{E}_0\cdot\vec{r} \ + \ \frac{\vec{p} \cdot\vec{r}}{4\pi\epsilon_0 r^3} \quad
\text{ mit } \quad \frac{\vec{p}}{4\pi\epsilon_0} = \vec{E}_0 \cdot R^3
\end{equation*}

Das äußere elektrische Feld induziert also offensichtlich ein Dipolmoment in der Kugel, welches für den zusätzlichen Term in $\varphi$ verantwortlich ist.
\end{enumerate}
\newpage

\section{Mehrere Leiter}

Wir betrachten mehrere Leiter im Raum mit den Oberflächen $\mathcal{S}_i$. Erneut gilt die Tatsache, dass es keine Raumladungen gibt ($\bigtriangleup\varphi = 0$) und die Randbedingungen für die Leiteroberflächen $\varphi_i = \varphi$ auf den $\mathcal{S}_i. \ (\varphi_0 = 0$ wird willkürlich festgelegt)\\
Nun gilt aufgrund der Linearität der \textsc{Maxwell}-Gleichungen für das Gesamtpotential:

\begin{equation*}
\varphi (\vec{r}) = \sum_k \ G_k(\vec{r}) \ \varphi_k
\end{equation*}

Die $G_k$ hängen dabei von der Geometrie der Leiteranordnung ab.\
Wenn wir nun die Quellen auf den $\mathcal{S}_i$ mit in die Betrachtung mit einbeziehen, erhalten wir: 

\begin{align*}
\sigma &= \epsilon_0 \vec{E}_n\big|_{\mathcal{S}_i}  = -\epsilon_0 \pdiff{\varphi}{n}\Bigg|_{\mathcal{S}_i} = - \epsilon_0 \sum_k \pdiff{G_K(\vec{r})}{n}\Bigg|_{\mathcal{S}_i} \ \varphi_k \\
\ \\
\Rightarrow Q_i &= \Oiint{\mathcal{S}_i}{}{A} \ \sigma = -\epsilon_0\sum_k\Oiint{\mathcal{S}_i}{}{\vec{A}}\  \cdot \ \pdiff{G_K(\vec{r})}{\vec{r}} \ \varphi_k\\
\alignedbox{\hphantom{a}Q_i}{=: \sum_k C_{ik} \varphi_k \quad \text{ mit } \quad C_{ik} = -\epsilon_0 \Oiint{\mathcal{S}_i}{\vphantom{A}}{\vec{A}} \ \cdot \ \pdiff{G_k}{\vec{r}}\hphantom{a}}
\end{align*}


Die $C_{ik}$ nennen wir die \textbf{Kapazitätskoeffizienten}. Für sie gilt $C_{ik} = C_{ki}$. Speziell für zwei sich umschließende Leiter ($\hat{=}$ Kondensator) folgt daraus:

\begin{equation*}
Q_1 = C_{11}\varphi_1 \quad (\text{und } Q_0 = -Q_1) \quad \hat{=} \quad \ Q=C \cdot U
\end{equation*}
\chapter{Stationäre Ströme}

\section{Grundgleichungen und Vektorpotential}

Wenn wir stationäre Ströme betrachten, dann gilt ebenso wie in der Elektrostatik, dass die Felder zeitunabhängig sind: $\dot{\vec{E}} = 0, \dot{\vec{B}}=0$. Außerdem ist div $\vec{j} = 0$.\
Da für das $\vec{B}$-Feld unter diesen Bedingungen gilt, dass rot $\vec{B} = \mu_0 \ \vec{j}$, ist es nicht möglich ein $\psi$ zu finden, sodass grad $\psi = \vec{B}$. Anstattdessen macht man sich die Quellenfreiheit eines Wirbelfelds zu nutze und führt ein Vektorpotential $\vec{A}(\vec{r})$ ein, sodass:

\begin{equation*}
\vec{B}(\vec{r}) = \rot \vec{A}(\vec{r})
\end{equation*}

$\vec{B}$ bestimmt $\vec{A}$ bis auf Eichtransformation $\vec{A}\ \rightarrow \ \vec{A} \ + \ \grad \chi$ eindeutig, da beide Vektorpotentiale das selbe $\vec{B}$-Feld liefern. Bei spezieller Wahl von $\chi$ spricht man von fixierter Eichung.\\
Mit der Einführung von $\vec{A}$ folgt mit $\frac{1}{\mu_0}\ \rot \vec{B} = \vec{j}$:

\begin{equation*}
\frac{1}{\mu_0}\ (\rot\rot\vec{A})=\vec{j} \quad \text{bzw.} \quad \grad\div \vec{A}\ - \ \laplace\vec{A}=\mu_0 \ \vec{j}
\end{equation*}

Es bietet sich an, die Eichung div $\vec{A} = 0$ (\textsc{Coulomb}-Eichung) zu wählen, sodass folgt:

\begin{equation*}
-\laplace\vec{A}=\mu_0 \ \vec{j}
\end{equation*}

Eine ähnliche Gleichung haben wir mit der \textsc{Poisson}-Gleichung $-\epsilon_0 \ \laplace\varphi = \rho$ in der Elektrostatik hergeleitet und diese mit $\varphi = \frac{1}{4\pi\epsilon_0}\Int{}{}{V'}\frac{\rho(\vec{r}')}{|\vec{r}-\vec{r}'|}$ gelöst.\ \\
Analog erhalten wir auch die Lösung für $\vec{A}$:

\begin{align*}
\vec{A}(\vec{r}) &= \frac{\mu_0}{4\pi}\Int{}{}{V'} \ \frac{\vec{j}(\vec{r}')}{|\vec{r}-\vec{r}'|} \qquad \text{mit div }\vec{A} = 0 \\
\ \\
\vec{B}(\vec{r}) &= \rot \vec{A}(\vec{r}) = \frac{\mu_0}{4\pi}\ \Int {}{}{V'} \ \frac{\vec{j}(\vec{r}')\times (\vec{r}-\vec{r}')}{|\vec{r}-\vec{r}'|^3}
\end{align*}

Die Kontrolle, ob div $\vec{A} =0$ ist, ergibt:

\begin{equation*}
\div\vec{A}(\vec{r}) = \frac{\mu_0}{4\pi} \ \Int{}{}{V''}\ \frac{1}{r''} \ \underbrace{\div\vec{j(\vec{r}+\vec{r}''})}_{=0}=0 \qquad \text{mit } \vec{r}'' = \vec{r}'-\vec{r}
\end{equation*}

\section{Leiterschleifen}

\begin{wrapfigure}[]{l}[0cm]{0cm}
	\raisebox{0pt}[\dimexpr\height-1\baselineskip\relax]{
		\colorbox{hgrey}{
			\begin{tikzpicture}
			
				\draw[ultra thick] plot[smooth, tension=.8] coordinates {(1.7,-0.1) (1.9,1.5) (2.4,3) };	
				\draw[->] (0,0) -- node[above left]{$\vec{r}'$} (2,2);
				\draw[->] (0,0) -- node[below right]{$\ \vec{r}$} (4,1) ;
			
			
			\end{tikzpicture}
		}
	}
	\caption{Ausschnitt einer Leiterschleife}
\end{wrapfigure}
Wir betrachten einen Leiter der Dicke $d$ an der Position $\vec{r}'$. Für "`dünne"' Leiter, d.h. $d \ll |\vec{r}-\vec{r}'|$, kann man d$V' \vec{j}(\vec{r})$ vereinfachen zu: d$\vec{r}' \cdot I$, wobei das Längenelement d$\vec{r}'$ entlang des Leiters verlaufen soll.\
Für mehrere Leiter $\mathcal{L}_n$ folgt demnach für das Vektorpotential:

\begin{align*}
\vec{A}(\vec{r}) &= \frac{\mu_0}{4\pi}\ \sum_n \ I_n \ \int\limits_{\mathcal{L}_n} \frac{\mathrm{d}\vec{r}'}{|\vec{r}-\vec{r}'|}\\
\ \\
\overset{\vec{B}= \rot\vec{A}}{\Rightarrow} \quad \vec{B}(\vec{r}) &= \frac{\mu_0}{4\pi} \ \sum_n \ I_n \int\limits_{\mathcal{L}_n} \frac{\mathrm{d}\vec{r}' \times (\vec{r}-\vec{r}')}{|\vec{r}-\vec{r}'|^3}
\end{align*}
\ \\
Diese Gleichung zur Bestimmung des $\vec{B}$-Feldes einer beliebigen Anordnung dünner Leiter nennt man das \textbf{\textsc{Biot-Savart}-Gesetz}.\
\\
\ \\
Betrachten wir nun bei mehreren geschlossenen Leiterschleifen den magnetischen Fluss auf deren Oberflächen $\mathcal{S}_m$:

\begin{equation*}
\Phi = \Iint{\mathcal{S}_m}{}{\vec{A}_{\mathcal{S}_m}}\cdot\vec{B} = \Oint{\partial\mathcal{S}_m}{}{\vec{r}}\cdot\vec{A}
\end{equation*}

\newpage
Mit \textsc{Biot-Savart} ergibt sich:

\begin{equation*}
\Phi = \frac{\mu_0}{4\pi}\ \sum_n \ I_n \ \Oint{\partial\mathcal{S}_m}{}{\vec{r}}\ \cdot \ \Oint{\partial\mathcal{S}_n}{}{\vec{r}'}\frac{1}{|\vec{r}-\vec{r}'|} \ =: \ \sum_n \ L_{mn} \ I_n
\end{equation*}

\begin{equation*}
\text{mit } L_{mn} = \frac{\mu_0}{4\pi} \ \Oint{\partial\mathcal{S}_m}{}{\vec{r}}\ \Oint{\partial\mathcal{S}_n}{}{\vec{r}'}\frac{1}{|\vec{r}-\vec{r}'|}
\end{equation*}

Die $L_{mn}$ sind die sogenannten \textbf{Induktionskoeffizienten}, welche ebenso wie die Kapazitätskoeefizienten symmetrisch sind: $L_{mn} = L_{nm}$. Für $m=n$ redet man von \textbf{Selbstinduktivitäten} der Leiter, welche allerdings  nicht mit der obigen Formel berechnet werden können, da dann die Näherung der "`dünnen"' Leiter zusammenbricht.

\section{Magnetischer Dipol}

Für eine geschlossene Leiterschleife der Fläche $\vec{A}_F$, durch die der Ringstrom $I$ fließt, definieren wir das \textbf{magnetische Dipolmoment} $\vec{m}$ wie folgt:

\begin{equation*}
\vec{m} \ := \ I \ \cdot \ \vec{A}_F
\end{equation*}
\begin{equation*}
\text{Dipollimit: } \quad |\vec{A}_F| \ \rightarrow \ 0, \ I \ \rightarrow \ \infty \ \Rightarrow \ |\vec{m}| \ = \ \text{const.}
\end{equation*}

Um das Vektorpotential

\begin{equation*}
\vec{A}(\vec{r}) = \frac{\mu_0 I}{4\pi}\oint\frac{\mathrm{d}\vec{r}'}{|\vec{r}-\vec{r}'|}
\end{equation*}

für diesen Dipol zu berechnen entwickeln wir dieses unter der Näherung großer Abstände zum Dipol( $r\gg a$, wobei $a$ die größte Ausdehnung des Dipols in eine Raumrichtung ist):

\begin{equation*}
\frac{1}{|\vec{r}-\vec{r}'|} \cong  \frac{1}{|\vec{r}|} + \frac{\vec{r}\cdot\vec{r}'}{|\vec{r}|^3} + \dotsc \quad \Rightarrow \quad \oint\frac{\mathrm{d}\vec{r}'}{|\vec{r}-\vec{r}'|} = \underbrace{\frac{1}{r} \Oint{}{}{\vec{r}'}}_{=0} \ + \ \frac{\vec{r}}{r^2}\oint\d\vec{r} \circ\vec{r}' \ + \dotsc
\end{equation*}
\newpage
Umformen ergibt:

\begin{align*}
\Oint{}{}{\vec{r}'} (\vec{r}'\cdot\vec{a}) &= \frac{1}{2}\Bigg[\Oint{}{}{\vec{r}'}(\vec{r}'\cdot\vec{a}) \ - \oint(\mathrm{d}\vec{r}'\cdot\vec{a})\vec{r}'\Bigg] \ + \ \frac{1}{2} \Bigg[\Oint{}{}{\vec{r}'}(\vec{r}'\cdot\vec{a}) \ + \ \oint(\mathrm{d}\vec{r}'\cdot\vec{a})\vec{r}'\Bigg]\\
&= \underbrace{\frac{1}{2}\oint(\vec{r}'\times\mathrm{d}\vec{r}')}_{\text{Fläche }A_F} \times\vec{a} \ + \ \frac{1}{2} \oint\mathrm{d}\left[\vec{r}'(\vec{r}'\cdot\vec{a})\right]\\ 
\ \\
& \Rightarrow \quad \vec{A}(\vec{r}) = \frac{\mu_0 I }{4\pi} \; \vec{A}_F\times\frac{\vec{r}}{r^3} = \frac{\mu_0}{4\pi} \cdot \frac{\vec{m}\times\vec{r}}{r^3}
\end{align*}
\ \\
(zum Vergleich das Potential eines Elektrischen Dipols: $\varphi(\vec{r}) = \frac{1}{4\pi\epsilon_0} \cdot \frac{\vec{p}\cdot\vec{r}}{r^3}$)
\ \\
\begin{align*}
\vec{B}(\vec{r}) = \rot\vec{A}(\vec{r}) &= - \frac{\mu_0 I}{4\pi} \ \nabla \times \left(A_F\times\nabla\frac{1}{r}\right) \ = \ \frac{\mu_0}{4\pi} \ \cdot \ 3(\vec{m}\cdot\vec{r})\vec{r} \ - \ \vec{m}r^2\\
\ \\
&= \ \underbrace{\vec{A}_F \ \laplace\frac{1}{r}}_{(*)} \quad - \quad \underbrace{(\vec{A}_F
\cdot\nabla)\ \nabla \ \frac{1}{r}}_{= \ \vec{A}_F (\nabla \circ \nabla) \frac{1}{r}}
\end{align*}

$(*) = 4\pi\delta(\vec{r})$ wird im Fernfeld vernachlässigt
\ \\
\ \\
\underline{\textbf{Ladung auf Umlaufbahn:}}$\qquad$ (Ladung $Q$, Masse $M$)

\begin{equation*}
I = \frac{Q}{\tau} \qquad \text{ mit } \qquad \tau \; \hat{=} \; \text{Umlaufzeit}
\end{equation*}
\begin{equation*}
\vec{A}_F = \frac{1}{2}\oint\vec{r}\times\mathrm{d}\vec{r} = \frac{1}{2}\Int{0}{\tau}{t} \ \left(\vec{r}\times\diff{\vec{r}}{t}\right) =\frac{1}{2}\tau \frac{\vec{L}}{M}
\end{equation*}

Das Magnetische Dipolmoment ist also bei einer Ladung auf einer Umlaufbahn eng mit dessen Drehimpuls $\vec{L}$ verknüpft:

\begin{equation*}
\vec{m}  \ =\ I \ \cdot \ \vec{A}_F \ = \ \frac{1}{2} \ \underbrace{\frac{Q}{M}}_{=: g_B} \ \vec{L}
\end{equation*}

(Zum Vergleich das \textsc{Bohr}'sche Magnetron: $\mu_B = \frac{e\hbar}{2m}$)
\newpage

\underline{\textbf{Allgemeine Stromverteilung:}}

\begin{equation*}
\vec{m} \ = \ \frac{1}{2} \ \Int{}{}{V} \; \vec{r}\ \times \ \vec{j}(\vec{r})
\end{equation*}

(Zum Vergleich der elektrische Dipol: $\vec{p} = \Int{}{}{V} \vec{r}\cdot\rho(\vec{r})$)
\input{chapter/05.tex}
\chapter[Energie- und Impulsbilanz]{Energie- und Impulsbilanz des elektromagnetischen Feldes}

\section{Bilanzgleichungen}

Wir betrachten in einem Volumen $V$ die \textbf{Observable} $A$, für die wir auch ganz allgemein eine Dichte definieren wollen:

\begin{equation*}
A=\Int{V}{}{V} \ a   \quad \Rightarrow \quad a := \diff{A}{V} \quad \text{ist die Dichte von } A
\end{equation*}

Anschaulich kann man sagen, dass sich die zeitliche Änderung von $A$ in den Volumen aus seiner Erzeugungsrate $N_A$ und seinem Strom $I_A$ aus dem Volumen heraus zusammensetzt:

\begin{equation*}
\dot{A}(t) \ = \ - I_A \ + \ N_A
\end{equation*}

Analog zur Dichte $a$ von $A$ wollen wir nun auch für den Strom $I_A$ eine Stromdichte $\vec{j}_a$ durch die Oberfläche $\partial V$ und für die Erzeugungsrate $N_A$ eine Erzeugungsdichte $\nu_a$ im Volumen $V$ definieren, sodass gilt:

\begin{equation*}
\Int{V}{}{V} \ \partial_t \ a = -\Oiint{\partial V}{}{\vec{F}} \cdot\vec{j}_a \ + \ \Int{V}{}{V} \ \nu_a \ = \ \Int{V}{}{V} \left(- \ \div\vec{j}_a \ + \ \nu_a\right)
\end{equation*} 

Daraus folgt die \textbf{allgemeine Bilanzgleichung}:

\begin{equation*}
\dot{a} \ +  \div\vec{j}_a \  =  \ \nu_a
\end{equation*}

\newpage
Falls $A$ eine Erhaltungsgröße ist, gilt:

\begin{equation*}
N_A = 0, \nu_a = 0 \quad \Rightarrow \quad \dot{a}  \ + \div \vec{j}_a \ = \ 0 \quad \Rightarrow \quad \dot{A} = -\Oiint{\partial V}{}{\vec{F}}\cdot\vec{j}_a
\end{equation*}

Für den Grenzfall, dass $V\rightarrow\infty$, folgt, dass $\dot{A}=0$ und somit $A =$ \textit{const.}, was das erwartete Verhalten einer Erhaltungsgröße widerspiegelt.

\section{Energiebilanz}

Auf eine Punktladung $Q$ wirkt die Kraft $\vec{F}_L = Q(\vec{v}\times\vec{B} \ + \ \vec{E})$ worüber man die Leistung des Feldes an der Ladung $N = \vec{F}\cdot\vec{v}$ ableiten kann.\
Für eine Energieänderung des elektromagnetischen Feldes gilt dementsprechend:

\begin{equation*}
\dot{W}_{em} \ = \ -\vec{v}\cdot\vec{F}_L = -Q \ \cdot \ \vec{v} \ \vec{E}
\end{equation*}

Für eine Änderung der Energiedichte $\nu_{em}$ folgt daraus bei mehreren Ladungsträgerarten:

\begin{equation*}
\nu_{em} \ =  \ - \sum_i \ \rho_i \ \vec{v}_i \ \vec{E} \ = \ - \vec{j}\cdot\vec{E}
\end{equation*}

Damit lautet die Bilanzgleichung, welche in diesem Zusammenhang auch  \textbf{\textsc{Poynting}-Theorem} genannt wird:

\begin{equation*}
\pdiff{w}{t} \ + \ \div \vec{S}_P \ = \ \nu \ = \ -\vec{j} \cdot \vec{E}
\end{equation*}

wobei $w$ die Energiedichte und $\vec{S}_P$ die Energiestromdichte (auch \textbf{\textsc{Poynting}-Vektor} genannt) ist.\
$w$ und $\vec{S}_P$ hängen vom $\vec{E}$- und $\vec{B}$-Feld ab, also sind diese nach \textsc{Maxwell} zu bestimmen:

\begin{align*}
\nu \ &= -\vec{j}\cdot\vec{E} = \epsilon_0 \ \dot{\vec{E}} \ \vec{E} \ - \ \frac{1}{\mu_0} \left(\nabla\times\vec{B}\right) \cdot \vec{E}\\
&= \partial_t \left(\frac{\epsilon_0}{2} \vec{E}^2\right) \ - \frac{1}{\mu_0} \nabla \cdot (\vec{B}\times\vec{E}) \ - \ \frac{1}{\mu_0} \vec{B}\cdot \underbrace{(\nabla\times\vec{B})}_{= \dot{\vec{B}}}\\
&= \frac{1}{2}\partial_t \left(\epsilon_0\vec{E}^2 \ + \ \frac{1}{\mu_0}\vec{B}^2\right) \ - \ \frac{1}{\mu_0}\nabla \cdot (\vec{B}\times\vec{E})
\end{align*}

\newpage
Der Vergleich mit dem \textsc{Poynting}-Theorem ergibt:

\begin{align*}
w \ &= \ \frac{1}{2} \left(\epsilon_0\vec{E}^2 \ + \ \frac{1}{\mu_0}\vec{B}^2\right)\\
\vec{S}_P \ &= \ \frac{1}{\mu_0} \ \vec{E}\times\vec{B} 
\end{align*}

\ \\
\ \\
\underline{Beispiel zur Erzeugungsdichte $\nu$:}$\qquad$ \textsc{Ohm}'sches Gesetz $\vec{j} = \sigma \cdot \vec{E}$

\begin{equation*}
\nu \ = \ - \sigma \ \cdot \ \vec{E}^2 \ = \ - \frac{\vec{j}^2}{\sigma}
\end{equation*}

Der erhaltene Ausdruck für die Erzeugungsdichte entspricht der \textbf{\textsc{Ohm}'schen Wärme}.

\section{Elektrostatische Feldenergie}

\begin{equation*}
W_e \ = \ \Int{}{}{V} \frac{\epsilon_0}{2} \vec{E}^2 \ = \ - \Int{}{}{V} \frac{\epsilon_0}{2} \vec{E} \ \grad\varphi
\end{equation*}

Nutze zur Umformung partielle Integration mit dem Satz von \textsc{Gauss}:

\begin{align*}
\Rightarrow W_e &= \Int{}{}{V} \frac{\epsilon_0}{2} \ (\nabla \cdot \vec{E}) \varphi \ - \ \underbrace{\Oiint{}{}{\vec{A}} \cdot \frac{\epsilon_0}{2}\vec{E}\varphi}_{=0 \text{ im gesamten Raum}} \\
\\
&= \ \frac{1}{2} \ \Int{}{}{V} \varphi \cdot \rho \quad = \quad \frac{1}{2} \Int{}{}{Q} \cdot \varphi
\end{align*}

Dies entspricht auch der Anschauung, dass Energie = Ladung $\cdot$ Potential.\
Umschreiben ergibt:

\begin{equation*}
W_e \ = \ \frac{1}{2} \ \Int{}{}{V} \rho \cdot \varphi \ = \ \frac{1}{8\pi\epsilon_0} \ \int\d V\int\d V' \; \frac{\rho(\vec{r}) \ \rho(\vec{r}')}{|\vec{r}-\vec{r}'|}
\end{equation*}

\ \\
Für eine Punktladung ergibt die erhaltene Gleichung: 

\begin{align*}
W_e \quad &= \quad \sum_{i\neq j} \ \frac{Q_i \ Q_j}{8\pi\epsilon_0 \ |\vec{r}-\vec{r}'|} \; + \; \text{\textbf{ Selbstenergie} für i = j}\\
&= \quad \sum_{i<j} \ \frac{Q_i \ Q_j}{8\pi\epsilon_0 \ |\vec{r}-\vec{r}'|} \; + \; \text{ Selbstenergie für i = j}\\
\end{align*}

Für die Selbstenergie gilt zunächst für eine geladene Kugel mit dem Radus $a$: berechnen:

\begin{equation*}
W_e \ = \ \alpha \cdot \frac{Q^2}{8\pi\epsilon_0 \ a} \qquad \text{mit} \quad \alpha = \begin{cases}
	\frac{6}{5} \quad \text{für homogene Kugel}\\
	1 \quad \text{für Hohlkugel}
  \end{cases}	
\end{equation*}	

Wenn man nun für diese Kugel den Grenzübergang zu einer Punktladung machen möchte und $a$ gegen 0 gehen lässt, so erhält man als Ergebnis, dass die Selbstenergie einer Punktladung unendlich sein müsste. An dieser Stelle ist die klassische Elektrodynamik nicht anwendbar, da sie als Kontinuumstheorie an ihre Grenzen stößt. Für Selbstenergie von Elementarteilchen ist also eine Erweiterung der Theorie der Elektrodynamik, welche ausschließlich auf den \textsc{Maxwell}-Gleichungen beruht, vonnöten, so wie es in der Quantenelektrodynamik behandelt wird.

\section{Elektrostatische Energie einer Leiteranordnung}

Da wir eine feste Leiteranordnung betrachten, folgt daraus, dass es keine Raumladungen gibt, sondern diese an die Leiteroberflächen gebunden sind.

\begin{align*}
& \quad W_e \ = \ \frac{1}{2} \ \Oiint{}{}{A} \sigma\cdot\varphi \  =  \frac{1}{2} \sum_i \ \varphi_i \ Q_i\\
&\quad \text{wobei die }\varphi_i = \varphi \text{ auf den Leiteroberflächen konstant sind}
\end{align*}

\underline{Beispiel:} $\quad Q=Q_1=Q_2 \quad\Rightarrow\quad  W_e =\frac{1}{2}Q(\varphi_1-\varphi_2 ) = \frac{1}{2}QU = \frac{1}{2}CU^2 = \frac{1}{2}\frac{Q^2}{C}$\\
\ \\
allgemein gilt: $Q_i \ = \ \sum_i \ C_{ik} \ \varphi_k$, sodass für die elektrostatische Energie folgt:

\begin{equation*}
\Rightarrow \quad W_e = \frac{1}{2} \sum_{ik} \ \varphi_i \ C_{ik} \ \varphi_k \ = 
\ \frac{1}{2} \sum_{ik} \ Q_i \ \tilde{C}_{ik} \ Q_k
\end{equation*}

\newpage
Da $W_e$ aufgrund von $W_e = \int\d V \ \frac{\epsilon_0}{2} \vec{E}^2$ immer größer oder gleich 0 ist, folgt daraus, dass die $C_{ik}$ bzw. $\tilde{C}_{ik}$ positiv definit sein müssen (insbesondere gilt sogar: $C_{ii} > 0$ und $\tilde{C}_{ii} > 0$)\\
Wenn wir nun kleine Ladungsänderungen $\rho \rightarrow \rho +\d \rho, \varphi \rightarrow\varphi + \d\varphi$ betrachten erhalten wir:

\begin{align*}
\delta\varphi &= \ \frac{1}{4\pi\epsilon_0} \ \Int{}{}{V} \frac{\delta\rho(\vec{r})}{|\vec{r}-\vec{r}'|}\\
\delta W_e &= \ \frac{1}{2} \ \Int{}{}{V} (\d\rho \ \varphi \; + \; \rho \ \d\varphi) \ = \ \frac{1}{4\pi\epsilon_0} \cdot\frac{1}{2}\cdot 2 \ \int\d V \d V' \ \frac{\rho(\vec{r}) \ \delta\rho(\vec{r}')}{|\vec{r}-\vec{r}'|}
\end{align*}

\begin{align*}
\text{für Flächenladungen: } & \quad\delta W_e \ = \ \frac{1}{2}\Int{}{}{A}  \delta\sigma \ \varphi \ = \ \Int{}{}{A}  \sigma \ \delta\varphi\\
\text{für Leiter: } & \quad\delta W_e \ = \frac{1}{2}\sum_i \ \delta(Q_i\varphi_i) \ = \ \sum_i \varphi_i \delta Q_i \ = \ \sum_i \ Q_i \ \delta\varphi_i
\end{align*}

\ \\
\underline{\textbf{Spezialfälle:}}\\

\begin{enumerate}

\item Verschiebung von Ladungen entlang der Leiteroberfläche

\begin{equation*}
\delta Q_i = 0 \quad \Rightarrow \quad \delta W_e = 0
\end{equation*}

Da Verschiebung $\perp$ Kraft, ist auch die Arbeit 0.\\
Daraus folgt, dass $W_e$ im Gleichgewicht Extremum (i.A. Minimum) annimmt (\textbf{\textsc{Thompson}'scher Satz}).

\item Transport von Ladungen zwischen Leitern

\begin{equation*}
\delta Q_i \ \neq \ 0 \quad\Rightarrow\quad \delta W_e = \sum_i \ Q_i \ \delta\varphi_i \ = \ \sum_i \ \varphi_i \ \delta Q_i
\end{equation*} 

Beachte:

\begin{align*}
C_{ik} \ &= \ \frac{\partial^2 W_e}{\partial \varphi_i \ \partial\varphi_k} \qquad(\Rightarrow \ C_{ik} \ = \ C_{ki})\\
Q_i \ &= \ \pdiff{W_e}{\varphi_i} \ = \ \sum_k \ C_{ik} \varphi_k\\
\delta W_e \ &= \ \sum_i \ \pdiff{W_e(\varphi_k)}{\varphi_i} \ \delta\varphi_i \ = \ \d W_e \qquad (\text{totales Differential})
\end{align*}
\end{enumerate}

\section{Energie des stationären Magnetfelds}

\begin{align*}
W_m & \ \quad = \quad  \ \Int{}{}{V}\frac{1}{2\mu_0}\vec{B}^2 \ =  \ \Int{}{}{V} \frac{1}{2\mu_0} \vec{B}\times\rot\vec{A} \ = \ \Int{}{}{V} \frac{1}{2\mu_0} \ \left(\vec{B}\times\nabla\right) \overset{\downarrow}{\vec{A}}\\
& \overset{\text{part. Int.}}{=}  \  \Int{}{}{V} \frac{1}{2\mu_0} \vec{A} \cdot\left(\nabla\times\vec{B}\right) \quad + \quad \text{Oberflächenintegral} \left(\rightarrow 0 \text{ für } V \rightarrow \infty\right)\\
& \;\;\;\;\overset{\dot{\vec{E}}=0}{=} \ \frac{1}{2} \Int{}{}{V} \ \vec{j}\cdot\vec{A}
\end{align*}

Analog zum elektrostatischen Fall ergibt Umschreiben:

\begin{equation*}
W_m = \frac{\mu_0}{8\pi} \ \int\d V \int\d V' \ \frac{\vec{j}(\vec{r}) \ \vec{j}(\vec{r}')}{|\vec{r}-\vec{r}'|}
\end{equation*}
\ \\

Für dünne linienförmige und geschlossene Leiterschleifen $\mathcal{L}_i$ gilt mit $\int\d V\vec{j} \rightarrow \int\d \vec{r} \cdot I$ und unter Anwendung des Satzes von \textsc{Stokes}:

\begin{equation*}
W_m = \frac{1}{2} \cdot \sum_i I_i \Int{\mathcal{L}_i}{}{\vec{r}} \cdot \vec{A} \ = \ \frac{1}{2} \sum_i I_i \Phi_i
\end{equation*}
\ \\
\ \\
Allgemein folgt somit aus $\Phi_i = \sum_k L_{ik} I_k$:

\begin{align*}
W_m \ &= \ \frac{1}{2} \ \sum_{ik}  I_i L_{ik} I_k \ = \ \frac{1}{2} \ \sum_{ik} \Phi_i \tilde{L}_{ik} \Phi_k\\
&= \ \frac{1}{2} \ \sum_{i\neq k} I_i I_k \ \underbrace{\frac{\mu_0}{4\pi} \ \Int{\mathcal{L}_i}{}{\vec{r}} \Int{\mathcal{L}_k}{}{\vec{r}'} \ \frac{1}{|\vec{r}-\vec{r}'|}}_{L_{ik}} \; + \; \text{Selbstenergie für } (i=k)
\end{align*}

Ähnlich wie im elektrostatischen Analogon stößt die klassische Elektrodynamik bei der Berechnung der Selbstenergien für ``dünne'' und somit sonst ideale Leiter an ihre Grenzen. Für eine Leiterschleife endlicher Dicke kann man die Selbstenergie jedoch wieder berechnen, sie beträgt:

\begin{align*}
& \ W_m  \ = \ \frac{1}{2} \ \cdot  \ L \ \cdot \ I^2 \\
\text{mit } \ & \ L \ = \ \frac{\mu_0}{4\pi I^2}\ \Int{}{}{V'} \frac{\vec{j}(\vec{r})\vec{j}(\vec{r}')}{|\vec{r}-\vec{r}'|} \ = \ \frac{1}{I^2} \ \Int{}{}{V} \frac{\vec{B}^2}{\mu_0} \quad
 \left( = \ \frac{\Phi}{I}\right) 
\end{align*}

\section{Beispiele für Energiestromdichten}

\begin{enumerate}
\item \textbf{ Stromdurchflossener gerader Leiter}\\
\ \\
\begin{equation*}
\frac{1}{\mu_0} \rot \vec{B} \ = \ \vec{j} \ + \ \epsilon_0 \dot{\vec{E}}
\end{equation*}

Für $\dot{\vec{E}}=0$ und der integralen Formulierung $\Oint{}{}{\vec{r}} \cdot \vec{B} = \mu_0 I$ ebenjener \textsc{Maxwell}-Gleichung folgt, dass um den geraden Leiter ein tangentiales $\vec{B}$-Feld existiert:

\begin{equation*}
B \ = \ \frac{\mu_0 \ I}{2\pi \ r_{\perp}}
\end{equation*}

Mit dem \textsc{Ohm}'schen Gesetz $\vec{E} = \sigma \cdot\vec{j}$ folgt, dass das $\vec{E}$-Feld entlang des Leiters gerichtet sein muss. Somit gilt für die Energiestromdichte $\vec{S}_P = \frac{1}{\mu_0} \left(\vec{E}\times\vec{B}\right)$, dass sie radial nach innen gerichtet sein muss.\\
\ \\
Bei einem einfachen Stromkreis wird demnach die Energie nicht entlang der Leiter sondern über die erzeugten Feldern von der Spannungsquelle zum Verbraucher transportiert!\\
\ \\
Berechnet man nun außerdem das Flächenintegral über die Energiestromdichte, erhält man für den geraden Leiter:

\begin{equation*}
\Iint{}{}{\vec{A}_F}\cdot\vec{S}_P \ = \ 2\pi r_{\perp} l \ S_P \ = \ 2\pi r_{\perp} l \frac{1}{\mu_0}E\frac{\mu_0 I}{2\pi r_{\perp}} \ = \ l \cdot E \cdot I
\end{equation*}

Der erhaltene Ausdruck $N := l \cdot E \cdot I = U \cdot I$ ist somit anschaulich die abgestrahlte Energie pro Zeiteinheit und ist auch als \textbf{\textsc{Ohm}'scher Verlust} oder \textbf{\textsc{Ohm}'sche Wärme} bekannt.


\item \textbf{ ideale parallele Doppelleiter mit entgegengesetzten Stromrichtungen}
\ \\

Hier betrachten wir gleich zu Beginn das Flächenintegral über der Energiestromdichte und setzen nur die Querschnittsfläche ein:

\begin{align*}
N \ &= \ \Int{}{}{\vec{A}_F} \cdot \vec{S}_P \ = \ \frac{1}{\mu_0} \left(\vec{E}\times\vec{B}\right) \ \overset{\text{stationär}}{=} \ - \frac{1}{\mu_0} \Int{}{}{\vec{A}_F} \cdot \left(\nabla \varphi \times \vec{B}\right)\\
&= \ - \frac{1}{\mu_0} \Int{}{}{\vec{A}_F} \cdot \left(\nabla\times\left(\varphi\vec{B}\right)\right) \; + \; \Int{}{}{\vec{A}_F}\cdot\varphi\cdot\underbrace{\left(\nabla\times\vec{B}\right)}_{=\vec{j}}
\end{align*}

Nach Umformen mit \textsc{Stokes} erhält man für den ersten Summanden:

\begin{equation*}
\Oint{\partial A_F}{}{\vec{r}}  \cdot \varphi \vec{B}
\end{equation*}

doch dieser Anteil geht für $\partial A_F \rightarrow \infty$ schnell genug gegen Null. Somit folgt für die Leistung:

\begin{equation*}
\Rightarrow \; N \ = \ \Int{}{}{\vec{A}_F} \cdot \varphi \vec{j} \ = \ I \left(\varphi_1 - \varphi_2 \right) \ = \ I \cdot U_{12}
\end{equation*}
Beim Doppelleiter wird die Leistung entlang der Leiter transportiert.
\end{enumerate}


\section{Energie einer ebenen harmonischen Welle}

Wir betrachten:

\begin{align*}
\vec{E} \ &= \ \vec{E}_0 \cdot \operatorname{Re} e^{i\left(\vec{k}\vec{r}-\omega t\right)} \qquad \text{mit} \qquad \vec{E}\perp\vec{k}, \; \vec{E}_0 \text{ reell}\\
\vec{B} \ &= \ \frac{1}{c}\left(\vec{e}_k\times\vec{E}\right) \qquad \text{mit} \qquad \vec{e}_k = \frac{\vec{k}}{k}
\end{align*}

\textbf{Energiedichte:}\\

\begin{align*}
w \ &= \ \frac{\epsilon_0}{2} \vec{E}^2 \; + \; \frac{1}{2\mu_0}\vec{B}^2 = \left[\frac{\epsilon_0}{2}\vec{E}_0^2 + \frac{1}{2\mu_0 c^2} \ \left(\vec{e}_k\times\vec{E}\right)^2\right] \cos^2\left(\vec{k}\vec{r}-\omega t\right)\\
\ \\
w \ &= \ \epsilon_0 \vec{E}_0^2 \ \cos^2\left(\vec{k}\vec{r}-\omega t\right)
\end{align*}

Räumliche oder zeitliche Mittelung: $\qquad\left(\langle\cos^2(.)\rangle = \frac{1}{2}(.)\right)$

\begin{equation*}
\langle w\rangle \ = \ \frac{\epsilon_0}{2} \ \vec{E}_0^2
\end{equation*}

\ \\
\textbf{Energiestromdichte:}

\begin{align*}
\vec{S}_P \ &= \ \frac{1}{\mu_0} \vec{E}\times\vec{B} \ = \ \frac{1}{\mu_0 c} \vec{E}_0 \times \left(\vec{e}_k\times\vec{E}_0\right) \ \cos^2\left(\vec{k}\vec{r}-\omega t\right)\\
&= \ \epsilon_0 c \; \Bigg(\vec{e}_k\vec{E}_0^2 \ - \ \vec{E}_0\underbrace{\left(\vec{e}_k\cdot\vec{E}_0\right)}_{=0}\Bigg) \ \cos^2\left(\vec{k}\vec{r}-\omega t\right)\\
\vec{S}_p \ &= \ c \ \vec{e}_k \ \epsilon_0 \ \vec{E}_0 \ \cos^2 \left(\vec{k}\vec{r}-\omega t\right) \ = \ c \cdot  \vec{e}_k \cdot w
\end{align*}

nach Mittelung:

\begin{equation*}
\vec{S}_P \ = \ c \cdot \vec{e}_k \cdot \langle w\rangle \ = \ c \cdot \vec{e}_k \ \frac{\epsilon_0}{2} \ \vec{E}_0 \ \left( = \frac{1}{2\mu_0} \operatorname{Re} \ (\vec{E}\times\vec{B}^*)\right)
\end{equation*}
\ \\

\section{Impulsbilanz des elektromagnetischen Feldes}
\ \\
Impulsänderung = Kraft

\begin{equation*}
\Rightarrow \ \vec{F}_L \ = \ \dot{\vec{p}}_{\text{mech}} \ = \  \dot{\vec{p}}_{\text{elm}} \quad \text{(Impuls im em. Feld)}
\end{equation*}
\ \\

Bilanzgleichung pro Volumen:

\begin{align*}
\pdiff{\vec{g}}{t} \; + \; \div \tens{T} & \ = \ - \vec{f}_L \\
\ \\
\text{mit} \qquad \vec{g} & \ - \ \text{Impulsdichte}\\
\vec{f}_L & \ - \ \text{Lorentzkraftdichte} \quad \left( = \rho\vec{E} \ + \ \vec{j}\times\vec{B}\right)\\
\tens{T} & \ - \ \text{Impulsstromdichte}
\end{align*}

$ - \vec{f}_L$ ist dementsprechend die elektromagnetische Impulserzeugungsrate pro Volumen $\nu_g$.
$\vec{g}$ und $\tens{T}$ hängen im Allgemeinen von Feldern ab. \textsc{Maxwell} liefert uns:

\begin{align*}
- \vec{f}_L \ &= \ \epsilon_0 ( \nabla \cdot\overset{\downarrow}{\vec{E}} ) \ + \ \frac{1}{\mu_0} \ \vec{B} \times (\nabla \times \vec{B}) \ + \ \epsilon_0 \dot{\vec{E}} \times \vec{B}\\
\overset{\text{Umformen}}{\Longrightarrow} \quad\vec{E}\times\vec{B} \ &= \ \partial_t ( \vec{E} \times \vec{B} ) \ - \ \vec{E}\times\dot{\vec{B}} \ =  \ \partial_t (\vec{E}\times\vec{B}) \ + \ \vec{E}\times (\nabla\times\vec{E})\\
\vec{E}\times ( \nabla\times \overset{\downarrow}{\vec{E}} ) \ &= \ \nabla \ (\overset{\downarrow}{\vec{E}}\cdot\vec{E} ) \ - \ ( \vec{E}\cdot\nabla ) \ \overset{\downarrow}{\vec{E}} \ = \ \frac{1}{2}\nabla\vec{E}^2 \ - \ ( \vec{E}\cdot\nabla) \ \overset{\downarrow}{\vec{E}}
\end{align*}

Für $\vec{B}\times(\nabla\times\vec{B})$ analog, mit $(\nabla\cdot\overset{\downarrow}{\vec{B}})\vec{B}=0$. Insgesamt erhalten wir:

\begin{equation*}
(-\vec{f}_L)_k \ = \ \partial_t \ \epsilon_0 \ (\vec{E}\times\vec{B})_k \ - \ \pdiff{}{x_i} \epsilon_0 \ \left(\frac{\vec{E}^2}{2} \delta_{ik} \ - \ \vec{E}_i \vec{E}_k \right) \; + \; \pdiff{}{x_i} \frac{1}{\mu_0} \ \left(\frac{\vec{B}^2}{2}\delta_{ik} \ - \ \vec{B}_i \vec{B}_k \right) 
\end{equation*}

\ \\
Der Vergleich mit $\dot{\vec{g}} + \div\tens{T} = - \vec{f}_L$ ergibt:

\begin{align*}
\text{Impulsdichte} \qquad & \vec{g} \ = \ \epsilon_0 \ (\vec{E}\times\vec{B})\\
\text{Impulsstromdichte} \qquad & \tens{T} \ = \ \epsilon_0 \ \left(\frac{\vec{E}^2}{2} \mathbbm{1} \ - \ \vec{E}\circ\vec{E}\right) \ + \ \frac{1}{\mu_0} \ \left(\frac{\vec{B}^2}{2}\mathbbm{1} \ - \ \vec{B}\circ\vec{B}\right)\\
\text{Impulserzeugungsrate} \quad \ - & \vec{f}_L \ = \ \nu_g
\end{align*}

\ \\
\underline{\textbf{Diskussion:}}

\begin{enumerate}
\item \textbf{Impulsdichte:}

\begin{equation*}
\vec{g} \ = \ \epsilon_0\mu_0 \vec{S}_P \ = \ \frac{1}{c^2}\vec{S}_P
\end{equation*} 

Allgemeingültig für Feldtheorien bei Ausbreitung mit $c$, vgl. Relativitätstheorie.
\begin{equation*}
\text{für Welle gilt: } |\vec{S}_P| \ = \ c \cdot w \quad \Rightarrow \quad |\vec{g}| \ = \ \frac{w}{c} 
\end{equation*}

Zusammenhang mit Strahlungsdruck (Absorption einer em. Welle): 

\begin{align*}
\text{Impulsübertrag:  } \qquad & \Delta \vec{p} \ = \ \vec{g} \ \Delta V \ = \ \vec{g} \ c \ \Delta t \ \Delta A_F\\
\text{Druck: } \qquad & \frac{|\Delta\vec{p}|}{\Delta t \ \Delta A_F} \ = \ c \ |\vec{g}| \ = \ w
\end{align*}
\ \\
\item \textbf{Impulsstromdichte:} \\

Tensor $\tens{T}$ mit folgenden Eigenschaften:

\begin{itemize}
\item $T_{ik}$ mit $k$ = Impulskomponente, $i$ = Transportrichtung
\item $T_{ik} \ = \ T_{ki}$
\item $ [T_{ik}] \ = \ [\vec{f}] \cdot [l] \ = \ \frac{[\vec{F}]}{[l]^2} \quad \Rightarrow \quad$ Druck, Spannung
\item Stationäre Felder: $\quad \dot{\vec{g}} \ = \ 0 \quad \Rightarrow \quad \div \tens{T} \ = \ -\vec{f}_L$\\
\ \\
Volumenintegral + Satz von \textsc{Gauss} $ \quad \Rightarrow \quad \Oiint{}{}{\vec{A}_F} \cdot \tens{T} \ = \ - \vec{F}_L \quad$\\
\ \\
\textbf{Oberflächenkräfte}, Interpretation des Vorzeichens: Fläche schließt Ströme/Ladungen ein, auf die die Kräfte wirken
\end{itemize}
\end{enumerate}

\section{Beispiele für Impulsbilanz}

\begin{enumerate}
\item \textbf{ Plattenkondensator:}\\
\ \\
Wir betrachten einen unendlich ausgedehnten Plattenkondensator (Vernachlässigung von Randeffekten) welcher mit der Ladung $\pm Q$ beladen ist und das in ihm erzeugte $\vec{E}$-Feld entlang der x-Achse ausgerichtet ist. Es gilt:

\begin{align*}
& T_{ik} \ = \ \epsilon_0 \left(\frac{1}{2}\vec{E}^2\mathbbm{1} \ - \ \vec{E}\circ\vec{E}\right)_{ik}\\
\Rightarrow \qquad & T_{xx} \ = \ -\frac{\epsilon_0}{2}\vec{E}^2, \ T_{yy} \ = \ T_{zz} \ = \ \frac{\epsilon_0}{2} \vec{E}^2\\
& T_{xy} \ = \ T_{xz} \ = \ T_{yz} \ = \ 0
\end{align*}

Somit erhält man für die Kraft auf die linke Platte:

\begin{equation*}
(\vec{F}_L)_x \ = \ - \left( \Oiint{}{}{\vec{A}_F} \cdot \tens{T}\right)_x \ = \ {\vec{A}_F}_x \ T_{xx} \ = \ \frac{\epsilon_0}{2} A_F \vec{E}^2 \ = \ \frac{Q\cdot E}{2}  
\end{equation*}
\ \\
\item \textbf{ Lange Spule:}\\
\ \\
Wir betrachten eine Spule in der x-y-Ebene welche mit ihrer Symmetrieachse (längs) entlang der x-Achse ausgerichtet ist.
Für das $\vec{B}$-Feld entlang dieser Symmetrieachse gilt:

\begin{align*}
\vec{B} \ =& \ \mu_0 \ I \ \frac{N}{l} \ \vec{e}_x\\
\Rightarrow \qquad T_{xx} \ =& \ - T_{yy} \ = \ -T_{zz} \ = \ - \frac{1}{2\mu_0}\vec{B}^2
\end{align*}

\begin{itemize}
\item Spule quer teilen, Kraft auf linken Teil $(x<0)$:

\begin{equation*}
(\vec{F}_L)_x \ = \ - \underbrace{{A_F}_x}_{= \pi R^2} \ T_{xx} \ = \ \frac{A_F}{2\mu_0}\vec{B}^2 \ = \ \frac{A_F \ \mu_0 \ (I \cdot N)^2}{2\ l^2} \quad \text{Anziehung}
\end{equation*}

\item Spule längs teilen, Kraft auf unteren Teil $(y<0)$:

\begin{equation*}
(\vec{F}_L)_x \ = \ -\underbrace{{A_F}_y}_{=\pi R l} \ T_{yy} \ = \ -\frac{R \ l \ \vec{B}^2}{\mu_0} \quad \text{Abstoßung}
\end{equation*}

Interpretieren wir nun diese Ergebnisse mithilfe von Feldlinien, so können wir insgesamt verallgemeinern, dass parallel zu den Feldlinien eine Zugkraft herrscht, wohingegen senkrecht zu ihnen gedrückt wird. 
\end{itemize}

\item \textbf{ Isotrope Strahlung in einem Hohlraum}

Wir betrachten das Zeit- und Ortsmittel der Impulsstromdichte $\tens{T}$:

\begin{align*}
\langle\tens{T}\rangle \ &= \ \Big\langle \epsilon_0 \ \left(\frac{1}{2}\vec{E}^2\cdot\mathbbm{1} \ - \ \vec{E}\circ\vec{E}\right) \ + \ \frac{1}{\mu_0} \ \left(\frac{1}{2}\vec{B}^2\cdot\mathbbm{1} \ - \ \vec{B}\circ\vec{B}\right) \Big\rangle \ = \ T_0 \cdot \mathbbm{1}\\
\operatorname{Spur} \tens{T} \ &= \ 3 T_0 \ = \ \epsilon_0 \ \left(\frac{3}{2}\vec{E}^2 \ - \ \vec{E}^2\right) \ + \ \frac{1}{\mu_0} \ \left(\frac{3}{2}\vec{B}^2 \ - \ \vec{B}^2\right) \ = \ \frac{\epsilon_0}{2}\vec{E}^2 \ + \ \frac{1}{2\mu_0}\vec{B}^2 \\
\text{mit } T_0 \ &= \ \frac{1}{3} w 
\end{align*}

\newpage
Somit ist die Kraft auf die Wand:

\begin{equation*}
\Delta\vec{F}_L \ = \ -\Oiint{}{}{\vec{A}_F} \cdot\tens{T} \ = \ - \Delta A_F \cdot \langle\tens{T} \rangle \ = \ - \Delta A_F \cdot \frac{w}{3}
\end{equation*}

Dies entspricht dem Strahlungsdruck $p$ beim gerichtetem senkrechten Einfall auf die Wände in alle drei Raumrichtungen: $\quad p = \frac{1}{3}w$
\end{enumerate}
\input{chapter/07.tex}
\chapter[Zeitabhängige Quellenverteilungen]{Felder zeitabhängiger Ladungs- und Stromverteilungen}

Nun suchen nach allgemeinen Lösungen der \textsc{Maxwell}-Gleichungen:

\begin{align*}
\div \vec{B}  \ &= \ 0 & \epsilon_0 \ \div\vec{E}  \ &= \ \rho\\
\rot\vec{E} \ + \ \dot{\vec{B}} \ &= \ 0 & \frac{1}{\mu_0}\rot\vec{B}\ - \ \epsilon_0\dot{\vec{E}}  \ &= \ \vec{j}
\end{align*}

\section{Viererpotential}

\ \\
Die Gleichung $\div \vec{B} = 0 $ wird erfüllt durch $\vec{B}=\rot\vec{A}$.\\
Die Gleichung $\rot\vec{E} \ + \ \dot{\vec{B}} = 0 \; \Rightarrow \; \rot (\vec{E} + \dot{\vec{A}}) =0$ wird erfüllt durch $\vec{E} + \dot{\vec{A}}= - \grad\varphi$\\
\ \\
\ \\
Somit können alle Felder durch das \textbf{Viererpotential} $(\varphi,\vec{A})$ ausgedrückt werden, sodass im Endeffekt immer 4 skalare Felder bestimmt werden müssen:

\begin{empheq}[box=\highlightbox]{align*}
\vec{B} \ &= \ \rot \vec{A}\vphantom{\big|}\\
\vec{E} \ &= \ - \grad \varphi - \dot{\vec{A}}
\end{empheq}

\newpage
Das Einsetzen in die \textsc{Maxwell}-Gleichungen und Ausnutzung des \textsc{d'Alembert}-Operators  $\Dalembert  =  \frac{1}{c^2}\pddiff{}{t} - \laplace$ liefert:

\begin{align*}
\div\vec{E}  \ &= \  \frac{\rho}{\epsilon_0} \qquad  & \rot \vec{B} \ - \ \epsilon_0\mu_0\vec{E}  \ = \ \mu_0\vec{j}\\
-\laplace\varphi \ - \ \div\dot{\vec{A}}  \ &= \ \frac{\rho}{\epsilon_0}	\qquad	& \rot\rot\vec{A} \ + \ \frac{1}{c^2}\grad\dot{\varphi} \ + \ \frac{1}{c^2} \ddot{\vec{A}}  \ = \ \mu_0\vec{j}\\
&& \nabla(\nabla\vec{A}) \ - \ \laplace\vec{A} \ + \ \frac{1}{c^2}\ddot{\vec{A}} \ + \ \frac{1}{c^2}\partial_t \ \nabla\varphi  \ = \ \mu_0\vec{j}\\
\ \\
\Dalembert\varphi  \ - \ \partial_t\left(\frac{1}{c^2}\partial_t\ \varphi \ + \ \nabla\vec{A}\right)  \ &= \ \frac{\rho}{\epsilon_0}  \qquad &
\Dalembert\vec{A} \ + \ \nabla\left(\frac{1}{c^2} \partial_t \ \varphi \ + \ \nabla\vec{A}\right) \ = \ \mu_0\vec{j}
\end{align*}

\ \\

Die Potentiale sind damit aber nicht eindeutig, sondern nur bis auf eine beliebige Eichung der Form $\vec{A}\rightarrow\vec{A}+\grad\chi$ und $\varphi \rightarrow\varphi-\partial_t\chi$ genau bestimmt. Eine \textbf{gleichwertige Umeichung} von $\vec{A}$ und $\varphi$ lässt die Felder unter solche einer Transformation invariant. Die Eichtransformation enthält genau eine skalare Funktion $\chi$, anders gesprochen eine skalare Bedingung. Für uns günstig ist die sogenannte \textbf{\textsc{Lorentz}-Eichung}, da sie die Felder invariant unter \textsc{Lorentz}-Transformation lässt und sie somit geeignet bleiben für relativistische Probleme. Die \textsc{Lorentz}-Transformation hat folgende Gestalt:

\begin{equation*}
\frac{1}{c^2} \ \pdiff{\varphi}{t} \ + \ \div \vec{A}  \ = \ 0
\end{equation*} 

Mit der \textsc{Lorentz-Eichung} erhält man als Gleichungen für die Potentiale:

\begin{equation*}
\Dalembert\varphi  \ = \ \frac{\rho}{\epsilon_0}	\qquad  \qquad		\Dalembert\vec{A}  \ = \ \mu_0\vec{j} 
\end{equation*}

Hieran lässt sich auch einfach überprüfen, dass man die Gleichungen für die statischen Probleme leicht aus denen mit Zeitabhängigkeit erhalten kann mittels $\partial_t \rightarrow 0; \ \Dalembert \rightarrow \laplace$:

\begin{equation*}
\laplace\varphi  \ = \ -\frac{\rho}{\epsilon_0} \quad\text{(s. Kap.3)} \qquad	\laplace\vec{A} \ = \ -\mu_0\vec{j} \quad\text{(s. Kap. 4)} 	
\end{equation*}

\newpage
\underline{\textbf{Wichtige Eichungen:}}

\begin{enumerate}

\item \textbf{\textsc{Lorentz}-Eichung}

\begin{equation*}
\frac{1}{c^2} \partial_t \ \varphi \ + \ \div\vec{A} \ = \ 0
\end{equation*}

Die \textsc{Lorentz}-Eichung fixiert die Potentiale nicht; eine Umeichung der Form $\Dalembert\chi=0$ ist immer noch möglich.

\item \textbf{\textsc{Coulomb}-Eichung}

\begin{align*}
\div\vec{A} \ = \ 0 \qquad\qquad \Rightarrow \qquad\qquad -\laplace\varphi \ &= \ \frac{\rho}{\epsilon_0}\\
\Dalembert\vec{A} \ &= \ \mu_0\vec{j} \  - \ \frac{1}{c^2}\ \frac{\partial^2 \ \varphi}{\partial t \partial \vec{r}} 
\end{align*}

Die \textsc{Coulomb}-Eichung ist hier dieselbe wie in der Elektrostatik plus entsprechende Korrekturen.

\item \textbf{Transversale Wellen}

\begin{align*}
\varphi \ = \ 0 \qquad\qquad\Rightarrow\qquad\qquad \frac{1}{c^2}\ddot{\vec{A}}\ + \ \rot\rot\vec{A} \ &= \ \mu_0\vec{j}\\
- \frac{\partial^2 \ \vec{A}}{\partial t \partial \vec{r}}  \ &= \ \frac{\rho}{\epsilon_0} 
\end{align*}
\end{enumerate}

\section{Retardierte Potentiale}

\ \\
Wir haben nun eine inhomogene, lineare Differentialgleichung der Form $\Dalembert u \ = \ \xi$ vorliegen, zu deren Lösung wir die \textsc{Green}'sche Funktion $G(\vec{r},\vec{r}',t,t')$ heranziehen, welche die DGL $\;\Dalembert G = 4\pi \delta(\vec{r}-\vec{r}')\delta(t-t')$ löst.\\
Da $G$ translationsinvariant sein soll, kann es nur von $\vec{r}-\vec{r}'$ und $t-t'$ abhängen. Weiterhin erhalten wir aus der Rotationssymmetrie des Problems, dass $G$ nur von $|\vec{r}-\vec{r}'|=:|\vec{R}|$ abhängen kann.
\newpage
Zur weiteren Lösung der DGL $\; \Dalembert G = 4\pi\delta(\vec{r}-\vec{r}')\delta(t-t')$ bilden wir nun ihre \textsc{Fourier}-Transformierte (s. Kap. 1):

\begin{align*}
\left(\pddiff{}{\vec{R}} \ - \ \frac{1}{c^2}(i\omega)^2\right)G\left(\vec{R},\omega\right)  \ &= \ 4\pi\delta(\vec{R})\;\Bigg | \; \frac{\omega}{c} \ = \ k; \text{ benutze Kugelkoord.}\\
\ddiff{}{R}G_k(R) \ + \ k^2 G_k(R) \ &= \ 4\pi\delta(R) \ \Bigg |\cdot R \neq 0\\
\ddiff{}{R}(R \ G_k) \ + \ k^2 \cdot (R \ G_k) \ &= \ 0 \quad\qquad\text{homogene DGL}\\
\ \\
\text{Lsg.: } R \ G_k(R) \ &= \ A\cdot e^{ikR} \ + \ A\cdot e^{-ikR}
\end{align*}

Die Inhomogenität $\delta(\vec{R})$ ist daher sehr wichtig nahe $\vec{R}=0$. Dort ist $k \cdot R \ll 1$, wodurch $k^2 \cdot R \ G_k$ vernachlässigbar wird gegenüber $\ddiff{}{R}(R\cdot G_k)$. Dann reduziert sich die DGL auf:

\begin{equation*}
\laplace_R G_k(R)  \ = \ - 4 \pi \delta(\vec{R})
\end{equation*}

Im Grenzwert $\lim\limits_{kR \rightarrow 0}{G_k(R)}  \ = \ \frac{1}{R}$ ist die allgemeine Lösung für $G$ also:

\begin{equation*}
G_k  \ = \ A \cdot G_k^+(R) \ + \ B\cdot G_k^- (R), \quad G_k^{\pm}  \ = \ \frac{e^{\pm i k R}}{R},\quad A+B=1
\end{equation*}

Nun können wir $G_k^{\pm}(R)$ rücktransformieren zu $G^{\pm}(\vec{R},\tau)$:

\begin{equation*}
G^{\pm}(\vec{R},\tau)  \ = \  \frac{1}{2\pi} \Int{-\infty}{\infty}{\omega} \frac{e^{-\pm i\omega\tau}}{R}\cdot e^{-i\omega\tau}  \ = \ \frac{1}{R}\delta\left(\tau\mp \frac{R}{c}\right) \qquad \left(\text{mit }k=\frac{\omega}{c}\right)
\end{equation*}

Bezogen auf unser Anfangsproblem entspräche diese Lösung:

\begin{equation*}
G^{\pm}(\vec{r},t,\vec{r}',t')  \ = \ \frac{\delta\left(t' \ - \ \left(t \mp \frac{|\vec{r}-\vec{r}'|}{c}\right)\right)}{|\vec{r}-\vec{r}'|}
\end{equation*}

Der Unterschied zwischen $G^+$ und $G^-$ liegt in den Randbedingungen in der Zeit. Anschaulich beschreibt $G$ die Reaktion des Systems bei $(\vec{r},t)$ aufgrund einer Störung (Inhomogenität) bei $(\vec{r}',t')$. Um die Kausalität nicht zu verletzen, muss demzufolge $G(t<t')=0$ gelten. Dies ist erfüllt für die \textbf{retardierte \textsc{Green}'sche Funktion} $G^+$, da hier die Wirkung \underline{nach} der Ursache auftritt und sich mit Lichtgeschwindigkeit ausbreitet (Verzögerung $\tau = \frac{R}{c}$). $G^-$ nennt man auch die \textbf{avancierte \textsc{Green}'sche Funktion}, aber aus naheliegenden Gründen wird sie hier nicht weiter behandelt.\\
Die (kausale) Lösung unserer inhomogenen DGL vom Anfang $\Dalembert u = \xi$ lautet damit:

\begin{equation*}
u(\vec{r},t) \ = \ \underbrace{u_0(\vec{r},t)}_{\text{homogene Lsg.}} \ + \ \frac{1}{4\pi}\int\d V' \d t' G^+(\vec{r},t,\vec{r}',t')\xi (\vec{r}',t')
\end{equation*}

Diese Lösung können wir nun auf unsere DGLn zur Bestimmung des Viererpotentials $\Dalembert\varphi = \frac{\rho}{\epsilon_0}$ und $\Dalembert\vec{A}= \mu_0\vec{j}$ anwenden:\\
Für eine räumlich begrenzte Quellenverteilung und der Randbedingung, dass die Felder im Unendlichen gegen Null gehen, erhalten wir, wenn wir als homogene Lösungen $\varphi_0=0$ und $\vec{A}_0=0$ setzen, folgende allgemeine Lösung der \textsc{Maxwell}-Gleichungen:

\begin{empheq}[box=\highlightbox]{align*}
\varphi(\vec{r},t)  \ &= \ \frac{1}{4\pi\epsilon_0} \ \Int{}{\vphantom{\big|}}{V'} \ \frac{\rho\left(\vec{r'}, t \ - \ \frac{|\vec{r}-\vec{r}'|}{c}\right)}{|\vec{r}-\vec{r}'|}\\
\ \\
\vec{A}(\vec{r},t)  \ &= \ \ \frac{\mu_0}{4\pi} \ \ \Int{\vphantom{A}}{}{V'} \ \frac{\vec{j}\left(\vec{r}',t \ - \ \frac{|\vec{r}-\vec{r}'|}{c}\right)}{|\vec{r}-\vec{r}'|} 
\end{empheq}

Die obigen Gleichungen beschreiben \textbf{retardierte Potentiale}, welche folgendermaßen interpretiert werden können:\\
$\rho$ und $\vec{j}$ sind die Ursachen für die Wirkungen $\varphi$ und $\vec{A}$, welche allerdings eine Laufzeitverzögerung von $\frac{|\vec{r}-\vec{r}'|}{c}$ aufweisen.\\
\ \\

Die Überprüfung der gefundenen Lösung erfolgt leicht durch Einsetzen in $ \ \Dalembert\varphi=\frac{\rho}{\epsilon_0}$ und $\ \Dalembert\vec{A} = \mu_0\vec{j}$. Setzt man sie außerdem in die \textsc{Lorentz}-Eichung $\frac{1}{c^2}\partial_t \ \varphi + \div\vec{A} =0$ ein, so führt dieses auf die Kontinuitätsgleichung $\dot{\rho} + \div\vec{j}=0$.\\
\ \\
\underline{Bemerkung:}\\
Auch die avancierte \textsc{Green}-Funktion $G^-$ erfüllt die inhomogenen Wellengleichungen $\ \Dalembert\varphi=\frac{\rho}{\epsilon_0}$ und $\ \Dalembert\vec{A} = \mu_0\vec{j}$. Dies liegt mathematisch daran, dass die Wellengleichungen $c$ quadratisch enthalten, das Potential aber nur linear. Da diese Lösung aber akausal ist und nur $G^+$ die Kausalität erhält, zeichnet ebenjene Wahl von $G^+$ die Richtung der Zeit aus.


\section{\textsc{Hertz}'scher Dipol}

Wir betrachten nun als konkretes Beispiel für eine zeitabhängige Quellenverteilung einen oszillierenden Dipol: zwei Ladungen $\pm Q$ befinden sich entlang einer Achse in variablen Abstand $\vec{a}(t)$ voneinander entfernt. Somit gilt für die Stromdichte $\vec{j} := \vec{J}\cdot\delta(\vec{r})$, dass $\vec{j}=\dot{\vec{a}}\cdot Q \cdot\delta(\vec{r}-\vec{r}_a)$ ist. Im Dipollimit $\vec{a}\rightarrow 0$ folgt somit $\vec{j}=\dot{\vec{p}}\cdot\delta(\vec{r})$.\\
Allgemein gilt somit: $\vec{J}(t) = \Int{}{}{V} \vec{j}(\vec{r},t) = \dot{\vec{p}}$, oder genauer:

\begin{align*}
\dot{\vec{p}} \ = \ &\Int{}{}{V}\vec{r}\dot{\rho} \ = \ -\Int{}{}{V}\vec{r}\div\vec{j} \ = \ \Int{}{}{V}\left(\vec{j}\cdot\nabla\right)\cdot\vec{r} \ + \ \underbrace{\text{Oberflächenintegral}}_{\rightarrow 0} \\
\overset{\nabla\circ\vec{r}=\mathbbm{1}}{=} \ &\Int{}{}{V}\vec{j}  \ = \ \vec{J}
\end{align*}

Nun wollen wir die  (abgestrahlten) Felder des oszillierenden Dipols berechnen, wozu wir zunächst die retardierten Potentiale aufstellen:

\begin{equation*}
\vec{A}(\vec{r},t) \ = \ \frac{\mu_0}{4\pi}\ \Int{}{}{V'} \ \frac{\delta(\vec{r})\vec{J}\left(t \ - \ \frac{|\vec{r}-\vec{r}'|}{c}\right)}{|\vec{r}-\vec{r}'|} \ = \ \frac{\mu_0}{4\pi}\ \frac{\dot{\vec{p}}\left(t\ - \ \frac{r}{c}\right)}{r}
\end{equation*}

$\varphi$ erhalten wir aus der Ladungsverteilung zu $\vec{j}$ und aus der \textsc{Lorentz}-Eichung:

\begin{equation*}
\varphi(\vec{r},t)  \ = \ -\frac{1}{4\pi\epsilon_0} \ \frac{\vec{r}}{r} \ \pdiff{}{r} \ \frac{\vec{p}\left(t\ - \ \frac{r}{c}\right)}{r} \ + \ \text{zeitunabhängiges Potential}
\end{equation*}

Jetzt können die Felder $\vec{B}=\rot\vec{A}$ und $\vec{E}=-\grad\varphi-\dot{\vec{A}}$ berechnet werden.\\
$\left(\text{Notationshinweis: }\vec{p}|_{\text{ret}} \text{ steht für }\vec{p}\left(t-\frac{r}{c}\right)\right):$\\

\begin{align*}
\vec{B} \ &= \ \frac{\vec{r}}{r}\times\pdiff{\vec{A}}{r} \ = \ - \frac{\mu_0}{4\pi} \ \frac{\vec{r}}{r}\times \left(\frac{\ddot{\vec{p}}}{c}\ + \ \frac{\dot{\vec{p}}}{r}\right)_{\text{ret}}\\
\ \\
\vec{E} \ &= \ - \grad\varphi - \dot{\vec{A}} \ = \ -\grad\left(\frac{1}{4\pi\epsilon_0} \ \frac{\vec{r}}{r} \left[\frac{\vec{p}}{r^2} \ + \ \frac{\dot{\vec{p}}}{cr}\right]_{\text{ret}}\right) \ - \ \dot{\vec{A}}\\
\ \\
&= \ -\frac{1}{4\pi\epsilon_0 \cdot r} \left(\frac{\ddot{\vec{p}}}{c^2} \ - \ \frac{\left(\ddot{\vec{p}}\cdot\vec{r}\right)\vec{r}}{c^2 \ r^2} \ + \ \frac{\dot{\vec{p}}}{c \ r} \ - \ 3 \frac{\left(\dot{\vec{p}}\cdot\vec{r}\right)\vec{r}}{c \ r^3} \ + \ \frac{\vec{p}}{r^2} \ - \ 3 \frac{\left(\vec{p}\cdot\vec{r}\right)\vec{r}}{r^4}\right)_{\text{ret}}
\end{align*}

\newpage
\underline{Spezialfälle:}
\begin{enumerate}
\item \textbf{statischer Dipol} \qquad $\partial_t  \ = \ 0$

\begin{align*}
\Rightarrow \qquad \vec{B}  \ &= \ 0\\
\vec{E} \ &= \ - \frac{1}{4\pi\epsilon_0 \ r^3}\left(\vec{p} \ - \ \frac{3(\vec{p}\cdot\vec{r})\cdot\vec{r}}{r^2}\right)
\end{align*}

\item \textbf{harmonische Schwingung} \qquad $\vec{p} \ \sim \ e^{-i\omega t}$

\begin{align*}
\Rightarrow \qquad \partial_t \ &\rightarrow \ -i\omega\\
\frac{1}{c}\partial_t \ &\rightarrow \ - \frac{i\omega}{c}  \ = \ -i \frac{2\pi}{\lambda}  \ = \ -ik
\end{align*}

Für den Fall der harmonischen Schwingung von $\vec{p}$ wollen wir nun die Abstandsabhängigkeit der Feldbeiträge betrachten. Dazu sortieren wir die Beiträge nach ihren Ordnungen $\sigma(.)$:

\begin{align*}
\vec{B} \ &\sim \ - \frac{\mu_0 \vec{r}}{4\pi \ r^2} \times \Bigg(\underbrace{\sigma\left(\frac{\dot{\vec{p}}}{\lambda}\right)}_{\text{Fernfeld}} \; + \; \underbrace{\sigma\left(\frac{\dot{\vec{p}}}{r}\right)}_{\text{Nahfeld}}\Bigg)\\
\vec{E} \ &\sim \ - \frac{1}{4\pi\epsilon_0} \cdot \left(\sigma\left(\frac{\vec{p}}{\lambda^2}\right) \; + \; \sigma\left(\frac{\vec{p}}{r \cdot \lambda}\right) \; + \; \sigma\left(\frac{\vec{p}}{r^2}\right)\right)
\end{align*}

\ \\
\textbf{Nahfeld:} $\quad r \ll \lambda$

\begin{empheq}[box=\highlightbox]{align*}
\vec{B}(\vec{r},t)  \ &= \ \frac{\mu_0\vphantom{\big|}}{4\pi\ r^2} \ \left(\dot{\vec{p}} \times \frac{\vec{r}}{r}\right)_{\text{ret}}\\
\vec{E}(\vec{r},t)  \ &= \ \frac{1}{4\pi\epsilon_0 \ r^2\vphantom{\big|}} \ \left(\frac{3(\vec{p}\cdot\vec{r})\vec{r}}{r^2} \ - \ \vec{p}\right)_{\text{ret}} 
\end{empheq}

Für das Nahfeld verzichtet man häufig auf die Retardierung, da sie kaum ins Gewicht fällt:

\begin{equation*}
\vec{p} \ \sim \ e^{-i\omega\left(t-\frac{r}{c}\right)}  \ = \ e^{-i\omega t} \cdot \underbrace{\; e^{2\pi i \frac{r}{\lambda}}\;}_{1+2\pi i \frac{r}{\lambda}+\ldots \approx 1}
\end{equation*}

\ \\
\textbf{Fernfeld:} $\quad r \gg \lambda \qquad\left(\frac{\vec{r}}{r} \ = \ \vec{e}_r\right)$

\begin{empheq}[box=\highlightbox]{align*}	
\vec{B}(\vec{r},t)  \ &= \ \frac{\mu_0\vphantom{\big|}}{4\pi \ r \ c} \left(\ddot{\vec{p}}\times\vec{e}_r\right)_{\text{ret}}\\
\vec{E}(\vec{r},t)  \ &= \ \frac{\mu_0}{4\pi \ r\vphantom{\big|}} \left(\left(\ddot{\vec{p}}\cdot\vec{e}_r\right)\cdot\vec{e}_r \ - \ \ddot{\vec{p}}\right)_{\text{ret}}  \ = \ \frac{\mu_0}{4\pi \ r}\left(\ddot{\vec{p}}\times\vec{e}_r\right)_{\text{ret}}\times\vec{e}_r  \ = \ c\cdot\vec{B}\times\vec{e}_r
\end{empheq}

Bei harmonisch schwingendem $\vec{p}$ breiten sich $\vec{E}$ und $\vec{B}$ also radial als transversale Welle aus:

\begin{equation*}
e^{-i\omega\left(t-\frac{r}{c}\right)}  \ = \ e^{i(kr-\omega t)} \qquad \text{mit } k  \ = \  \frac{\omega}{c}
\end{equation*}

Dabei fallen $|\vec{E}|$ und $|\vec{B}|$ nur mit $\frac{1}{r}$ ab!
\end{enumerate}


\section{Energieabstrahlung des \textsc{Hertz}'schen Dipols}

\textbf{Energiedichte:}

\begin{equation*}
w  \ = \ \frac{\epsilon_0}{2} \vec{E}^2  \ +  \ \frac{1}{2\mu_0} \vec{B}^2  \ = \ \frac{\vec{B^2}}{\mu_0}
\end{equation*}

\ \\
\textbf{Energiestromdichte:}

\begin{align*}
\vec{S}_P  \ &= \ \frac{1}{\mu_0} (\vec{E}\times\vec{B})  \ = \ \frac{c}{\mu_0} (\vec{B}\times\vec{e}_r\times\vec{B})  \ = \  \frac{c}{\mu_0}\Big(\vec{e}_r\vec{B}^2 \ - \ \vec{B}(\underbrace{\vec{n}\cdot\vec{B}}_{=0})\Big)  \ = \ c\cdot\vec{e}_r\cdot w\\
\alignedbox{\hspace{.5em}|\vec{S}_P|  \ }{= \ \frac{\mu_0}{(4\pi)^2c}\cdot \frac{\left(\ddot{\vec{p}}\times\vec{e}_r\right)^2}{r^2} \ = \ \frac{\mu_0}{(4\pi)^2c}\cdot\frac{\ddot{\vec{p}}^2\sin^2\theta}{r^2} \hspace{.5em}} \qquad \left(\text{mit } \theta = \sphericalangle(\ddot{\vec{p}},\vec{e}_r)\right)
\end{align*}

Man sieht hieran leicht, dass senkrecht zur Dipolachse $\vec{a}$ am stärksten ausgestrahlt wird (da auch $\ddot{\vec{p}}\perp \vec{a}$).

\ \\
\textbf{Frequenzabhängigkeit der abgestrahlten Leistung:}

\begin{equation*}
|\vec{S}_P|  \ \sim \ |\ddot{\vec{p}}|^2 \ \sim p^2 \ \omega^4
\end{equation*}

Diese Frequenzabhängigkeit ist charakteristisch für Dipolstrahlung.

\newpage
\textbf{Abgestrahlte Leistung:}

\begin{align*}
N  \ &= \ \Iint{}{}{\vec{A}_F} \cdot \vec{S}_P  \ = \ \Int{}{}{\Omega}r^2 \ \frac{\mu_0}{(4\pi)^2c}\cdot\frac{\ddot{\vec{p}}^2\sin^2\theta}{r^2}  \ = \ \frac{\mu_0\ddot{\vec{p}}^2}{(4\pi)^2c} \ \Int{}{}{\Omega}\sin^2\theta\\
&= \ \frac{\mu_0\ddot{\vec{p}}^2}{(4\pi)^2c} \Int{0}{2\pi}{\phi}\Int{-1}{1}{(\cos\theta)} \ (1 \ - \ \cos^2\theta) \ = \ \frac{\mu_0\ddot{\vec{p}}^2}{(4\pi)^2c} \cdot 2\pi \cdot \left(2-\frac{2}{3}\right) \ = \ \frac{2}{3} \ \frac{\mu_0}{4\pi c} \ddot{\vec{p}}^2_{\text{ret}}
\end{align*}

Wenn wir einen harmonisch oszillierenden Dipol betrachten, so gilt für $\ddot{\vec{p}}$:

\begin{align*}
\vec{p}  \ &= \ \vec{p}_0 \cos\omega t \qquad \Rightarrow \qquad \ddot{\vec{p}}^2  \ = \ \omega^4\vec{p}_0^2\cos^2\omega t\\
\Rightarrow \langle\ddot{\vec{p}}^2\rangle_T  \ &= \  \omega^4\vec{p}_0^2 \langle\cos^2\omega t\rangle_T  \ = \ \frac{1}{2} \omega^4 \vec{p}_0^2
\end{align*}

Damit gilt also für die (über eine Periode gemittelte) abgestrahlte Leistung:

\begin{empheq}[box=\highlightbox]{equation*}
\langle N \rangle_T  \ = \  \frac{\mu_0 \ \omega^4\ \vec{p}_0^2\vphantom{\big|}}{12\pi \ c\vphantom{\big|}}
\end{empheq}

\section[Strahlung räumlich begrenzter Quellen]{Strahlungsfeld einer räumlich begrenzten Quellenverteilung}

Wir betrachten nun eine beliebige Quellenverteilung am Ort $\vec{r}'$ mit der maximalen räumlichen Ausdehnung $a$. Es gilt ganz allgemein für das Vektorpotential am Ort $\vec{r}$:

\begin{equation*}
\vec{A}(\vec{r},t) \ = \ \frac{\mu_0}{4\pi}\Int{}{}{V'} \ \frac{\vec{j}\left(\vec{r}',t\ - \ \frac{|\vec{r}-\vec{r}'|}{c}\right)}{|\vec{r}-\vec{r}'|}
\end{equation*}

Wir entwickeln nun den Ausdruck $\frac{1}{|\vec{r}-\vec{r}'|}$ für das Fernfeld $(r\gg a, |\vec{r}| \gg |\vec{r}'|)$:

\begin{align*}
\frac{1}{|\vec{r}-\vec{r}|}  \ &= \ \frac{1}{r} \ + \ \frac{\vec{r}\cdot\vec{r}'}{r^3} \ + \ \ldots \ \approx \ \frac{1}{r} \quad \text{(Dipolnäherung)}\\
|\vec{r}-\vec{r}'|  \ &= \ r \ - \ \frac{\vec{r}\cdot\vec{r}'}{r} \ + \ \ldots \ \approx \ r \ - \ \vec{e}_r \cdot \vec{r}'\\
\Rightarrow \quad \vec{A}(\vec{r},t)  \ &= \ \frac{\mu_0}{4\pi \ r} \underbrace{\Int{}{}{V'} \vec{j} \left(\vec{r}',t \ - \ \frac{r}{c} \ + \ \frac{\vec{e}_r\cdot\vec{r}'}{c}\right)}_{\equiv \dot{\vec{q}}\left(t \ - \ \frac{r}{c},\vec{e}_r\right)}
\end{align*}

Zum Vergleich: Das Vektorpotential für einen \textsc{Hertz}'schen Dipol ergab:

\begin{equation*}
\vec{A}  \ = \ \frac{\mu_0}{4\pi \ r} \ \dot{\vec{p}}\left(t \ - \ \frac{r}{c}\right)
\end{equation*}

Das $\vec{B}$-Feld erhalten wir aus dem eben gewonnenen Ausdruck für $\vec{A}$ durch $\vec{B}=\rot\vec{A}$:

\begin{align*}
\pdiff{}{\vec{r}}\left[\frac{1}{r} f \left(t-\frac{r}{c},\vec{e}_r\right)\right] \ &= \ \Bigg[\underbrace{-\frac{\vec{e}_r}{c}\partial_t}_{\sigma\left(\frac{1}{\lambda}\right)} \ - \ \underbrace{\frac{\vec{e}_r}{r}}_{\sigma\left(\frac{1}{r}\right)} \ + \ \underbrace{\pdiff{}{\vec{r}}\left(\vec{e}_r\cdot\pdiff{}{\vec{e}_r}\right)}_{\sigma\left(\frac{1}{r}\right)}\Bigg ] \left(\frac{1}{r}f\right) \ \approx \ - \frac{\vec{e}_r}{c} \partial_t \left(\frac{1}{r}f\right)\\
\ \\
\overset{r\gg\lambda}{\Rightarrow} \qquad\quad \alignedbox{\hspace{.5em}\vec{B}  \ }{= \ -\frac{\vec{e}_r}{c}\times\dot{\vec{A}}  \ = \  \frac{\mu_0}{4\pi \ c} \cdot \frac{\ddot{\vec{q}}\times\vec{e}_r}{r}\hspace{.5em}}
\end{align*}

Das $\vec{E}$-Feld erhalten wir aus der inhomogenen \textsc{Maxwell}-Gleichung: $\frac{1}{\mu_0}\rot\vec{B}= \vec{j} + \epsilon_0\dot{\vec{E}}$. Unter Beachtung, dass $\vec{j}=0$ außerhalb der Quellenverteilung ist, erhalten wir zunächst für $\dot{\vec{E}}$:

\begin{align*}
\dot{\vec{E}}  \ &= \ c^2 \ \rot\vec{B} \ \approx \ - \frac{\vec{e}_r}{c}\partial_t\times c^2 \vec{B}\\
\Rightarrow\qquad\quad \alignedbox{\hspace{.5em}\vec{E} \ }{= \ c\vec{B}\times\vec{e}_r  \ = \ \frac{\mu_0}{4\pi} \cdot \frac{\left(\ddot{\vec{q}}\times\vec{e}_r\right)\times\vec{e}_r}{r} \hspace{.5em}}
\end{align*}

Wir erhalten also wieder eine transversale Welle der Felder $\vec{E}$ und $\vec{B}$, welche beide mit $\frac{1}{r}$ abfallen. (Korrekturen aus höheren Termen fallen dabei schneller ab.)\\
Für die Energiestromdichte gilt damit:

\begin{equation*}
\vec{S}_P \ = \ \frac{1}{\mu_0} (\vec{E}\times\vec{B})  \ = \  \vec{e}_r \ \frac{\mu_0}{(4\pi)^2 c} \ \frac{\left(\ddot{\vec{q}}\times\vec{e}_r\right)^2}{r^2}
\end{equation*}

Der direkte Vergleich zwischen dem Fernfeld des \textsc{Hertz}'schen Dipols und dem Strahlungsfeld einer beliebigen Ladungsverteilung zeigt uns, dass mit $\dot{\vec{p}}\leftrightarrow\dot{\vec{q}}$ alle Fernfeldformeln identisch sind:

\begin{align*}
&\text{\textsc{Hertz}'scher Dipol:} \qquad\qquad\qquad\ \; \text{allgemein:}&\\
&\dot{\vec{p}}_{\text{ret}}  \ = \ \Int{}{}{V'} \vec{j}\left(\vec{r}',t-\frac{r}{c}\right) \qquad \qquad \dot{\vec{q}}  \ = \ \Int{}{}{V'}\vec{j}\left(\vec{r}',t-\frac{r}{c}+\frac{\vec{e}_r\cdot\vec{r}'}{c}\right)
\end{align*}

Den Ausdruck $\frac{\vec{e}_r\cdot\vec{r}'}{c}$ in $\dot{\vec{q}}$ kann man als die Laufzeit innerhalb der Quellen verstehen.


\section{Multipolentwicklung des Fernfelds}

Im vorigen Kapitel haben wir das Strahlungsfeld einer räumlich begrenzten Quellenverteilung mit beliebiger Zeitabhängigkeit betrachtet. Jetzt setzen wir uns mit dem Spezialfall auseinander, dass $\vec{j}\sim e^{-i\omega t}$ gilt:

\begin{equation*}
\vec{j}\left(t-\frac{r}{c} + \frac{\vec{e}_r \cdot\vec{r}'}{c}\right) \ \sim \ e^{-i\omega\left(t-\frac{r}{c}\right)} \ e^{-i\frac{\omega}{c} \vec{e}_r\cdot\vec{r}'}
\end{equation*}\\
\ \\

Wir entwickeln diesen Ausdruck für das Fernfeld $(r\gg a, r \gg \lambda)$ mit der zusätzlichen Annahme, dass es sich um eine "kleine Quelle" handelt $(\lambda\gg a)$:

\begin{align*}
&\frac{\omega}{c} \vec{e}_r \cdot \vec{r} \sim \frac{\omega}{c}a' \ = \  \frac{2\pi a}{\lambda} \ll 1 \quad\text{(Laufzeit in Quelle $\ll$ Periodendauer)}\\
\Rightarrow\quad &\vec{j}\left(t-\frac{r}{c}+\frac{\vec{e}_r\cdot\vec{r}'}{c}\right) \ \approx \ \vec{j}\left(t-\frac{r}{c}\right) \ + \ \frac{\vec{e}_r\cdot\vec{r}'}{c}\partial_t\vec{j}\left(t-\frac{r}{c}\right) \ + \ \ldots\\
&\dot{\vec{q}}  \ = \ \underbrace{\Int{}{}{V'}\vec{j}\left(\vec{r}',t-\frac{r}{c}\right)}_{=\dot{\vec{p}}_{\text{ret}}} \ + \ \frac{1}{c} \ \partial_t \Int{}{}{V'}\left(\vec{e}_r\cdot\vec{r}'\right) \ \vec{j} \left(\vec{r}', t- \frac{r}{c}\right) \ + \ \ldots
\end{align*}

Umformen ergibt:

\begin{align*}
\Int{}{}{V'} \left(\vec{e}_r\cdot\vec{r}'\right) \ \vec{j}  \ &= \ \frac{1}{2}\Int{}{}{V'} \left[\left(\vec{e}_r\cdot\vec{r}'\right)\vec{j} \ - \ \left(\vec{e}_r \cdot \vec{j}\right)\vec{r}'\right] \ + \ \Int{}{}{V'} \left[\left(\vec{e}_r\cdot\vec{r}'\right)\vec{j} \ + \ \left(\vec{e}_r\cdot\vec{j}\right)\vec{r}'\right]\\
&= \ -\vec{e}_r \times\underbrace{\frac{1}{2}\Int{}{}{V'}\left(\vec{r}'\times\vec{j}\right)}_{=\vec{m}} \ + \ \frac{\vec{e}_r}{2} \ \frac{1}{3} \dot{\tens{D}}'
\end{align*}

Denn:

\begin{align*}
\dot{\tens{D}} \ &= \ \Int{}{}{V}\underbrace{\dot{\rho}}_{-\nabla\vec{j}} \left(3\vec{r}\circ\vec{r} \ - \ \mathbbm{1} \vec{r}^2\right)  \ \overset{\text{part. Int.}}{=} \ \Int{}{}{V} \left(\vec{j}\cdot\nabla\right) \ \left(3\vec{r}\circ\vec{r} \ - \ \mathbbm{1}\vec{r}^2\right)\\
&= \ \Int{}{}{V}\left(3\vec{j}\circ\vec{r} \ + \ 3\vec{r}\circ\vec{j} \ - \ 2 \cdot \mathbbm{1} \ \left(\vec{j}\cdot\vec{r}\right)\right) \qquad\qquad(*)
\end{align*}

\newpage
Wir definieren:
\begin{align*}
\tens{D}'  \ &= \ \tens{D} \ + \ \mathbbm{1}\frac{\operatorname{Spur} \tens{D}}{3}  \ = \  \Int{}{}{V}\rho \ 3 \vec{r}\circ\vec{r}\\
\dot{\tens{D}} \ &= \ \Int{}{}{V}\left(3\vec{j}\circ\vec{r} \ + \ 3 \vec{r}\circ\vec{j}\right)
\end{align*}

Somit gilt für die Komponenten von $(*)$:\\

\begin{align*}
\left[\left(\vec{j}\cdot\nabla\right) \ \left(3\vec{r}\circ\vec{r} \ - \ \mathbbm{1} \vec{r}^2\right)\right]_{ij}  \ &= \ j_k \partial_k\ \left(3 r_i r_j \ - \ \delta_{ij} r_l r_l\right)\\
&= \ j_k \left(3\delta_{ij}r_j \ + \ 3r_i\delta_{jk} \ - \ \delta_{ij} 2 r_l \delta_{lk}\right)\\
&= \ 3j_i r_j \ + \ 3 r_i j_j \ - \ \delta_{ij} 2 j_k r_k
\end{align*}

Nun brauchen wir noch:\\

\begin{equation*}
\big[\left(\vec{m}\times\vec{e}_r\right)\times\vec{e}_r\big]\times\vec{e}_r  \ = \ \big[\vec{e}_r(\vec{m}\cdot\vec{e}_r) \ - \ \vec{m}(\underbrace{\vec{e_r}\cdot\vec{e}_r}_{=1})\big] \times \vec{e}_r \ = \ - \vec{m}\times\vec{e}_r
\end{equation*}\\

Und:\\
\begin{equation*}
\vec{e_r}\cdot\tens{D}'\times\vec{e}_r  \ = \ \vec{e}_r \cdot\tens{D}'\times\vec{e}_r \ + \ \underbrace{\vec{e}_r \ \mathbbm{1} \times\vec{e}_r}_{=\vec{e}_r\times\vec{r}_r =0} \ \frac{\operatorname{Spur}\tens{D}'}{3}  \ = \ \vec{e}_r\cdot\tens{D}\times\vec{e}_r
\end{equation*}

\begin{equation*}
\dot{\vec{q}}  \ = \  \left(\dot{\vec{p}} \ + \ \dot{\vec{m}} \times\frac{\vec{e}_r}{c} \ + \frac{\vec{e}_r}{6c}\cdot \ddot{\tens{
D}}\right)_{\text{ret}}
\end{equation*}\\

Somit folgt für die Felder einer kleinen Quelle:

\begin{empheq}[box=\highlightbox]{align*}
\vec{B}  \ &= \ \frac{\mu_0\vphantom{\big|}}{4\pi \ c r} \left(\ddot{\vec{p}}\times\vec{e}_r \ + \ \frac{1}{c} \left(\ddot{\vec{m}}\times
\vec{e}_r\right) \times\vec{e}_r \ + \ \frac{1}{6c}\vec{e}_r\cdot\dddot{\tens{D}} \times\vec{e}_r\right)_{\text{ret}}\\
\vec{E}  \ &= \ \frac{\mu_0}{4\pi \ c} \Bigg(\underbrace{\left(\ddot{\vec{p}}\times\vec{e}_r\right)\times\vec{e}_r}_{\text{el. Dipol}} \ - \ \underbrace{\frac{1}{c}\left(\ddot{\vec{m}}\times\vec{e}_r\right)}_{\text{mag. Dipol}} \ + \ \underbrace{\frac{1}{6c}\left(\vec{e}_r\cdot\dddot{\tens{D}}\times\vec{e}_r\right)\times\vec{e}_r}_{\text{el. Quadrupol}}\Bigg)_{\text{ret}}
\end{empheq}

\newpage
Die Felder des magnetischen Dipols und des elektrischen Quadrupols kommen eine Ordnung später als der elektrische Dipol; sie sind also um $\frac{a}{\lambda}$ kleiner. Die Richtungsabhängigkeit der Felder (und damit auch die der Energiestromdichte) hat folgende Form:

\begin{figure}[ht]
\centering
\colorbox{hgrey}{
	\begin{tikzpicture}[scale=.8]
		\draw [ultra thick,->] (0,-1.5) -- (0,1.5);
		\draw (0,0) .. controls (2.5,1.75) and (2.5,-1.75) .. (0,0);
		\draw (0,0) .. controls (-2.5,1.75) and (-2.5,-1.75) .. (0,0);
		\draw (0,-2) node{\textbf{Dipol}};

		\draw [ultra thick,->] (7.5,-1.5) -- (7.5,1.5);
		\draw [ultra thick,->] (6,0) -- (9,0);
		\draw [thick] (7.5,0) circle (0.2cm);
		\fill (7.5,0) circle (0.1cm);
		\begin{scope}[shorten <= .2cm, shorten >= .2cm]
			\draw (7.5,0) .. controls +(25:2cm) and +(65:2cm) .. (7.5,0);
			\draw (7.5,0) .. controls +(115:2cm) and +(155:2cm) .. (7.5,0);
			\draw (7.5,0) .. controls +(205:2cm) and +(245:2cm) .. (7.5,0);
			\draw (7.5,0) .. controls +(295:2cm) and +(335:2cm) .. (7.5,0);
		\end{scope}
		\draw (10,-2) node{\textbf{Quadrupol}};
		\draw (10,0) node{bzw.};

		\draw [ultra thick,->] (11,0,0) -- (14,0,0);
		\draw [ultra thick,->] (12.5,-1.5,0) -- (12.5,1.5,0);
		\draw [ultra thick,->] (12.5,0,-1.5) -- (12.5,0,1.5);
		\draw (12.5,0,0) .. controls (14,.8,0) and (14,-.8,0) .. (12.5,0,0);
		\draw (12.5,0,0) .. controls (11,.8,0) and (11,-.8,0) .. (12.5,0,0);
		\draw (12.5,0,0) .. controls (12.5,.8,1.5) and (12.5,-.8,1.5) .. (12.5,0,0);
		\draw (12.5,0,0) .. controls (12.5,.8,-1.5) and (12.5,-.8,-1.5) .. (12.5,0,0);
	\end{tikzpicture}}
\caption{Richtungsabhängigkeit der Felder}
\end{figure}

\section{Strahlung einer bewegten Punktladung}

Wir wollen nun die im letzten Kapitel hergeleiteten Formeln auf das Beispiel einer bewegten Punktladung anwenden. Es gilt für die retardierten Potentiale:

\begin{equation*}
\begin{pmatrix}
\varphi\\
\vec{A}
\end{pmatrix}
\left(\vec{r},t\right) \ = \ \begin{pmatrix}
\frac{1}{\epsilon_0}\\
\mu_0
\end{pmatrix}
\ \Int{}{}{V'}\frac{1}{4\pi|\vec{r}-\vec{r}'|} \ \begin{pmatrix}
\rho\\
\vec{j}
\end{pmatrix}
\left(\vec{r}', t \ - \ \frac{|\vec{r}-\vec{r}'|}{c}\right)
\end{equation*}

Für den Spezialfall einer Punktladung auf einer Bahn $\vec{R}(t)$ gilt:

\begin{align*}
\rho(\vec{r},t)  \ &= \ Q \ \delta\left(\vec{r}-\vec{R}(t)\right)\\
\vec{j}(\vec{r},t)  \ &= \ Q \ \vec{R}(t) \ \delta\left(\vec{r}-\vec{R}(t)\right) 
\end{align*}

Einsetzen ergibt:

\begin{align*}
\varphi(\vec{r},t)  \ &= \ \frac{Q}{4\pi\epsilon_0} \ \int\d V'\d t' \frac{\delta(\vec{r}-\vec{R}(t'))}{|\vec{r}-\vec{r}'|} \ \delta\left(t'-t + \frac{|\vec{r}-\vec{r}'|}{c}\right) \\ 
&= \ \frac{Q}{4\pi\epsilon_0}\Int{}{}{t'} \ \frac{1}{|\vec{r}-\vec{R}(t')|} \ \delta \Bigg(t'\underbrace{-t+\frac{|\vec{r}-\vec{R}(t')|}{c}}_{=: -t_0}\Bigg)
\end{align*}

\newpage
\ \\
\ \\
\begin{wrapfigure}[]{c}[0cm]{0cm}
	\raisebox{-.5cm}[\dimexpr\height-1\baselineskip\relax]{
		\colorbox{hgrey}{
			\begin{tikzpicture}[scale=.7]
				\draw [ultra thick,->] (0,0)--(6,0);
				\draw [ultra thick,->] (0,0)--(0,4);
				\fill (5,2) circle(0.1cm) node[right]{Beobachter};
				\draw (5,0.2)--(5,-0.2) node[below]{$t$};
				\draw (0.2,2)--(-0.2,2) node[left]{x};
				\draw (1,0.3)--(5,2);
				\draw (1,3.7)--(5,2);
				\fill [red] (2.4,0.9) circle(0.1cm);
				\draw (0.6,0.75) node[red]{$\vec{R}(t)$};
				\draw [very thick, red] (0.6,1.15) to[out=30, in=140] (2.4,0.9);
				\draw [very thick, ->, red] (2.4,0.9) to[out=320, in=200] (4,0.8) node[right]{Quelle};
				\draw (2.4,0.2)--(2.4,-0.2) node[below]{$t_0$};
				\draw [color=hgrey] (0,4.21) node{a};
			\end{tikzpicture}
		}
	}
	\caption{Bewegte Punktladung}
\end{wrapfigure}

Dies kann man sich mithilfe eines Lichtkegels wie in Abb. 8.2 veranschaulichen.\\
Um die Rechnung weiterzuführen benötigt man:

\begin{equation*}
\delta(f(x))  \ = \  \sum_i \ \frac{\delta(x-x_i)}{\hspace{.65cm}\left|\pdiff{f}{x}\right|_{x=x_i}}
\end{equation*}

wobei die $x_i$ die Nullstellen von $f$ sind.\\
Somit gilt für das Potential:

\begin{equation*}
\varphi(\vec{r},t)  \ = \ \frac{Q}{4\pi\epsilon_0}\cdot \frac{1}{|\vec{r}-\vec{R}(t_0)|} \cdot \left(1\ - \ \frac{\dot{\vec{R}}(t_0) \cdot (\vec{r}-\vec{R}(t_0))}{c \ |\vec{r}-\vec{R}(t_0)|}\right)^{-1}
\end{equation*}\\

Setzen wir nun für den Ausdruck $\vec{r}-\vec{R}$ die Abkürzung $\vec{r}''$ und für $\dot{\vec{R}}$ den Ausdruck $\vec{V}$ ein, erhalten wir die sogenannten \textbf{Liénard-Wiechert-Potentiale}:

\begin{empheq}[box=\highlightbox]{align*}
\varphi(\vec{r},t)  \ &= \ \frac{Q^{\vphantom{\big|}}}{4\pi\epsilon_0} \ \left[\frac{1}{|\vec{r}-\vec{R}| \ - \ \frac{\dot{\vec{R}}}{c} (\vec{r}-\vec{R})}\right]_{\text{ret}}  \ = \ \frac{Q}{4\pi\epsilon_0} \ \left(\frac{1}{r'' \ - \ \frac{ \vec{V}\cdot \vec{r}''}{c}}\right)_{\text{ret}}\\
\ \\
\vec{A}(\vec{r},t) \ &= \ \frac{\mu_0 \ Q}{4\pi} \ \left[\frac{\dot{\vec{R}}}{|\vec{r}-\vec{R}| \ - \ \frac{\dot{\vec{R}}}{c}(\vec{r}-\vec{R})}\right]_{\text{ret}}  \ = \  \frac{\mu_0 \ Q}{4\pi} \ \left(\frac{\vec{V}}{r'' \ - \ \frac{\vec{V}\cdot\vec{r}''}{c}}\right)_{\text{ret}} 
\end{empheq}	

\newpage
\section{Strahlungsbremsung}

Wir wollen in diesem Kapitel eine Punktladung auf einer Bahn 
$\vec{R}(t)$ im nichtrelativistischen Falle betrachten. Dafür berechnen wir zunächst alle benötigten Größen:

\begin{align*}
\rho  \ &= \ Q \delta(\vec{r}-\vec{R}(t))\\
\vec{j}  \ &= \ Q \  \dot{\vec{R}} \ \delta(\vec{r}-\vec{R}(t))\\
\vec{p} \ &= \ \Int{}{}{V}r\rho  \ = \  Q \ \vec{R}(t)\\
\vec{m}  \ &= \ \frac{1}{2} \Int{}{}{V} \vec{r}\times\vec{j}  \ = \ \frac{1}{2}\Int{}{}{V}\vec{r}\times\dot{\vec{R}}\ Q \ \delta(\vec{r}-\vec{R}(t)) \ = \ \frac{Q}{2} \vec{R}\times\dot{\vec{R}}\\
\dot{\tens{D}}'  \ &= \ 3 \Int{}{}{V} \left(\vec{r}\circ\vec{j} \ + \ \vec{j}\circ\vec{r}\right)  \ = \ 3Q \ \left(\vec{R}\circ\dot{\vec{R}} \ + \ \dot{\vec{R}}\circ\vec{R}\right)
\end{align*}

Wie man leicht erkennt hat eine \underline{bewegte} Punktladung auch Multipolmomente. Nun nutzen wir die Dipolnäherung aus Kapitel 8.4 für die abgestrahlte Leistung:

\begin{equation*}
N  \ = \  \frac{\mu_0}{6\pi \ c} \ddot{\vec{p}}^2  \ = \  \underbrace{\frac{\mu_0 \ Q^2}{6\pi \ c}}_{:=\alpha} \ \dot{\vec{v}}^2 \qquad \left(\text{Gültig für } \frac{v}{c}\ll 1\right)
\end{equation*}

Für $\dot{\vec{v}}\neq 0$ verliert das Teilchen also Bewegungsenergie in Form von abgestrahlter elektromagnetischer Leistung. Es wirkt also eine Bremskraft:

\begin{align*}
\Delta W_{\text{elm}}  \ = \ \Int{}{}{t} N  \ &= \ - \Delta W_{\text{mech}}  \ = \  - \Int{}{}{t} \vec{F}_S\cdot\vec{v}\\
\alpha \Int{}{}{t}\dot{\vec{v}}^2  \ &= \ - \Int{}{}{t}\vec{v}\cdot\vec{F}_s \ \overset{\text{part. Int.}}{=} \ - \alpha\Int{}{}{t}\ddot{\vec{v}}\vec{v} \ + \ \left.\alpha\vec{v}\dot{\vec{v}}\right|^{t_2}_{t_1} 
\end{align*}

Der Randterm verschwindet dabei für oszillierende Bewegungen, sodass man die Bremskraft $\vec{F}_S$ explizit ausrechnen kann mit:

\begin{empheq}[box=\highlightbox]{equation*}
\vec{F}_S \ = \ \alpha \ddot{\vec{v}}  \ = \ \frac{Q^2\vphantom{\big|}}{6\pi\epsilon_0 \ c^3\vphantom{\big|}} \dddot{\vec{R}}
\end{empheq}

Diese Formel ist allerdings nur anwendbar, wenn $\vec{F}_S$ eine kleine Korrektur ist.
\input{chapter/09.tex}
\input{chapter/10.tex}
\input{chapter/11.tex}
\input{chapter/12.tex}
\input{chapter/13.tex}

\end{document}