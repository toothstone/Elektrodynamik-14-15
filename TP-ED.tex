\documentclass[a4paper,12pt,portrait]{book}
\title{Theoretische Elektrodynamik}
\author{Matthias Vojta\\ \\ \"ubertragen von\\Sebastian Schmidt und Lukas K\"orber}
\date{Wintersemester 2014/2015}

  
\usepackage[ngerman]{babel}
\usepackage[utf8]{inputenc}
\usepackage{amsmath}
\usepackage{mathtools}
\usepackage{amssymb}
\usepackage{fancyhdr}
\usepackage{xcolor}
\usepackage{fourier}
\usepackage[frak=mma]{mathalfa}
\usepackage{sectsty}
\usepackage{lipsum}
\usepackage{bm}
\usepackage{bbm}
\usepackage{enumitem}
\usepackage{eucal}


\usepackage{wrapfig}
\usepackage{tikz}
\usetikzlibrary{patterns}
\tikzset{
	partial ellipse/.style args={#1:#2:#3}{
		insert path={+ (#1:#3) arc (#1:#2:#3)}
	}
}


%layout options
\definecolor{dblue}{HTML}{183CE0}
\definecolor{hblue}{HTML}{0E75F7}
\definecolor{hgrey}{HTML}{FAFAFA}
\sectionfont{\color{hblue}}
\chapterfont{\color{dblue}}
\pagestyle{headings}
\setlength\parindent{0pt}

%defined commands
%differential operators
\newcommand{\diff}[2]{\frac{\mathrm{d}{#1}}{\mathrm{d}{#2}}}
\newcommand{\ddiff}[2]{\frac{\mathrm{d}^2{#1}}{\mathrm{d}{#2}^2}}
\newcommand{\dddiff}[2]{\frac{\mathrm{d}^3{#1}}{\mathrm{d}{#2}^3}}
\newcommand{\ndiff}[2]{\frac{\mathrm{d}^n{#1}}{\mathrm{d}{#2}^n}}
\newcommand{\pdiff}[2]{\frac{\partial{#1}}{\partial{#2}}}
\newcommand{\pddiff}[2]{\frac{\partial ^2{#1}}{\partial{#2}^2}}
\renewcommand{\d}{\mathrm{d}}


%vectors and tensors
\newcommand{\tens}[1]{\bm{\hat{#1}}}
\renewcommand{\vec}[1]{\bm{#1}}


%integrals
\newcommand{\Int}[3]{\int\limits_{{#1}}^{{#2}}\mathrm{d}{#3}}
\newcommand{\Iint}[3]{\iint\limits_{{#1}}^{{#2}}\mathrm{d}{#3}}
\newcommand{\Iiint}[3]{\iiint\limits_{{#1}}^{{#2}}\mathrm{d}{#3}}
\newcommand{\Oint}[3]{\oint\limits_{{#1}}^{{#2}}\mathrm{d}{#3}}
\newcommand{\Oiint}[3]{\oiint\limits_{{#1}}^{{#2}}\mathrm{d}{#3}}


%Vector-analytic operators
\newcommand{\grad}{\text{grad }}
\renewcommand{\div}{\text{div }}
\newcommand{\rot}{\text{rot }}

	
\begin{document}

\maketitle

\tableofcontents

\chapter{Einleitung}
Gegenstand der Vorlesung ist die (klassische) Theorie der Elektrischen Felder ausgehend von den \textsc{Maxwell}-Gleichungen (1864):

\begin{equation*}
\div \vec{B}=0
\end{equation*}

\begin{equation*}
\rot  \vec{E}+\pdiff{\vec{B}}{t}=0
\end{equation*}

\begin{equation*}
\varepsilon_0\div \vec{E}=\rho
\end{equation*}

\begin{equation*}
\frac{1}{\mu_0}\rot \vec{B}-\varepsilon_0\pdiff{\vec{E}}{t}=\vec{j}
\end{equation*}

für die Felder $\vec{E}$ und $\vec{B}$ in Abhängigkeit von Ladungs- und Stromverteilung $\rho(\vec{r},t)$ und $\vec{j}(\vec{r},t)$  sollen physikalische Erscheinungen geschildert werden.\\
Die Elektrodynamik ist ein Teil des Standardmodells der Teilchenphysik, das einheitlich Teilichen und ihre Wechselwirkungen beschreibt.\\
Klassische Elektrodynamik ist ein Grenzfall der Quantenelektrodynamik (gültig für kleine Impuls- und Energiebeträge, große Brechungszahlen für Photonen).\\
Sie ist im Einklang mit der der speziellen Relativitätstheorie (c ist implizit in den \textsc{Maxwell}-Gleichungen enthalten). Viele interessante Effekte von Materie können mit klassischer Theorie nicht beschrieben werden.\\
Zum Beispiel: Wann sind Atome stabil? Wann ist Eisen ferromagnetisch? Warum wird z.B. Blei bei tiefen Temperaturen supraleitend? Für diese Fragen werden Quanteneffekte wichtig.

\chapter{Mathematische Hilfsmittel}

\section{Skalar- und Vektorfelder}

Felder entsprechen Größen, die an jedem Raumpunkt einen bestimmten Wert haben, der zeitabhängig sein kann.\\ \linebreak

\begin{wrapfigure}[]{r}[0cm]{0cm}
	\raisebox{0pt}[\dimexpr\height-1\baselineskip\relax]{
		\colorbox{hgrey}{
			\begin{tikzpicture}
			\draw[->](0,0)-- (4,0) node[right]{$x$};
			\draw[->](0,0)--(0,3) node[above]{$y$};
			\draw (3,3) node[right]{$T(x,y)$};
			\draw plot[smooth, tension=.8] coordinates {(0.5,1.5)(2,0.7)(3.5,0.4)} node[right]{$T=20^{\circ}$} ;
			\draw plot[smooth, tension=.8] coordinates {(0.5,2.2)(2,1.1)(3.5,0.8)} node[right]{$T=40^{\circ}$} ;
			\draw plot[smooth, tension=.8] coordinates {(0.5,3)(2,1.6)(3.5,1.2)} node[right]{$T=60^{\circ}$} ;
			\end{tikzpicture}
		}
	}
	\caption{Isothermen}
\end{wrapfigure}

\textbf{a. skalare Felder}\ $\phi=\phi(x,y,z,t)$\\ 


Jedem Raumpunkt wird ein Wert in Form einer (reellen) Zahl zugeordnet, wie zum Beispiel Temperatur, Druck, Ladung oder Energie. Flächen oder Linien mit konstantem Wert nennt man Äquipotentialflächen beziehungsweise -linien.\\ \linebreak\linebreak

\begin{wrapfigure}[]{r}[0cm]{0cm}
	\raisebox{0pt}[\dimexpr\height-1\baselineskip\relax]{
		\colorbox{hgrey}{
			\begin{tikzpicture}
			
			\draw[->](0,0)-- (4,0) node[right]{$x$};
			\draw[->](0,0)--(0,3) node[above]{$y$};
			\draw (3,3) node[right]{$\vec{F}(x,y)$};
			\foreach \x/\angle in {0.5/20, 1.5/40} {
				\foreach \y in {0.5, 1.5, 2.5} {
					\fill (\x,\y) circle[radius=1pt];
					\draw[->,thick]  (\x, \y) -- ++(\angle:1);
				}
			}
			\foreach \x/\angle in {2.5/60, 3.5/80} {
				\foreach \y in {0.5, 1.5} {
					\fill (\x,\y) circle[radius=1pt];
					\draw[->,thick]  (\x, \y) -- ++(\angle:1);
				}
			}
			\end{tikzpicture}
		}
	}
	\caption{Kraftfeld}
\end{wrapfigure}


\textbf{b. Vektorfelder}\ $\vec{E}=\vec{E}(x,y,z,t)$\\

Jedem Raumpunkt wird ein Vektor zugeordnet, der lokal die Richtung des Feldes beschreibt, wie etwa ein Geschwindigkeits- oder Kraftfeld. Vektorfelder lassen sich durch Feldlinien veranschaulichen, entlang derer sich zum Beispiel ein Teilchen bewegt, das die entsprechende Kraft erfährt.


\section{Integrale auf Feldern}
Integrale über skalare Felder werden wie bekannt bebildet; sie sind zu vermeiden.\\
Integriert man über ein Vektorfeld, spielt die Richtungsinformation eine entscheidende Rolle. Man unterscheidet je nach Dimension des Parameterbereichs von Linien-, Flächen- und Volumenintegralen.\\
\linebreak


\textbf{a. Linienintegrale}

\begin{equation*}
\varphi=\int\limits_{C}\vec{E}(\vec{r})\d\vec{r}
\end{equation*}

\begin{wrapfigure}[]{r}[0cm]{0cm}
	\raisebox{0pt}[\dimexpr\height-1\baselineskip\relax]{
		\colorbox{hgrey}{
			\begin{tikzpicture}
			
			\draw[->](0,0)-- (4,0) node[right]{$x$};
			\draw[->](0,0)--(0,3) node[above]{$y$};
			\draw[->](1.2,0.8)--(1,2) node[above]{$\vec{E}$};
			\draw[->](2,2.3)--(2.4,3) node[right]{$\vec{E}$};
			\draw(2,1.8) node{$C$};
			\draw[thick] plot[smooth, tension=.8] coordinates {(1.2,0.8)(2,2.3)(3.5,3)};
			\filldraw[black] (3.5,3) circle (2pt) node[right]{$\vec{r}(\tau_2)$};
			\filldraw[black] (1.2,0.8) circle (2pt) node[below left]{$\vec{r}(\tau_1)$};
			
			\end{tikzpicture}
		}
	}
	\caption{Linienintegral}
\end{wrapfigure}
Wir parametrisieren die Kurve durch $\vec{r}=\vec{r}(\tau)$ und erhalten somit
\begin{equation*}
\varphi=\int\limits_{\tau_0}^{\tau_1}\vec{E}(\vec{r}(\tau))\diff{\vec{r}}{\tau}\d \tau
\end{equation*}
Ein Speziallfall des Linienintegrals ist das sogenannte \textbf{geschlossene Linienintegral}, welches durch $\oint$ gekennzeichnet wird.\\
\linebreak


\begin{wrapfigure}[10]{r}[0cm]{0cm}
	\raisebox{0pt}[\dimexpr\height-1\baselineskip\relax]{
		\colorbox{hgrey}{
			\begin{tikzpicture}
			
			\draw plot[smooth, tension=.8] coordinates {(3,0) (5,1.4) (1.5,2) (0,1) (0,0) (3,0) };	
			\draw[->] (2,1)--(2,3) node[left]{$\Delta\vec{A}_i$};
			\draw[->] (2,1)--(3,2.8) node[right]{$\vec{E}(\vec{r}_i)$};
			\draw (0.5,0.5) node[right]{$\mathcal{F}$};
			
			
			\end{tikzpicture}
		}
	}
	\caption{Flächenintegral}
\end{wrapfigure}


\textbf{b. Flächenintegrale}\\
\begin{equation*}
\Phi=\iint\limits_{S}\vec{B}\cdot\d \vec{A} \ \ \ \text{mit } \d\vec{A}=\d A\cdot\vec{n}
\end{equation*}


Ganz analog zu \textbf{a.} kann die Fläche $\vec{r}=\vec{r}(u,v)$ parametrisiert werden. Es ist jedoch beim Bilden der Funktionaldeterminante auf die Richtung des Flächenelements zu achten. Die beiden möglichen Lösungen unterscheiden sich natürlich nur um ein Vorzeichen. Wir erhalten also
\begin{equation*}
\Phi=\int\limits_{v_1}^{v_2}\int\limits_{u_1}^{u_2}\vec{B}(u,v)\cdot\left(\pdiff{\vec{r}}{u}\times\pdiff{\vec{r}}{v}\right)\d u\d v
\end{equation*}

\textbf{c. Volumenintegrale}\\
\linebreak
\begin{equation*}
Q=\iiint\limits_G\d V\cdot\rho(\vec{r})=\iiint\limits_G\d ^3 r\cdot\rho(\vec{r})=
\end{equation*}
Beim Volumenintegral wird wiederum (nicht wie beim Flächenintegral) das Vorzeichen des Volumenelements vernachlässigt, da physikalisch die \textbf{Richtung} des Volumens nur sehr selten wirklich von Bedeutung ist. Mit entsprechender Parametrisierung $\vec{r}=\vec{r}(u,v,w)$ ergibt sich
\begin{equation*}
q=\int\limits_{w_1}^{w_2}\int\limits_{v_1}^{v_2}\int\limits_{u_1}^{u_2}\rho(u,v,w)\cdot\left|\pdiff{\vec{r}}{u}\cdot\left(\pdiff{\vec{r}}{v}\times\pdiff{\vec{r}}{w}\right)\right|\d u\d v\d w
\end{equation*}

\section{Vektorielle Ableitungen und Integrale}
\textbf{a. Gradient}\\
\linebreak
Der Gradient $\grad\varphi\ $ eines Skalarfeldes beschreibt dessen Änderung und steht senkrecht auf den Äquipotentialflächen (oder allgemeiner: Niveaumengen). Der Gradient lässt sich durch den Nabla-Operator ausdrücken und lautet in karthesischen Koordinaten: 
\begin{equation*}
\nabla=\pdiff{}{x}\vec{e}_x+\pdiff{}{y}\vec{e}_y+\pdiff{}{z}\vec{e}_z
\end{equation*}
Wichtig ist, dass $\nabla$ ein vektorieller Differenzialoperator ist. Er folgt Ableitungsregeln, wie etwa der Kettenregel, und $\nabla\varphi$ verhält sich unter Koordinatentransformation wie ein Vektor.\\
\linebreak
Andere Schreibweisen: $\pdiff{}{\vec{r}},\ \partial_{\vec{r}},\ \nabla_{\vec{r}}$\\
\linebreak
\underline{Beispiele:}\\
\linebreak
$\nabla |\vec{r}|=\frac{\vec{r}}{|\vec{r}|}=\vec{e}_r\\
\nabla \frac{1}{|\vec{r}|}=-\frac{1}{r^2}\vec{e}_r$\\
\linebreak\linebreak
\textbf{b. Divergenz} (Quellenstärke eines Vektorfeldes)\\
\linebreak
Die Divergenz div $\vec{E}=\nabla\cdot\vec{E}$ ist ein Skalar unter Koordinatentransformation und kann als \textbf{lokale Quellenstärke} interpretiert werden. Häufig benötigt man auch den \textsc{Laplace}-Operator, der die \textbf{zweite Ableitung} repräsentiert.\\
\begin{equation*}
\div\grad \varphi \ = \ \nabla^2\varphi \ = \ \laplace\varphi
\end{equation*}
\underline{Beispiele:}\\
\linebreak
div $\vec{r}=3$ (Anzahl der Dimensionen)\\
div $(\varphi\vec{A})=\nabla\cdot(\varphi\vec{A})=\vec{A}(\nabla\varphi)+\varphi(\nabla\vec{A})=\vec{A}\cdot\grad \varphi+\varphi\cdot\div\vec{A}$\\
\linebreak
\textbf{c. Rotation} (Wirbelstärke eines Vektorfeldes)\\
\linebreak
Die Rotation rot $\vec{B}=\nabla\times\vec{B}$

\begin{equation*}
\nabla\times\vec{B}=\begin{vmatrix}
\vec{e}_x & \vec{e}_y & \vec{e}_z \\
\pdiff{}{x} & \pdiff{}{y} & \pdiff{}{z}\\
B_x & B_y & B_z
\end{vmatrix}
\end{equation*}

kann als \textbf{lokale Wirbelstärke} verstanden werden. Ihre Komponenten lassen sich auch als

\begin{equation*}
(\nabla\times\vec{B})_i=\sum\limits_{j,k}\epsilon_{ijk}\cdot\pdiff{}{x_j}\cdot B_k
\end{equation*}

darstellen wobei $\epsilon_{ijk}\ $ der total antisymetrische Tensor 3. Stufe ist.\\
\linebreak
\underline{Beispiele:}\\
\linebreak
$\vec{v}=\vec{\omega}\times\vec{r} \ \Rightarrow \ \nabla\times\vec{v}=2\vec{\omega}$\\
$\nabla\times\vec{r}=0$\\
\linebreak\linebreak
\textbf{d. \textsc{Gauss}'scher Satz}\\
\begin{equation*}
\iiint\limits_V\div\vec{E}\cdot\d V=\oiint\limits_{\partial V}\vec{E}\cdot\d \vec{A}
\end{equation*}
Der Satz von \textsc{Gauss} verknüpft Eigenschaften im Inneren eines Volumens mit dem Verhalten auf dem Rand.\\

Über den Satz von \textsc{Gauss} lässt sich auch die partielle Integration in drei Dimensionen umformen zu:

\begin{equation*}
\Int{V}{}{V} \ \pdiff{}{\vec{r}} (u\cdot v) = \Int{V}{}{V} \  \pdiff{u}{\vec{r}}\cdot v \ + \ \Int{V}{}{V} \  u \cdot\pdiff{v}{\vec{r}} \ = \ \Oiint{\partial V}{}{A} \  (u\cdot v)
\end{equation*}

\ \\
\textbf{e. \textsc{Green}'scher Satz}\\
\begin{equation*}
\Int{V}{}{(\varphi\laplace\psi-\psi\laplace\varphi)}{V}=\Oint{\partial V}{}{(\varphi\nabla\psi-\psi\nabla\varphi)}{\vec{A}}
\end{equation*}
\linebreak
\textbf{f. \textsc{Stokes}'scher Satz}\\
\begin{equation*}
\iint\limits_S\rot\vec{B}\cdot\d\vec{A}=\oint\limits_{\partial A}\vec{B}\cdot\d\vec{r}
\end{equation*}
Analog zu \textsc{Gauss}'schen Satz verknüft der Satz von \textsc{Stokes} das Verhalten eines Feldes auf einer Fläche mit dem auf dem Rand der Fläche. Für geschlossene Flächen gilt
\begin{equation*}
\oiint\limits_{S=\partial V}\rot\vec{B}\cdot\d\vec{A}=0
\end{equation*}

\section{Differentialoperatoren in krummlinigen Koordinaten}
Karthesische /Kugel-/Zylinderkoordinaten sind hier wichtig.\\
\linebreak
z.B: \ $\nabla_x\psi=\partial_x\psi\vec{e}_x+\partial_y\psi\vec{e}_y+\partial_z\psi\vec{e}_z$\\
\linebreak
$\nabla_\theta\psi=\pdiff{}{r}\psi\vec{e}_r+\frac{1}{r}\pdiff{}{\theta}\psi\vec{e}_\theta+\frac{1}{r\sin\theta}\pdiff{}{\phi}\psi\vec{e}_\phi$\\
\linebreak
Generell: $(\nabla\psi)_u\equiv(\nabla\psi)\vec{e}_u=\frac{1}{g_u}\pdiff{\psi}{u}$ \ mit \ $g_u=|\pdiff{\psi}{u}|$\\

\section{\textsc{Fourier}-Transformation}
\begin{equation*}
\tilde{f}(\omega)=\frac{1}{\sqrt{2\pi}}\Int{-\infty}{\infty}{f(t)e^{-i\omega t}}{t}
\end{equation*}
\begin{equation*}
f(t)=\frac{1}{\sqrt{2\pi}}\Int{-\infty}{\infty}{\tilde{f}(t)e^{i\omega t}}{\omega}
\end{equation*}
Verallgemeinert auf $n$ Dimensionen ergibt sich:\\
\begin{equation*}
\tilde{f}(\vec{k})=\frac{1}{({2\pi})^{\frac{n}{2}}}\Int{-\infty}{\infty}{f(\vec{r})e^{-i\vec{k}\vec{r}}}{^nr}
\end{equation*}
\linebreak
\textbf{a. Differentiation}\\
\begin{equation*}
\diff{}{t}f(t)=\frac{1}{\sqrt{2\pi}}\Int{-\infty}{\infty}{i\omega\tilde{f}(\omega)e^{i\omega t}}{\omega}
\end{equation*}
\linebreak
\textbf{b. Faltung}\\
\begin{equation*}
(f*g)(t)=\frac{1}{\sqrt{2\pi}}\Int{-\infty}{\infty}{f(t-s)G(s)}{s}
\end{equation*}
\begin{equation*}
\widetilde{(f*g)}(\omega)=\tilde{f}(\omega)\tilde{g}(\omega)
\end{equation*}
\textbf{c. Rechenregeln}\\
\begin{align*}
f'(t) & \leftrightarrow i\omega\tilde{f}(\omega)\\
-itf(t) & \leftrightarrow \tilde{f}'(\omega)\\
f(t+a) & \leftrightarrow  e^{i\omega a}\tilde{f}(\omega)\\
e^{i\omega t}f(t) &\leftrightarrow & \tilde{f}(\omega-a)\\
f(at) & \leftrightarrow \frac{1}{|a|}\tilde{f}\left(\frac{\omega}{a}\right)\\
f^*(t) & \leftrightarrow \tilde{f}^*(\omega)\\
\tilde{\tilde{f}}(t) &\leftrightarrow & f(-t)\\
\end{align*}

\section{Delta-Distribution}

Die Delta-Distribution ist über folgende Eigenschaften definiert:

\begin{enumerate}
\item
\begin{equation*}
\delta(\vec{r}) = \begin{cases}
0 & \text{für }\vec{r}\neq\vec{r}_0\\
\infty & \text{für } \vec{r} = \vec{r}_0
\end{cases}
\end{equation*}

\item
\begin{equation*}
\int\limits_{\vec{r}_0\in V}\d V \ \delta({\vec{r}-\vec{r}_0}) = 1
\end{equation*}
\end{enumerate}

Alle Aussagen gelten analog für die Delta-Distribution $\delta(x)$ in einer Dimension.\
Bei höherdimensionalen Deltadistributionen gilt allerdings nur in kartesischen Koordinaten:

\begin{equation*}
\delta(\vec{r} - \vec{r}_0) = \delta(x-x_0)\cdot\delta(y-y_0)\cdot\delta(z-z_0)
\end{equation*}
\ \\
Faltet man die Delta-Distribution mit einer Funktion $f(\vec{r})$, so ergibt sich aus ihren Eigenschaften:

\begin{equation*}
\int\limits_{\vec{r}_0\in V}\d V \ \delta({\vec{r}-\vec{r}_0}) \ f(\vec{r}) = f(\vec{r}_0)
\end{equation*}

\section{\textsc{Green}'sche Funktion zur Lösung inhomogener linearer DGL}

Wir betrachten die lineare, inhomogene Differentialgleichung

\begin{equation*}
L \ \phi (x_1,\dotsc,x_n) = \rho (x_1,\dotsc,x_n) \; \text{ oder kurz } \; L\phi = \rho
\end{equation*}

wobei $L$ ein linearer Operator und $\rho$ die Inhomogenität sein soll.\
\\
Die \textsc{Green}'sche Funktion $G(x,x)$ zum Operator $L$ ist die Lösung der Differentialgleichung mit $\delta$-förmiger Inhomogenität.

\begin{equation*}
L \ G(x,x') = \delta (x-x') \; [= \delta(x_1-x_1')\cdot\dotsc\cdot\delta(x_n - x_n')]
\end{equation*}
\ \\
Wenn $g$ bekannt ist, dann kann die Lösung für beliebige Inhomogenität durch Superposition gewonnen werden.

\begin{equation*}
\phi (x) = \int\d x' \ G(x,x') \rho(x')
\end{equation*}

Den Beweis hierfür erhält man leicht durch Einsetzen:

\begin{equation*}
L \ \phi(x) = \int\d x' \ L \ G(x,x') \rho(x') = \rho(x)
\end{equation*}


\chapter{Grundbegriffe und \textsc{Maxwell}-Gleichungen}
\section{Kräfte und Punktladungen}

Aus der Erfahrung ergibt sich für eine ruhende Ladung

\begin{equation*}
\vec{F}(\vec{r},t)=Q\cdot\vec{E}(\vec{r},t)
\end{equation*}

Dabei ist die Ladung $Q$ eine Körpereigenschaft und $\vec{E}$ eine Eigenschaft, die die Umwelt charakterisiert. Über den Vergleich der Kraft auf zwei Körper $\vec{F}_1(\vec{r},t)=\frac{Q_1}{Q_2}\vec{F}_2(\vec{r},t)$ lässt sich so eine Einheit für die Ladung definieren.\\
\linebreak

Bei bewegten Ladungen beobachten wir etwas anderes. Die Kraft hat hier die Form

\begin{equation*}
\vec{F}=Q(\vec{E}+\vec{v}\times\vec{B})
\end{equation*}

\section{Ladungs- und Stromdichte, Ladungserhaltung}

Über eine Ladung in einem Volumenelement lässt sich der Begriff der Ladungsdichte definieren.

\begin{equation*}
\rho(\vec{r},t)=\diff{Q}{V}
\end{equation*}

Eine Ladungsänderung nennen wir schließlich den elektrischen Strom.

\begin{equation*}
-I:=\dot{Q}=\diff{}{t}\Int{V}{}{\rho(\vec{r},t)}{V}=\Int{V}{}{\pdiff{\rho}{t}}{V}
\end{equation*}		
Betrachten wir nun den Stromfluss durch ein Oberflächenelement d$\vec{A}$. Die Ladungsträger, welche durch diese Fläche wandern haben die Geschwindigkeit $\vec{v}$, sodass anschaulich ein kleines Volumenelement dV$ = \vec{v}\mathrm{d}t\cdot\mathrm{d}\vec{A}$ aufgespannt wird:

\begin{align*}
\mathrm{d}Q &= \rho(\vec{r},t)\vec{v}(\vec{r},t)\mathrm{d}t\mathrm{d}\vec{A}\\
\diff{Q}{t} =-I &= \rho\vec{v}\cdot\mathrm{d}\vec{A}=:\vec{j}(\vec{r},t)\cdot\mathrm{d}\vec{A}
\end{align*}

Wir nennen $\vec{j} = \rho\vec{v}$ der Anschaulichkeit nach die \textbf{Stromdichte}, denn man sieht leicht:

\begin{equation*}
\iint\limits_A\vec{j}\cdot\mathrm{d}\vec{A}=I
\end{equation*}

Setzen wir nun dies in die Gleichung für die Ladungserhaltung ein:

\begin{equation*}
0 = \dot{Q} + I = \iiint\limits_V\mathrm{d}V\pdiff{\rho}{t} \ + \oiint\limits_{\partial V}\mathrm{d}\vec{A}\cdot\vec{j} = \iiint\limits_V \mathrm{d}V\left(\pdiff{\rho}{t} + \pdiff{\vec{j}}{\vec{r}}\right) 
\end{equation*}

Da dies für für alle möglichen Volumina gelten soll, folgt daraus die \textbf{Kontinuitätsgleichung}:

\begin{equation*}
\dot{\rho} + \text{div} \ \vec{j} = 0
\end{equation*} 

Für den Grenzfall eines unendlich großen Volumens gilt zunächst $\vec{j}\rightarrow 0$ auf der Oberfläche, woraus man auf die für diesen Grenzfall logische Konsequenz schließen kann, dass

\begin{equation*}
\dot{Q} = -\oiint\vec{j}\mathrm{d}\vec{A} = 0
\end{equation*}

die Ladung im gesamten Raum erhalten ist.\ \\


Mit der eingeführten Stromdichte $\vec{j}$ kann man nun auch den Ausdruck der \textsc{Lorentz}kraft-Dichte $\vec{f} := \frac{\vec{F}}{V}$ definieren:

\begin{align*}
\mathrm{d}\vec{F} & = \mathrm{d}Q(\vec{E} + \vec{v}\times\vec{B}) \\
\Rightarrow \vec{f} & = \rho(\vec{r},t)\cdot(\vec{E}(\vec{r},t) + \vec{v}(\vec{r},t)\times\vec{B}(\vec{r},t) = \rho\vec{E}+ \vec{j}\times\vec{B}
\end{align*}

\section{Die \textsc{Maxwell}-Gleichungen}
Die \textsc{Maxwell}-Gleichungen wurden 1864 vom schottischen Physiker James Clerk \textsc{Maxwell} aufgestellt und bilden ein Differentialgleichungssystem für die Felder  $\vec{B}(\vec{r})$ und $\vec{E}(\vec{r})$. Zusammen mit der Kontinuitätsgleichung beschreiben sie die gesamte (klassische) Elektrodynamik, da $\rho$ und $\vec{j}$ die Quellen und Wirbel des $ \ \vec{B} \ $- und $ \ \vec{E} \ $-Feldes eindeutig bestimmen:

\begin{align*}
\text{div} \ \vec{B} &= 0 \qquad\qquad\qquad\quad \epsilon_0\text{div} \ \vec{E} = \rho \\
\text{rot} \ \vec{E} + \dot{\vec{B}} &= 0 \qquad\qquad \frac{1}{\mu_0}\text{rot} \ \vec{B} - \epsilon_0\dot{\vec{E}} = \vec{j}
\end{align*}


Nun könnte man fragen, ob die  Beschreibung der Elektrodynamik über lokale Felder denn zweckmäßig ist oder ob man sie nicht eliminieren könnte. Das \textsc{Coulomb}-Gesetz wäre ein Beispiel für diese Fernwirkungstheorie. Zwei Gründe sprechen für die lokale Feldtheorie: sie ist zum einen schlichtweg einfacher mathematisch zu beschreiben und zum anderen unabhängig vom Vorhandensein von Materie und demzufolge Ladungsträgern.

\section{Konstruktion der \textsc{Maxwell}-Gleichungen}
Versucht man die Elektrodynamik zu beschreiben, so kann man sich zu Beginn von phänomenologischen Seite diesem Problem nähern und fordern, dass Symmetrien in Zeit und Raum die Gültigkeit der Gleichungen erhalten sollen. Dies ist eine gängige physikalische Vorgehensweise; man verlangt, dass die beschriebene (reale) Physik unabhängig von der Wahl der Koordinaten sein soll.
Wir fordern also zunächst, dass die die Form der Gleichungen unter den Symmetrietransformationen der Rauminversion $(\vec{r}\rightarrow-\vec{r})$ und der zeitlichen Reversibilität ($t\rightarrow-t$) invariant ist. Zudem wollen wir uns als Ziel setzen, die Gesetze möglichst einfach zu formulieren, das heißt, es sollen maximal Differentialgleichungen 1. Ordnung auftauchen.
Betrachten wir nun also zunächst das Transformationsverhalten verschiedener Objekte:\ \\
\ \\


\begin{tabular}{c|c|c|l}
Objekt & $t\rightarrow-t$ & $\vec{r}\rightarrow-\vec{r}$ & Bemerkung\\
\hline $t, \pdiff{}{t}$ & - & + & Definition\\
$\vec{r},\pdiff{}{\vec{r}}$ & + & -& Definition\\
$\dot{\vec{r}}$ & -& - & durch Multiplikation der Vorzeichen erhalten\\
$\ddot{\vec{r}},\vec{F},\vec{f}$ & + & - & Erfahrung aus Mechanik:\ $\ddot{\vec{r}}=\frac{\vec{F}}{m}$\\
$Q,\rho$ & + & + & Annahme\\
$\vec{j}\ (=\rho\cdot\dot{\vec{r}})$ & - & - & \\
$\vec{E}$ & + & - & Vektor, erhalten aus: $\vec{F}=Q(\vec{E}+\vec{v}\times\vec{B})$\\
$\vec{B}$ & - & + & Pseudovektor\\
$\pdiff{}{\vec{r}}\cdot\vec{E}$ & + & + & Skalar\\
$\pdiff{}{\vec{r}}\cdot\vec{B}$ & - & - & Pseudoskalar\\
$\pdiff{}{\vec{r}}\times\vec{E}$ & + & + & Pseudovektor\\
$\pdiff{}{\vec{r}}\times\vec{B}$ & - & - & Vektor\\
$\pdiff{}{t}\vec{E}$ & - & - & Vektor\\
$\pdiff{}{t}\vec{B}$ & + & + & Pseudovektor
\end{tabular}
\ \\
\ \\
\ \\
\ \\
\ \\
\ \\
Da wir gefordert hatten, dass unsere gewünschten Gleichungen invariant unter den Transformationen sein sollten, dürfen wir nun nur die Größen mit dem gleichen Transformationsverhalten verknüpfen:\ \\


\begin{enumerate}
\item
\underline{$++$ Skalar} \quad $\rho,\text{div} \ \vec{E}\\
\\
\Rightarrow \rho = \epsilon_0 \cdot \text{div} \ \vec{E} \qquad (\epsilon_0$  ist beliebige Konstante)

\item
\underline{$--$ Vektor} \quad $\vec{j}, \text{rot} \ \vec{B},\dot{\vec{B}}\\
\\
\Rightarrow \vec{j} = \alpha\cdot\dot{E} + \frac{1}{\mu_0}\cdot\text{rot} \ \vec{B}  \qquad (\alpha,\frac{1}{\mu_0}$ sind beliebige Konstanten)

\item
\underline{$--$ Skalar} \quad $\text{div} \ \vec{B}\\ 
\\
\Rightarrow \ 0 = \text{div} \ \vec{B}$

\item
\underline{$++$ Vektor} \quad $\text{rot} \ \vec{E},\dot{\vec{B}}\\
\\
\Rightarrow \ 0 = \text{rot} \ \vec{E} + \beta\cdot\dot{\vec{B}} \qquad (\beta$  ist beliebige Konstante)
\\
\\

\item
\underline{$+-$ Vektor}  \quad $\vec{E},(\vec{r},\ddot{\vec{r}})\\
\\
\Rightarrow \ 0 =  \vec{E}$

\item
\underline{$-+$ Vektor} \quad $\vec{B}\\
\\
\Rightarrow \ 0 = \vec{B}$
\end{enumerate}
\ \\

Das System 1-4 ist ein widerspruchsfreies und vollständiges System von Differentialgleichungen für das $\vec{E}$- und das $\vec{B}$-Feld, da diese durch ihre Quellen und Wirbel jeweils eindeutig bis auf Konstanten bestimmt sind. Diese werden problemabhängig aus den gegebenen Randbedingungen bestimmt. Die Gleichungen 5 und 6 werden aus naheliegenden Gründen weggelassen; sie stehen zwar nicht im Widerspruch zu den ersten 4 Gleichungen, doch würde das Differentialgleichungssystem mit ihnen nur noch die Triviallösung ohne physikalisch interessante Bedeutung liefern.\
\\
\ \\
\ \\

\underline{\textbf{Konstantendiskussion:}}
\ \\
\begin{enumerate}
\item Die Konstante $\epsilon_0$ ist zunächst frei wählbar, da die Ladung $Q$ nur bis auf einen Faktor genau bestimmt ist. Für die Wahl von $\epsilon_0$ gibt es verschiedene Ansätze:\
\begin{enumerate}
\item $\epsilon_0$ wird als 1 definiert. Diese Defintion wird im cgs-System umgesetzt.\
\\
\item $4\pi\cdot\epsilon_0$ wird 1 gesetzt. Das sich aus dieser Definition ergebende Einheitensystem nennt man das \textsc{Gauss}-System.\
\end{enumerate}
\ \\
Im SI-System wird dagegen $\epsilon_0$ über $\mu_0$ festgelegt, wobei für $\mu_0$ gilt:

\begin{equation*}
[\mu_0] = \frac{[\vec{E}]}{[I]}\frac{[l]^2}{[l]}=\frac{[\vec{f}]}{[\vec{j}]}\frac{[l]}{[I]}=\frac{[\vec{F}]}{[I]^2}=\frac{N}{A^2}
\end{equation*}
\begin{equation*}
\mu_0 = 4\pi\cdot 10^{-7} \frac{N}{A^2}
\end{equation*}

$\epsilon_0$ erhält man nun daraus über die Fundamentalbeziehung im SI-System:
\begin{equation*}
\epsilon_0\mu_0 = \frac{1}{c^2}
\end{equation*}

\item
Die Konstante $\alpha$ erhalten wir, in dem wir von Gleichung (2) die Divergenz bilden und dann div $\vec{j}$ aus der Kontinuitätsgleichung einsetzen:

\begin{equation*}
(\epsilon_0 + \alpha) \ \pdiff{}{t} \ \text{div} \ \vec{E} \overset{!}{=} 0 \; \Rightarrow \; \alpha = -\epsilon_0 
\end{equation*}

\item
Dass die Konstante $\beta$ im SI-System gleich 1 sein musss, erhält man aus Überlegungen, dass die \textsc{Maxwell}-Gleichungen von Inertialsystem zu Inertialsystem invariant sein müssen.
\end{enumerate}
\ \\
\underline{Bemerkung:}
Im \textsc{Gauss}-System erhält man aufgrund der Wahl der Konstanten für die \textsc{Lorentz}-Kraft:
\begin{equation*}
\vec{F} = Q (\vec{E} + \frac{\vec{v}}{c}\times\vec{B})
\end{equation*}
woraus folgt:
\begin{equation*}
\epsilon_0\mu_0\cdot\beta = \frac{1}{c^2} \; \text{ und } \; \beta = \frac{1}{c}, \mu_0 = \frac{4\pi}{c}
\end{equation*}

\section{Integrale Fromulierung der \textsc{Maxwell}-Gleichungen}
Die integrale Formulierung der \textsc{Maxwell}-Gleichungen ist äquivalent zu der differentiellen und ergibt sich entweder aus Volumen- oder Flächenintegration auf beiden Seiten der entsprechenden Gleichung und dann der Anwendung der Integralsätze von \textsc{Gauss} oder \textsc{Stokes}:
\ \\
\begin{align*}
\text{i)} \quad & \epsilon_0 \ \text{div} \ \vec{E} = \rho & \Leftrightarrow \qquad\qquad & \epsilon_0\oiint \mathrm{d}\vec{A} \cdot\vec{E} = Q_{\text{in}} \\
\text{ii)} \quad & \text{div} \ \vec{B} = 0 & \Leftrightarrow \qquad\qquad & \oiint \mathrm{d}\vec{A}\cdot\vec{B} = 0 \\
\text{iii)} \quad & \text{rot} \ \vec{E} + \dot{\vec{B}} = 0 & \Leftrightarrow \qquad\qquad & \oint\limits_{\partial A} \mathrm{d}\vec{r}\cdot\vec{E} \ + \ \iint\limits_A \mathrm{d}\vec{A}\cdot\dot{\vec{B}} = 0 \\
\text{iv)} \quad & \frac{1}{\mu_0} \ \text{rot} \ \vec{B} - \epsilon_0 \dot{\vec{E}} = \vec{j} & \Leftrightarrow \qquad\qquad & \frac{1}{\mu_0}\oint\limits_{\partial A}\mathrm{d}\vec{r}\cdot\vec{B} \ - \ \epsilon_0\iint\limits_A\mathrm{d}\vec{A}\cdot\dot{\vec{E}} = I_{\text{in}}
\end{align*}

\underline{Bemerkung:}

$\vec{r}$ und $t$ sind unabhängige Variablen, das heißt, dass die Felder $\vec{B}$ und $\vec{E}$ jeweils von $\vec{r}$ und $t$ abhängen, nicht aber von $\dot{\vec{r}}$.
Zudem ist es aufgrund unserer Forderungen bei der Konstruktion der \textsc{Maxwell}-Gleichungen verboten, dass eine explizite Abhängigkeit der Grundgleichungen von $\vec{r}$ und $t$ vorliegt, da es sonst außergewöhnliche Zeiten und Orte gäbe, was aber die geforderte Homogenität verletzen würde.
\section{Induktionsgesetz für Leiterschleifen}

Zunächst definieren wir den magnetischen Fluss $\Phi$ durch eine Fläche $\vec{A}$ im Raum:

\begin{equation*}
\Phi := \iint\limits_A\mathrm{d}\vec{A}\cdot\vec{B}
\end{equation*}

Man sieht leicht, dass sich der Fluss $\Phi$ bei Flächenänderung und Änderung der magnetischen Flussdichte $\vec{B}$ ändert:

\begin{align*}
\Delta\Phi &=  \Delta\left(\iint\mathrm{d}\vec{A}\cdot\vec{B}\right) = \iint\limits_A\mathrm{d}\vec{A}\cdot\Delta\vec{B} \; + \; \iint\limits_{\Delta A}\mathrm{d}\vec{A}\cdot\vec{B}\\
&= \Delta t \iint\limits_A\mathrm{d}\vec{A}\cdot\pdiff{\vec{B}}{t} \; + \; \oint\limits_{\partial A}\left(\vec{v}\Delta t\times\mathrm{d}\vec{r}\right)\cdot\vec{B}\\
&= \Delta t\left(\iint\limits_A\mathrm{d}\vec{V}\cdot\dot{\vec{B}} \; - \; \oint\limits_{\partial A}\mathrm{d}\vec{r}\cdot\left(\vec{v}\times\vec{B}\right)\right)\\
\Rightarrow \dot{\Phi} &= \iint\limits_A\mathrm{d}\vec{A}\cdot\dot{\vec{B}} \; - \; \oint\limits_{\partial A}\mathrm{d}\vec{r} \ \left(\vec{v}\times\vec{B}\right)
\end{align*}

Nach Anwenden der dritten \textsc{Maxwell}-Gleichung erhält man das \textbf{Induktionsgesetz}:
\begin{equation*}
\dot{\Phi} = - \oint\limits{\partial A}\mathrm{d}\vec{r} \ \left(\vec{E} + \vec{v}\times\vec{B}\right) = - U_{\mathrm{induziert}}
\end{equation*}

Das letzte Minuszeichen nennt man auch die \textbf{\textsc{Lenz}`sche Regel}, welche besagt, dass ein induzierter Strom immer ein Magnetfeld erzeugt, welches seiner eigenen Ursache ($U_{\mathrm{induziert}}$) entgegengerichtet ist.
\ \\
Auffällig bei dem Induktionsgesetz ist seine Ähnlichkeit mit der auf eine freie Ladung wirkende Kraft $\vec{F} = Q(\vec{E} + \vec{v}\times\vec{B})$. Darin liegt auch die Begründung für ebenjenes Gesetz:\

Wir stellen uns eine Leiterschleife vor, welche an einer Stelle durchbrochen ist, damit kein Strom durch die Schleife fließen könnte. Auf einen sich in dieser Schleife bewegenden Ladungsträger wirkt die Kraft:

\begin{equation*}
\vec{F} = Q(\vec{E} + \vec{v}\times\vec{B}) =: Q\vec{E'}
\end{equation*} 

Man sieht, dass das $\vec{E}$-Feld abhängig vom Bezugsystem ist, daher haben wir für $\vec{E'}$ ein Bezugssystem konstruiert, welches sich mit der Geschwindigkeit $\vec{v}$ gegenüber dem Laborsystem bewegt. Damit haben wir im mitbewegeten Bezugssystem erreicht, dass $\vec{v'} = 0$ ist. Bilden wir nun das Weginteral für ein Teilchen entlang der Leiterschleife im $\vec{E}$-Feld erhalten wir:

\begin{equation*}
\oint\limits_{\mathrm{Schleife}}\mathrm{d}\vec{r}\cdot\left(\vec{E} + \vec{v}\times\vec{B}\right) = \oint\limits_{\mathrm{Schleife}}\mathrm{d}\vec{r}\cdot\vec{E'} = \int\limits_{\mathrm{Beginn}}^{\mathrm{Ende}}\mathrm{d}\vec{r}\cdot\vec{E'} = U_{\mathrm{induziert}}
\end{equation*}

\chapter{Elektrostatik}

\section{Grundgleichungen und elektrostatisches Potential}

In der Elektrostatik betrachten wir, wie der Name schon andeutet, zeitunabhängige Felder. Dementsprechend kann man als erste Konsequenz daraus folgern, dass $\dot{\vec{E}} = 0$ und $\dot{\vec{B}} = 0$ ist. Fallen nun in den \textsc{Maxwell}-Gleichungen alle Beiträge mit $\dot{\vec{E}}$ und $\dot{\vec{B}}$ weg, kann man die Felder $\vec{E}$ und $\vec{B}$ getrennt voneinander betrachten. Laienhaft gesprochen entkoppeln wir die Phänomene \grqq Elektrizität\grqq und "Magnetismus". Des Weiteren betrachten wir in der Elektrostatik nur ruhende Ladungen, woraus folgt, dass außerdem $\vec{j}=0 \Rightarrow \vec{B}=0$ ist.\

Damit erhalten wir aus der dritten \textsc{Maxwell}-Gleichung, dass rot $\vec{E} = 0$ gilt, wodurch das Einführen eines Potentials für $\vec{E}$ möglich wird:

\begin{equation*}
\vec{E} =: \ -\grad \varphi
\end{equation*}

Mit div $\vec{E} = \frac{\rho}{\epsilon_0}$ erhält man daraus die \textbf{\textsc{Poisson}-Gleichung} der Elektrostatik:

\begin{equation*}
\bigtriangleup\varphi = - \frac{\rho}{\epsilon_0}
\end{equation*}

Für $\bigtriangleup\varphi = 0$ nennt man die \textsc{Poisson}-Gleichung auch \textbf{\textsc{Laplace}-Gleichung}.

\section{Kugelsymmetrische Ladungsverteilung}

Für eine kugelsymmetrische Ladungsverteilung gilt:

\begin{equation*}
\rho(\vec{r}) = \rho(|\vec{r}|) = \rho(r) \; \Rightarrow \; \varphi(\vec{r}) = \varphi(r)
\end{equation*}

Dem kann man entnehmen, dass die Äquipotentialflächen Kugelflächen sein müssen und somit der Gradient von $\varphi$ auch parallel zum Ortsvektor stehen muss.($\vec{E}(\vec{r}) = \vec{E}(r)\vec{e}_r$ \
Für das $\vec{E}$-Feld gilt weiterhin:

\begin{equation*}
\epsilon_0\oiint\limits_{\partial Kugel}\mathrm{d}\vec{A}\cdot\vec{E} \; \overset{\vec{A}\parallel\vec{E}}{=} \; \epsilon_0\oiint\limits_{\partial Kugel}\mathrm{d} A \cdot E = 4\pi\epsilon_0\cdot r^2 \cdot E(r) = Q_{\mathrm{in}}(r)
\end{equation*}

Damit ergibt sich für das $\vec{E}$-Feld und das Potential:

\begin{align*}
\vec{E}(r) &= \frac{Q_{\mathrm{in}}(r)}{4\pi\epsilon_0\cdot r^2} \cdot \vec{e}_r\\
\ \\
\varphi(r) &= \frac{Q_{\mathrm{in}}(r)}{4\pi\epsilon_0\cdot r} + \varphi_0 \; \text{ mit } \; \varphi_0 = \varphi(r \rightarrow 0)
\end{align*}

\section{Feld einer beliebigen räumlich begrenzten Ladungsverteilung}

\begin{enumerate}
\item Punktladung bei $\vec{r}_0$:
\begin{equation*}
\varphi(\vec{r}) = \frac{Q}{4\pi\epsilon_0 \ |\vec{r}-\vec{r}_0|}
\end{equation*}


\item Mehrere Punktladungen (Superpositionsprinzip anwendbar wegen Linearität der \textsc{Maxwell}-Gleichungen):
\begin{equation*}
\varphi(\vec{r}) = \sum\limits_i \ \frac{Q_i}{4\pi\epsilon_0 \ |\vec{r}-\vec{r}_i|}
\end{equation*}

\item Kontinuierliche Ladungsverteilung:
\begin{equation*}
\varphi(\vec{r}) = \int\mathrm{d}V' \ \frac{Q(\vec{r}')}{4\pi\epsilon_0 \ |\vec{r}-\vec{r}'|}
\end{equation*}
\end{enumerate}

Die allgemeine Gleichung für die kontinuerliche Ladugnsverteilung ergibt sich aus der Lösung der \textsc{Poisson}-Gleichung mithilfe der bekannten \textsc{Green}'schen Funktion für eine Punktladung der Größe 1:\ $G(\vec{r}) = \frac{1}{4\pi\epsilon_0 \cdot |\vec{r}|}$\

\begin{align*}
-\epsilon_0 \cdot \bigtriangleup\varphi &= \rho\\
\Rightarrow -\epsilon_0 \cdot \bigtriangleup G(\vec{r}) &= \delta(\vec{r})\\
\end{align*}

Dabei gilt: $G(\vec{r},\vec{r}') = G(\vec{r}-\vec{r}')$ aufgrund der Translationsinvarianz der \textsc{Green}-Funktion.

\begin{equation*}
\Rightarrow \varphi(\vec{r}) = \int\mathrm{d}V' \ G(\vec{r}-\vec{r}')\cdot\rho(\vec{r}') = \frac{1}{4\pi\epsilon_0} \ \int\mathrm{d}V' \ \frac{\rho(\vec{r}')}{|\vec{r}-\vec{r}'|}
\end{equation*}
\ \\

Aus dieser allgemeinen Form lässt sich natürlich auch im umgekehrten Falle das $\vec{E}$-Feld einer Punktladung in $\vec{r}_0$ herleiten. Dafür muss nur $\rho(\vec{r}) = Q\cdot\delta(\vec{r}-\vec{r}_0)$ gesetzt werden:

\begin{equation*}
\varphi(\vec{r}) = \int\mathrm{d}V' \ \frac{\rho(\vec{r}')}{4\pi\epsilon_0 \cdot |\vec{r}-\vec{r}'|} = \ \frac{Q}{4\pi\epsilon_0} \ \underbrace{\int\mathrm{d}V'\frac{\delta(\vec{r}'-\vec{r}_0)}{|\vec{r}-\vec{r}'|}}_{=\frac{1}{|\vec{r}-\vec{r}_0|}}
\end{equation*}

\section{Feld eines elektrischen Dipols}

Ein Dipol besteht aus zwei gleich großen, entgegengesetzt geladenen Ladungen $\pm Q $, welche  einen festen Abstand $\vec{a}$ voneinander entfernt sind. Daher ergibt es Sinn, als charakteristische Eigenschaft des Dipols das \textbf{Dipolmoment} $\vec{p}$ wie folgt zu definieren:

\begin{equation*}
\vec{p} := Q \cdot \vec{a}
\end{equation*}
\begin{equation*}
\text{Dipollimit: } \quad |\vec{a}| \ \rightarrow \ 0, \ Q \ \rightarrow \ \infty \ \Rightarrow \ |\vec{p}| = \ \text{const.}
\end{equation*}

Für das Potentialfeld eines solchen Dipols gilt offensichtlich:

\begin{equation*}
\varphi(\vec{r}) = \frac{Q}{4\pi\epsilon_0}\cdot\left(\frac{1}{|\vec{r}|} - \frac{1}{|\vec{r}+\vec{a}|}\right)
\end{equation*}

Für große Abstände von diesem Dipol, d.h. $\vec{r}\gg\vec{a}$ wollen wir das Potentialfeld \textsc{Taylor}-entwickeln, um besser mit ihm arbeiten zu können.\
Dazu betrachten wir den Term $\frac{1}{|\vec{r}+\vec{a}|}$ ein wenig genauer:

\begin{equation*}
\frac{1}{|\vec{r}+\vec{a}|} \cong \frac{1}{|\vec{r}|} + \left(\vec{a}\cdot\pdiff{}{\vec{r}}\right) \ \frac{1}{|\vec{r}|} = \frac{1}{|\vec{r}|}-\vec{a}\cdot\frac{\vec{r}}{|\vec{r}|^3}
\end{equation*}
\ \\
Damit gilt für das Potential:

\begin{equation*}
\varphi(\vec{r})=\frac{Q}{4\pi\epsilon_0}\left(\frac{1}{r}-\frac{1}{r}-\left(\vec{a}\cdot\pdiff{}{\vec{r}}\right)\frac{1}{r}\right) = \frac{\vec{p}\cdot\vec{r}}{4\pi\epsilon_o\cdot r^3}
\end{equation*}

und das $\vec{E}$-Feld:

\begin{align*}
\vec{E}(\vec{r}) &= - \nabla \varphi = \frac{1}{4\pi\epsilon_0}\ \nabla \left(\vec{p}\cdot\nabla\right)\ \frac{1}{r} = \frac{\vec{p}}{4\pi\epsilon_0}\ \underbrace{\left(\nabla \circ \nabla\right)\ \frac{1}{r}}_{(*)}\\
&= \frac{1}{4\pi\epsilon_0}\ \frac{3(\vec{p}\cdot\vec{r})\vec{r} - \vec{p}r^2}{r^5}\\
\ \\
\text{mit}\quad (*) &= \left(\pdiff{}{\vec{r}} \circ \pdiff{}{\vec{r}}\right)\frac{1}{|\vec{r}|} = - \pdiff{}{\vec{r}}\circ\frac{\vec{r}}{|\vec{r}|^3} = \frac{3\vec{r}\circ\vec{r}-\mathbbm{1}\cdot\vec{r}^2}{|\vec{r}|^5}
\end{align*}
\section{Fernfeld einer räumlich eingegrenzten Ladungsverteilung}

Wenn man das Fernfeld einer räumlich begrenzten Ladungsverteilung ermitteln möchte, spricht man in diesem Zusammenhang auch immer von der sogenannten \textbf{Multipolentwicklung}.

Wir betrachten nun eine räumlich eingegrenzte Ladunugsverteilung der Dichte $\rho$, für die zunächst einmal allgemein gilt:

\begin{equation*}
\varphi\left(\vec{r}\right) = \frac{1}{4\pi\epsilon_0}\cdot\int\d V \frac{\rho(\vec{r}'}{|\vec{r}-\vec{r}'|}
\end{equation*}

Unter der Annahme, dass $|\vec{r}| \gg \ a$ gilt (wobei $a$ die größte räumliche Ausdehnungsrichtung der Ladungsverteilung ist), werden wir nun den Term $\frac{1}{|\vec{r}-\vec{r}'|}$ entwickeln:

\begin{equation*}
\frac{1}{|\vec{r}-\vec{r}'|} = \sum_{n=0}^{\infty} \; \frac{1}{n!}\left(-\vec{r}'\cdot\pdiff{}{\vec{r}}\right)^n \ \frac{1}{r} = \frac{1}{r} + \frac{\vec{r}'\cdot\vec{r}}{r^3} + \frac{1}{2} \ \frac{3 (\vec{r}'\cdot\vec{r})^2 - \vec{r}'^2 \ \vec{r}^2}{r^5} + \dotsc
\end{equation*}

\begin{align*}
\Rightarrow \quad \varphi(\vec{r}) &= \frac{1}{4\pi\epsilon_0} \left[ \frac{1}{r}\int\d V' \rho(\vec{r}') + \frac{\vec{r}}{r^3} \int\d V' \vec{r}' \rho(\vec{r}') + \sum_{i,j} \ \frac{x_i x_j}{r^5} \ \int\d V' \rho(\vec{r}') \ (3x_i'x_j' - \delta_{ij}\vec{r}'^2) + \dotsc\right] \\
&= \frac{1}{4\pi\epsilon_0} \Big[\underbrace{\frac{Q}{r}}_{\sim\frac{1}{r}} \quad + \quad \underbrace{\frac{\vec{r}\cdot\vec{p}}{r^3}}_{\sim\frac{1}{r^2}} \quad + \quad \underbrace{\frac{1}{2}\frac{\vec{r}\cdot \tens{D} \cdot\vec{r}}{r^5}}_{\sim\frac{1}{r^3}} \quad + \quad \dotsc \quad \Big]
\end{align*}

Die einzelnen Summanden bezeichnet man auch als \textbf{Multipolmomente} einer Ladungsverteilung:

\begin{align*}
\mathrm{Monopol:}& \qquad Q = \int\d V \rho(\vec{r})\\
\mathrm{Dipol:}& \qquad \vec{p} = \int\d V \rho(\vec{r})\vec{r}\\
\mathrm{Quadrupol:}& \qquad \tens{D} = \int\d V \rho(\vec{r}) \ (3\vec{r}a\vec{r}-\mathbb{1}\vec{r}^2)\\
\mathrm{Oktupol:}& \qquad \dotsc\\
\vdots \qquad &
\end{align*}

Im Allgemeinen hängen die Multipolmomente vom Bezugspunkt ab, nur das erste nicht verschwindende Moment ist unabhängig vom selbigen.

Der Quadrupol-Tensor $\tens{D}$ hat dabei folgende Eigenschaften:

\begin{align*}
- \qquad & \bm{D}_{ij} = \bm{D}_{ji}\text{, insbesondere } \; \text{ Spur } \tens{D} = \sum_j \bm{D}_{jj} = \int\d V (3\vec{r}^2 - 3\vec{r}^2)=0 \\
- \qquad & \tens{D} \text{ hat 5 unabhängige Komponenten}\\
- \qquad & \tens{D} \text{ kann hauptachsentransformiert werden}\\
- \qquad & \text{Spur } \tens{D} =0 \text{ ist } \tens{D}=0 \text{ für Kugel und Kegel}
\end{align*}

\ \\

Aufgrund der der charakteristischen Richtungsabhängigkeit ist es sinnvoll, das Potential der Ladungsverteilung mit Kugelflächenfunktionen zu entwickeln. Ausgangspunkt ist hierbei wieder das allgemeine Potential für eine beliebige Ladungsverteilung: 

\begin{equation*}
\varphi(\vec{r}) = \frac{1}{4\pi\epsilon_0}\int\d V' \frac{\rho(\vec{r})}{|\vec{r}-\vec{r}'|} = \int\d V' G(\vec{r}-\vec{r}')\rho(\vec{r}')
\end{equation*}

Wobei $G$ die \textsc{Green}'sche Funktion ist, welche die \textsc{Poisson}-Gleichung mit $\delta$-förmiger Inhomogenität löst:

\begin{equation*}
-\epsilon_0 \ \bigtriangleup G(\vec{r}) = \delta(\vec{r})
\end{equation*}

Als nächstes separieren wir die Winkel- und Richtungsabhängigkeit des Differentialoperators $\bigtriangleup$, welches sich am besten explizit in Kugelkoordinaten vornehmen lässt.

\begin{equation*}
\bigtriangleup = \frac{1}{r} \pddiff{}{r}  \; +\; \underbrace{\frac{1}{r^2 \sin\theta} \pdiff{}{\theta} \sin\theta \; \pdiff{}{\theta} \; + \; \frac{1}{r^2 \sin^2\theta} \ \pddiff{}{\phi}}_{=: \frac{1}{r^2}\Lambda(\theta,\phi)}
\end{equation*}

Nun wenden wir auf die Differentialgleichung

\begin{equation*}
\Lambda \ Y(\theta,\phi) \; = \; -l(l+1) \ Y(\theta,\phi) \qquad l \in \mathbb{N}
\end{equation*}

den Separationsansatz $Y(\theta,\phi) = P(\theta)\cdot Q(\phi)$ an und erhalten zunächst für $Q(\phi)$:

\begin{align*}
\ddiff{}{\phi}Q(\phi) &= - m^2 Q(\phi)\\
\Rightarrow Q &= e^{im\phi} \qquad \text{ mit } \quad m \in [-l,l] \subset \mathbb{Z}
\end{align*}

Substituieren wir nun oben $\cos\theta = x$, so führt dies auf eine verallgemeinerte \textbf{\textsc{Legendre}-Differentialgleichung} für $P(x)$

\begin{equation*}
\left(\diff{}{x} \ \left(1+x^2\right) \ \diff{}{x} \ \left(-\frac{m^2}{1-x^2} + l(l+1)\right)\right) \ P_l^m(x) = 0
\end{equation*}

Es genügt diese für $m=1$ zu lösen, denn:

\begin{equation*}
P_l^m(x) = (-1)^m(1-x)^{\frac{m}{2}} \ \left(\diff{}{x}\right)^{|m|} P_l(x)
\end{equation*}

Somit bleibt nur noch folgende \textsc{Legendre}-Differentialgleichung übrig:

\begin{equation*}
(1-x^2) P_l'' \ - \ 2x \ P_l' \ + \ l(l+1) \ P_l \ = \ 0
\end{equation*}

Deren Lösungen $P_l$ sind sogenannte \textbf{\textsc{Legendre}-Polynome}:

\begin{equation*}
P_l(x) = \frac{1}{2^l l!} \ \left(\diff{}{x}\right)^l \ (x^2+1)^l \qquad l\in \mathbb{N}
\end{equation*}

(Die ersten $P_l$ lauten explizit: $P_0(x) =1, \; P_1(x) = x, \; P_2(x) = \frac{1}{2}(3x-1), \dotsc$)
Nun erhalten wir aus $P$ und $Q$ unsere ursprüngliche, separierte Funktion $Y(\theta,\phi)$:

\begin{equation*}
Y_{lm}(\theta,\phi) = \sqrt{\frac{2l + 1}{4\pi} \ \frac{(l-|m|)!}{(l+|m|)!}} \ P_l(\cos\theta) \ e^{im\phi}
\end{equation*}

(Die ersten $Y_{lm}$ lauten explizit: $Y_{00} = \frac{1}{\sqrt{4\pi}}, \; Y_{10} = \sqrt{\frac{3}{4\pi}}\cos\theta, \; Y_{1,\pm1} = \mp \sqrt{\frac{3}{8\pi}}\sin\theta \ e^{i\phi}$\
\\
\ \\
\underline{Bemerkung zu den $Y_{lm}$:}\
\\
\ \\
Die $Y_{lm}$ sind sogenannte \textbf{Kugelflächenfunktionen} und Lösungen der Differentialgleichung

\begin{align*}
\left(\frac{1}{\sin\theta}\pdiff{}{\theta} \ + \ \sin\theta \pdiff{}{\theta} \ + \ \frac{1}{\sin^2\theta} \pddiff{}{\phi} \ + \l(l+1)\right) \ Y_{lm}(\theta,\phi) = 0\\
l \in \mathbb{N}, m \in [-l,l]
\end{align*}

Sie stellen zudem eine Orthonormalbasis für alle Funktionen auf Kugeloberflächen dar. Dazu überprüfen wir zunächst die Orthogonalität der Basiselemente zueinander:

\begin{equation*}
\langle Y_{lm},Y_{l'm'}\rangle =: \int_{-1}^1\d(\cos\theta)\int_{0}^2\pi\d\phi \ Y^{*}_{lm}(\theta\phi) Y_{lm}(\theta\phi) = \delta_{ll'}\delta_{mm'}
\end{equation*}

Nach bekannter Vorgehensweise lässt sich nun jede beliebige Funktion $f$ auf einer Kugeloberfläche aus den $Y_{lm}$ darstellen:

\begin{align*}
f = (\theta,\phi) &= \sum_{l=0}^{\infty} \sum_{m=-l}^{l} f_{lm} Y_{lm}(\theta,\phi)\\
\text{mit } f_{lm} &= \langle Y_{lm},f\rangle = \int\d(\cos\theta) \int\d\phi \ Y^{*}_{lm}(\theta,\phi)f(\theta,\phi)
\end{align*}

Somit lässt sich auch mit ihnen die allgemeine Lösung der \textsc{Laplace}-Gleichung $\bigtriangleup\varphi=0$ darstellen: 

\begin{equation*}
\varphi(r,\theta,\phi) = \sum_{l=0}^{\infty}\sum_{m=-l}^{\infty} \left(A_l \cdot r^l + B_l \cdot r^{-l-1}\right) Y_{lm}(\theta,\phi)
\end{equation*}

Wir können nun zur Entwicklung von $\frac{1}{|\vec{r} - \vec{r}'|}$ zurückkehren:

\begin{align*}
\frac{1}{|\vec{r} - \vec{r}'|} = \sum_{l=0} \left(A_l \cdot r^l + B_l \cdot r^{-l-1}\right) P_l(\cos\gamma) \; \text{ mit } \gamma = \angle\left(\vec{r},\vec{r}'\right)\\
(\gamma\text{ ohne $\varphi$-Abhängigkeit wegen axialer Symmetrie)}
\end{align*}

Wähle nun für die $A_l, B_l$, dass $\vec{r}\parallel\vec{r}'$ ist und führe so die Entwicklung fort

\begin{align*}
\frac{1}{|\vec{r}-\vec{r}'|} &= \sum_{l=0}^{\infty} \ \frac{1}{l!} \left(\-r_{<}\diff{}{r_{<}}\right)^l \ \frac{1}{r_{>}} \qquad \text{ mit } r_{<} := \min\{r,r'\}, \ r_{>} \text{ analog}\\
&= \frac{1}{r_{<}}\sum_{l=0}^{\infty} \ \left(\frac{r_{>}}{r_{<}}\right)^l\\
&=  \sum_{l=0}^{\infty} \frac{r_{>}}{r_{<}^{l+1}} \ P_l (\cos\gamma)\\
\\
P_l (\cos\gamma) &= \frac{4\pi}{2l+1}\sum_{m=-l}^{l}  Y^{*}_{lm}(\theta',\varphi')Y_{lm}(\theta,\phi)\\
& \quad\left(\cos\gamma = \cos\theta\cos\theta' \ + \ \sin\theta\sin\theta'\cos(\phi-\phi')\right)\\
\\
\Rightarrow \frac{1}{|\vec{r}-\vec{r}'|} &= 4\pi \sum_{l=0}^{\infty}\sum_{m=-l}^{l} \frac{1}{2l+1}\ \frac{r_{>}^l}{r_{<}^{l+1}} Y^{*}_{lm}(\theta',\phi')Y_{lm}(\theta,\phi)
\end{align*}

Wir haben nun $\frac{1}{|\vec{r}-\vec{r}'|}$ vollständig faktorisiert und können nun das Potential einer Ladungsverteilung aufstellen:

\begin{equation*}
\varphi (\vec{r}) = \frac{1}{4\pi\epsilon_0} \sum_{l,m} \ \sqrt{\frac{4\pi}{2l +1}} \ \frac{Y_{lm}(\theta,\phi)}{r^{l+1}} \ \underbrace{\int\d V' \  \rho(\vec{r}) \  Y_{lm}^{*}(\theta',\phi') \ {r'}^l \sqrt{\frac{4\pi}{2l+1}}}_{q_{lm} \hat{=} \text{ Multipolmomente}}
\end{equation*}

Aus dem allgemeinen Ausdruck $q_{lm}$ für die Multipolmomente können wir nun auch die uns bereits bekannten Momente ableiten:

\begin{align*}
q_{00} &= \sqrt{4\pi}\int\d V' \ \rho(\vec{r}') \ Y_{00} = Q
\\
q_{10} &= \int\d V' \ \rho(\vec{r}') \ \underbrace{r'\cos\theta'}_{z'} = p_z\\
q_{1,\pm 1} &= \pm \frac{1}{\sqrt{2}} \ \int\d V' \ \rho(\vec{r}')  \ r'\sin\theta' \; e^{i\phi'} = (p_x \mp i p_y) \cdot \frac{1}{\sqrt{2}}\\
\\q_{2m} &\rightarrow \text{ 5 skalare Komponenten } \rightarrow \text{ Quadrupol}
\end{align*}
\section{Randbedingungen}

Die allgemeine Lösung der \textsc{Poisson}-Gleichung $\epsilon_0 \bigtriangleup \varphi = \rho$ hängt von ihren Randbedingungen ab. Die vollständige Lösung erhält man durch Addition der allgemeinen Lösung der zugehörigen homogenen Differentialgleichung und einer partikulären Lösung der inhomogenen Gleichung: $\quad \varphi = \varphi_p + \varphi_h$. Es bietet sich an die Randbedingungen in den homogenen Teil einzubauen (bisher haben wir immer angenommen, dass $\varphi (\infty) = 0$). Mathematisch liefert uns eine einzige Randbedingung auf einem geschlossenen Rand R eine physikalisch eindeutige Lösung für eine Differentialgleichung 2. Ordnung, da es sich durch den geschlossenen Rand effektiv um zwei Randbedingungen handelt. Wir unterscheiden dabei verschiedene gängige Varianten sich dem Problem zu nähern:


\begin{enumerate}
\item $\varphi(R)$ ist gegeben\
\\
Diese Variante nennt man auch die \textsc{Dirichlet}-Randbedingung\

\item $\pdiff{\varphi}{n}(R)$ ist gegeben\
\\
\ \\
Diese Variante nennt man auch die \textsc{von-Neumann}-Randbedingung\
\\
\ \\
($\pdiff{\varphi}{n} := \pdiff{\varphi}{\vec{r}}\cdot\vec{e}_n = - \vec{E}_n$ ist dabei die Normalenableitung)\

\item $\alpha \varphi \ + \ \beta\pdiff{\varphi}{n}$ ist gegeben
\end{enumerate}

\section{Leiter im elektrischen Feld}

Bis jetzt hatten wir in der Elektrostatik nur ruhende Ladungen betrachtet. In Leitern  gibt es allerdings bewegliche Ladungen im Inneren. Diese befinden sich im Gleichgewicht bei $\vec{F}=0 \ \Rightarrow \vec{E}= 0$
Daraus kann man dieser folgern, dass $\varphi =$ const. im Inneren des Leiters und auf der Leiteroberfläche gilt. Dafür muss gelten, dass $\rho = 0$ im Leiterinneren ist. Außerdem folgt direkt, dass $\vec{E} = -\pdiff{\varphi}{\vec{r}}$ senkrecht zur Oberfläche stehen muss und das es ausschließlich von \textbf{Oberflächenladungen} erzeugt wird. Um diese zu definieren betrachten wir ein Volumen $\Delta V$ auf dem Leiteroberflächenstück $\Delta \vec{A}$, welches die Ladung $\Delta Q$ in sich trägt.

\begin{equation*}
\epsilon_0 \oiint_{\partial\Delta V}\d\vec{A}\cdot\vec{E} = \Delta Q \quad \Rightarrow \quad \epsilon_0 \ \Delta\vec{A}\cdot\vec{E}_n = \Delta Q
\end{equation*}

Darüber können wir uns die \textbf{Flächenladungsdichte} $\sigma$ definieren, um die Oberflächenladungen beschreiben zu können:

\begin{align*}
\sigma &:= \frac{\Delta Q}{\Delta A} = \epsilon_0 \ E_n\\
Q  &= \iint\d A \cdot \sigma
\end{align*}

Die Oberflächenladungen werden durch äußere elektrische Felder bestimmt und schirmen das Leiterinnere von diesen Feldern ab.\
\\
\ \\
Betrachten wir nun den Innenraum eines Hohlleiters. Hier gilt genau wie bei einem normalen Leiter, dass auf der Leiteroberfläche das Potential konstant ist. Zudem ist der Innenraum ladungsfrei, woraus folgt, dass auch dort $\varphi$ = const. gilt uns somit auch $\vec{E} = 0$. Dieses Prinzip ist auch als \textbf{\textsc{Faraday}'scher Käfig} bekannt.\
\\
Die Begründung für dieses Prizip kann man auch direkt aus den \textsc{Maxwell}-Gleichungen herleiten, denn es gilt div $\vec{E} = 0$ und rot $\vec{E} = 0$ im Inneren des Hohlleiters. Jede Feldlinie im Inneren müsste demzufolge auf dem Rand anfangen und enden. Für eine Integration entlang einer Feldlinie $\int\d \vec{r} \cdot \vec{E} = \Delta \varphi$ würde dies jedoch ein endliches $\Delta \varphi$ zwischen Anfangs- und Endpunkt liefern, welches im Widerspruch zu $\varphi$ = const. auf dem Rand stehen würde. Also muss $\vec{E} = 0$ im Inneren des Hohlleiters gelten.

\section{Beispiele}
\ \\
\underline{A)  Punktladung und ebene Leiterfläche}\
\\
\ \\
Wir betrachten eine Punktladung $Q$, welche sich im Abstand $a$ von einer ebenen Leiteroberfläche befindet. Letztere sei entlang der y-Achse unseres Koordinatensystems ausgerichtet, während sich $Q$ auf der x-Achse befindet.
Demzufolge erhalten wir die \textsc{Poisson}-Gleichung:

\begin{equation*}
-\epsilon_0 \ \bigtriangleup\varphi = Q \ \delta(\vec{r}-a\vec{e}_x)
\end{equation*}

mit der Randbedingung $\varphi(x=0) = 0$ auf der Leiteroberfläche.\
Wir wissen, dass die Feldlinien der Punktladung senkrecht auf die Leiteroberfläche aufkommen müssen. Daher können wir uns fragen, wie man eben jenes Feldlinienbild beschreiben könnte. Man erhält es durch das Einbringen einer zweiten, gedachten Ladung $-Q$ bei $-a\vec{e}_x$, sodass die gesamte Anordnung für x$>0$ das gesuchte Feldlinienbild ergibt. Die imaginäre Punktladung bei $-a\vec{e}_x$ nennt man \textbf{Spiegelladung}. Die Begründung für dieses Phänomen ist, das das Einbringen einer Leiteroberfläche in ein gegebenes Potential $\varphi (\vec{r})$ entlang einer Äquipotentialfläche das Feld außerhalb des Leiters nicht ändert. Dort gilt weiterhin $-\epsilon_0 \bigtriangleup\varphi = \rho$ unverändert und die Randbedingungen sind effektiv identisch zu der Gleichung, welche das Feldlinienbild mithilfe der Spiegelladung beschreibt. Diese lautet hier:

\begin{equation*}
- \epsilon_0\bigtriangleup\varphi = Q \left(\delta(\vec{r}-a\vec{e}_x) \ - \ \delta(\vec{r}+a\vec{e}_x)\right)
\end{equation*}

welche für das Potential liefert:

\begin{equation*}
\varphi = \frac{Q}{4\pi\epsilon_0} \left(\frac{1}{|\vec{r}-a\vec{e}_x|} \ - \ \frac{1}{|\vec{r}+a\vec{e}_x|}\right)
\end{equation*}
\ \\
\ \\

\underline{B) Kugeloberfläche in einem asymptotisch homogenen Feld}\
\\
\ \\
Wir betrachten eine leitende Kugel mit dem Radius $R$, welche sich im Ursprung des Koordinatensystems in einem homogenen elektrischen Feld $\vec{E}_0$ befindet. Für $|\vec{r}|>R$ gilt dementsprechend: $\bigtriangleup\varphi = 0$ mit den Randbedingunge $\varphi(|\vec{r}| = R) = \varphi_0 := 0$ und $\varphi(|\vec{r}|\rightarrow\infty) = -\vec{E}_0\cdot\vec{r}$ (Homogenität des Feldes). Aus Symmetrieüberlegungen erhalten wir außerdem, dass $\varphi(\vec{r},R,\vec{E}_0)$ linear in $\vec{E}_0$ sein muss.\
Dementsprechend wählen wir den Ansatz $\varphi = -\vec{E}_0 \cdot\vec{r} \ G(r,R)$ mit den resultierenden Randbedingungen $G(r=R)=0$ und $G(r\rightarrow\infty
) =1$, welcher nach Einsetzen in die \textsc{Laplace}-Gleichung folgende homogene DGL für $G$ liefert:

\begin{equation*}
\bigtriangleup\varphi = -\vec{E}_0\cdot\vec{r} \ \left(\frac{4}{r} \ \diff{G}{r} \ + \ \ddiff{G}{r}\right) = 0
\end{equation*}
\ \\
Diese lösen wir mit dem Ansatz $ G \sim r^n$:


\begin{align*}
& 4n + n(n+1) = 0\\
\Rightarrow & n_1 = 0, n_2=3\\
\Rightarrow & G = C_1 + \frac{C_2}{r^3}\\
\ \\
Rb.: & G(r\rightarrow\infty) = 1 \quad \Rightarrow \quad C_1 = 1\\
& G(r=R) = 0 \quad \Rightarrow \quad C_2 = -R^3 
\end{align*}

Also ergibt sich für $\varphi$:

\begin{equation*}
\varphi = \vec{E}_0 \ + \ \frac{\vec{p} \cdot\vec{r}}{4\pi\epsilon_0 r^3} \quad
 \text{ mit } \quad \frac{\vec{p}}{4\pi\epsilon_0} = \vec{E}_0 \cdot R^3
\end{equation*}

Das äußere elektrische Feld induziert also offensichtlich ein Dipolmoment in der Kugel, welches für den zusätzlichen Term in $\varphi$ verantwortlich ist.

\section{Mehrere Leiter}

Wir betrachten mehrere Leiter im Raum mit den Oberflächen $\mathcal{S}_i$. Erneut gilt die Tatsache, dass es keine Raumladungen gibt ($\bigtriangleup\varphi = 0$) und die Randbedingungen für die Leiteroberflächen ($\varphi = \varphi_i$ auf den $\mathcal{S}_i, \ \varphi_0 = 0$ wird willkürlich festgelegt).\
Nun gilt aufgrund der Linearität der \textsc{Maxwell}-Gleichungen für das Gesamtpotential:

\begin{equation*}
\varphi (\vec{r}) = \sum_k \ G_k(\vec{r}) \ \varphi_k
\end{equation*}

Die $G_k$ hängen dabei von der Geometrie der Leiteranordnung ab.\
Wenn wir nun die Quellen auf den $\mathcal{S}_i$ mit in die Betrachtung mit einbeziehen, erhalten wir: 

\begin{align*}
\sigma &= \epsilon_0 \vec{E}_n\big|_{\mathcal{S}_i}  = -\epsilon_0 \pdiff{\varphi}{n}\Bigg|_{\mathcal{S}_i} = - \epsilon_0 \sum_k \pdiff{G_K(\vec{r})}{n}\Bigg|_{\mathcal{S}_i} \ \varphi_k \\
\ \\
\Rightarrow Q_i &= \Oiint{\mathcal{S}_i}{}{A} \ \sigma = -\epsilon_0\sum_k\Oiint{\mathcal{S}_i}{}{\vec{A}}\  \cdot \ \pdiff{G_K(\vec{r})}{\vec{r}} \ \varphi_k\\
&=: \sum_k C_{ik} \varphi_k \quad \text{ mit } \quad C_{ik} = -\epsilon_0 \Oiint{\mathcal{S}_i}{}{\vec{A}} \ \cdot \ \pdiff{G_k}{\vec{r}}
\end{align*}


Die $C_{ik}$ nennen wir die \textbf{Kapazitätskoeffizienten}. Für sie gilt $C_{ik} = C_{ki}$. Speziell für zwei sich umschließende Leiter ($\hat{=}$ Kondensator) folgt daraus:

\begin{equation*}
Q_1 = C_{11}\varphi_1 \quad (\text{und } Q_0 = -Q_1) \quad \hat{=} \ Q=C \cdot U
\end{equation*}
\chapter{Stationäre Ströme}

\section{Grundgleichungen und Vektorpotential}

Wenn wir stationäre Ströme betrachten, dann gilt ebenso wie in der Elektrostatik, dass die Felder zeitunabhängig sind: $\dot{\vec{E}} = 0, \dot{\vec{B}}=0$. Außerdem ist div $\vec{j} = 0$.\
Da für das $\vec{B}$-Feld unter diesen Bedingungen gilt, dass rot $\vec{B} = \mu_0 \ \vec{j}$, ist es nicht möglich ein $\psi$ zu finden, sodass grad $\psi = \vec{B}$. Anstattdessen macht man sich die Quellenfreiheit eines Wirbelfelds zu nutze und führt ein Vektorpotential $\vec{A}(\vec{r})$ ein, sodass:

\begin{equation*}
\vec{B}(\vec{r}) = \text{rot } \vec{A}(\vec{r})
\end{equation*}

$\vec{B}$ bestimmt $\vec{A}$ bis auf Eichtransformation $\vec{A}\ \rightarrow \ \vec{A} \ + \ \text{grad } \vec{chi}$ eindeutig, da beide Vektorpotentiale das selbe $\vec{B}$-Feld liefern. Bei spezieller Wahl von $\chi$ spricht man von fixierter Eichung.\
Mit der Einführung von $\vec{A}$ folgt mit $\frac{1}{\mu_0}\ \text{rot } \vec{B} = \vec{j}$:

\begin{equation*}
\frac{1}{\mu_0}\ (\text{rot rot }\vec{A})=\vec{j} \quad \text{bzw.} \quad \text{grad div }\vec{A}\ - \ \Delta\vec{A}=\mu_0 \ \vec{j}
\end{equation*}

Es bietet sich an, die Eichung div $\vec{A} = 0$ (\textsc{Coulomb}-Eichung) zu wählen, sodass folgt:

\begin{equation*}
-\Delta\vec{A}=\mu_0 \ \vec{j}
\end{equation*}

Eine ähnliche Gleichung haben wir mit der \textsc{Poisson}-Gleichung $-\epsilon_0 \ \Delta\varphi = \rho$ in der Elektrostatik hergeleitet und diese mit $\varphi = \frac{1}{4\pi\epsilon_0}\Int{}{}{V'}\frac{\rho(\vec{r})}{|\vec{r}-\vec{r}'|}$ gelöst.\ \\
Analog erhalten wir auch die Lösung für $\vec{A}$:

\begin{align*}
\vec{A}(\vec{r}) = \frac{\mu_0}{4\pi}\Int{}{}{V'} \ \frac{\vec{j}(\vec{r})}{|\vec{r}-\vec{r}'|} \qquad \text{mit div }\vec{A} = 0 \\
\ \\
\vec{B}(\vec{r}) = \text{rot } \vec{A}(\vec{r}) = \frac{\mu_0}{4\pi}\ \Int {}{}{V'} \ \frac{\vec{j}(\vec{r})\times (\vec{r}-\vec{r}')}{|\vec{r}-\vec{r}'|^3}
\end{align*}

Die Kontrolle, ob div $\vec{A} =0$ ist, ergibt:

\begin{equation*}
\text{div }\vec{A}(\vec{r}) = \frac{\mu_0}{4\pi} \ \Int{}{}{V''}\ \frac{1}{r''} \ \underbrace{\text{div }\vec{j(\vec{r}+\vec{r}''})}_{=0}=0 \qquad \text{mit } \vec{r}'' = \vec{r}'-\vec{r}
\end{equation*}

\section{Leiterschleifen}

Wir betrachten einen Leiter der Dicke $d$ an der Position $\vec{r}'$. Für "dünne" Leiter, d.h. $d \ll |\vec{r}-\vec{r}'|$, kann man d$V' \vec{j}(\vec{r})$ vereinfachen zu: d$\vec{r}' \cdot I$, wobei das Längenelement d$\vec{r}'$ entlang des Leiters verlaufen soll.\
Für mehrere Leiter $\mathcal{L}_n$ folgt demnach für das Vektorpotential:

\begin{align*}
\vec{A}(\vec{r}) &= \frac{\mu_0}{4\pi}\ \sum_n \ I_n \ \int\limits_{\mathcal{L}_n} \frac{\mathrm{d}\vec{r}'}{|\vec{r}-\vec{r}'|}\\
\ \\
\overset{\vec{B}= \text{rot } \vec{A}}{\Rightarrow} \quad \vec{B}(\vec{r}) &= \frac{\mu_0}{4\pi} \ \sum_n \ I_n \int\limits_{\mathcal{L}_n} \frac{\mathrm{d}\vec{r}' \times (\vec{r}-\vec{r}')}{|\vec{r}-\vec{r}'|^3}
\end{align*}
\ \\
Diese Gleichung zur Bestimmung des $\vec{B}$-Feldes einer beliebigen Anordnung dünner Leiter nennt man das \textsc{Biot-Savart}-Gesetz.\
\\
\ \\
Betrachten wir nun bei mehreren geschlossenen Leiterschleifen den magnetischen Fluss auf deren Oberflächen $\mathcal{F}_m$:

\begin{equation*}
\Phi = \Iint{\mathcal{F}_m}{}{\vec{A}_{\mathcal{F}_m}}\cdot\vec{B} = \Oint{\partial\mathcal{F}_m}{}{\vec{r}}\cdot\vec{A}
\end{equation*}

Mit \textsc{Biot-Savart} ergibt sich:

\begin{equation*}
\Phi = \frac{\mu_0}{4\pi}\ \sum_n \ I_n \ \Oint{\partial\mathcal{F}_m}{}{\vec{r}}\ \cdot \ \Oint{\partial\mathcal{F}_n}{}{\vec{r}'}\frac{1}{|\vec{r}-\vec{r}'|} \ =: \ \sum_n \ L_{mn} \ I_n
\end{equation*}

\begin{equation*}
\text{mit } L_{mn} = \frac{\mu_0}{4\pi} \ \Oint{\partial\mathcal{F}_m}{}{\vec{r}}\ \Oint{\partial\mathcal{F}_n}{}{\vec{r}'}\frac{1}{|\vec{r}-\vec{r}'|}
\end{equation*}

Die $L_{mn}$ sind die sogenannten \textbf{Induktionskoeffizienten}, welche ebenso wie die Kapazitätskoeefizienten symmetrisch sind: $L_{mn} = L_{nm}$. Für $m=n$ redet man von \textbf{Selbstinduktivitäten} der Leiter, welche allerdings  nicht mit der obigen Formel berechnet werden können, da dann die Näherung der "dünnen" Leiter zusammenbricht.

\section{Magnetischer Dipol}

Für eine geschlossene Leiterschleife der Fläche $\vec{A}_F$, durch die der Ringstrom $I$ fließt, definieren wir das \textbf{magnetische Dipolmoment} $\vec{m}$ wie folgt:

\begin{equation*}
\vec{m} \ := \ I \ \cdot \ \vec{A}_F
\end{equation*}
\begin{equation*}
\text{Dipollimit: } \quad |\vec{A}_F| \ \rightarrow \ 0, \ I \ \rightarrow \ \infty \ \Rightarrow \ |\vec{m}| \ = \ \text{const.}
\end{equation*}

Um das Vektorpotential

\begin{equation*}
\vec{A}(\vec{r}) = \frac{\mu_0 I}{4\pi}\oint\frac{\mathrm{d}\vec{r}'}{|\vec{r}-\vec{r}'|}
\end{equation*}

für diesen Dipol zu berechnen entwickeln wir dieses unter der Näherung großer Abstände zum Dipol( $r\gg a$, wobei $a$ die größte Ausdehnung des Dipols in eine Raumrichtung ist):

\begin{equation*}
\frac{1}{|\vec{r}-\vec{r}'|} \cong  \frac{1}{|\vec{r}|} + \frac{\vec{r}\cdot\vec{r}'}{|\vec{r}|^3} + \dotsc \quad \Rightarrow \quad \oint\frac{\mathrm{d}\vec{r}'}{|\vec{r}-\vec{r}'|} = \underbrace{\frac{1}{r} \Oint{}{}{\vec{r}'}}_{=0} \ + \ \frac{\vec{r}}{r^2}\Oint{}{}{\vec{r}} \circ\vec{r}' \ + \dotsc
\end{equation*}

Umformen ergibt:

\begin{align*}
\Oint{}{}{\vec{r}'} (\vec{r}'\cdot\vec{a}) &= \frac{1}{2}\Bigg[\Oint{}{}{\vec{r}'}(\vec{r}'\cdot\vec{a}) \ - \oint(\mathrm{d}\vec{r}'\cdot\vec{a})\vec{r}'\Bigg] \ + \ \frac{1}{2} \Bigg[\Oint{}{}{\vec{r}'}(\vec{r}'\cdot\vec{a}) \ + \ \oint(\mathrm{d}\vec{r}'\cdot\vec{a})\vec{r}'\Bigg]\\
&= \underbrace{\frac{1}{2}\oint(\vec{r}'\times\mathrm{d}\vec{r}')}_{\text{Fläche }A_F} \times\vec{a} \ + \ \frac{1}{2} \oint\mathrm{d}\left[\vec{r}'(\vec{r}'\cdot\vec{a})\right]\\ 
\ \\
& \Rightarrow \quad \vec{A}(\vec{r}) = \frac{\mu_0 I }{4\pi} \; \vec{A}_F\times\frac{\vec{r}}{r^3} = \frac{\mu_0}{4\pi} \cdot \frac{\vec{m}\times\vec{r}}{r^3}
\end{align*}
\ \\
(zum Vergleich das Potential eines Elektrischen Dipols: $\varphi(\vec{r}) = \frac{1}{4\pi\epsilon_0} \cdot \frac{\vec{p}\cdot\vec{r}}{r^3}$)
\ \\
\begin{align*}
\vec{B}(\vec{r}) = \text{rot } \vec{A}(\vec{r}) &= - \frac{\mu_0 I}{4\pi} \ \nabla \times \left(A_F\times\nabla\frac{1}{r}\right) \ = \ \frac{\mu_0}{4\pi} \ \cdot \ 3(\vec{m}\cdot\vec{r})\vec{r} \ - \ \vec{m}r^2\\
\ \\
&= \ \underbrace{\vec{A}_F \ \Delta\frac{1}{r}}_{(*)} \quad - \quad \underbrace{(\vec{A}_F
\cdot\nabla)\ \nabla \ \frac{1}{r}}_{= \ \vec{A}_F (\nabla \circ \nabla) \frac{1}{r}}
\end{align*}

$(*) = 4\pi\delta(\vec{r})$ wird im Fernfeld vernachlässigt
\ \\
\ \\
\underline{Ladung auf Umlaufbahn}:$\qquad$ (Ladung $Q$, Masse $M$)

\begin{equation*}
I = \frac{Q}{\tau} \qquad \text{ mit } \qquad \tau \; \hat{=} \; \text{Umlaufzeit}
\end{equation*}
\begin{equation*}
\vec{A}_F = \frac{1}{2}\oint\vec{r}\times\mathrm{d}\vec{r} = \frac{1}{2}\Int{0}{\tau}{t} \ \left(\vec{r}\times\diff{\vec{r}}{t}\right) =\frac{1}{2}\tau \frac{\vec{L}}{M}
\end{equation*}

Das Magnetische Dipolmoment ist also bei einer Ladung auf einer Umlaufbahn eng mit dessen Drehimpuls $\vec{L}$ verknüpft:

\begin{equation*}
\vec{m}  \ =\ I \ \cdot \ \vec{A}_F \ = \ \frac{1}{2} \ \underbrace{\frac{Q}{M}}_{=: g_B} \ \vec{L}
\end{equation*}

(Zum Vergleich das \textsc{Bohr}'sche Magnetron: $\mu_b = \frac{e\hbar}{2m}$)
\ \\
\ \\
\ \\
\ \\
\underline{Allgemeine Stromverteilung}

\begin{equation*}
\vec{m} \ = \ \frac{1}{2} \ \Int{}{}{V} \; \vec{r}\ \times \ \vec{j}(\vec{r})
\end{equation*}

(Zum Vergleich der elektrische Dipol: $\vec{p} = \Int{}{}{V} \vec{r}\rho(\vec{r})$)
\chapter{Elektromagnetische Wellen}

\section{Wellengleichung}

Bisher haben wir in der Elektrostatik und in unserer Betrachtung von Stationären Strömen nur Fälle behandelt, bei denen galt:

\begin{equation*}
\rho(\vec{r})\neq 0, \vec{j}(\vec{r})\neq 0, \dot{\vec{E}}=0, \dot{\vec{B}}=0
\end{equation*}

Jetzt wollen wir elektromagnetische Wellen im Vakuum , also ohne Quellen, betrachten. Daher muss gelten: 

\begin{equation*}
\rho=0, \vec{j}=0,\dot{\vec{E}}\neq 0,\dot{\vec{B}}\neq 0
\end{equation*}

Daraus folgt zunächst für die \textsc{Maxwell}-Gleichungen:

\begin{align*}
\div \vec{E} &=0 \qquad \qquad\div \vec{B} = 0\\
\rot  \vec{E} &= -\dot{\vec{B}} \qquad\quad \ \rot \vec{B} = \epsilon_0\mu_0 \dot{\vec{E}}\\
\end{align*}

\begin{align*}
\Rightarrow & \rot\dot{\vec{B}} = \epsilon_0\mu_0\ddot{\vec{E}} = -\rot\rot\vec{E} = -\nabla\underbrace{(\nabla\cdot\vec{E})}_{\div\vec{E}=0} \ + \ \nabla^2 \vec{E}\\
\ \\
\overset{\epsilon_0\mu_0 = \frac{1}{c^2}}{\Rightarrow} &\qquad \left(\frac{1}{c^2} \ \pddiff{}{t} \ - \ \bigtriangleup\right)\vec{E} =: \Box\vec{E} = 0
\end{align*}

Die erhaltene partielle Differentialgleichung ist die sogenannte \textbf{Wellengleichung}, welche sich auch analog für das $\vec{B}$-Feld herleiten lässt. Das Symbol $\Box$ wird auch als \textbf{Wellen-} oder \textbf{\textsc{D'Alembert}-Operator} bezeichnet.


\section{Lösungen der Wellengleichungen}

$\Box U = 0$

\ \\
\begin{enumerate}\bfseries
\item \textbf{eindimensionale Lösung} ($\vec{r}\  \rightarrow \ x$)
\begin{equation*}
\left(\frac{1}{c^2} \ \pddiff{}{t} \ - \ - \pddif{}{x}\right)\ U(x,t) \ = \ 0 \ = \ \left(\frac{1}{c}\ \partial_t \ - \ \partial_x\right)\left(\frac{1}{c} \ \partial_t \ + \ \partial_x\right) \ U(x,t)
\end{equation*}

\end{enumerate}
\chapter{Energie- und Impulsbilanz des em. Feldes}

\section{Bilanzgleichungen}

Wir betrachten in einem Volumen $V$ die \textbf{Observable} $A$, für die wir auch gan allgemein eine Dichte definieren wollen:

\begin{equation}
A=\Int{V}{}{V} \ a =  \quad \Rightarrow \quad a := \diff{A}{V} \quad \text{ist die Dichte von } A
\end{equation}

Anschaulich kann man sagen, dass sich die zeitliche Änderung von $A$ in den Volumen aus seiner Erzeugungsrate $N_A$ und seinem Strom $I_A$ aus dem Volumen heraus zusammensetzt:

\begin{equation*}
\dot{A}(t) \ = \ - I_A \ + \ N_A
\end{equation*}

Analog zur Dichte $a$ von $A$ wollen wir nun auch für den Strom $I_A$ eine Stromdichte $\vec{j}_a$ durch die Oberfläche $\partial V$ und für die Erzeugungsrate $N_A$ eine Erzeugungsdichte $\nu_a$ im Volumen $V$ definieren, sodass gilt:

\begin{equation*}
\Int{V}{}{V} \ \partial_t \ a = -\Oiint{\partial V}{}{\vec{F}} \cdot\vec{j}_a \ + \ \Int{V}{}{V} \ \nu_a \ = \ \Int{V}{}{V} \left(- \ \div\vec{j}_a \ + \ \nu_a\right)
\end{equation*} 

Daraus folgt die \textbf{allgemeine Bilanzgleichung}:

\begin{equation*}
\dot{a} \ +  \div\vec{j}_a \  =  \ \nu_a
\end{equation*}

Falls $A$ eine Erhaltungsgröße ist, gilt:

\begin{equation*}
N_A = 0, \nu_a = 0 \quad \Rightarrow \quad \dot{a}  \ + \div \vec{j}_a \ = \ 0 \quad \Rightarrow \quad \dot{A} = -\Oiint{\partial V}{}{\vec{F}}\cdot\vec{j}_a
\end{equation*}

FÜr den Grenzfall, dass $V\rightarrow\infty$, folgt, dass $\dot{A}=0$ und somit $A =$ const., was das erwartete Verhalten einer ERhaltungsgröße widerspiegelt.

\section{Energiebilanz}

Auf eine Punktladung $Q$ wirkt die Kraft $\vec{F}_L = Q(\vec{v}\times\vec{B} \ + \ \vec{E})$ worüber man die Leistung des Feldes an der Ladung $N = \vec{F}\cdot\vec{v}$ ableiten kann.\
Für eine Energieänderung des em. Feldes gilt dememtsprechend:

\begin{equation*}
\dot{W}_{em} \ = \ -\vec{v}\cdot\vec{F}_L = -Q \ \cdot \ \vec{v} \ \vec{E}
\end{equation*}

Für eine Änderung der Energiedichte $\nu_{em}$ folgt daraus bei mehreren Ladungsträgerarten:

\begin{equation*}
\nu_{em} \ =  \ - \sum_i \ \rho_i \ \vec{v}_i \ \vec{E} \ = \ - \vec{j}\cdot\vec{E}
\end{equation*}

Damit lautet die Bilanzgleichung, welche in diesem Zusammenhang auch  \textbf{\textsc{Poynting}-Theorem} genannt wird:

\begin{equation*}
\pdiff{w}{t} \ + \ \div \vec{S}_P \ = \ \nu \ = \ -\vec{j} \cdot \vec{E}
\end{equation*}

wobei $w$ die Energiedichte und $\vec{S}_P$ die Energiestromdichte (auch \textbf{\textsc{Poynting}-Vektor} genannt) ist.\
$w$ und $\vec{S}_P$ hängen vom $\vec{E}$- und $\vec{B}$-Feld ab, also sind diese nach \textsc{Maxwell} zu bestimmen:

\begin{align*}
\nu \ &= -\vec{j}\cdot\vec{E} = \epsilon_0 \ \dot{\vec{E}} \ \vec{E} \ - \ \frac{1}{\mu_0} \left(\nabla\times\vec{B}\right) \cdot \vec{E}\\
&= \partial_t \left(\frac{\epsilon_0}{2} \vec{E}^2\right) \ - \frac{1}{\mu_0} \nabla \cdot (\vec{B}\times\vec{E}) \ - \ \frac{1}{\mu_0} \vec{B}\cdot \underbrace{(\nabla\times\vec{B})}_{= \dot{\vec{B}}}\\
&= \frac{1}{2}\partial_t \left(\epsilon_0\vec{E}^2 \ + \ \frac{1}{\mu_0}\vec{B}^2\right) \ - \ \frac{1}{\mu_0}\nabla \cdot (\vec{B}\times\vec{E})
\end{align*}

Der Vergleich mit dem \textsc{Poynting}-Theorem ergibt:

\begin{align*}
w \ &= \ \frac{1}{2} \left(\epsilon_0\vec{E}^2 \ + \ \frac{1}{\mu_0}\vec{B}^2\right)\\
\vec{S}_P \ &= \ \frac{1}{\mu_0} \ \vec{E}\times\vec{B} 
\end{align*}

\ \\
\ \\
\underline{Beispiel zur Erzeugungsdichte $\nu$:}$\qquad$ \textsc{Ohm}'sches Gesetz $\vec{j} = \sigma \cdot \vec{E}$

\begin{equation*}
\nu \ = \ - \sigma \ \cdot \ \vec{E}^2 \ = \ - \frac{\vec{j}^2}{\sigma}
\end{equation*}

Der erhaltene Ausdruck für die Erzeugungsdichte entspricht der \textbf{\textsc{Ohm}'schen Wärme}.

\section{Elektrostatische Feldenergie}

\begin{equation*}
W_e \ = \ \Int{}{}{V} \frac{\epsilon_0}{2} \vec{E}^2 \ = \ - \Int{}{}{V} \frac{\epsilon_0}{2} \vec{E}^2 \ \grad\varphi
\end{equation*}

Nutze zur Umformung partielle Integration mit dem Satz von \textsc{Gauss}:

\begin{align*}
\Rightarrow W_e &= \Int{}{}{V} \frac{\epsilon_0}{2} \ (\nabla \cdot \vec{E}) \varphi \ - \ \underbrace{\Oiint{}{}{\vec{A}} \cdot \frac{\epsilon_0}{2}\vec{E}\varphi}_{=0 \text{ im gesamten Raum}} \\
\\
&= \ \frac{1}{2} \ \Int{}{}{V} \varphi \cdot \rho \quad = \quad \frac{1}{2} \Int{}{}{Q} \cdot \varphi
\end{align*}

Dies entspricht auch der Anschauung, dass Energie = Ladung $\cdot$ Potential.\
Umschreiben ergibt:

\begin{equation*}
W_e \ = \ \frac{1}{2} \ \Int{}{}{V} \rho \cdot \varphi \ = \ \frac{1}{8\pi\epsilon_0} \ \int\d V\d V' \; \frac{\rho(\vec{r}) \ \rho(\vec{r}')}{|\vec{r}-\vec{r}'|}
\end{equation*}

\ \\
\ \\
Für eine Punktladung ergibt die erhaltene Gleichung: 

\begin{align*}
W_e \quad &= \quad \sum_{i\neq j} \ \frac{Q_i \ Q_j}{8\pi\epsilon_0 \ |\vec{r}-\vec{r}'|} \; + \; \text{\textbf{ Selbstenergie} für i = j}\\
&= \quad \sum_{i<j} \ \frac{Q_i \ Q_j}{8\pi\epsilon_0 \ |\vec{r}-\vec{r}'|} \; + \; \text{ Selbstenergie für i = j}\\
\end{align*}
\ \\
Für die Selbstenergie gilt zunächst für eine geladene Kugel mit dem Radus $a$: berechnen:

\begin{equation*}
W_e \ = \ \alpha \cdot \frac{Q^2}{8\pi\epsilon_0 \ a} \qquad \text{mit} \quad \alpha = \begin{cases}
	\frac{6}{5} \quad \text{für homogene Kugel}\\
	1 \quad \text{für Hohlkugel}
  \end{cases}	
\end{equation*}	

Wenn man nun für diese Kugel den Grenzübergang zu einer Punktladung machen möchte und $a$ gegen 0 gehen lässt, so erhält man als Ergebnis, dass die Selbstenergie einer Punktladung unendlich sein müsste. An dieser Stelle ist die klassische Elektrodynamik nicht anwendbar, da sie als Kontinuumstheorie an ihre Grenzen stößt. Für Selbstenergie von Elementarteilchen ist also eine Erweiterung der Theorie der Elektrodynamik, welche ausschließlich auf den \textsc{Maxwell}-Gleichungen beruht, vonnöten, so wie es in der Quantenelektrodynamik behandelt wird.

\section{Elektrostatische Energie einer Leiteranordnung}

Da wir eine feste Leiteranordnung betrachten, folgt daraus, dass es keine Raumladungen gibt, sondern diese an die Leiteroberflächen gebunden sind.

\begin{align*}
& \quad W_e \ = \ \frac{1}{2} \ \Oiint{}{}{A} \sigma\cdot\varphi \  =  \frac{1}{2} \sum_i \ \varphi_i \ Q_i\\
&\quad \text{wobei die }\varphi_i = \varphi \text{ auf den Leiteroberflächen konstant sind}
\end{align*}

\underline{Beispiel:} $\quad Q=Q_1=Q_2 \quad\Rightarrow\quad  W_e =\frac{1}{2}Q(\varphi_1-\varphi_2 ) = \frac{1}{2}QU = \frac{1}{2}CU^2 = \frac{1}{2}\frac{Q^2}{C}$\

\ \\
allgemein gilt: $Q_i \ = \ \sum_i \ C_{ik} \ \varphi_k$, sodass für die elektrostatische Energie folgt:


\begin{equation*}
\Rightarrow \quad W_e = \frac{1}{2} \sum_{ik} \ \varphi_i \ C_{ik} \ \varphi_k \ = 
\ \frac{1}{2} \sum_{ik} \ Q_i \ \tilde{C}_{ik} \ Q_k
\end{equation*}

Da $W_e$ aufgrund von $W_e = \int\d V \ \frac{\epsilon_0}{2} \vec{E}^2$ immer gößer oder gleich 0 ist, folgt daraus, dass die $C_{ik}$ bzw. $\tilde{C}_{ik}$ positiv definit sein müssen (insbesindere gilt sogar: $C_{ii} > 0$ und $\tilde{C}_{ii} > 0$)\\
\ \\

Wenn wir nun kleine Ladungsänderungen $\rho \rightarrow \rho +\d \rho, \varphi \rightarrow\varphi + \d\varphi$ betrachten erhalten wir:

\begin{align*}
\delta\varphi &= \ \frac{1}{4\pi\epsilon_0} \ \Int{}{}{V} \frac{\delta\rho(\vec{r})}{|\vec{r}-\vec{r}'|}\\
\delta W_e &= \ \frac{1}{2} \ \Int{}{}{V} (\d\rho \ \varphi \; + \; \rho \ \d\varphi) \ = \ \frac{1}{4\pi\epsilon_0} \cdot\frac{1}{2}\cdot 2 \ \int\d V \d V' \ \frac{\rho(\vec{r}) \ \delta\rho(\vec{r}')}{|\vec{r}-\vec{r}'|}
\end{align*}

\begin{align*}
\text{für Flächenladungen: } & \quad\delta W_e \ = \ \frac{1}{2}\Int{}{}{A}  \delta\sigma \ \varphi \ = \ \Int{}{}{A}  \sigma \ \delta\varphi\\
\text{für Leiter: } & \quad\delta W_e \ = \frac{1}{2}\sum_i \ \delta(Q_i\varphi_i) \ = \ \sum_i \varphi_i \delta Q_i \ = \ \sum_i \ Q_i \ \delta\varphi_i
\end{align*}

\ \\
\underline{Spezialfälle:}\\

\begin{enumerate}

\item Verschiebung von Ladungen entlang der Leiteroberfläche

\begin{equation*}
\delta Q_i = 0 \quad \Rightarrow \quad \delta W_e = 0
\end{equation*}

Da Verschiebung $\perp$ Kraft, ist auch die Arbeit 0.\\
Daraus folgt, dass $W_e$ im Gleichgewicht Extremum (i.A. Minimum) annimmt (\textbf{\textsc{Thompson}'scher Satz}).

\item Transport von Ladungen zwischen Leitern

\begin{equation*}
\delta Q_i \ \neq \ 0 \quad\Rightarrow\quad \delta W_e = \sum_i \ Q_i \ \delta\varphi_i \ = \ \sum_i \ \varphi_i \ \delta Q_i
\end{equation*} 

Beachte:

\begin{align*}
C_{ik} \ &= \ \frac{\partial^2 W_e}{\partial \varphi_i \ \partial\varphi_k} \qquad(\Rightarrow \ C_{ik} \ = \ C_{ki})\\
Q_i \ &= \ \pdiff{W_e}{\varphi_i} \ = \ \sum_k \ C_{ik} \varphi_k\\
\delta W_e \ &= \ \sum_i \ \pdiff{W_e(\varphi_k)}{\varphi_i} \ \delta\varphi_i \ = \ \d W_e \qquad (\text{totales Differential})
\end{align*}



\end{enumerate}

\section{Energie des stationären Magnetfelds}

\begin{align*}
W_m & \ \quad = \quad  \ \Int{}{}{V}\frac{1}{2\mu_0}\vec{B}^2 \ =  \ \Int{}{}{V} \frac{1}{2\mu_0} \vec{B}\times\rot\vec{A} \ = \ \Int{}{}{V} \frac{1}{2\mu_0} \ \left(\vec{B}\times\nabla\right) \overset{\downarrow}{\vec{A}}\\
& \overset{\text{part. Int.}}{=}  \  \Int{}{}{V} \frac{1}{2\mu_0} \vec{A} \cdot\left(\nabla\times\vec{B}\right) \quad + \quad \text{Oberflächenintegral} \left(\rightarrow 0 \text{ für } V \rightarrow \infty\right)\\
& \;\;\;\;\overset{\dot{\vec{E}}=0}{=} \ \frac{1}{2} \Int{}{}{V} \ \vec{j}\cdot\vec{A}
\end{align*}

Analog zum elektrostatischen Fall ergibt Umschreiben:

\begin{equation*}
W_m = \frac{\mu_0}{8\pi} \ \int\int\d V \d V' \ \frac{\vec{j}(\vec{r}) \ \vec{j}(\vec{r}')}{|\vec{r}-\vec{r}'|}
\end{equation*}
\ \\

Für dünne linienförmige und geschlossene Leiterschleifen $\mathcal{L}_i$ gilt mit $\int\d V\vec{j} \rightarrow \int\d \vec{r} \cdot I$ und unter Anwendung des Satzes von \textsc{Stokes}:

\begin{equation*}
W_m = \frac{1}{2} \cdot \sum_i I_i \Int{\mathcal{L}_i}{}{\vec{r}} \cdot \vec{A} \ = \ \frac{1}{2} \sum_i I_i \Phi_i
\end{equation*}
\ \\
\ \\
Allgemein folgt somit aus $\Phi_i = \sum_k L_{ik} I_k$:

\begin{align*}
W_m \ &= \ \frac{1}{2} \ \sum_{ik}  I_i L_{ik} I_k \ = \ \frac{1}{2} \ \sum_{ik} \Phi_i \tilde{L}_{ik} \Phi_k\\
&= \ \frac{1}{2} \ \sum_{i\neq k} I_i I_k \ \underbrace{\frac{\mu_0}{4\pi} \ \Int{\mathcal{L}_i}{}{\vec{r}} \Int{\mathcal{L}_k}{}{\vec{r}'} \ \frac{1}{|\vec{r}-\vec{r}'|}}_{L_{ik}} \; + \; \text{Selbstenergie für } (i=k)
\end{align*}

Ähnlich wie im elektrostatischen Analogon stößt die klassische Elektrodynamik bei der Berechnung der Selbstenergien für ``dünne'' und somit sonst ideale Leiter an ihre Grenzen. Für eine Leiter schleife endlicher Dicke kann man die Selbstenergie jedoch wieder berechnen, sie beträgt:

\begin{align*}
& \ W_m  \ = \ \frac{1}{2} \ \cdot  \ L \ \cdot \ I^2 \\
\text{mit } \ & \ L \ = \ \frac{\mu_0}{4\pi I^2}\ \Int{}{}{V'} \frac{\vec{j}(\vec{r})\vec{j}(\vec{r}')}{|\vec{r}-\vec{r}'|} \ = \ \frac{1}{I^2} \ \Int{}{}{V} \frac{\vec{B}^2}{\mu_0} \quad
 \left( = \ \frac{\Phi}{I}\right) 
\end{align*}

\section{Beispiele für Energiestromdichten}

\begin{enumerate}[label=\roman*]
\item \underline{Stromdurchflossener gerader Leiter}\\
\ \\
\begin{equation*}
\frac{1}{\mu_0} \rot \vec{B} \ = \ \vec{j} \ + \ \epsilon_0 \dot{\vec{E}}
\end{equation*}

Für $\dot{\vec{E}}=0$ und der integralen Formulierung $\Oint{}{}{\vec{r}} \cdot \vec{B} = \mu_0 I$ ebenjener \textsc{Maxwell}-Gleichung folgt, dass um den geraden Leiter ein tangentiales $\vec{B}$-Feld existiert:

\begin{equation*}
B \ = \ \frac{\mu_0 \ I}{2\pi \ r_{\perp}}
\end{equation*}

Mit dem \textsc{Ohm}'schen Gesetz $\vec{E} = \sigma \cdot\vec{j}$ folgt, dass das $\vec{E}$-Feld entlang des Leiters gerichtet sein muss. Somit gilt für die Energiestromdichte $\vec{S}_P = \frac{1}{\mu_0} \left(\vec{E}\times\vec{B}\right)$, dass sie radial nach innen gerichtet sein muss.\\
\ \\
Bei einem einfachen Stromkreis wird demnach die Energie nicht entlang der Leiter sondern über die erzeugten Feldern von der Spannungsquelle zum Verbraucher transportiert!\\
\ \\
Berechnet man nun außerdem das Flächenintegral über die Energiestromdichte, erhält man für den geraden Leiter:

\begin{equation*}
\Iint{}{}{\vec{A}_F}\cdot\vec{S}_P \ = \ 2\pi r_{\perp} l \ S_P \ = \ 2\pi r_{\perp} l \frac{1}{\mu_0}E\frac{\mu_0 I}{2\pi r_{\perp}} \ = \ l \cdot E \cdot I
\end{equation*}

Der erhaltene Ausdruck $N := l \cdot E \cdot I = U \cdot I$ ist somit anschaulich die abgestrahlte Energie pro Zeiteinheit und ist auch als \textbf{\textsc{Ohm}'scher Verlust} oder \textbf{\textsc{Ohm}'sche Wärme} bekannt.

\ \\

\item \underline{ideale parallele Doppelleiter mit entgegengesetzten Stromrichtungen}
\ \\

Hier betrachten wir gleich zu Beginn das Flächenintegral über der Energiestromdichte und setzen nur die Querschnittsfläche ein:

\begin{align*}
N \ &= \ \Int{}{}{\vec{A}_F} \cdot \vec{S}_P \ = \ \frac{1}{\mu_0} \left(\vec{E}\times\vec{B}\right) \ \overset{\text{stationär}}{=} \ - \frac{1}{\mu_0} \Int{}{}{\vec{A}_F} \cdot \left(\nabla \varphi \times \vec{B}\right)\\
&= \ - \frac{1}{\mu_0} \Int{}{}{\vec{A}_F} \cdot \left(\nabla\times\left(\varphi\vec{B}\right)\right) \; + \; blub
\end{align*}



\end{enumerate}
\chapter{Kraftwirkung auf Ladungen und Ströme}

Erinnerung: \textbf{Lorentzkraftdichte}

\begin{equation*}
\vec{f}_L \ = \ \rho \ \vec{E} \ + \ \vec{j}\times\vec{B}
\end{equation*}

\section{Elektrischer Dipol}

Wir betrachten einen elektrischen Dipol am Ort $\vec{r}$, dessen Ladungen den Abstand $\vec{a}$ voneinander haben. Die Kraft auf ihn beträgt:

\begin{equation*}
\vec{F} \ = \ Q \cdot \vec{E}\left(\vec{r} \ + \ \frac{\vec{a}}{2}\right) \ - \ Q \cdot\vec{E} \left(\vec{r} \ - \ \frac{\vec{a}}{2}\right) \qquad (=0 \text{ für $\vec{E}$ homogen})
\end{equation*}

Wir entwickeln diesen Ausdruck für das Dipollimit $|\vec{a}| \rightarrow 0$:

\begin{align*}
\vec{F}  \ &= \ Q \cdot \left( \vec{E} (\vec{r}) \ + \ \frac{1}{2} \left(\vec{a}\cdot\pdiff{}{\vec{r}}\right) \vec{E}(\vec{r}) \ - \ \vec{E}(\vec{r}) \ + \ \frac{1}{2} \left(\vec{a}\cdot\pdiff{}{\vec{r}}\right)\vec{E}(\vec{r})\right) \\
\vec{F} \ &= \ Q\cdot \left(\vec{a}\cdot\nabla\right) \ \vec{E}  \ = \ \left( \vec{p} \cdot \nabla\right) \ \vec{E}
\end{align*}

Den Ausdruck für das Drehmoment auf einen Dipol im elektrischen Feld erhalten wir analog:

\begin{align*}
\vec{M}  \ &= \ Q \cdot \left(\frac{\vec{a}}{2} \times \vec{E}\left(\vec{r} \ + \ \frac{a}{2}\right) \ + \ \frac{\vec{a}}{2} \times \vec{E}\left(\vec{r} \ - \ \frac{\vec{a}}{2}\right)\right)\\
\vec{M}  \ &= \ Q \cdot \vec{a}\times\vec{E}  \ = \ \vec{p}\times\vec{E} 
\end{align*}

\section{Magnetischer Dipol}

Wir betrachten einen Kreisstrom $I$, dessen Mittelpunkt sich am Ort $\vec{r}$ befindet und welcher die Fläche $\vec{A}_F$ umschließt und somit ein Dipolmoment von $\vec{m} = I \cdot \vec{A}_F$ erzeugt. Die Kraft auf diesen magnetischen Dipol beträgt: ($\vec{r}'$ ist dabei ein Ort auf dem Rand des Kreisstroms)

\begin{align*}
\vec{F} \ &= \ \Int{}{}{V'} \ \vec{j}(\vec{r}') \times \vec{B}  \ = \ I \ \Oint{}{}{\vec{r}'} \times\vec{B} \qquad (= 0 \text{ für homogenes Feld})\\
\vec{F} \ &= \ I \ \Oint{}{}{\vec{r}'} \times \vec{B}(\vec{r'} - \ \vec{r}'') \qquad\qquad \text{mit }\  \vec{r}''  = \ \vec{r}' - \ \vec{r} 
\end{align*}

Wir entwickeln für $|\vec{r}''| \ll |\vec{r}|$: 

\begin{align*}
\vec{F} \ &= \ I \ \Oint{}{}{\vec{r}'} \times \Bigg[ \underbrace{\vec{B}(\vec{r})}_{=0} \ + \ \left(\vec{r}'' \cdot\pdiff{}{\vec{r}}\right)\vec{B}(\vec{r})\Bigg]\\
&= \ I \ \underbrace{\Oint{}{}{\vec{r}'}\times\left(\vec{r}'' \cdot \pdiff{}{\vec{r}}\right)}_{(*)}\vec{B}(\vec{r}) 
\end{align*}

Lösung des Integrals $(*)$:

\begin{align*}
\Oint{}{}{\vec{r}'} (\vec{r}''\cdot \ \vec{a}) & \quad = \quad \ \oint{}{}{\vec{r}'} \cdot (\vec{r'}- \ \underbrace{\vec{r}}_{=0}) \vec{a}\\
& \overset{\text{Kap. 5.3}}{=} \ \frac{1}{2}\oint(\vec{r}'\times\d\vec{r}')\times\vec{a} \ + \ \frac{1}{2}\Oint{}{}{\vec{r}'}(\vec{r}'\cdot\vec{a})\\
& \quad = \quad \ \vec{A}_F \times\vec{a}
\end{align*}

$\Rightarrow\quad$ Einsetzen:

\begin{align*}
\vec{F}  \ &= \ \underbrace{I \ (\vec{A}_F}_{\vec{m}} \times \nabla)\times \vec{B} \ \overset{\text{bac-cab}}{=} \ \nabla(\vec{m}\cdot\overset{\downarrow}{\vec{B}}) \ - \ \vec{m}(\underbrace{\nabla\cdot\overset{\downarrow}{\vec{B}}}_{=0})\\
\vec{F}  \ &= \ \nabla(\vec{m}\cdot\vec{B}) \ \overset{\text{bac-cab}}{=} \ \vec{m} \times (\underbrace{\nabla\times\vec{B}}_{(**)}) \ + \ (\vec{m}\cdot\nabla)\vec{B}
\end{align*}

$(**)=0$, da $\dot{\vec{E}}=0$ und $\mu_0\vec{j}\rightarrow 0$ außerhalb der Quellen.\\
Somit :

\begin{equation*}
\vec{F}  \ = \  (\vec{m}\cdot\nabla) \vec{B} \qquad\qquad (\text{vgl. el. Dipol: } \vec{F}  \ = \ (\vec{p}\cdot\nabla)\vec{E})
\end{equation*}

Für das Drehmoment auf den magnetischen Dipol gilt:

\begin{align*}
\vec{M}  \ &= \ \Int{}{}{V'} \vec{r}\times\left[\vec{j}(\vec{r}') \times \vec{B}(\vec{r}')\right] \ = \  I  \ \Oint{}{}{\vec{r}''} \times \left[\d\vec{r}' \times\vec{B}(\vec{r}')\right]\\
 &= \ I \ \Oint{}{}{\vec{r}'}(\vec{r}''\cdot\vec{B}(\vec{r}')) \ - \ I \ \oint(\d\vec{r}'\cdot\vec{r}'')\vec{B}(\vec{r}')
\end{align*}

Näherung: $\vec{B}$ ist homogen. Auflösung der Integrale:

\begin{align*}
&\Oint{}{}{\vec{r}'} \cdot \vec{r}'' \ = \ \Oint{}{}{\vec{r}'}\cdot\vec{r}'  \ = \ \oint\frac{1}{2}\d(\vec{r}'^2)  \ = \ 0\\
&\Oint{}{}{\vec{r}'} (\vec{r}''\cdot\ \vec{B})  \ = \ \vec{A}_F \times\vec{B} \qquad \text{wie in } (*)
\end{align*}

\begin{equation*}
\Rightarrow \quad \vec{M}  \ = \ I\cdot \vec{A}_F \times \vec{B} \ = \ \vec{m}\times\vec{B} \qquad\qquad (\text{vgl. el. Dipol: } \vec{M}  \ = \ \vec{p}\times\vec{E})
\end{equation*}

\section{Multipolentwicklung der elektrischen Wechselwirkungsenergie}

Wir betrachten zwei Ladungsverteilungen mit den Dichten $\rho_1$ und $\rho_2$, welche sich im Abstand $l$ voneinander befinden. Die gemeinsame elektrische Feldenergie beträgt:

\begin{equation*}
W_e \ = \ \frac{1}{8\pi\epsilon_0}\int\d V \d V' \frac{\left(\rho_1(\vec{r}) \ + \ \rho
_2(\vec{r})\right)\left(\rho_1 (\vec{r}') \ + \ \rho_2(\vec{r}')\right)}{|\vec{r} \ - \ \vec{r}'|}
\end{equation*}

Betrachten wir nun die Wechselwirkungsenergie zwischen 1 und 2, wozu wir annehmen, dass $\rho_1$ ein "äußeres" Potential $\varphi_1$ erzeugt, welches mit $\rho_2$ wechselwirkt:

\begin{equation*}
W_{12} \ = \  \frac{1}{4\pi\epsilon_0} \ \int\d V \d V' \ \frac{\rho_1(\vec{r})\rho_2(\vec{r}')}{|\vec{r}-\vec{r}'|} \ = \ \Int{}{}{V} \rho_2\varphi_1
\end{equation*}

Wir benennen nun der Einfachheit halber $\varphi_1$ in $\varphi$ und $\rho_2$ in $\rho$ um und entwickeln nun das Potential:

\begin{align*}
\varphi(\vec{r})  \ &= \  \varphi(0) \ + \ \left.\left(\vec{r}\cdot\pdiff{}{\vec{r}}\right)\varphi\right|_0 \ + \ \frac{1}{2}\ \sum_{i,j} x_i x_j \ \left.\frac{\partial ^2}{\partial x_i  \partial x_j}\varphi\right|_0 \ + \ \ldots \\
\ \\
\Rightarrow \quad W \ &= \ \Int{}{}{V}\rho(\vec{r})\left[\varphi(0) \ + \ \left.\left(\vec{r}\cdot\pdiff{}{\vec{r}}\right)\varphi\right|_0 \ + \ \frac{1}{2} \sum_{i,j}x_i x_j \left.\frac{\partial^2}{\partial x_i \partial x_j}\varphi\right|_0 \ + \ \ldots\right]\\
&= \ Q\cdot\varphi(0) \ + \ \vec{p}\cdot\left.\pdiff{}{\vec{r}}\varphi\right|_0 \ + \ \frac{1}{2}\sum_{i,j}\left(D_{ij} \ + \ \delta_{ij}\Int{}{}{V}\rho\vec{r}\right)\left.\frac{\partial^2}{\partial x_i \partial x_i}\varphi\right|_0 \ + \ \ldots\\
\text{mit} \quad D_{ij} \ &= \ \Int{}{}{V} \rho \ \left(3x_i x_j\ - \ \delta_{ij}\vec{r}^2\right)
\end{align*}

Nach dem Umformen des Ausdrucks:

\begin{equation*}
\sum_{i,j} \delta_{ij}\frac{\partial^2 \ \varphi}{\partial x_i \partial x_j} \ = \ \laplace\varphi  \ = \ -\frac{\rho_1}{\epsilon_0}  \ = \ 0 \quad \text{bei } \vec{r}=0
\end{equation*}

erhalten wir als Ausdruck für die Wechselwirkungsenergie

\begin{equation*}
W \ = \ Q\cdot\varphi(\vec{r}) \ + \ \vec{p}\cdot\nabla\varphi(\vec{r})  \ = \ \frac{1}{6}\left(\nabla\cdot\tens{D}\cdot\nabla\right)\varphi(\vec{r}) \ + \ \ldots \qquad(\vec{r}\equiv \text{Ort von }\rho(\vec{r}))
\end{equation*}

\ \\
\textbf{Verschieben} dieser Ladungsverteilung von $\vec{r}$ nach $\vec{r} + \delta\vec{r}$ liefert uns den mulitpolentwickelten Ausdruck für die Kraft auf die Ladungsverteilung. Dabei bleibt allerdings der Bezugspunkt für $\vec{p},\tens{D}$ unverändert. Es gilt für die verrichtete Arbeit:

\begin{align*}
\delta A \ &= \ \vec{F} \cdot \delta\vec{r} \ = \ -\delta W \\
\Rightarrow \quad \vec{F}  \ &= \  -\pdiff{\varphi}{\vec{r}}  \ = \ -Q\ \pdiff{\varphi}{\vec{r}} \ - \ \pdiff{}{\vec{r}} \left(\vec{p}\cdot\pdiff{}{\vec{r}}\varphi\right) \ - \ \frac{1}{6} \ \left(\pdiff{}{\vec{r}}\cdot\tens{D}\cdot\pdiff{}{\vec{r}}\right)\pdiff{\varphi}{\vec{r}}\\
\Rightarrow \quad \vec{F} \ &= \ Q \ \vec{E} \ + \ \left(\vec{p}\cdot\nabla\right)\vec{E} \ + \ \frac{1}{6}\left(\nabla\cdot\tens{D}\cdot\nabla\right)\vec{E}
\end{align*}

\ \\
\textbf{Drehen} der Ladungsverteilung um $\vec{r}$ mit dem Winkel $\delta\vec{\alpha}$ liefert uns den multipolentwickelten Ausdruck für das Drehmoment auf die Ladungsverteilung. es gilt für die verrichtete Arbeit:

\begin{equation*}
\delta A  \ = \ \vec{M}\cdot\delta\vec{\alpha}  \ = \ - \delta W \quad\Rightarrow \quad \vec{M}  \ = \ - \pdiff{W}{\vec{\alpha}}
\end{equation*}

Es gilt außerdem:

\begin{equation*}
\delta Q  \ = \ 0 ; \quad \delta\vec{p}  \ = \  \delta\vec{\alpha}\times\vec{p}; \quad \delta\tens{D}  \ = \ \delta\vec{\alpha}\times\tens{D} \ - \ \tens{D}\times\delta\vec{\alpha}
\end{equation*}
\ \\

Somit gilt für $\delta W$ und schlussendlich für das Drehmoment:

\begin{align*}
\delta W  \ &= \  \delta\vec{p} \pdiff{\varphi}{\vec{r}} \ + \ \frac{1}{6}\left(\nabla\cdot\delta\tens{D}\cdot\nabla\right)\varphi \ + \ \ldots\\
&= \ - \left(\delta\vec{\alpha}\times\vec{p}\right)\vec{E} \ - \ \frac{1}{6}\nabla\left(\delta\vec{\alpha}\times\tens{D} \ - \ \tens{D} \times\delta\vec{\alpha}\right)\vec{E} \ +  \ \ldots\\
\Rightarrow\quad \vec{M} \ &= \ \vec{p}\times\vec{E} \ + \ \frac{1}{6}\left(\nabla\cdot\tens{D}\times\vec{E} \ - \ \nabla\times\tens{D}\cdot\vec{E}\right) \ + \ \ldots
\end{align*}

Achtung: Die analoge Prozedur für den magnetischen Fall (Kraft/Drehmoment aus Änderung der Feldenergie bestimmen) liefert ein falsches Vorzeichen! Dies hat seine Ursache in der zusätzlichen Energie aus der Spannungsquelle durch Induktion.
\chapter{Felder zeitabhängiger Ladungs- und Stromverteilungen}

Nun suchen nach allgemeinen Lösungen der \textsc{Maxwell}-Gleichungen:

\begin{align*}
\div \vec{B}  \ &= \ 0 \qquad ; \qquad \epsilon_0 \div\vec{E}  \ = \ \rho\\
\rot\vec{E} \ + \ \dot{\vec{B}} \ &= \ 0 \qquad ; \qquad \frac{1}{\mu_0}\rot\vec{B}\ - \ \epsilon_0\dot{\vec{E}}  \ = \ \vec{j}
\end{align*}

\section{Viererpotential}

Die Gleichung $\div \vec{B} = 0 $ wird erfüllt durch $\vec{B}=\rot\vec{A}$.\\
Die Gleichung $\rot\vec{E} \ + \ \dot{\vec{B}} = 0 \; \Rightarrow \; \rot (\vec{E} + \dot{\vec{A}}) =0$ wird erfüllt durch $\vec{E} + \dot{\vec{A}}= - \grad\varphi$\\
\ \\
Somit können alle Felder durch das \textbf{Viererpotential} $(\varphi,\vec{A})$ ausgedrückt werden, sodass im Endeffekt immer 4 skalare Felder bestimmt werden müssen:

\begin{align*}
\vec{B} \ &= \ \rot \vec{A}\\
\vec{E} \ &= \ - \grad \varphi - \dot{\vec{A}}
\end{align*}

Das Einsetzen in die \textsc{Maxwell}-Gleichungen und Ausnutzung des \textsc{d'Alembert}-Operators  $\Dalembert  =  \frac{1}{c^2}\pddiff{}{t} - \laplace$ liefert:

\begin{align*}
\div\vec{E}  \ &= \  \frac{\rho}{\epsilon_0} \qquad  & \rot \vec{B} \ - \ \epsilon_0\mu_0\vec{E}  \ = \ \mu_0\vec{j}\\
-\laplace\varphi \ - \ \div\dot{\vec{A}}  \ &= \ \frac{\rho}{\epsilon_0}	\qquad	& \rot\rot\vec{A} \ + \ \frac{1}{c^2}\grad\dot{\varphi} \ + \ \frac{1}{c^2} \ddot{\vec{A}}  \ = \ \mu_0\vec{j}\\
&& \nabla(\nabla\vec{A}) \ - \ \laplace\vec{A} \ + \ \frac{1}{c^2}\ddot{\vec{A}} \ + \ \frac{1}{c^2}\partial_t \ \nabla\varphi  \ = \ \mu_0\vec{j}\\
\ \\
\Dalembert\varphi  \ - \ \partial_t\left(\frac{1}{c^2}\partial_t\ \varphi \ + \ \nabla\vec{A}\right)  \ &= \ \frac{\rho}{\epsilon_0}  \qquad &
\Dalembert\vec{A} \ + \ \nabla\left(\frac{1}{c^2} \partial_t \ \varphi \ + \ \nabla\vec{A}\right) \ = \ \mu_0\vec{j}
\end{align*}

\ \\

Die Potentiale sind damit aber nicht eindeutig, sondern nur bis auf eine beliebige Eichung der Form $\vec{A}\rightarrow\vec{A}+\grad\chi$ und $\varphi \rightarrow\varphi-\partial_t\chi$ genau bestimmt. Eine \textbf{gleichwertige Umeichung} von $\vec{A}$ und $\varphi$ lässt die Felder unter solche einer Transformation invariant. Die Eichtransformation enthält genau eine skalare Funktion $\chi$, anders gesprochen eine skalare Bedingung. Für uns günstig ist die sogenannte \textbf{\textsc{Lorentz}-Eichung}, da sie die Felder invariant unter \textsc{Lorentz}-Transformation lässt und sie somit geeignet bleiben für relativistische Probleme. Die \textsc{Lorentz}-Transformation hat folgende Gestalt:

\begin{equation*}
\frac{1}{c^2} \ \pdiff{\varphi}{t} \ + \ \div \vec{A}  \ = \ 0
\end{equation*} 

Mit der \textsc{Lorentz-Eichung} erhält man als Gleichungen für die Potentiale:

\begin{equation*}
\Dalembert\varphi  \ = \ \frac{\rho}{\epsilon_0}	\qquad ; \qquad		\Dalembert\vec{A}  \ = \ \mu_0\vec{j} 
\end{equation*}

Hieran lässt sich auch einfach überprüfen, dass man die Gleichungen für die statischen Probleme leicht aus denen mit Zeitabhängigkeit erhalten kann mittels $\partial_t \rightarrow 0; \ \Dalembert \rightarrow \laplace$:

\begin{equation*}
\laplace\varphi  \ = \ \frac{\rho}{\epsilon_0}\text{ (s. Kap.4)}; \qquad	\laplace\vec{A} \ = \ -\mu_0\vec{j} \text{ (s. Kap.5)} 	
\end{equation*}

\ \\
\ \\

\underline{Wichtige Eichungen:}

\begin{enumerate}[label=\roman*)]

\item \textbf{\textsc{Lorentz}-Eichung}

\begin{equation*}
\frac{1}{c^2} \partial_t \ \varphi \ + \ \div\vec{A} \ = \ 0
\end{equation*}

Die \textsc{Lorentz}-Eichung fixiert die Potentiale nicht; eine Umeichung der Form $\Dalembert\chi=0$ ist immer noch möglich.

\item \textbf{\textsc{Coulomb}-Eichung}

\begin{align*}
\div\vec{A} \ = \ 0 \qquad\qquad \Rightarrow \qquad\qquad -\laplace\varphi \ &= \ \frac{\rho}{\epsilon_0}\\
\Dalembert\vec{A} \ &= \ \mu_0\vec{j} \  - \ \frac{1}{c^2}\ \frac{\partial^2 \ \varphi}{\partial t \partial \vec{r}} 
\end{align*}

Die \textsc{Coulomb}-Eichung ist hier dieselbe wie in der Elektrostatik plus entsprechende Korrekturen.

\item \textbf{Transversale Wellen}

\begin{align*}
\varphi \ = \ 0 \qquad\qquad\Rightarrow\qquad\qquad \frac{1}{c^2}\ddot{\vec{A}}\ + \ \rot\rot\vec{A} \ &= \ \mu_0\vec{j}\\
- \frac{\partial^2 \ \vec{A}}{\partial t \partial \vec{r}}  \ &= \ \frac{\rho}{\epsilon_0} 
\end{align*}
\end{enumerate}

\section{Retardierte Potentiale}

Wir haben nun eine inhomogene, lineare Differentialgleichung der Form $\Dalembert u \ = \ \xi$ vorliegen, zu deren Lösung wir die \textsc{Green}'sche Funktion $G(\vec{r},\vec{r}',t,t')$ heranziehen, welche die DGL $\;\Dalembert G = 4\pi \delta(\vec{r}-\vec{r}')\delta(t-t')$ löst.\\
Da $G$ translationsinvariant sein soll, kann es nur von $\vec{r}-\vec{r}'$ und $t-t'$ abhängen. Weiterhin erhalten wir aus der Rotationssymmetrie des Problems, dass $G$ nur von $|\vec{R}|$ abhängen kann.\\
Zur weiteren Lösung der DGL $\; \Dalembert G = 4\pi\delta(\vec{r}-\vec{r}')\delta(t-t')$ bilden wir nun ihre \textsc{Fourier}-Transformierte (s.Kap.2):

\begin{align*}
\left(\pddiff{}{\vec{R}} \ - \ \frac{1}{c^2}(i\omega)^2\right)G\left(\vec{R},\omega\right)  \ &= \ 4\pi\delta(\vec{R})\;\Bigg | \; \frac{\omega}{c} \ = \ k; \text{ benutze Kugelkoord.}\\
\ddiff{}{R}G_k(R) \ + \ k^2 G_k(R) \ &= \ 4\pi\delta(R) \ \Bigg |\cdot R \neq 0\\
\ddiff{}{R}(R \ G_k) \ + \ k^2 \cdot (R \ G_k) \ &= \ 0 \quad\qquad\text{homogene DGL}\\
\ \\
\text{Lsg.: } R \ G_k(R) \ &= \ A\cdot e^{ikR} \ + \ A\cdot e^{-ikR}
\end{align*}

Die Inhomogenität $\delta(\vec{R})$ ist daher sehr wichtig nahe $\vec{R}=0$. Dort ist $k \cdot R \ll 1$, wodurch $k^2 \cdot R \ G_k$ vernachlässigbar wird gegenüber $\ddiff{}{R}(R\cdot G_k)$. Dann reduziert sich die DGL auf:

\begin{equation*}
\laplace_R G_k(R)  \ = \ - 4 \pi \delta(\vec{R})
\end{equation*}

Im Grenzwert $\lim\limits_{kR \rightarrow 0}{G_k(R)}  \ = \ \frac{1}{R}$ ist die allgemeine Lösung für $G$ also:

\begin{equation*}
G_k  \ = \ A \cdot G_k^+(R) \ + \ B\cdot G_k^- (R), \quad G_k^{\pm}  \ = \ \frac{e^{\pm i k R}}{R},\quad A+B=1
\end{equation*}

Nun können wir $G_k^{\pm}(R)$ rücktransformieren zu $G^{\pm}(\vec{R},\tau)$:

\begin{equation*}
G^{\pm}(\vec{R},\tau)  \ = \  \frac{1}{2\pi} \Int{-\infty}{\infty}{\omega} \frac{e^{-\pm i\omega\tau}}{R}\cdot e^{-i\omega\tau}  \ = \ \frac{1}{R}\delta\left(\tau\mp \frac{R}{c}\right) \qquad \left(\text{mit }k=\frac{\omega}{c}\right)
\end{equation*}

Bezogen auf unser Anfangsproblem entspräche diese Lösung:

\begin{equation*}
G^{\pm}(\vec{r},t,\vec{r}',t')  \ = \ \frac{\delta\left(t' \ - \ \left(t \mp \frac{|\vec{r}-\vec{r}'|}{c}\right)\right)}{|\vec{r}-\vec{r}'|}
\end{equation*}

Der Unterschied zwischen $G^+$ und $G^-$ liegt in den Randbedingungen in der Zeit. Anschaulich beschreibt $G$ die Reaktion des Systems bei $(\vec{r},t)$ aufgrund einer Störung (Inhomogenität) bei $(\vec{r}',t')$. Um die Kausalität nicht zu verletzen, muss demzufolge $G(t<t')=0$ gelten. Dies ist erfüllt für die \textbf{retardierte \textsc{Green}'sche Funktion} $G^+$, da hier die Wirkung \underline{nach} der Ursache auftritt und sich mit Lichtgeschwindigkeit ausbreitet (Verzögerung $\tau = \frac{R}{c}$). $G^-$ nennt man auch die \textbf{avancierte \textsc{Green}'sche Funktion}, aber aus naheliegenden Gründen wird sie hier nicht weiter behandelt.\\
Die (kausale) Lösung unserer inhomogenen DGL vom Anfang $\Dalembert u = \xi$ lautet damit:

\begin{equation*}
u(\vec{r},t) \ = \ \underbrace{u_0(\vec{r},t)}_{\text{homogene Lsg.}} \ + \ \frac{1}{4\pi}\int\d V' \d t' G^+(\vec{r},t,\vec{r}',t')\xi (\vec{r}',t')
\end{equation*}

\ \\
\ \\

Diese Lösung können wir nun auf unsere DGLn zur Bestimmung des Viererpotentials $\Dalembert\varphi = \frac{\rho}{\epsilon_0}$ und $\Dalembert\vec{A}= \mu_0\vec{j}$ anwenden:\\
Für eine räumlich begrenzte Quellenverteilung und der Randbedingung, dass die Felder im Unendlichen gegen Null gehen, erhalten wir, wenn wir als homogene Lösungen $\varphi_0=0$ und $\vec{A}_0=0$ setzen, folgende allgemeine Lösung der \textsc{Maxwell}-Gleichungen:

\begin{align*}
\varphi(\vec{r},t)  \ &= \ \frac{1}{4\pi\epsilon_0} \ \Int{}{}{V'} \ \frac{\rho\left(\vec{r'}, t \ - \ \frac{|\vec{r}-\vec{r}'|}{c}\right)}{|\vec{r}-\vec{r}'|}\\
\ \\
\vec{A}(\vec{r},t)  \ &= \ \ \frac{\mu_0}{4\pi} \ \ \Int{}{}{V'} \ \frac{\vec{j}\left(\vec{r}',t \ - \ \frac{|\vec{r}-\vec{r}'|}{c}\right)}{|\vec{r}-\vec{r}'|} 
\end{align*}

Die obigen Gleichungen beschreiben \textbf{retardierte Potentiale}, welche folgendermaßen interpretiert werden können:\\
$\rho$ und $\vec{j}$ sind die Ursachen für die Wirkungen $\varphi$ und $\vec{A}$, welche allerdings eine Laufzeitverzögerung von $\frac{|\vec{r}-\vec{r}'|}{c}$ aufweisen.\\
\ \\

Die Überprüfung der gefundenen Lösung erfolgt leicht durch Einsetzen in $ \ \Dalembert\varphi=\frac{\rho}{\epsilon_0}$ und $\ \Dalembert\vec{A} = \mu_0\vec{j}$. Setzt man sie außerdem in die \textsc{Lorentz}-Eichung $\frac{1}{c^2}\partial_t \ \varphi + \div\vec{A} =0$ ein, so führt dieses auf die Kontinuitätsgleichung $\dot{\rho} + \div\vec{j}=0$.\\
\ \\
Bemerkung:\\
Auch die avancierte \textsc{Green}-Funktion $G^-$ erfüllt die inhomogenen Wellengleichungen $\ \Dalembert\varphi=\frac{\rho}{\epsilon_0}$ und $\ \Dalembert\vec{A} = \mu_0\vec{j}$. Dies liegt mathematisch daran, dass die Wellengleichungen $c$ quadratisch enthalten, das Potential aber nur linear. Da diese Lösung aber akausal ist und nur $G^+$ die Kausalität erhält, zeichnet ebenjene Wahl von $G^+$ die Richtung der Zeit aus.


\section{\textsc{Hertz}'scher Dipol}

Wir betrachten nun als konkretes Beispiel für eine zeitabhängige Quellenverteilung einen oszillierenden Dipol: zwei Ladungen $\pm Q$ befinden sich entlang einer Achse in variablen Abstand $\vec{a}(t)$ voneinander entfernt. Somit gilt für die Stromdichte $\vec{j} := \vec{J}\cdot\delta(\vec{r})$, dass $\vec{j}=\dot{\vec{a}}\cdot Q \cdot\delta(\vec{r}-\vec{r}_a)$ ist. Im Dipollimit $\vec{a}\rightarrow 0$ folgt somit $\vec{j}=\dot{\vec{p}}\cdot\delta(\vec{r})$.\\
Allgemein gilt somit: $\vec{J}(t) = \Int{}{}{V} \vec{j}(\vec{r},t) = \dot{\vec{p}}$, oder genauer:

\begin{align*}
\dot{\vec{p}} \ = \ &\Int{}{}{V}\vec{r}\dot{\rho} \ = \ -\Int{}{}{V}\vec{r}\div\vec{j} \ = \ \Int{}{}{V}\left(\vec{j}\cdot\nabla\right)\cdot\vec{r} \ + \ \underbrace{\text{Oberflächenintegral}}_{\rightarrow 0} \\
\overset{\nabla\circ\vec{r}=\mathbbm{1}}{=} \ &\Int{}{}{V}\vec{j}  \ = \ \vec{J}
\end{align*}

Nun wollen wir die  (abgestrahlten) Felder des oszillierenden Dipols berechnen, wozu wir zunächst die retardierten Potentiale aufstellen:

\begin{equation*}
\vec{A}(\vec{r},t) \ = \ \frac{\mu_0}{4\pi}\ \Int{}{}{V'} \ \frac{\delta(\vec{r})\vec{J}\left(t \ - \ \frac{|\vec{r}-\vec{r}'|}{c}\right)}{|\vec{r}-\vec{r}'|} \ = \ \frac{\mu_0}{4\pi}\ \frac{\dot{\vec{p}}\left(t\ - \ \frac{r}{c}\right)}{r}
\end{equation*}

$\varphi$ erhalten wir aus der Ladungsverteilung zu $\vec{j}$ und aus der \textsc{Lorentz}-Eichung:

\begin{equation*}
\varphi(\vec{r},t)  \ = \ -\frac{1}{4\pi\epsilon_0} \ \frac{\vec{r}}{r} \ \pdiff{}{r} \ \frac{\vec{p}\left(t\ - \ \frac{r}{c}\right)}{r} \ + \ \text{zeitunabhängiges Potential}
\end{equation*}

Jetzt können die Felder $\vec{B}=\rot\vec{A}$ und $\vec{E}=-\grad\varphi-\dot{\vec{A}}$ berechnet werden.\\
$\left(\text{Notationshinweis: }\vec{p}|_{\text{ret}} \text{ steht für }\vec{p}\left(t-\frac{r}{c}\right)\right):$\\

\begin{align*}
\vec{B} \ &= \ \frac{\vec{r}}{r}\times\pdiff{\vec{A}}{r} \ = \ - \frac{\mu_0}{4\pi} \ \frac{\vec{r}}{r}\times \left(\frac{\ddot{\vec{p}}}{c}\ + \ \frac{\dot{\vec{p}}}{r}\right)_{\text{ret}}\\
\ \\
\vec{E} \ &= \ - \grad\varphi - \dot{\vec{A}} \ = \ -\grad\left(\frac{1}{4\pi\epsilon_0} \ \frac{\vec{r}}{r} \left[\frac{\vec{p}}{r^2} \ + \ \frac{\dot{\vec{p}}}{cr}\right]_{\text{ret}}\right) \ - \ \dot{\vec{A}}\\
\ \\
&= \ -\frac{1}{4\pi\epsilon_0 \cdot r} \left(\frac{\dot{\vec{p}}}{c^2} \ - \ \frac{\left(\ddot{\vec{p}}\cdot\vec{r}\right)\vec{r}}{c^2 \ r^2} \ + \ \frac{\dot{\vec{p}}}{c \ r} \ - \ 3 \frac{\left(\dot{\vec{p}}\cdot\vec{r}\right)\vec{r}}{c \ r^3} \ + \ \frac{\vec{p}}{r^2} \ - \ 3 \frac{\left(\vec{p}\cdot\vec{r}\right)\vec{r}}{r^4}\right)_{\text{ret}}
\end{align*}

\ \\
\ \\
\underline{Spezialfälle:}
\begin{enumerate}[label=\roman*)]
\item $\partial_t  \ = \ 0 \quad\ \; \qquad$ statischer Dipol

\begin{align*}
\Rightarrow \qquad \vec{B}  \ &= \ 0\\
\vec{E} \ &= \ - \frac{1}{4\pi\epsilon_0 \ r^3}\left(\vec{p} \ - \ \frac{3(\vec{p}\cdot\vec{r})\cdot\vec{r}}{r^2}\right)
\end{align*}

\item $\vec{p} \ \sim \ e^{-i\omega t} \qquad$ harmonische Schwingung

\begin{align*}
\Rightarrow \qquad \partial_t \ &\rightarrow \ -i\omega\\
\frac{1}{c}\partial_t \ &\rightarrow \ - \frac{i\omega}{c}  \ = \ -i \frac{2\pi}{\lambda}  \ = \ -ik
\end{align*}

Für den Fall der harmonischen Schwingung von $\vec{p}$ wollen wir nun die Abstandsbhängigkeit der Feldbeiträge betrachten. Dazu \grqq sortieren\grqq   wir die Beiträge nach ihren Ordnungen $\sigma(.)$:

\begin{align*}
\vec{B} \ &\sim \ - \frac{\mu_0 \vec{r}}{4\pi \ r^2} \times \Bigg(\underbrace{\sigma\left(\frac{\dot{\vec{p}}}{\lambda}\right)}_{\text{Fernfeld}} \; + \; \underbrace{\sigma\left(\frac{\dot{\vec{p}}}{r}\right)}_{\text{Nahfeld}}\Bigg)\\
\vec{E} \ &\sim \ - \frac{1}{4\pi\epsilon_0} \cdot \left(\sigma\left(\frac{p}{\lambda^2}\right) \; + \; \sigma\left(\frac{r}{r \cdot \lambda}\right) \; + \; \sigma\left(\frac{p}{r^2}\right)\right)
\end{align*}

\ \\
\textbf{Nahfeld:} $\quad r \ll \lambda$

\begin{align*}
\vec{B}(\vec{r},t)  \ &= \ \frac{\mu_0}{4\pi\ r^2} \ \left(\dot{\vec{p}} \times \frac{\vec{r}}{r}\right)_{\text{ret}}\\
\vec{E}(\vec{r},t)  \ &= \ \frac{1}{4\pi\epsilon_0 \ r^2} \ \left(\frac{3(\vec{p}\cdot\vec{r})\vec{r}}{r^2} \ - \ \vec{p}\right)_{\text{ret}} 
\end{align*}

Für das Nahfeld verzichtet man häufig auf die Retardierung, da sie kaum ins Gewicht fällt:

\begin{equation*}
\vec{p} \ \sim \ e^{-i\omega\left(t-\frac{r}{c}\right)}  \ = \ e^{-i\omega t} \ \underbrace{\; e^{2\pi i \frac{r}{\lambda}}\;}_{1+2\pi i \frac{r}{\lambda}+\ldots \approx 1}
\end{equation*}

\ \\
\textbf{Fernfeld:} $\quad r \gg \lambda \qquad\left(\frac{\vec{r}}{r} \ = \ \vec{e}_r\right)$

\begin{align*}	
\vec{B}(\vec{r},t)  \ &= \ \frac{\mu_0}{4\pi \ r \ c} \left(\dot{\vec{p}}\times\vec{e}_r\right)_{\text{ret}}\\
\vec{E}(\vec{r},t)  \ &= \ \frac{\mu_0}{4\pi \ r} \left(\left(\ddot{\vec{p}}\cdot\vec{e}_r\right)\cdot\vec{e}_r \ - \ \ddot{\vec{p}}\right)_{\text{ret}}  \ = \ \frac{\mu_0}{4\pi \ r}\left(\ddot{\vec{p}}\times\vec{e}_r\right)_{\text{ret}}\times\vec{e}_r  \ = \ c\cdot\vec{B}\times\vec{e}_r
\end{align*}

Bei harmonisch schwingendem $\vec{p}$ breiten sich $\vec{E}$ und $\vec{B}$ also radial als transversale Welle aus:

\begin{equation*}
e^{-i\omega\left(t-\frac{r}{c}\right)}  \ = \ e^{i(kr-\omega t)} \qquad \text{mit } k  \ = \  \frac{\omega}{c}
\end{equation*}

Dabei fallen $|\vec{E}|$ und $|\vec{B}|$ nur mit $\frac{1}{r}$ ab!
\end{enumerate}


\section{Energieabstrahlung des \textsc{Hertz}'schen Dipols}

\textbf{Energiedichte:}

\begin{equation*}
w  \ = \ \frac{\epsilon_0}{2} \vec{E}^2  \ +  \ \frac{1}{2\mu_0} \vec{B}^2  \ = \ \frac{\vec{B^2}}{\mu_0}
\end{equation*}

\ \\
\textbf{Energiestromdichte:}

\begin{align*}
\vec{S}_P  \ &= \ \frac{1}{\mu_0} (\vec{E}\times\vec{B})  \ = \ \frac{c}{\mu_0} (\vec{B}\times\vec{e}_r\times\vec{B})  \ = \  \frac{c}{\mu_0}\Big(\vec{e}_r\vec{B}^2 \ - \ \vec{B}(\underbrace{\vec{n}\cdot\vec{B}}_{=0})\Big)  \ = \ c\cdot\vec{e}_r\cdot w\\
|\vec{S}_P|  \ &= \ \frac{\mu_0}{(4\pi)^2c}\cdot \frac{\left(\ddot{\vec{p}}\times\vec{e}_r\right)^2}{r^2} \ = \ \frac{\mu_0}{(4\pi)^2c}\cdot\frac{\ddot{\vec{p}}^2\sin^2\theta}{r^2} \qquad \left(\text{mit } \theta = \sphericalangle(\ddot{\vec{p}},\vec{S}_P)\right)
\end{align*}

Man sieht hieran leicht, dass senkrecht zur Dipolachse am stärksten ausgestrahlt wird (da auch $\ddot{\vec{p}}\perp \vec{a}$).

\ \\
\textbf{Frequenzabhängigkeit der abgestrahlten Leistung:}

\begin{equation*}
|\vec{S}_P|  \ \sim \ |\ddot{\vec{p}}|^2 \ \sim p^2 \ \omega^4
\end{equation*}

Diese Frequenzabhängigkeit ist charakteristisch für Dipolstrahlung.

\ \\
\textbf{Abgestrahlte Leistung:}

\begin{align*}
N  \ &= \ \Iint{}{}{\vec{A}_F} \cdot \vec{S}_P  \ = \ \Int{}{}{\Omega}r^2 \ \frac{\mu_0}{(4\pi)^2c}\cdot\frac{\ddot{\vec{p}}^2\sin^2\theta}{r^2}  \ = \ \frac{\mu_0\ddot{\vec{p}}^2}{(4\pi)^2c} \ \Int{}{}{\Omega}\sin^2\theta\\
&= \ \frac{\mu_0\ddot{\vec{p}}^2}{(4\pi)^2c} \Int{0}{2\pi}{\phi}\Int{-1}{1}{\cos\theta} \ (1 \ - \ \cos^2\theta) \ = \ \frac{\mu_0\ddot{\vec{p}}^2}{(4\pi)^2c} \cdot 2\pi \cdot \left(2-\frac{2}{3}\right) \ = \ \frac{2}{3} \ \frac{\mu_0}{4\pi c} \ddot{\vec{p}}^2_{\text{ret}}
\end{align*}

Wenn wir einen harmonisch oszillierenden Dipol betrachten, so gilt für $\ddot{\vec{p}}$:

\begin{align*}
\vec{p}  \ &= \ \vec{p}_0 \cos\omega t \qquad \Rightarrow \qquad \ddot{\vec{p}}  \ = \ \omega^4\vec{p}_0^2\cos^2\omega t\\
\Rightarrow \langle\ddot{\vec{p}}\rangle_T  \ &= \  \omega^4\vec{p}_0^2 \langle\cos^2\omega t\rangle_T  \ = \ \frac{1}{2} \omega^4 \vec{p}_0^2
\end{align*}

Damit gilt also für die (über eine Periode gemittelte) abgestrahlte Leistung:

\begin{equation*}
\langle N \rangle_T  \ = \  \frac{\mu_0 \ \omega^4\ \vec{p}_0^2}{12\pi \ c}
\end{equation*}

\section{Strahlungsfeld einer räumlich begrenzten Quellenverteilung}

Wir betrachten nun eine beliebige Quellenverteilung am Ort $\vec{r}'$ mit der maximalen räumlichen Ausdehnung $a$. Es gilt ganz allgemein für das Vektorpotential am Ort $\vec{r}$:

\begin{equation*}
\vec{A}(\vec{r},t) \ = \ \frac{\mu_0}{4\pi}\Int{}{}{V'} \ \frac{\vec{j}\left(\vec{r}',t\ - \ \frac{|\vec{r}-\vec{r}'|}{c}\right)}{|\vec{r}-\vec{r}'|}
\end{equation*}

Wir entwickeln nun den Ausdruck $\frac{1}{|\vec{r}-\vec{r}'|}$ für das Fernfeld $(r\gg q, |\vec{r}| \gg |\vec{r}'|)$:

\begin{align*}
\frac{1}{|\vec{r}-\vec{r}|}  \ &= \ \frac{1}{r} \ + \ \frac{\vec{r}\cdot\vec{r}'}{r^3} \ + \ \ldots \ \approx \ \frac{1}{r} \quad \text{(Dipolnäherung)}\\
|\vec{r}-\vec{r}'|  \ &= \ r \ - \ \frac{\vec{r}\cdot\vec{r}'}{r} \ + \ \ldots \ \approx \ r \ - \ \vec{e}_r \cdot \vec{r}'\\
\Rightarrow \quad \vec{A}(\vec{r},t)  \ &= \ \frac{\mu_0}{4\pi \ r} \underbrace{\Int{}{}{V'} \vec{j} \left(\vec{r}',t \ - \ \frac{r}{c} \ + \ \frac{\vec{e}_r\cdot\vec{r}'}{c}\right)}_{\equiv \dot{\vec{q}}\left(t \ - \ \frac{r}{c},\vec{e}_r\right)}
\end{align*}

Zum Vergleich: Das Vektorpotential für einen \textsc{Hertz}'schen Dipol ergab:

\begin{equation*}
\vec{A}  \ = \ \frac{\mu_0}{4\pi \ r} \ \dot{\vec{p}}\left(t \ - \ \frac{r}{c}\right)
\end{equation*}

Das $\vec{B}$-Feld erhalten wir aus dem eben gewonnenen Ausdruck für $\vec{A}$ durch $\vec{B}=\rot\vec{A}$:

\begin{align*}
\pdiff{}{\vec{r}}\left[\frac{1}{r} f \left(t-\frac{r}{c},\vec{e}_r\right)\right] \ &= \ \Bigg[\underbrace{-\frac{\vec{e}_r}{c}\partial_t}_{\sigma\left(\frac{1}{\lambda}\right)} \ - \ \underbrace{\frac{\vec{e}_r}{r}}_{\sigma\left(\frac{1}{r}\right)} \ + \ \underbrace{\pdiff{}{\vec{r}}\left(\vec{e}_r\cdot\pdiff{}{\vec{e}_r}\right)}_{\sigma\left(\frac{1}{r}\right)}\Bigg ] \left(\frac{1}{r}f\right) \ \approx \ - \frac{\vec{e}_r}{c} \partial_t \left(\frac{1}{r}f\right)\\
\ \\
\overset{r\gg\lambda}{\Rightarrow} \qquad \vec{B}  \ &= \ -\frac{\vec{e}_r}{c}\times\dot{\vec{A}}  \ = \  \frac{\mu_0}{4\pi \ c} \cdot \frac{\ddot{\vec{q}}\times\vec{e}_r}{r}
\end{align*}

Das $\vec{E}$-Feld erhalten wir aus der inhomogenen \textsc{Maxwell}-Gleichung: $\frac{1}{\mu_0}\rot\vec{B}= \vec{j} + \epsilon_0\dot{\vec{E}}$. Unter Beachtung, dass $\vec{j}=0$ außerhalb der Quellenverteilung ist, erhalten wir zunächst für $\dot{\vec{E}}$:

\begin{align*}
\dot{\vec{E}}  \ &= \ c^2 \ \rot\vec{B} \ \approx \ - \frac{\vec{e}_r}{c}\partial_t\times c^2 \vec{B}\\
\Rightarrow\qquad \vec{E} \ &= \ c\vec{B}\times\vec{e}_r  \ = \ \frac{\mu_0}{4\pi} \cdot \frac{\left(\ddot{\vec{q}}\times\vec{e}_r\right)\times\vec{e}_r}{r}
\end{align*}

Wir erhalten also wieder eine transversale Welle der Felder $\vec{E}$ und $\vec{B}$, welche beide mit $\frac{1}{r}$ abfallen. (Korrekturen aus höheren Termen fallen dabei schneller ab.)\\
Für die Energiestromdichte gilt damit:

\begin{equation*}
\vec{S}_P \ = \ \frac{1}{\mu_0} (\vec{E}\times\vec{B})  \ = \  \vec{e}_r \ \frac{\mu_0}{(4\pi)^2 c} \ \frac{\left(\ddot{\vec{q}}\times\vec{e}_r\right)^2}{r^2}
\end{equation*}

Der direkte Vergleich zwischen dem Fernfeld des \textsc{Hertz}'schen Dipols und dem Strahlungsfeld einer beliebigen Ladungsverteilung zeigt uns, dass mit $\dot{\vec{p}}\leftrightarrow\dot{\vec{q}}$ alle Fernfeldformeln identisch sind:

\begin{align*}
&\text{\textsc{Hertz}'scher Dipol:} \qquad\qquad\qquad\ \; \text{allgemein:}&\\
&\dot{\vec{p}}_{\text{ret}}  \ = \ \Int{}{}{V'} \vec{j}\left(\vec{r}',t-\frac{r}{c}\right) \qquad \qquad \dot{\vec{q}}  \ = \ \Int{}{}{V'}\vec{j}\left(\vec{r}',t-\frac{r}{c}+\frac{\vec{e}_r\cdot\vec{r}'}{c}\right)
\end{align*}

Den Ausdruck $\frac{\vec{e}_r\cdot\vec{r}'}{c}$ in $\dot{\vec{q}}$ kann man als die Laufzeit innerhalb der Quellen verstehen.

\chapter{Elektromagnetische Felder in Substanzen}

Bisher haben wir die mikroskopischen \textsc{Maxwell}-Gleichungen betrachtet, die im Vakuum auch auf makroskopischen Längenskalen ihre Gültigkeit behalten. Dabei haben wir angenommen, dass $\rho$ und $\vec{j}$ alle Quellen enthalten. \\
Um nun den Schritt zu Betrachtung von Feldern in Substanzen zu machen, müssen wir weitere Feldquellen (Polarisationsladungen, Abschirmströme) betrachten. Eine sehr ausführliche Diskussion dazu findet man im \emph{Jackson} Kapitel 7.

\section{Elektrische Polarisation}

Man stellt fest, dass sich die makroskopische Ladungsdichte 
\begin{equation*}
\rho = \rho_0 + \rho_p
\end{equation*}
aus der Dichte der freien Ladungen $\rho_0$ und der der sogenannten Polarisationsladungen $\rho_P$ zusammensetzt. Letztere werden durch ein äußeres Feld induziert und sind im Experiment allgemein nicht bekannt. \\
Aufgrund der Ladungserhaltung muss die Polarisationsladung im \emph{gesamten }Raum verschwinden.
\begin{equation*}
\rho_P\neq 0, \qquad \text{aber}\quad \int\mathrm{d}V \rho_P = Q_P = 0
\end{equation*}
Das ist insofern intuitiv, da ein äußeres Feld Ladungen voneinander trennen und somit lokal Dichteschwankungen hervorrufen, aber nie Ladungen vernichten kann. \\
Auch die Polarisationsladungen erfüllen die Kontinuitätsgleichung, die so zur Definition der Polarisationsstromdichte $\vec{j}_P$ dient.
\begin{equation*}
\dot{\rho}_P+\div\vec{j}_P=0
\end{equation*}
Wir erhalten so 
\begin{equation*}
\rho_P = -\nabla\underbrace{\int\limits_0^t\mathrm{d}t'\ \vec{j}_P}_{\vec{=:P}} + \underbrace{\rho_P(0)}_{=0}.
\end{equation*}
Wobei wir $\vec{P}$ als elektrische Polarisation bezeichnen wollen.
\begin{align*}
\vec{P} &=\int\vec{j}_P\mathrm{d}t\\
\rho_P &= -\div\vec{P}\\
Q_P &=-\int\mathrm{d}V\ \nabla\vec{P}=-\oiint\mathrm{d}\vec{A}_F\cdot\vec{P}=0
\end{align*}
Nun gilt aber außerdem
\begin{equation*}
\int\vec{j}_P\mathrm{d}V = \int\dot{\vec{P}} = \dot{\vec{p}}.
\end{equation*}
Man sieht, dass man die Polarisation auch als Dipoldichte auffassen kann. Das wollen wir anhand des Potentials einer Polarisationsladungsverteilung nachprüfen.
\begin{align*}
\varphi_P(\vec{r}) &=\frac{1}{4\pi\epsilon_0} \int\mathrm{d}V'\ \frac{\rho_P(\vec{r'})}{|\vec{r}-\vec{r}'|} = -\frac{1}{4\pi\epsilon_0}\int\mathrm{d}V'\ \frac{1}{|\vec{r}-\vec{r}'|}\nabla_{\vec{r}'}\vec{P}(\vec{r}') = \\
&=\frac{1}{4\pi\epsilon_0}\int\mathrm{d}V'\ \vec{P}(\vec{r}')\cdot\nabla_{\vec{r}'}\frac{1}{|\vec{r}-\vec{r}'|} =-\frac{1}{4\pi\epsilon_0}\int\mathrm{d}V' \vec{P}(\vec{r}')\nabla_{\vec{r}'}\frac{1}{|\vec{r}-\vec{r}'|}
\end{align*}
Vergleicht man das mit dem Potential eines Dipols
\begin{equation*}
\varphi(\vec{r})=-\frac{1}{4\pi\epsilon_0}\cdot\vec{p}\cdot\nabla\frac{1}{r},
\end{equation*}
so bestätigt sich unsere Auffassung. Makroskopisch entspricht die elektrische Polarisation
\begin{equation*}
\vec{P} = \diff{\vec{p}}{V}.
\end{equation*}
Auf atomarer Ebene sind die Dipolmomente natürlich nicht kontinuierlich, sondern diskret verteilt. In diesem Fall muss zur Summe übergegangen werden.
\begin{equation*}
\vec{P} = \frac{1}{|V|}\sum\limits_{i\in V}\vec{p}_i
\end{equation*}
Da wir nun die Polarisationsladungen durch die Dipoldichte ausdrücken können, wird die erste \textsc{Maxwell}-Gleichung zu
\begin{equation*}
\epsilon_0\div\vec{E}=\rho_0+\rho_p = \rho_0 - \div\vec{P}.
\end{equation*}
Nach umstellen erhalten wir die erste makroskopische \textsc{Maxwell}-Gleichung
\begin{align*}
\div\left(\epsilon_0\vec{E}+\vec{P}\right)&=\rho_0\\
\div\vec{D} &= \rho_0 ,
\end{align*}
wobei $\vec{D} = \epsilon_0\vec{E} +\vec{P}$ das elektrische Verschiebefeld ist, das allein von den freien Ladungen $\rho_0$ abhängt.

\section{Magnetisierung}

Im makroskopischen Fall setzt sich auch die Stromdichte aus mehreren Quellen zusammen.
\begin{equation*}
\vec{j} = \underbrace{\vec{j}_k+\vec{j}_l}_{\vec{j}_0} + \vec{j}_P + \vec{j}_M
\end{equation*}
Dabei ist $\vec{j}_k$ der Konvektionsstrom bewegter Teilchen, $\vec{j}_l$ der Leitungsstrom, $\vec{j}_P$ der Polarisationsstrom und $\vec{j}_M$ ein Strom ohne Ladungstrennung. \\
Die Kontinuitätsgleichung lautet
\begin{equation*}
\dot{\rho}+\div\vec{j} = \dot{\rho}_0 + \dot{\rho}_P + \div\vec{j}_0 + \div\vec{j}_P + \div\vec{j}_M = 0.
\end{equation*}
Da $\vec{j}_0$ und $\vec{j}_P$ jeweils separat eine Kontinuitätsgleichung erfüllen, muss $\div\vec{j}_P=0$ sein. Ein Magnetisierungsstrom hat also keine Ladungsträger. Da er divergenzfrei ist, können wir mit Einführung der Magnetisierung $\vec{M}$
\begin{equation*}
\vec{j}_M  = \rot\vec{M}
\end{equation*}
schreiben.\\
Das resultierende Vektorpotential ist dann
\begin{align*}
\vec{A}_M(\vec{r}) &= \frac{\mu_0}{4\pi}\int\mathrm{d}V'\  \frac{\vec{j}_M(\vec{r}')}{|\vec{r}-\vec{r}'|} = \frac{\mu_0}{4\pi}\int\frac{\mathrm{d}V'}{|\vec{r}-\vec{r}'|}\nabla_{\vec{r}'}\times\vec{M}(\vec{r}') = \\
&=-\frac{\mu_0}{4\pi}\int\mathrm{d}V'\ \vec{M}(\vec{a}')\times\nabla_{\vec{r}'}\frac{1}{|\vec{r}-\vec{r}'|}
\end{align*}
Vergleichen wir das nun wieder mit dem magnetischen Dipolpotential
\begin{equation*}
\vec{A}(\vec{r}) = -\frac{\mu_0}{4\pi}\vec{m}\times\nabla\frac{1}{r},
\end{equation*}
so können wir analog zum elektrischen Fall, die Magnetisierung als magnetische Dipoldichte
\begin{equation*}
\vec{M} = \diff{\vec{m}}{V}
\end{equation*}
auffassen. Daraus ergibt sich nun mit
\begin{equation*}
\frac{1}{\mu_0}\rot\vec{B}=\vec{j} + \epsilon_0\dot{\vec{E}} = \vec{j}_0 + \dot{\vec{P}} + \rot\vec{M} + \epsilon_0\dot{\vec{E}}
\end{equation*}
das \textsc{Ampére}sche Gesetz für Substanzen
\begin{align*}
\rot\left(\frac{\vec{B}}{\mu_0}-\vec{M}\right) & = \vec{j}_0 + \pdiff{}{t}\left(\epsilon_0\vec{E}+\vec{P}\right)\\
\rot\vec{H} &= \vec{j}_0 + \dot{\vec{D}}.
\end{align*}
Zusammengefasst erhalten wir so die makroskopischen \textsc{Maxwell}-Gleichungen
\begin{align*}
\div\vec{B} &=0 &\div\vec{D} &= \rho_0\\
\rot \vec{E} + \dot{\vec{B}} &= 0 &\rot \vec{H}-\dot{\vec{D}}&=0
\end{align*}
mit den Materialeigenschaften
\begin{align*}
\vec{D}&=\epsilon_0\vec{E}+\vec{P} &\vec{H} &= \frac{\vec{B}}{\mu_0} -\vec{M}.
\end{align*}

\section{Materialgesetze}

Sowohl Polarisation $\vec{P}$, Magnetisierung $\vec{M}$ als auch der Leitungsstrom $\vec{j}_l$ werden durch die Felder $\vec{E}$ und $\vec{B}$ hervorgerufen und sind im allgemeinen materialabhängig, insbesondere auch von Druck und Temperatur. Häufig erhält man jedoch einfache Gesetze, die abgebrochenen \textsc{Taylor}-Entwicklungen entsprechen, also für Felder gültig sind, die schwach gegen die interatomaren Kräfte sind.\\
Dazu wollen wir ein paar einfache Beispiele betrachten.

\begin{enumerate}
	\item[\textbf{a.}] \textbf{Ohmsches Gesetz}
	\begin{equation*}
	I = \vec{A}_F\cdot\vec{j} = A_F\sigma E = A_F\sigma\frac{U}{l} = \frac{U}{R}
	\end{equation*}
	\item[\textbf{b.}] \textbf{Verschiebefeld}
	\begin{equation*}
	\vec{D}=\epsilon_0\vec{E}+\vec{P}=(1+\chi_l)\epsilon_0\vec{E}=\epsilon_0\epsilon_r \vec{E}
	\end{equation*}
	\item[\textbf{b.}] \textbf{Magnetische Dipoldichte}
	\begin{equation*}
	\vec{H} = \frac{1}{\mu_0}\vec{B}-\vec{M} = (1-\chi_m)\frac{1}{\mu_0}\vec{B}  = \frac{1}{\mu_0\mu_r}\vec{B}
	\end{equation*} 
\end{enumerate}

So werden die \textsc{Maxwell}-Gleichungen mit \\\
{linearen} Materialgesetzen zu
\begin{align*}
\div\vec{B}&=0 &\div(\epsilon_0\epsilon_r\vec{E})&=\rho_0\\
\rot \vec{E}+\dot{\vec{B}}&=0 &\rot\frac{\vec{B}}{\mu_0}-\epsilon_0\epsilon_r\dot{\vec{E}}&=\vec{j}_0.
\end{align*}

Natürlich können $\mu=\mu_0\mu_r$ und $\epsilon=\epsilon_0\epsilon_r$ auch richtungsabhängig sein und müssen dann durch Tensoren $\tens{\mu}$ und $\tens{\epsilon}$ ausgedrückt werden.\\

Untersuchen wir nun, wie sich uns bereits aus dem Vakuum bekannte Größen in Substanzen verhalten.
\begin{enumerate}
		\item[\textbf{a.}] \textbf{Phasengeschwindigkeit}
		\begin{equation*}
		u_P=\frac{\omega}{k}=\frac{1}{\sqrt{\mu\epsilon}}=\frac{c}{\sqrt{\mu_r\epsilon_r}} =: \frac{c}{n}\quad\quad\text{mit}\ n=\sqrt{\mu_r\epsilon_r}
		\end{equation*}
		\item[\textbf{b.}] \textbf{Energiedichte}
		\begin{equation*}
		w=\frac{\epsilon}{2}E^2 + \frac{1}{2\mu}B^2 = \frac{1}{2}\left(\vec{D}\cdot\vec{E}+\vec{B}\cdot\vec{H}\right)
		\end{equation*}
		\item[\textbf{c.}] \textbf{Energiestromdichte}
		\begin{equation*}
		\vec{S}_P = \frac{1}{\mu}\vec{E}\times\vec{B} = \vec{E}\times\vec{H}
		\end{equation*}
		\item[\textbf{d.}] \textbf{Impulsdichte}
		\begin{equation*}
		\vec{g}=\epsilon\vec{E}\times\vec{B}=\vec{D}\times\vec{B}=\frac{u_P^2}{c^2}\vec{S}_P
		\end{equation*}
\end{enumerate}
Es ist zu beachten, dass die Ausdrücke, die auf die Hilfsfelder zurückgreifen nur für lineare Medien Richtigkeit haben.
\newpage
\section{Verhalten an Grenzflächen}

\begin{wrapfigure}[]{r}[0cm]{0cm}
	\raisebox{0pt}[\dimexpr\height-1\baselineskip\relax]{
		\colorbox{hgrey}{
			\begin{tikzpicture}
			\draw(0,0)-- (0,4) ;
			\draw[|-|] (2,0.5)--node[right]{$A_F / L$}(2,3.5);
			\draw[ultra thick] (-1,0.5)--(1,0.5);
			\draw[ultra thick] (-1,3.5)--(1,3.5);
			\draw[ultra thick] (-1,0.5)--(-1,3.5);
			\draw[ultra thick] (1,0.5)--(1,3.5);
			\draw[|-|] (-1,-0.5)--node[below]{$d\rightarrow 0$}(1,-0.5);
			\draw (-1.5,4.5) node{\textbf{1}};
			\draw (1.5,4.5) node{\textbf{2}};
			\draw[color=hgrey] (-2.3,4.5) node{f};
			\draw[->] (0,2)--(0.5,2) node[below]{$\vec{n}$};
			\draw[->] (-2,2)--(-2,2.5) node[left]{$\vec{t}$};
			\fill[pattern=north east lines] (0,0)--(0,4)--(-0.5,4)--(-0.5,0);
			\end{tikzpicture}
		}
	}
	\caption{Integrationsvolumen  bzw. Fläche}
\end{wrapfigure}
Wir wollen nun das Verhalten der Felder an Grenzflächen zwischen zwei Medien untersuchen. Dazu definieren wir uns ein unendlich flaches Integrationsvolumen, das den Übergang zwischen den beiden Substanzen einschließt.\\
Aus der Quellenfreiheit des $B$-Feldes folgt
\begin{equation*}
\oiint\mathrm{d}\vec{A}_F\cdot\vec{B}=0
\end{equation*}
Wir halten die Fläche $A_F$ konstant und da die Quellenfreiheit für alle Volumina gilt, muss 
\begin{equation*}
\vec{n}A_F(\vec{B}_2-\vec{B}_1) = A_F(B_{2n}-B_{1n}) = 0
\end{equation*}
sein. Daraus können wir ablesen, dass die Normalkomponente von $\vec{B}$ beim Grenzübergang zwischen zwei Medien stetig sein muss.\\

Die elektrische Polarisation $\vec{D}$ enthält als Quellen die freien Ladungen $Q_0$. Analog zu oben erhalten wir durch dieselben Überlegungen jedoch
\begin{equation*}
D_{2n}-D_{1n}=\frac{Q_0}{A_F}=\sigma_0.
\end{equation*}

$\vec{D}$ ist also nur stetig an Grenzflächen, wenn es keine Oberflächenladungen gibt.\\

Ähnlich zum obigen Integrationsvolumen verwenden wir nun eine Fläche mit fast verschwindender Breite $d\rightarrow 0$, und der Höhe $L$, die parallel zu $\vec{n}$ aus der Fläche zeigt - man stelle sich Abbildung 10.1 mit $L$ statt $A_F$ vor.\\

Wir nehmen an, dass sich das Magnetfeld über die Zeit nicht ändert ($\dot{\vec{B}}=0$), so erhalten wir aus $\rot \vec{E} = 0$
\begin{equation*}
\oint\mathrm{d}\vec{r}\cdot\vec{E} = 0 
\end{equation*}
und damit
\begin{equation*}
\vec{t}L(\vec{E}_2-\vec{E}_1)=L(E_{2t}-E_{1t})=0.
\end{equation*}
Die Tangentialkomponente von $\vec{E}$ ist also genau dann stetig, wenn $\vec{B}$ zeitunabhängig ist. \\

Nun gehen wir davon aus, dass sich die Dipoldichte zeitlich nicht ändert ($\dot{\vec{D}}=0$) und erhalten so analog aus $\rot \vec{H}=\vec{j}_0$
\begin{equation*}
L(H_{2t}-H_{1t})=I_{0,\parallel},
\end{equation*}
wobei $I_{0,\parallel}$ der Anteil des Stroms entlang der Grenzfläche ist. $\vec{H}$ ist also stetig, wenn an der Oberfläche keine Ströme fließen und $\vec{D}$ zeitunabhängig ist.\\
Für $\vec{B}$ ergibt sich unter Berücksichtigung der Magnetisierungsströme
\begin{equation*}
\frac{L}{\mu_0}(B_{2t}-B_{1t})=I_{0,\parallel}+I_{M,\parallel}.
\end{equation*}
Für stromfreie Grenzflächen gilt also
\begin{equation*}
\Delta B_{t} = \mu_0\frac{I_{M,\parallel}}{L}.
\end{equation*}

Dieselben Ergebnisse für $E_t$ hätte man auch aus der Überlegung heraus erzielt, dass im Falle statischer Felder ($\dot{\vec{D}}=0,\ \dot{\vec{B}}=0$) das Potential stetig sein muss. Es muss nämlich gelten
\begin{align*}
\left.\varphi_1(\vec{r})\right|_\textit{Grenzfl.} &= \left.\varphi_2(\vec{r})\right|_\textit{Grenzfl.}\\
\varphi_1(\vec{r}+\mathrm{d}\vec{r}) &=\varphi_2(\vec{r}+\mathrm{d}\vec{r})	\\
\Rightarrow
\pdiff{\varphi_1}{\vec{t}} &=\pdiff{\varphi_2}{\vec{t}}\\
E_{1t} &= E_{2t}
\end{align*}

\textbf{a.\ Plattenkondensator mit Dielektrikum}\\

\begin{wrapfigure}[]{l}[0cm]{0cm}
	\raisebox{0pt}[\dimexpr\height-1\baselineskip\relax]{
		\colorbox{hgrey}{
			\begin{tikzpicture}
			%Platten
			\draw[ultra thick] (0,0)--(0,3);
			\draw[ultra thick] (3,0)--(3,3);
			\draw (0,1.5)--(3,1.5);
			\fill[opacity=0.2,pattern=north east lines] (0,0)--(3,0)--(3,1.5)--(0,1.5);

			%Felder
			\draw[->](0,2)--node[above]{$\vec{E}_1$}(2.9,2);
			\draw[->](0,0.5)--node[above]{$\vec{E}_2$}(2.9,0.5);
			
			%Ladungen
			\draw (-0.5,0.5) node{+};
			\draw (-0.5,2) node{+};
			\draw (3.5,0.5) node{-};
			\draw (3.5,2) node{-};
			\end{tikzpicture}
		}
	}
	\caption{Kondensator unten gefüllt}
\end{wrapfigure}

Die Tangentialkomponente von $\vec{E}$ muss stetig sein, da sich $\vec{B}$ nicht ändert. Die Polarisation muss jedoch für $\epsilon_1\neq\epsilon_2$ in beiden Bereichen unterschiedlich groß sein.\\ \linebreak\linebreak\linebreak\linebreak\linebreak\linebreak\linebreak

\begin{wrapfigure}[]{r}[0cm]{0cm}
	\raisebox{0pt}[\dimexpr\height-1\baselineskip\relax]{
		\colorbox{hgrey}{
			\begin{tikzpicture}
			%Platten
			\draw[ultra thick] (0,0)--(0,3);
			\draw[ultra thick] (3,0)--(3,3);
			\draw (1.5,0)--(1.5,3);
			\fill[opacity=0.2,pattern=north east lines] (1.5,0)--(3,0)--(3,3)--(1.5,3);
			
			%Felder
			\draw[->](0,2)--node[above]{$\vec{E}_1$}(1.4,2);
			\draw[->](0,.5)--(1.4,.5);
			\draw[->](1.5,2)--node[above]{$\vec{E}_2$}(2.9,2);
			\draw[->](1.5,.5)--(2.9,.5);
			
			%Ladungen
			\draw (-0.5,0.5) node{+};
			\draw (-0.5,2) node{+};
			\draw (3.5,0.5) node{-};
			\draw (3.5,2) node{-};
			\end{tikzpicture}
		}
	}
	\caption{Kondensator rechts gefüllt}
\end{wrapfigure}
Da $D_n$ aufgrund der fehlenden Oberflächenladungen stetig  ist, müssen die E-Felder unterschiedlich groß sein.\\ \linebreak\linebreak\linebreak\linebreak\linebreak

\textbf{b.\ Grenzfläche schräg zu den Feldlinien}\\

\begin{wrapfigure}[10]{l}[0cm]{0cm}
	\raisebox{0pt}[\dimexpr\height-1\baselineskip\relax]{
		\colorbox{hgrey}{
			\begin{tikzpicture}
			\draw (-2,0)--(2,0);
			\draw[->] (-1,1)node[below left]{$\vec{E}_1$}--(0,0);
			\draw (-0.5,0.5) arc(135:168:1cm);
			\draw (-0.5,0.4) node[below]{$\alpha$};
			\draw[->] (0,0)--(0.5,-1)node[above right]{$\vec{E}_2$};
			\draw (0.25,-0.5) arc(-60:-22:1cm);
			\draw (0.35,-0.25) node{$\beta$};
			\end{tikzpicture}
		}
	}
	\caption{schräge Feldlinien}
\end{wrapfigure}
Aus der Stetigkeit der Tangetialkomponente von $\vec{E}$ und der Normalkomponente von $\vec{D}$ folgt
\begin{align*}
E_{1t}&=E_{2t} &\Rightarrow& & E_1\sin\alpha &=E_2\sin\beta\\
D_{1n}&=D_{2n} &\Rightarrow& &\epsilon_1E_1\cos\alpha &= \epsilon_2E_2\cos\beta.
\end{align*}
Diese beiden Gleichungen können wir dividieren und erhalten
\begin{equation*}
\frac{\tan\alpha}{\tan\beta}=\frac{\epsilon_1}{\epsilon_2}.
\end{equation*}

\section{Dielektrische Kugel im homogenen Feld}

Eine dielektrische Kugel mit Radius $a$ möge sich in einem homogenen elektrischen Feld $\vec{E}_0$ befinden. Wir legen den Koordinatenursprung in die Kugelmitte. 
\begin{align*}
\rot \vec{E} &=0 &\Rightarrow & &\vec{E}=-\grad\varphi\\
\div\vec{D} &=0 &\Rightarrow& &-\div(\epsilon\grad\varphi)=0
\end{align*}
Wenn das Potential an der Grenzfläche $r=a$ stetig ist, dann ist es auch $E_t$. Ebenso ist $\epsilon\pdiff{\varphi}{r}$ stetig, wenn $D_n$ stetig ist. Wir wählen deshalb ähnlich wie in Kapitel 4.8. den Ansatz
\begin{equation*}
\varphi(\vec{r})=-\vec{E}_0\cdot\vec{r}\cdot g(r,a,\epsilon_i,\epsilon_a).
\end{equation*}
Durch Separation erhalten wir
\begin{equation*}
\frac{4}{r}\diff{g}{r}+\ddiff{g}{r}=0
\end{equation*}
und schließlich
\begin{equation*}
g(r)=\left\lbrace\begin{matrix}
\alpha + \beta\frac{a^3}{r^3} \quad\text{für}\ r<a\\
\gamma + \delta\frac{a^3}{r^3}\quad\text{für}\ r>a
\end{matrix}\right. .
\end{equation*}
Aus den Randbedingungen, dass $g(r=0)$ regulär, $g(r\rightarrow a)=1$ und $\varphi$ stetig sein, muss folgt jeweils $\beta=1$, $\gamma=1$ und $\alpha=1+\delta$. Da
\begin{equation*}
\left.\pdiff{\varphi}{r}\right|_\textit{Kugeloberfl.} = -\pdiff{}{r}\left(\vec{E}_0\cdot\vec{n}\cdot r\cdot g(r)\right)
\end{equation*}
muss $D_n$ stetig sein. Das heißt also
\begin{equation*}
\epsilon_i \alpha = \epsilon_i(1+\delta)\ \stackrel{!}{=}\ \epsilon_a(1-2\delta).
\end{equation*}
Das können wir eindeutig nach $\alpha$ und $\delta$ auflösen und erhalten
\begin{align*}
\delta & = \frac{\epsilon_a-\epsilon_i}{\epsilon_i+2\epsilon_a}, &\alpha&=\frac{3\epsilon_a}{\epsilon_i+2\epsilon_a}.
\end{align*}
womit wir das Potential bestimmt hätten.
\begin{equation*}
\varphi(r)=\left\lbrace\begin{matrix}
-\vec{E}_0\cdot\vec{r}\ (1+\delta) \qquad\ \ \text{für}\ r<a\\
\ \ \ -\vec{E}_0\cdot\vec{r}  (1+\delta\frac{a^3}{r^3})\quad\text{für}\ r>a
\end{matrix}\right. 
\end{equation*}
Das Feld im inneren der Kugel ist
\begin{equation*}
\vec{E}_i = \vec{E}_0(1+\delta) = \vec{E}_0 + \Delta\vec{E}_i,
\end{equation*}
wobei
\begin{equation*}
\Delta\vec{E}_i = \frac{\epsilon_a-\epsilon_i}{\epsilon_i+2\epsilon_a}\vec{E}_0
\end{equation*}
ein entelektrisierendes Feld ist. Analog gilt
\begin{equation*}
\vec{D}_i = \epsilon_i\vec{E}_i = 3\frac{\epsilon_i}{\epsilon_i+2\epsilon_a}\vec{D}_0.
\end{equation*}
Der Außenraum der Kugel sieht natürlich bis auf das homogene äußere Feld aus, wie das Feld eines Dipols mit Moment
\begin{equation*}
\vec{p} = -4\pi\epsilon_0\vec{E}_0\delta a^3
\end{equation*}
aus. \\
Es gibt einige Spezialfälle zu betrachten. Im Fall, dass $\epsilon_i$ unendlich groß wird, verschwindet das Feld im Inneren der Kugel. Es wird $\delta=-1$ und das Potential im Außenraum zu
\begin{equation*}
\varphi_a(r)=\vec{E}_0\cdot\vec{r}\left(1-\frac{a^3}{r^3}\right).
\end{equation*}
Das sieht genauso aus, wie das Potential einer leitenden Kugel. Ein Leiter ist also nicht anderes als ein Dielektrikum mit $\epsilon\rightarrow\infty$.\\
Sollte $\epsilon_a=\epsilon_0$ sein (wie im Vakuum), so wird das Entelektrisierungsfeld
\begin{equation*}
\Delta\vec{E}_i=-\frac{\epsilon_i-\epsilon_0}{\epsilon_i+2\epsilon_0}\vec{E}_0
\end{equation*}
und die Polarisation
\begin{align*}
\vec{P}_i &= \vec{D}_i - \epsilon_0\vec{E}_i = -3\epsilon_0\Delta\vec{E}_i\\
\Rightarrow \Delta\vec{E}_i &= -\frac{1}{3\epsilon_0}\vec{P}_i.
\end{align*}
Den Faktor $\frac{1}{3}$ bezeichnet man als Entelektrisierungsfaktor. Dieser ist immer von der Geometrie abhängig. 

\section{Atomare Polarisierbarkeit und Suszeptibilität}

Wir haben bereits gesehen, dass das elektrische Feld lokal Dipolmoment induzieren kann.
\begin{equation*}
\vec{p}\left(\vec{E}_\textit{lokal}\right) = \alpha\epsilon_0\vec{E}_\textit{lokal}
\end{equation*}
Man nennt $\alpha$ die atomare Polarisierbarkeit. Es ist ganz wichtig zu beachten, dass dieses lokale Feld nicht dem makroskopischen (gemittelten) Feld entspricht, da es nicht das Feld des Dipols selbst enthält! \\
\begin{equation*}
\vec{E}_\textit{lokal}=\vec{E}_\textit{gemittelt} + \frac{1}{3\epsilon_0}\vec{P}
\end{equation*}
Sei $n$ die Anzahl der Dipole pro Volumen, dann können wir $\vec{P}$ schreiben als
\begin{equation*}
\vec{P}=n\vec{p} = \underbrace{n\cdot\alpha}_{=:\kappa}\cdot\epsilon_0\left(\vec{E}+\frac{1}{3\epsilon_0}\vec{P}\right) =\frac{\kappa}{1-\frac{\kappa}{3}}\epsilon_0\vec{E}.
\end{equation*}
Vergleichen wir das nun mit $\vec{P}=\chi_\textit{el}\epsilon_0\vec{E}$, sehen wir sofort
\begin{align*}
\chi_\textit{el}&=\frac{\kappa}{1-\frac{\kappa}{3}} &\Rightarrow& &\epsilon_r = 1+ \chi_\textit{el}=\frac{2\kappa+3}{3-\kappa}.
\end{align*}
Das lässt sich in die \textsc{Clausius-Mosotti}-Formel umstellen
\begin{equation*}
\frac{\epsilon_r-1}{\epsilon_r+2}=\frac{\kappa}{3},
\end{equation*}
die Gültigkeit für alle homogenen, isotropen Medien hat.

\section{Reflexion und Brechung von Wellen an Grenzflächen}

\begin{wrapfigure}[11]{r}[0cm]{0cm}
	\raisebox{0pt}[\dimexpr\height-1\baselineskip\relax]{
		\colorbox{hgrey}{
			\begin{tikzpicture}
			\draw[ultra thick] (0,2)--(0,-2)node[below]{$x=0$};
			\draw (-2,0)--(2,0);
			\draw[->] (0,0)--(-2,2);
			\draw[->] (-2,-2)--(0,0);
			\draw[->] (0,0)--(2,1);
			\draw (-1,1) arc(135:225:1.41cm);
			\draw (1.5,0) arc(0:17.5:2.3cm);
			\draw (-0.8,0.3) node{$\alpha_r$};
			\draw (-0.8,-0.35) node{$\alpha_e$};
			\draw (1.1,0.2) node{$\alpha_d$};
			\draw (-1,2.5) node{$\epsilon_1\mu_1$};
			\draw (1,2.5) node{$\epsilon_2\mu_2$};
			\end{tikzpicture}
		}
	}
	\caption{Reflexion und Brechung}
\end{wrapfigure}
Um die Phänomene der Brechung und Reflexion zu untersuchen betrachten wir die folgenden Strahlen/Wellen:
\begin{align*}
\text{Einfallend:}\quad &\vec{E}^e e^{i(\vec{k}_e\vec{r}-\omega_e t)} &\qquad \omega_e =\frac{c}{n_1}k_e \\
\text{Reflektiert:}\quad&\vec{E}^r e^{i(\vec{k}_r\vec{r}-\omega_r t)} &\qquad \omega_r = \frac{c}{n_1}k_r \\
\text{Durchgehend:}\quad &\vec{E}^d e^{i(\vec{k}_d\vec{r}-\omega_d t)}&\qquad \omega_d=\frac{c}{n_2}k_d
\end{align*}
Natürlich muss $E_t$ bei $x=0$ stetig sein. 
\begin{equation*}
E_t^e e^{i(\vec{k}_e\vec{r}-\omega_e t)} + E_t^r e^{i(\vec{k}_r\vec{r}-\omega_r t)} = E_t^d e^{i(\vec{k}_d\vec{r}-\omega_d t)}
\end{equation*}
Wir sehen, so dass die $\omega_i$ im ganzen Raum identisch sein müssen. Genauso verhält es sich jeweils mit $k_{iy}$ und $k_{iz}$. Das gilt für beliebige Grenzflächenbedingungen! Aufgrund der Geometrie sehen wir nun
\begin{equation*}
k_{ey}=k_e\sin\alpha_e\ \stackrel{!}{=}\ k_{ry} = k_r\sin\alpha_r\ \stackrel{!}{=}\
k_{dy}=k_d\sin\alpha_d.
\end{equation*}
Mit $\omega_e=\omega_r=\omega_d$ und $\omega=\frac{c}{n}k$ folgt
\begin{align*}
k_e&=k_r &\frac{k_e}{n_1}&=\frac{k_d}{n_2}.
\end{align*}
So erhalten wir das \textbf{Reflexions- und Brechungsgesetz}
\begin{align*}
\sin\alpha_e &=\sin\alpha_r &\frac{\sin\alpha_e}{\sin\alpha_d}=\frac{n_2}{n_1}=\frac{c_1}{c_2}.
\end{align*}
Im folgenden schreiben wir $k_1$ für $k_e=k_r$, $k_2$ für $k_d$, sowie $\alpha$ für $\alpha_e=\alpha_r$ und $\beta$ für $\alpha_d$.\\

Werfen wir nun eine Blick auf das Verhalten der Amplituden an der Grenzfläche. Da $E_t$ und $H_t$ stetig sei müssen (keine Ströme!), also
\begin{align*}
E_t^e+E_t^r &= E_t^d\\
\frac{B_t^e}{\mu_1}+\frac{B_t^r}{\mu_1} &= \frac{B_t^d}{\mu_2}
\end{align*}
gelten muss und wir $\vec{B}=\frac{\vec{k}\times\vec{E}}{\omega}$ setzen können, ergibt sich
\begin{equation*}
\frac{(\vec{k}_e\times\vec{E}^e)_t}{\mu_1}+\frac{(\vec{k}_r\times\vec{E}^r)_t}{\mu_1} = \frac{(\vec{k}_d\times\vec{E}^d)_t}{\mu_2}.
\end{equation*}
\ \\
\textbf{a.\ senkrechter Einfall}\\

Der tritt mit $\alpha=0$ auf und so können wir, statt der Tangentialkomponenten auch einfach
\begin{equation*}
\vec{E}^e + \vec{E}^r = \vec{E}^d
\end{equation*}
schreiben. Es folgt außerdem
\begin{equation*}
\frac{k_1}{\mu_1}\left[\vec{e}_x\times\left(\vec{E}^e-\vec{E}^r\right)\right]_t = \frac{k_2}{\mu_2}\left[\vec{e}_x\times\vec{E}^d\right]_t,
\end{equation*}
was sogar auch ohne das Ziehen der Tangentialkomponente ( $]_t$ ) erfüllt wäre, da die Wellen transversal sind. 
\begin{equation*}
\frac{k_1}{\mu_1}\left(\vec{E}^e-\vec{E}^r\right) = \frac{k_2}{\mu_2}\vec{E}^d
\end{equation*}
Wir nehmen nun $\vec{E}^r=a^r\vec{E}^e$ und $\vec{E}^d=a^d\vec{E}^e$ an. Es wird sich später herausstellen, dass dadurch kein Widerspruch in der Gleichung entsteht. Setzen wir sie ein, erhalten wir
\begin{align*}
1+a^r&=a^d\\
1-a^r&=\frac{\mu_1k_2}{\mu_2k_1}a^d=:\nu a^d
\end{align*}
mit
\begin{equation*}
\nu:=\frac{k_2\mu_1}{k_1\mu_2}=\frac{n_2\mu_1}{k_1\mu_2}=\sqrt{\frac{\mu_1\epsilon1}{\mu_2\epsilon_2}}.
\end{equation*}
Wir erhalten damit die \textbf{\textsc{Fresnel}schen Formeln für senkrechten Einfall}
\begin{align*}
a^d&=\frac{2}{1+\nu} & a^r &=\frac{1-\nu}{1+\nu}.
\end{align*}
Achtung: Bei Reflexion am dichten Medium kann in $a^r$ ein Phasensprung entstehen.\\
\newpage
\textbf{b.\ Schräger Einfall}\\

Die eben gesehenen \textsc{Fresnel}-Gleichungen lassen sich für beliebige Winkel verallgemeinern (hier Rechnung). Man unterscheidet dabei, ob $\vec{E}$ senkrecht oder parallel zur Einfallsebene, die von einfallendem, reflektierten und durchgelassenem Strahl aufgespannt wird, steht.
\begin{align*}
a_\perp^d&=\frac{2}{1+\nu\xi} & a_\perp^r&=\frac{1-\nu\xi}{1+\nu\xi}\\
a_\parallel^d&=\frac{2}{\xi+\nu} &a_\parallel^r &=-\frac{\xi-\nu}{\xi+\nu}
\end{align*}
Dabei ist
\begin{equation*}
\xi=\frac{\cos\beta}{\cos\alpha}.
\end{equation*}
\ \\ \linebreak
\textbf{c.\ Energiebilanz für senkrechten Einfall}\\

Man erwartet natürlich
\begin{equation*}
S^e=S^d+S^r.
\end{equation*}
Es gilt auf jeden Fall
\begin{equation*}
|\vec{S}_P|=\frac{|\vec{E}\times\vec{B}|}{´\mu}=\frac{kE^2}{\omega\mu}=\frac{n}{c}\frac{E^2}{\mu}.
\end{equation*}
Daraus gewinnt man
\begin{align*}
cS_P^e &=\frac{n_1}{\mu_1}\left(E^e\right)^2 \\
cS_P^r &= \frac{n_1}{\mu_1}\left(a^rE^e\right)^2 = \left(a^t\right)^2cS_P^e\\
cS_P^d &=\frac{n_2}{\mu_2}\left(a^dE^e\right)^2 = \frac{n_1}{\mu_1}\nu\left(a^dE^e\right)^2 = \nu\left(a^d\right)^2cS_P^e=\left(1-\left(a^r\right)^2\right)cS_p^e.
\end{align*}
Zusammengefasst ist das
\begin{equation*}
S_P^d=S_P^e-S_P^r=TS_P^e
\end{equation*}
Man führt dann zum sogenannten Transmissionskoeffizienten $T$ noch den Reflexionskoeffizienten $R=\left(a^r\right)^2$ ein. Es gilt offensichtlich
\begin{equation*}
R+T=1.
\end{equation*}

\section{Totalflexion}

Beim Übergang vom optisch dichteren ins optisch dünnere Medium ($n_2<n_1$) kann es zum Phänomen der Totalflexion kommen, bei dem ein eintreffender Strahl so stark vom Lot weggebrochen wird, dass er in der Grenzfläche liegt. Diesen Grenzwinkel liefert das Brechungsgesetz:
\begin{equation*}
\sin\alpha_G = \frac{n_2}{n_1}
\end{equation*}
Für alle $\alpha > \alpha_G$ wird $\sin\beta>1$ und es gibt keine gebrochene Welle mehr. Die korrekte physikalische Interpretation ist
\begin{equation*}
\sin\beta=\frac{k_{2y}}{k_2}>1.
\end{equation*}
Mit $k^2=k_{2x}^2+k_{2y}^2$ folgt, dass $k_{2x}$ rein imaginär werden muss (wir schreiben $k_{2x}=i\kappa$). Das Feld im zweiten Medium hinter der Grenzfläche ist damit
\begin{equation*}
\vec{E}=\vec{E}^d e^{i(k_{2y}y-\omega t)}e^{-\kappa t}.
\end{equation*}
Die Welle parallel zur Oberfläche klingt also in $x$-Richtung exponentiell schnell ab. Es findet keine Dissipation statt, stattdessen wird die gesamte Energie reflektiert.\\
Das lässt sich auch anhand der \textsc{Fresnel}schen Formeln nachvollziehen. $\xi$ wird nämlich in dem Fall auch rein imaginär.
\begin{equation*}
\xi = \frac{\sqrt{1-\sin^2\beta}}{\cos\alpha}=:i\xi'
\end{equation*}
So werden die Formeln zu
\begin{align*}
a_\perp^r &=\frac{1-i\nu\zeta'}{1+i\nu\zeta'} &\Rightarrow& &|a_\perp^r|=1\\
a_\parallel^r &= -\frac{i\zeta'-\nu}{i\zeta'+\nu} &\Rightarrow& &|a_\parallel^r|=1
\end{align*}
Es wird also alles reflektiert, jedoch mit Phasenverschiebung. \\
Aber Vorsicht: Obwohl es in $x$-Richtung keine propagierende Welle gibt, können $a_\perp^d$ und $a_\parallel^d$ verschieden von Null sein. Ist das Medium $n_2$ sehr dünn, kann es durchaus zur Transmission kommen.
\chapter{Quasistationäre Ströme}

\section{Quasistationäre Näherung}

Ausgangspunkt der folgenden Betrachtungen sollen wieder die \textsc{Maxwell}-Gleichungen sein:

\begin{align*}
\div \vec{B} \ &= \ 0  &\epsilon_0\div\vec{E}  \ &= \ \rho\\
\rot\vec{E}+\dot{\vec{B}}  \ &= 0  &\frac{1}{\mu_0}\rot\vec{B}-\epsilon_0\dot{\vec{E}}  \ &= \ \vec{j}  
\end{align*}

Die Idee der Quasistationärität ist, dass die Zeitabhängigkeit langsam ist, wodurch man $\epsilon_0\dot{\vec{E}}\ll\vec{j}$ nähern kann. damit kommt es zur \textbf{effektiven Entkopplung} von $\vec{E}$ und $\vec{B}$ in der felderzeugenden Quelle.\\
Mit der quasistationären Näherung gilt also:

\begin{equation*}
\rot\vec{B} \ = \ \mu_0\vec{j} \quad\Rightarrow\quad -\laplace\vec{A}  \ = \ \mu_0\vec{j} \qquad \text{mit $\div \vec{A}=0$, (\textsc{Coumlomb}-Eichung)}
\end{equation*}

Exakt wäre:

\begin{equation*}
\Dalembert\vec{A}  \ = \  \left(\frac{1}{c^2}\partial^2_t \ - \ \laplace\right)\vec{A}  \ = \  \mu_0\vec{j}
\end{equation*}


Anschaulich entspricht also die Vernachlässigung von $\dot{\vec{E}}$ einer Vernachlässigung der Retardierdung: $\Dalembert \approx - \laplace$.\\
\ \\
Damit man die quasistationäre Näherung anwenden darf, muss folgenden Bedingung erfüllt sein:

\begin{align*}
& \partial_t \ \sim \ -i\omega \ \sim \ -i\frac{2\pi}{\tau}\\
& \partial_{\vec{r}} \ \sim \ \sigma\left(\frac{1}{l}\right) \qquad \text{l \ldots charakteristische Länge für Änderung des Feldes}\\
\Rightarrow \quad & \frac{1}{c^2}\partial_t^2\ll\laplace \quad \text{entspricht} \quad \frac{\omega^2}{c^2}\ll\frac{^1}{l^2}\\
\Rightarrow \quad &\left(\frac{2\pi \ l}{c\tau}\right)^2\ll 1 \quad\text{bzw.}\quad\left(\frac{2\pi \ l}{\lambda}\right)^2\ll 1
\end{align*}


\section{Leiterschleifen}

Wir betrachten nun mehrere Leiterschleifen $\mathcal{S}_i$ durch die  die Ströme $I_k$ fließen. Nach dem Induktionsgesetz gilt:

\begin{equation*}
(U_{\text{ind}})_i  \ = \ -\dot{\Phi}_i \quad \text{mit} \quad \Phi_i (t)  \ = \ \Int{\mathcal{S}_i}{}{\vec{A}_F}\cdot\vec{B}(t)  \ = \ \sum_k L_{ik} I_k(t)
\end{equation*}

Wir nehmen nun an, dass die Leiterschleife $\mathcal{S}_i$ über einen Widerstand $R_i$ und über eine Kapazität $C_i$ verfügt und an eine Spannungsquelle $U_i$ angeschlossen ist. Dann folgt mit den \textsc{Kirchhoff}-Gesetzen:

\begin{align*}
- & U_i \ + \ R_i \ I_i \ + \ \frac{Q_i}{C_i} \ = \ U_{\text{ind}}  \ = \ -\diff{}{t} \sum_k \ L_{ik} \ I_k\\
\Rightarrow \quad & \dot{U}_i  \ = \ R_i \ \dot{I}_i \ + \ \frac{I_i}{C_i} \ + \ \sum_k \ L_{ik} \ \ddot{I}_k
\end{align*}

Durch die Quasistationarität kommt es, wie man aus obiger Gleichung entnehmen kann, nur zu einer induktiven, aber keiner kapazitiven Kopplung.\\
Speziell für eine Schleife gilt: $\dot{U}=L\ddot{I}+R\dot{I}+\frac{1}{C}I$. Dabei kann man mehrere Fälle unterscheiden:
\ \\
\begin{enumerate}[label=\roman*]
\item $U=0\qquad$ Eigenschwingung:
\ \\
Wir wählen für die verbleibende DGL den Ansatz:
\begin{equation*}
I \ = \ I_0 \ e^{i\omega_0 t} \quad\Rightarrow\quad -\omega_0^2 L \ + \ i\omega_0 \ R \ + \ \frac{1}{C} \ = \ 0
\end{equation*}

Für $R=0$ gilt für $\omega_0=\sqrt{\frac{1}{LC}}$, ansonsten tritt eine Dämpfung der Schwingung auf.
\ \\\

\item $U=U_0 \ e^{i\omega t} \qquad$ erzwungene Schwingung:	
\ \\
Wir wählen erneut den Ansatz $I=I_0 e^{i\omega t}$

\begin{align*}
i\omega U_0  \ &= \ \left(-\omega^2 \ L \ + \ i \omega \ R \ + \ \frac{1}{C}\right) \cdot I_0\\
U_0 \ &= \ \underbrace{\Bigg[R \ + \ \left(i\omega \ L \ - \ \frac{i}{\omega \ C}\right)\Bigg]}_{=: Z \text{ komplexer Scheinwiderstand}} \cdot I_0\\
\ \\
\Rightarrow U_0 \ &= \ Z \cdot I_0 \quad\Rightarrow\quad U  \ = \ Z \ I  \ = \  Z \ I_0 \cos (\omega t \ + \phi)
\end{align*}
\end{enumerate}

Für die Energiebilanz einer solchen Schleife gilt:

\begin{align*}
U  \ &= \ L \ \dot{I} \ + \ R \ I \ + \ \frac{Q}{C}\\
\Rightarrow\quad \underbrace{U \ I}_{\text{Leistung}} \ &= \ \diff{}{t} \Bigg( \underbrace{\frac{1}{2}L \ I^2}_{W_{\text{mag}}} \ + \ \underbrace{\frac{1}{2} \ \frac{Q^2}{C}}_{W_{\text{el}}}\Bigg) \ + \ \underbrace{R \ I^2}_{\text{\textsc{Joule}'sche Wärme (Dissipation)}} 
\end{align*}

Die Mittelung dieser Energie über eine Periode liefert uns mit $\langle N \rangle = \langle N_{\textsc{Joule}} \rangle$:

\begin{align*}
\langle I^2 \rangle \ &= \ I_0^2 \ \langle\cos^2\omega t \rangle  \ = \  \frac{1}{2}I_0^2 \quad \Rightarrow\quad I_{\text{eff}} := \frac{1}{\sqrt{2}}I_0; \; \; U_{\text{eff}}:= \frac{1}{\sqrt{2}}U_0\\
\ \\
\langle N \rangle  \ &= \ U_0 \ I_0 \ \Big\langle\cos (\omega t) \ \cos(\omega t \ + \ \phi) \Big\rangle  \ = \  \frac{1}{2}U_0 \ I_0 \ \Big\langle\left(\cos(2\omega t \ + \ \phi) \ + \cos (\phi)\right)\Big\rangle  \ = \ U_{\text{eff}} \ I_{\text{eff}} \ \cos(\phi)
\end{align*}


\section{Drahtwellen}

[Büldschn]
Wir betrachten zwei parallele Leiter der Dicke $d$, durch die in entgegengesetzte Richtung der Strom $I$ fließt. Um einfacher über das Problem reden zu können, definieren wir uns zunächst die Größen der Leiter pro Längeneinheit:

\begin{align*}
&\text{Induktivität}  & l \ &:= \ \frac{\Delta L}{\Delta x}\\
&\text{Kapazität}   & \zeta  \ &:= \ \frac{\Delta C}{\Delta x}\\
&\text{Widerstand}   & r  \ &:= \ \frac{\Delta R}{\Delta x}\\
&\text{Ladung}   &q \ &:= \ \frac{\Delta Q}{\Delta x}\\
&\text{Leitwert}   &g   
\end{align*}
\ \\

Das Induktionsgesetz liefert uns für den Doppelleiter:

\begin{equation*}
\Oint{\rectangle}{}{\vec{r}}\cdot\vec{E} \ = \ - \diff{}{t} (\Delta\Phi(x,t))  \ = \ - \diff{}{t} (l \cdot \Delta x \cdot I(x,t))
\end{equation*}


Für die Spannungsbilanz einer Masche gilt:

\begin{align*}
& \underbrace{U(x+\Delta x) - U(x)}_{\pdiff{U}{x}\Delta x} \ + \ \Delta x \cdot r  \cdot I  \ = \ - \Delta x \ l \ \dot{I}\\
\Rightarrow\quad &\pdiff{U}{x} \ + \ r \ I \ + \ l \ \dot{I}  \ = \ 0
\end{align*}

Die Ladungsbilanz für einen Leiter erhalten wir ähnlich aus dem Kontinuitätsgesetz:

\begin{align*}
& \diff{}{t}(\Delta Q) \ + \ \Oiint{}{}{\vec{A}_F}\cdot\vec{j} \ = \ 0\\
& \Delta x \dot{q} \ + \ I(x+\Delta x)-I(x) \ + \ \underbrace{\Delta x \ g \ U(x)}_{\text{Verluste}} \ = \ 0\\
\overset{Q=CU}{\Rightarrow} & \zeta \ \dot{U} \ + \ \pdiff{I}{x} \ + \ g \ U  \ = \ 0
\end{align*}


Leiten wir die Spannungsbilanz nun noch einmal nach der Zeit und die  Ladungsbilanz nach dem Ort $x$ ab, so erhalten wir folgende zwei Gleichungen, welche beide $\frac{\partial^2 U}{\partial t \partial t}$ enthalten:

\begin{align*}
\frac{\partial^2 U}{\partial x \partial t} \ + \ r \ \dot{I} \ + \ l \ \ddot{I}  \ &= \ 0\\
\zeta\frac{\partial^2 U}{\partial x \partial t} \ + \ \pddiff{I}{x} \ + \ g \pdiff{U}{x}  \ &= \ 0
\end{align*}

Diese beiden Gleichungen lassen sich nun zu sogenannten \textbf{Telegraphengleichung} zusammensetzen:

\begin{equation*}
\pddiff{I}{x} \ - \ \zeta l \pddiff{I}{t} \ - \ \underbrace{\left(\zeta r + g l \right)}_{\text{Verlust}}\pdiff{I}{t} \ - \ \underbrace{g r}_{\text{Verlust}} \ I \ = \ 0
\end{equation*}


Diese Gleichung wollen wir nun für folgende zwei Fälle genauer untersuchen:

\begin{enumerate}[label=\roman*]
\item \underline{Ideale Leitung:} $\qquad r=0, \; \; g=0$
\ \\
\ \\
Für die ideale Leitung gilt:

\begin{equation*}
\pddiff{I}{x} \ - \ \zeta l \ \pddiff{I}{t} \ = \  	0 \quad \Rightarrow\quad I(x,t)  \ = \ I(x \mp v_0\cdot t)
\end{equation*}

Dabei gilt für die Ausbreitungsgeschwindigkeit: $v_0^2 = \frac{1}{\zeta l} \ll c^2$ (quasistationäre Näherung).\\
Weiterhin folgt:

\begin{align*}
\pdiff{U}{x} \ &= \ - l\pdiff{I}{t} \ = \ \pm v_0 l \pdiff{I}{x}\\
\Rightarrow \quad U \ &= \ \pm l \ v_0 \ I  \ = \ \pm \sqrt{\frac{l}{\zeta}}I 
\end{align*}

Den Ausdruck $\sqrt{\frac{l}{\zeta}}$ bezeichnet man der Anschauung nach auch als \textbf{Wellenwiderstand} $Z$.\\
Verbindet man nun die beiden Teile des Doppelleiters über einen Widerstand $R$, so kommt es bei $R\neq Z$ zu einer teilweisen Reflexion der Welle.\\
\ \\
\item \underline{Nichtideale Leitung:}
\ \\
\ \\

Zum Lösen der Telegraphengleichung unter nichtidealen Bedingungen wählen wir den Ansatz: $I=I_0 e^{-i(kx-\omega t)}$. Setzen wir diesen nun ein, erhalten wir daraus, dass $k$ komplex sein muss: $k=k_0 + ik_1$. Dementsprechend folgt auch für $I$:

\begin{equation*}
I  \ = \ I_0 \ e^{-k_1 x} \ e^{-i(k_0 x -\omega t)}
\end{equation*}
\ \\
\begin{equation*}
\Rightarrow k_0  \ = \ \frac{\omega}{v_0}\left[1 \ + \ \frac{1}{8\omega^2}\left(\frac{r}{l} \ - \ \frac{g}{\zeta}\right)^2\right]
\end{equation*}

Da aus obiger Gleichung folgt, dass $v= \frac{\omega}{k_0}\neq v_0$ gilt, liegt also eine Dispersion $v(k)$ vor, welche zwangsläufig zu einer Signalverzerrung führt. Diese Dispersion kann man \grqq ausschalten\grqq , indem man  $l$ anpasst. Dafür muss gelten: $\frac{r}{l} \overset{!}{=}   \frac{g}{\zeta}$ , sodass $k_0 =\frac{v_0}{\omega}$ folgt, und die Leitung wieder ideal wird.\\
Ebenfalls aus obiger Gleichung erhält man den Dämpfungsterm:

\begin{equation*}
k_1  \ = \ \frac{1}{2v_0} \ \left(\frac{r}{l} \ - \ \frac{g}{\zeta}\right)
\end{equation*}
\end{enumerate}

\section{Quasistationäre Ströme in Leitern}

Ausgangspunkt unserer Betrachtungen sollen in diesem Kapitel wieder die \textsc{Maxwell}-Gleichungen sein, wobei wir aber noch das \textsc{Ohm}'sche Gesetz hinzunehmen:

\begin{align*}
\frac{1}{\mu}\rot\vec{B}  \ &= \ \vec{j}_L \ \xout{+ \ \epsilon\vec{E}}  & \div\vec{B}  \ &= \ 0\\
\epsilon \ \div\vec{E} \ &= \ \rho_0  & \rot\vec{E} \ + \ \dot{\vec{B}} \ &= \ 0\\
\ \\
\vec{j}_L \ &= \ \sigma \vec{E}
\end{align*}

Wenn wir nun die Annahme machen, dass die Leitungsströme $\vec{j}_L$ nahezu sämtliche Stromdichten ausmachen $(\vec{j}_L\rightarrow\vec{j})$ erhalten wir:

\begin{align*}
\rot \vec{j} \ = \ \sigma \rot\vec{E}  \ = \ -\sigma\dot{\vec{B}}\\
\Rightarrow \frac{1}{\mu} \ \rot\rot \vec{B} \ = \ -\sigma \dot{\vec{B}}  \ = \ \frac{1}{\mu}\left(\grad\underbrace{\div\vec{B}}_{=0} \ - \ \laplace\vec{B}\right)
\end{align*}

Über analoge Vorgehensweisen für $\vec{E}$ und $\vec{j}$ erhalten wir schlussendlich folgende drei Differentialgleichungen:

\begin{align*}
\laplace \vec{B} \ - \ \mu\sigma\dot{\vec{B}} \ &= \ 0\\
\laplace \vec{E} \ - \ \mu\sigma\dot{\vec{E}} \ &= \ 0\\
\laplace \vec{j} \ - \ \mu\sigma\dot{\vec{j}} \ &= \ 0
\end{align*}

Die erhaltenen Gleichungen sind sogenannte \textbf{Diffusionsgleichungen}, da sie aufgrund ihrer nur einfach auftretenden Zeitableitung irreversible Prozesse beschreiben. Zu ihrer Lösung wählen wir den Ansatz einer ebenen Welle für die entsprechenden Größen: $\vec{B}(\vec{r},t)= \vec{B}_0 e^{i(\vec{k}\vec{r}-\omega t)}$. Das Einsetzen des Ansatzes in die DGL liefert uns:

\begin{align*}
-\vec{k}^2 \ &- \ i\mu\sigma\omega  \ = \ 0 \quad \Rightarrow \quad k  \ = \ \sqrt{-i\mu\sigma\omega} \ = \ k_0(1-i)\\
k_0  \ &= \ \sqrt{\frac{\mu\sigma\omega}{2}} \ =: \ \frac{1}{\delta}
\end{align*}

Wenn wir nun ohne Beschränkung der Allgemeinheit annehmen, dass $\vec{k}\parallel\vec{e}_x$ ist, erhalten wir für unser $\vec{B}$-Feld:

\begin{equation*}
\vec{B} \ = \ \vec{B}_0 e^{i(\omega t-k_0 x)} \ e^{-k_0 x}  \ = \  \vec{B}_0 e^{i(\omega t-k_0 x)} \ e^{\frac{x}{\delta}}
\end{equation*}

Die Felder und Ströme fallen innerhalb des Leiters also exponentiell ab. Der Ausdruck $\delta\sim \frac{1}{\sqrt{\omega}}$ lässt sich also dementsprechend als \textbf{Eindringtiefe} verstehen.

[nochn Bildschn]

Ein Beispiel für dieses Verhalten von Feldern und Strömen in Leitern ist die Entstehung von Wirbelströmen in von einem Magnetfeld durchsetzten Eisenkern. Dieser fungiert als Abschirmstrom und verhindert somit das tiefe Durchdringen des Kerns durch das Magnetfeld. In technischen Anwendungen wie z.B. dem Transformator wird dem entgegengewirkt, indem die Leitfähigkeit $\sigma$ durch Lamellierung stark abgesenkt wird.\\
\ \\
Ein anderes Beispiel ist der sogenannte \textbf{Skin-Effekt}, welcher dafür sorgt, dass Wechselströme an der Drahtoberfläche fließen. Der Widerstand eines Drahtes bei Skin-Effekt lässt sich folgendermaßen berechnen:

\begin{equation*}
Z \ = \ \frac{U}{I} \ = \ \frac{E \cdot l}{I} \quad \Rightarrow \quad \frac{Z}{l} \ = \ \frac{E}{I}
\end{equation*}

$E$ ist dabei ein von außen angelegtes elektrisches Feld. Weiterhin nehmen wir an, dass ein starker Skin-Effekt vorleigt $(\delta\ll r)$. Zunächst müssen wir also den Strom $I$ berechnen, um den Widerstand zu erhalten:


\begin{align*}
I  \ &= \ \Int{}{}{\vec{A}_F}\cdot\vec{j} \ = \ \sigma \ 2 \pi \ r \Int{0}{\infty}{x} E(x) \qquad\qquad\qquad\Big| \ E(x) = E\cdot e^{(i-1)k_0 x}\\
&= \ 2\pi \sigma \ E \ r \Int{0}{\infty}{x}e^{(i-1)k_0 x}  \ = \ 2\pi\sigma \ E \ r  \frac{1}{(i-1)k_0}\cdot(-1) 
\end{align*}

Somit erhalten wir für den Widerstand:

\begin{equation*}
\frac{Z}{l}  \ = \ \frac{(1-i)k_0}{2\pi\sigma \ r} \ = \ \frac{1-i}{2\pi \ r}\ \sqrt{\frac{\mu\omega}{2\sigma}}
\end{equation*}

Vergleicht man dies mit dem normalen \textsc{Ohm}'schen Widerstand eines Leiters, so stellt man fest, dass der Widerstand bei Skin-Effekt sehr viel größt als dieser ist:

\begin{equation*}
\frac{R}{l} \ = \ \frac{1}{\pi \ r^2 \ \sigma} \quad \Rightarrow \quad \frac{Z}{R}  \ = \ \frac{1-i}{2} \ \frac{r}{\delta} \ \gg \ 1
\end{equation*}

\chapter{Dispersion}

\section{Allgemeines über Wellen in leitenden Medien}

Ausgangspunkt unserer Betrachtungen in diesem Kapitel werden die linearen Materialgesetze und das \textsc{Ohm}'sche Gesetz sein:

\begin{equation*}
\vec{D}  \ = \ \epsilon \vec{E}, \qquad\qquad\vec{j}_0  \ = \ \sigma\vec{E}
\end{equation*}

Die \textsc{Maxwell}-Gleichungen liefern uns zusätzlich:

\begin{align*}
\frac{1}{\mu_0}\rot\vec{B}  \ &= \  \vec{j}_0 \ + \ \dot{\vec{D}}  \ = \ \sigma\vec{E} \ + \ \epsilon\vec{E} \qquad\Big| \ \partial_t\\
\frac{1}{\mu_0}\rot\dot{\vec{B}} \ &= \ \sigma\dot{\vec{E}} \ + \ \epsilon\ddot{\vec{E}}\\
\ \\
&\left(\rot\vec{E} \ = \ -\dot{\vec{B}}, \quad \epsilon\div\vec{E} \ = \ \rho_0  \ = \ 0\right)\\
\ \\
-\frac{1}{\mu_0} \rot\rot \vec{E}  \ &= \ - \frac{1}{\mu}\Big(\grad\underbrace{\div\vec{E}}_{=0} \ + \ \laplace\vec{E}\Big)  \ = \ \sigma\dot{\vec{E}} \ + \ \epsilon\ddot{\vec{E}}\\
\ \\
\Rightarrow \quad 0  \ &= \ \frac{1}{\mu_0}\laplace\vec{E} \ + \ \sigma\dot{\vec{E}}\ + \ \epsilon\ddot{\vec{E}} 
\end{align*}

Zur Lösung der erhaltenen DGL wählen wir den Ansatz der ebenen Welle für $\vec{E}$. $\quad(\vec{E} \ = \ \vec{E}_0 e^{i(\vec{k}\vec{r}-\omega t)})$\\
Einsetzen liefert uns:

\begin{align*}
&\frac{1}{\mu_0} \left(i\vec{k}\right)^2  \ = \ -i \omega \sigma \ + \  \left(i\omega\right)^2\epsilon\\
\Rightarrow\quad &\vec{k}^2 \ = \ \mu_0\epsilon_0\omega^2 \ \left(\epsilon_r \ + \ \frac{i\sigma}{\epsilon_0\omega}\right) \ = \ \frac{\omega^2}{c_0^2} \ \underbrace{\left(\epsilon_r \ + \ \frac{i\sigma}{\epsilon_0 \omega}\right)}_{=:\tilde{\epsilon}_r}
\end{align*}


Die Aufspaltung der gesamten Stromdichte $\vec{j}$ in die freien Ströme $\vec{j}_0$ und die Polarisationsströme $\vec{j}_P$ ist dabei für alle $\omega>0$ willkürlich, insbesondere aber für große $\omega$.\\
Die Auftrennung des komplexen Wertes $\tilde{\epsilon}_r(\omega)$ in ein reelles $\epsilon_r$ und ein imaginäres $\frac{i\sigma}{\epsilon_0\omega}$ ist dabei ebenso nur im Limes $\omega\rightarrow\infty$ eindeutig, da sonst bereits $\epsilon_r(\omega)$ und $\sigma(\omega)$ an sich schon komplex sein können.\\
\ \\
Wir definieren uns: 

\begin{equation*}
\vec{k}^2 \ = \ \frac{\omega^2}{c^2}\tilde{n}^2; \qquad \tilde{n}(\omega)  \ = \  \sqrt{\epsilon_r(\omega} \ = \ n \cdot (1 \ + \ i\kappa) \qquad\text{mit } n, \kappa \text{ reell}
\end{equation*}

In den Grenzfällen bedeutet dies:

\begin{align*}
\frac{\sigma}{\epsilon_0\omega}&\gg\epsilon_r \quad \text{bzw.} \quad \omega\ll \frac{\sigma}{\epsilon} \quad\Rightarrow\quad \epsilon_r \text{ vernachlässigen }\Rightarrow \text{ quasistatischer Fall}\\
\frac{\sigma}{\epsilon_0\omega}&\ll\epsilon_r \quad\text{bzw.}\quad \omega\gg\frac{\sigma}{\epsilon} \quad\Rightarrow\quad \sigma\text{ vernachlässigen }\Rightarrow\text{ Dielektrikum}
\end{align*}
\ \\
\underline{Interpretation von $\tilde{n}$:}\\
\ \\
Eine in den Leiter eindringende Welle besteht hauptsächlich aus zwei Komponenten: der Wellenausbreitung im Medium und dem exponentiellen Abklingen in ihm. Dies ist leicht zu sehen, da $\omega$ reell ist und $\vec{k}=\vec{k}_0 + i\vec{k}_1$ sich aus einem reellen und imaginären Part zusammensetzt:

\begin{align*}
|\vec{k}_0| \ &= \  \frac{2\pi}{\lambda} \qquad \qquad \lambda \ \ldots \ \text{ Wellenlänge}\\
|\vec{k}_1| \ &= \  \frac{1}{\delta} \qquad \qquad \delta \ \ldots\ \text{ Abklinglänge}\\
\ \\
\Rightarrow \quad \vec{E}  \ &= \ \vec{E}_0 \ e^{i\left(\vec{k}_0\vec{r}-\omega t\right)} \ e^{-\vec{k}_1 \vec{r}} 
\end{align*}

$\vec{k}_0$ und $\vec{k}_1$ müssen dabei nicht notwendigerweise parallel sein; sie ergeben sich stattdessen aus Randbedingungen, wie z.B. der Stetigkeit verschiedener Komponenten an Grenzflächen. Die Beträge hingegen müssen aus der Dispersionsrelation bestimmt werden.
\ \\
\ \\
\underline{Bemerkung:}\\
\ \\
Formal wären auch komplexe $\omega$ möglich, welche einem zeitlichen Abklingen entsprächen.
\ \\
\ \\
Betrachten wir nun den Grenzfall, das wir eine Welle mit einer Kreisfrequenz $\omega \rightarrow 0$ auf eine Leiteroberfläche schicken. Daraus folgt zunächst direkt:

\begin{equation*}
\frac{\sigma}{\omega} \rightarrow \infty, \quad \tilde{n} \ = \ \sqrt{\epsilon_r + \frac{i\sigma}{\epsilon_0 \omega}}\rightarrow\infty, \quad n\rightarrow\infty,\quad n\kappa\rightarrow\infty
\end{equation*}

Das Reflexionsverhalten erhalten wir nun mithilfe der \textsc{Fresnel}'schen Formeln (wobei $n_1$ in diesem Falle 1 sei):

\begin{equation*}
a^r  \ = \  \frac{1 \ - \ \frac{\mu_1 n_2}{\mu_2 n_1}}{1 \ + \ \frac{\mu_1 n_2}{\mu_2 n_1}} \ = \ \frac{1 \ - \ \tilde{n}}{1\ + \ \tilde{n}} \ \rightarrow\  -1
\end{equation*}

Es kommt also zur Vollständigen Reflexion, analog zur Totalreflexion. Der Leiter ist also undurchsichtig für Wellen mir $\omega \rightarrow 0$. Für sichtbares Licht können wir leider noch keine Aussage treffen, da es dort wesentlich komplizierter ist.

\section{Dispersion in Dielektrika}

Allgemein bedeutet Dispersion die Abhängigkeit der Brechzahl $n$ von der Frequenz der einfallenden Welle: $\qquad n = n (\omega)$\\
Als \textbf{"normale" Dispersion} bezeichnet man dabei ein Verhalten, bei dem $n(\omega)$ mit $\omega$ wächst.
Zur Erklärung der Dispersion nutzt man elementar die Theorie, dass  durch das elektrische Feld der elektromagnetischen Welle atomare Dipole aus ihrer anfänglichen Ruhelage induziert werden. Genauer gesagt werden atomare Ladungen um den Betrag $r$ ausgelenkt, sodass ein Dipolmoment von $\vec{p}=e\cdot\vec{r}$ entsteht. Damit können  wir für diese Ladungen im Potential $V$ folgende Bewegungsgleichung aufstellen:

\begin{equation*}
m \ \ddot{\vec{r}} \ + \ \pdiff{V(\vec{r})}{\vec{r}} \ = \ -e \ \vec{E}_{\text{lok}}
\end{equation*}

Für kleine Auslenkungen können wir das Potential, also die Bindungsenergie, harmonisch nähern:

\begin{align*}
V(\vec{r})  \ &= \ V(0) \ + \ \frac{m\omega_0^2}{2}\vec{r}^2 \ + \ \ldots\\
\ \\
\Rightarrow \qquad m \left(\ddot{\vec{r}} \ + \ \omega_0^2\vec{r}\right)  \ &= \ - e \ \vec{E}_{\text{lok}}
\end{align*}

Da das elektrische Feld $\vec{E}_{\text{lok}}$ durch die elektromagnetische Welle hervorgerufen wird und somit mit $\vec{E}_{\text{lok}} = \vec{E}_0 e^{-i\omega t}$ oszilliert, erhalten wir für $\vec{r}$ eine erzwungene Schwingung mit der Erregerfrequenz $\omega$. Daher wählen wir für $\vec{r}$ den Ansatz $\vec{r}= \vec{r}_0 e^{-i\omega t}$:

\begin{align*}
m\left(-\omega^2 \ + \ \omega_0^2\right) \ \vec{r}  \ &= \ -e \ \vec{E}_{\text{lok}}\\
\Rightarrow \qquad \vec{r}  \ &= \ -\frac{e}{m\left(\omega^2_0 \ - \ \omega^2\right)} \ \vec{E}_{\text{lok}}\\
\ \\
\Rightarrow \qquad \vec{p}  \ &= \ \frac{e^2}{m\left(\omega_0^2 \ - \ \omega^2\right)} \ \vec{E}_{\text{lok}}  \ =: \ \alpha (\omega) \ \epsilon_0 \ \vec{E}_{\text{lok}}
\end{align*}

Die Größe $\alpha$ wird auch als \textbf{atomare Polarisierbarkeit} bezeichnet.\\
Mit dem Gesetz von \textsc{Clausius - Mosotti} aus dem Kapitel 11 folgt nun das \textbf{Gesetz von \textsc{Lorenz - Lorentz}}:

\begin{align*}
\frac{\epsilon_r \ - \ 1}{\epsilon_r \ + \ 2} \ &= \ \frac{1}{3} \mathcal{N} \cdot \alpha \qquad\qquad \text{mit }\mathcal{N}= \text{ Dichte der atomaren Dipole}\\
\ \\
\Rightarrow\qquad \frac{n^2 \ - \ 1}{n^2 \ + \ 2} \ &= \ \frac{1}{3} \ \frac{\mathcal{N} \ e^2}{\epsilon_0 m \left(\omega_0^2 \ - \ \omega^2\right)}
\end{align*}

Stellen wir dies nun nach $n$ um erhalten wir somit die Abhängigkeit $n(\omega)$ für normale Dispersion:

\begin{equation*}
n^2 \ = \ \frac{\mathcal{N} \ e^2}{\epsilon_0 m \left(\omega_0^{,2} \ - \ \omega^2\right)} \qquad \qquad \text{mit} \qquad \omega_0^{,2} \ = \ \omega_0^2\ \left(1 \ - \ \frac{\mathcal{N} \ e^2}{3 \epsilon_0 m \omega_0^2}\right)
\end{equation*}

[Bild]
Anschaulich repräsentiert $\omega'$ den Einfluss der Inhomogenität des Feldes. Die erhaltene Abhängigkeit $n(\omega)$ können wir nun für verschiedene Fälle diskutieren:

\begin{align*}
\omega \ < \ \omega^{,}_0:& \qquad n^2 \ > \ 1 \qquad \Rightarrow\qquad \vec{p},\vec{E} \text{ in Phase}\\
\omega \ = \ \omega^{,}_0:& \qquad \text{Resonanz}\\
\omega \ > \ \omega^{,}_0:& \qquad \vec{p},\vec{E} \text{ antiphasig}\\
\omega \rightarrow \omega^{,}_0:& \qquad n^2 \ \rightarrow \ 1, \vec{p} \ \rightarrow \ 0 \; \text{ atomare Dipole können nicht  folgen}
\end{align*}

Zudem ist für einen kleinen Bereich oberhalb von $\omega'_0$ das Quadrat der Brechzahl negativ. Da daraus $k^2 = \frac{\omega^2}{c^2} n^2 < 0$ folgt, muss gelten, dass $k=ik'$ komplex ist. Damit wird aus $e^{ikx} \rightarrow e^{-k'x}$; das Feld fällt also im inneren des Dielektrikums analog zur Totalreflexion exponentiell ab.\\
\ \\
Bis hierher haben wir bei unseren Betrachtungen immer sehr ideale Bedingungen vorausgesetzt, nämlich dass die atomaren Dipole ungedämpft oszillieren. Im Realen muss diese Dämpfung allerdings mitbeachtet werden, da dem System dissipativ Energie verloren geht. Mit der Einführung einer Dämfungskonstanten $\gamma$ ehrhalten wir nun folgende DGL, welche wir abermals mit dem Ansatz $\vec{r}= \vec{r}_0 e^{-i\omega t}$ lösen:

\begin{align*}
m \ \left(\ddot{\vec{r}} \ + \ \omega_0^2 \vec{r} \ + \ \gamma \dot{\vec{r}}\right)  \ &= \ -e \ \vec{E}_{\text{lok}}\\
m \ \left(\omega_0^2 \ - \ \omega^2 \ - \ i \gamma\omega\right)\vec{r} \ = \ -e \ \vec{E}_{\text{lok}}
\end{align*}

Effektiv wird  also $\omega^2$ zu $\omega^2 + i\gamma\omega$. Stellen wir nun obige Gleichung wieder mithilfe des Gesetzes von \textsc{Clausius - Mosotti} um und verwenden dabei wieder wie zuvor den Ausdruck $\omega^{,}_0$ erhalten wir für die komplexen Brechungsindex $\tilde{n}= n (1+i\kappa)$ den Ausdruck:

\begin{equation*}
\tilde{n}^2 \ = \ 1 \ + \ \frac{\mathcal{N} \ e^2}{\epsilon_0 m \left(\omega^{,2}_0 - \omega^2 - i \gamma\omega\right)}
\end{equation*}

Weiterhin muss man bei einem realen Material bedenken, dass es nicht nur einen sondern mehrere Oszillatoren gibt. Damit verändern sich unsere Größen zu:

\begin{align*}
\omega^{,}_0 \ &\rightarrow \ \omega_k\\
\gamma \ &\rightarrow \ \gamma_k\\
\frac{\mathcal{N} \ e^2}{\epsilon_0 m} \ &\rightarrow \ a \cdot f_k \qquad\qquad \text{mit}  &\sum_k \ f_k  \ &= \ 1 \quad \text{  (Oszillatorenstärke)}\\
& &a \ &= \ \Bigg\langle\frac{\mathcal{N}\ e^2}{\epsilon_0 m}\Bigg\rangle
\end{align*}

Damit wird der Ausdruck für die Dispersion $\tilde{n}(\omega)$ schlussendlich zu:

\begin{equation*}
\tilde{n}^2 \ = \ 1 \ + \ a \cdot\sum_k \ \frac{f_k}{\omega_k^2 \ - \ \omega^2 \ - \ i\gamma_k\omega}
\end{equation*}

\section{Anomale Dispersion}

Wir betrachten nun die Umgebung einer beliebigen Resonanzstelle etwas genauer und nehmen an, dass die Dämpfung $\gamma_k$ vernachlässigbar gegenüber $\omega_k$ ist. Wenn wir nur diese eine Resonanzstelle untersuchen, reicht die Vereinfachung $\omega_k \rightarrow \Omega, \ \gamma_k \rightarrow\gamma, \ a \cdot f_k \rightarrow A$. Alle anderen Beiträge zum komplexen Brechungsindex $\tilde{n}(\omega)$ seien außerdem nur  langsam veränderlich und näherungsweise reell; wir bezeichnen sie als $\overline{n}(\omega)$. Für geringe $\gamma$ erhalten wir somit wieder den normalen Dispersionsfall:

\begin{equation*}
\tilde{n}^2  \ = \  \overline{n}^2 \ + \ \frac{A}{\Omega^2 \ - \ \omega^2 \ - \ i\gamma\omega}
\end{equation*}

Die Dämfung $\gamma$ wird dann wichtig, wenn das Produkt $\gamma\omega \ \mathsmaller{\mathsmaller{\gtrsim}} \ |\Omega^2 - \omega^2|$ wird. Beziehen wir dabei mit in die Betrachtung ein, dass $\omega\approx\Omega$ gilt, folgt daraus:

\begin{equation*}
\gamma\Omega \ \mathsmaller{\mathsmaller{\gtrsim}} \ |\Omega \ - \ \omega| \cdot 2\Omega \qquad \Rightarrow \qquad \frac{\gamma}{2} \ \mathsmaller{\mathsmaller{\gtrsim}} \ |\Omega \ - \ \omega|
\end{equation*}

[hier drei Bild]
Unter diesen Bedingungen folgt für $n(\omega)$, dass es an der Resonanzstelle bei gleichzeitig starker Dämpfung \emph{abfällt}. Dieses Verhalten wird auch als \textbf{anomale Dispersion} bezeichnet.\\
Mathematisch können wir diesen Fall der nicht zu kleinen Dämpfung, für die $\frac{A}{\gamma\Omega}\ll \overline{n}^2$ gilt, folgendermaßen behandeln:

\begin{align*}
\tilde{n} \ &= \ \left(\overline{n}^2 \ + \ \frac{A}{\Omega^2 \ - \ \omega^2 \ - \ i \gamma\omega}\right)^{\frac{1}{2}}  \ = \  \overline{n} \  \left( 1 \ + \ \frac{A}{\overline{n}^2 \ \left(\Omega^2 \ - \ \omega^2 \ - \ i \gamma \omega\right)}\right)^{\frac{1}{2}}\\
&\simeq \ \overline{n} \ \left(1 \ + \ \frac{1}{2}\;\frac{A}{\overline{n}^2\left(\Omega^2 \ - \ \omega^2 \ - \ i \gamma\omega\right)}\right)\\
\ \\
n  \ &= \ \overline{n} \ + \ \frac{A}{2\overline{n}} \; \frac{\Omega^2 \ - \ \omega^2}{\left(\Omega^2-\omega^2\right)^2 \ + \ \gamma^2\omega^2}\\
\ \\
n \cdot \kappa  \ &= \ \frac{A}{2\overline{n}} \; \frac{\gamma\omega}{\left(\Omega^2-\omega^2\right)^2 \ + \ \gamma^2\omega^2} 
\end{align*}
[vllt. noch ein Bild]

\section{Metalldispersion}

Bisher hatten wir nur dielektrische Isolatoren betrachtet, in denen nur gebundene Elektronen vorliegen, für welche $\omega_k \neq 0$ gilt. Nun wollen wir uns auch mit metallischen Leiter befassen, in welchen sowohl gebundene als auch freie Elektronen vorkommen. Für Letztere gilt $\omega_k = 0$, woraus gleich zu Beginn folgt:

\begin{equation*}
\tilde{n}^2  \ = \ \tilde{\epsilon}_r  \ = \ 1 \ + \ \underbrace{a\cdot\sum_k \ \frac{f_k}{\omega_k^2 - \omega^2 - i \gamma_k\omega}}_{\text{gebundene Elektronen}} \quad - \quad \underbrace{\frac{\left(\nicefrac{\mathcal{N}e^2}{\epsilon_0m}\right)_L}{\omega^2 + i\gamma_L\omega}}_{\text{Leitungselektronen}}
\end{equation*}

Aus Kapitel 12.1 wissen wir bereits, dass für $\tilde{\epsilon}_r$ außerdem noch $\tilde{\epsilon}_r = \epsilon_r + \frac{i\sigma}{\omega\epsilon_0}$ gilt. Setze man nun die beiden Gleichungen gleich und stellt um, so erhält man für die Leitfähigkeit $\sigma$ die Abhängigkeit:

\begin{equation*}
\sigma(\omega) \ = \ \frac{i\left(\nicefrac{\mathcal{N}e^2}{m}\right)_L}{\omega + i \gamma_L} \qquad \overset{\omega\rightarrow0}{\longrightarrow}\qquad \sigma_0  \ = \ \frac{\left(\nicefrac{\mathcal{N}e^2}{m}\right)_L}{\gamma_L}	
\end{equation*}

Der erhaltene Ausdruck $\sigma_0$ entspricht der Gleichstromleitfähigkeit, wie sie auch von der \textbf{\textsc{Drude}-Theorie der Metalle} abgeleitet werden kann. Diese geht von folgender Kräftebilanz auf die freien Elektronen im Leiter aus:

\begin{align*}
\underbrace{-e\ \vec{E}}_{\textsc{Coulomb}} \quad - \quad \underbrace{m \gamma_L\vec{v}}_{\text{Reibung}}  \ \overset{!}{=} \ 0\\
\Rightarrow \qquad \vec{v}  \ = \  -\frac{e \ \vec{E}}{m \gamma_L} 
\end{align*}

Die Geschwindigkeit $\vec{v}$ setzen wir nun in die Stromdichte $\vec{j} = \rho\vec{v} = -e \mathcal{N}_L \vec{v}$ ein:

\begin{align*}
\vec{j}  \ &= \ -e^2 \frac{\mathcal{N}_L}{m\gamma_L} \ \vec{E} \ \overset{!}{=} \ \underbrace{\sigma_0 \ \vec{E}}_{\text{\textsc{Ohm}'sches Gesetz}}\\
\Rightarrow \qquad \sigma_0  \ &= \ \frac{\mathcal{N}_L \ e^2}{m \ \gamma_L} \qquad\qquad\text{wie oben}
\end{align*}

Die Dämpfungskonstante $\gamma_L = \gamma$ kann im Zusammenhang mit diesem Modell als Frequenz der Stöße der Elektronen an den Atomrümpfen im Leiter verstanden werden. Das Reziproke dieser Stoßfrequenz $\frac{1}{\gamma}=: \tau$  ist dementsprechend die Stoßzeit, also die mittlere Dauer der "freien" Bewegung der Elektronen. Im Wechselfeld erhält man für die Leitfähigkeit in Abhängigkeit von der Frequenz ebenso wie oben $\sigma(\omega) = \frac{\sigma_0}{1-i\omega\gamma_L}$\\
\ \\
Mit diesem Wissen können wir nun für Leiter folgendes formulieren:

\begin{equation*}
\tilde{\epsilon}_r \ = \ n_0^2 \ - \ \frac{\omega_{\text{Pl}}^2}{\omega^2 \ + \ i\gamma\omega} \qquad \text{mit}\qquad \omega_{\text{Pl}}^2  \ = \  \left(\frac{\mathcal{N}\ e^2}{\epsilon_0 \ m}\right)_L \ = \ \frac{\sigma_0\gamma}{\epsilon_0}
\end{equation*}

$n_0$ repräsentiert dabei den Beitrag der gebundenen Elektronen im Leiter und der neu eingeführte Ausdruck $\omega_{\text{Pl}}$ ist die sogenannte \textbf{Plasmafrequenz}, welche im folgenden Kapitel näher erläutert werden soll. Typische Materialfrequenzen $\gamma\ll\omega_{\text{Pl}}\ll\omega_L$ sind z.B. für Kupfer:

\begin{equation*}
\gamma \ \approx \ 10^{14} \ s^{-1}, \quad \omega_{\text{Pl}} \ \approx \ 3\cdot 10^{16} \ s^{-1}
\end{equation*}

\ \\
Damit lässt sich nun das gesamte Frequenzspektrum  in drei Bereiche aufteilen:\\
\ \\
\begin{enumerate}[label=\roman*)]
\item \underline{$\omega \ \ll \gamma$} \qquad\qquad\qquad \; Radiowellen:\\

\begin{align*}
\tilde{\epsilon}_r \ &\approx \ n_0^2 \ - \ \frac{\omega_{\text{Pl}^2}}{i\gamma\omega} \ \approx \ i \frac{\sigma_0}{\epsilon_0\omega} \qquad \text{(quasistatisch, s. Kap. 12.1)}\\
k \ &= \ \frac{\omega}{c} \ n  \ = \ \frac{\omega}{c} \ \sqrt{\frac{i\sigma}{\epsilon_0\omega}} \ = \ \sqrt{i\gamma_0\omega\sigma_0} \quad \Rightarrow \quad \text{Skin-Effekt}
\end{align*}
\ \\

\item \underline{$\gamma\ll\omega\ll\omega_{\text{Pl}}$} \qquad \qquad sichtbares Licht:\\

\begin{equation*}
\tilde{\epsilon}_r \ \approx \ n_0^2 \ - \ \underbrace{\frac{\omega_{\text{Pl}}^2}{\omega^2}}_{\gg 1} \ \approx \ - \frac{\omega_{\text{Pl}}^2}{\omega^2} \quad \Rightarrow\quad \tilde{n} \ = \ i \ n \kappa, \; k \ = \ i k'
\end{equation*}
 
Analog zur Totalreflexion kommt es hier also auch zu einem exponentiellen Abfall im Leiter.\\
Die räumliche Dispersion $\epsilon(\omega,k)$ wurde hierbei vernachlässigt und es kommt zum sogenannten \textbf{anomalen Skin-Effekt}, da die Eindringtiefe $\delta$ viel kleiner als die mittlere freie Weglänge $\frac{v}{\gamma}$ ist.\\ 

\item \underline{$\omega \gg \omega_{\text{Pl}}$} \qquad\qquad\qquad Röntgenwellen:\\
\begin{equation*}
\left(\frac{\omega_{\text{Pl}}}{\omega}\right)^2 \ \ll \ 1 \quad\Rightarrow\quad \epsilon_r \ \approx \ n_0^2
\end{equation*}

In diesem Falle wir der Einfluss der Leitungselektronen unwichtig und das Material erhält sich wie ein Dielektrikum.
\end{enumerate}

\section{Longitudinale Wellen}

Wir betrachten einen Leiter ohne makroskopische Ladung, sodass für die \textsc{Maxwell}-Gleichung gilt: $\div\vec{D}=0$. Für eine elektromagnetische Welle in dem Leiter mit $\vec{E}=\vec{E}_ e^{i(\vec{k}\vec{r}-\omega t)}$ folgt damit:

\begin{equation*}
\div \vec{B} \ = \ \div(\epsilon\vec{E})  \ = \ i\vec{k}\cdot \epsilon\vec{E} \ \overset{!}{=} \ 0
\end{equation*}

Bisher hatten wir daraus immer gefolgert, dass $\vec{k}\cdot\vec{E}=0$ sein muss und die Welle daher transversal ist. Allerdings wäre rein mathematisch auch die Lösung $\epsilon=0$ für diese Gleichung möglich. Könnte die Welle dann also auch longitudinal sein?\\
Der Versuch zeigt, dass $\vec{k}\parallel\vec{E}$ und somit $\epsilon =0$ möglich ist, aber bei welchen Frequenzen ist dies der Fall? Dazu betrachten wir:


\begin{equation*}
\tilde{\epsilon}_r  \ = \ n_0^2 \ - \ \frac{\omega_{\text{Pl}}^2}{\omega^2 \ + \ i \gamma\omega}
\end{equation*}

Im \emph{idealen Fall} ist $n_0=1$ und es liegt keine Dämpfung vor ($\gamma=0$) dann ergibt sich:

\begin{equation*}
\tilde{\epsilon}_r \ = \ 1 \ - \left(\frac{\omega_{\text{Pl}}}{\omega}\right)^2 \quad\Rightarrow\quad \epsilon_r \ = \ 0 \text{ bei } \omega \ = \ \omega_{\text{Pl}}
\end{equation*}

Dies gilt unabhängig von $\vec{k}$ und könnte eine longitudinale Welle repräsentieren. Diese hätte kein Magnetfeld, da $\dot{\vec{B}}= - \rot\vec{E} = - \vec{k}\times\vec{E}= 0$ und damit $\vec{B}=0$ bis auf Integrationskonstante gilt.\\
Im \emph{realen Fall} existiert allerdings Dämpfung und und räumliche Dispersion $\epsilon(\omega,\vec{k})$, daher muss eine longitudinale Welle einen anderen physikalischen Ursprung haben. Um diesen zu klären, kehren wir noch einmal zum Begriff \ \grqq Plasmafrequenz\grqq{}  zurück.  Dieser rührt von der Tatsache her, dass sich die Elektronenwolke im Leiter wie ein Plasma, also ein Gas aus ionisierten Teilchen verhält. Lenkt man nun diese Leitungselektronen um den Betrag $\Delta x \sim e^{ikx}$ aus, so entstehen lokale Ladungsdichtegradienten. Diese rufen wiederum ein $\vec{E}$-Feld hervor, welches seinerseits eine rücktreibende Kraft auf die ausgelenkten Elektronen ausübt. Dies hat zur Folge, dass es zu longitudinalen Oszillationen des Elektronengases mit der Plasmafrequenz als Eigenfrequenz kommt, welche man dann als \grqq longitudinale Welle\grqq{} oder auch \textbf{\grqq Plasmon\grqq{}} bezeichnet.  Es handelt sich dabei also um eine Plasmaschwingung durch eine Dichtewelle der Leitungselektronen.
[Bild]

\section{Gruppengeschwindigkeit}

Für eine harmonische Welle der Form $U \sim e^{ik(x-\frac{\omega}{k}t)}$ hatten wir uns bereits den Begriff der Phasengeschwindigkeit $c_{\text{Ph}} = \frac{\omega}{k}$ definiert, welche im Vakuum $c_{\text{Ph}}=c_0$ und im Medium $c_{\text{Ph}}= \frac{C_0}{n(\omega)}$ ist.\\
\ \\
Nun betrachten wir die Überlagerung zweier Wellen mit den Frequenzen $\omega_{\nicefrac{1}{2}} = \omega \pm \frac{\Delta\omega}{2}$. Weiterhin sei $\Delta\omega\ll\omega, \; k_{\nicefrac{1}{2}} = k \pm \frac{\Delta k}{2}, \; \Delta k \ll k$, sodass wir für die durch die Überlagerung entstandene Welle erhalten:

\begin{align*}
U(x,t) \ &= \ e^{i(k_1 x -\omega_1 t)} \; + \; e^{i(k_2 x - \omega_2 t)} \; = \; e^{i(kx-\omega t)} \cdot \left(e^{i\left(\frac{\Delta k}{2}x - \frac{\Delta \omega}{2}t \right)} \; + \; e^{-i\left(\frac{\Delta k}{2}x - \frac{\Delta \omega}{2}t\right)}\right)\\
&= \ 2 \ e^{ik(x-\frac{\omega}{k} t)} \ \cos\left(\frac{\Delta k}{2}\left(x \ - \ \frac{\Delta \omega}{\Delta k}t\right)\right)
\end{align*}

Der erhaltene Ausdruck besteht dementsprechend aus zwei Teilen:\\
\begin{enumerate}[label=\roman*)]
\item einem schnell veränderlichen Teil, dessen Oszillation sich mit der Phasengeschwindigkeit $c_{\text{Ph}} =\frac{\omega}{k}$ verschiebt und

\item einem langsam veränderlichen Teil, auch \textbf{Modulation} genannt, dessen Oszillation sich mit der \textbf{Gruppengeschwindigkeit} $c_{\text{Gr}} = \frac{\Delta \omega}{\Delta k}$ verschiebt.
\end{enumerate}

[hier definitiv ein Bild hin]

Für ein allgemeines Wellenpaket gilt:

\begin{equation*}
U(\vec{r},t)  \ = \ \int\d^3 k  \tilde{U}(\vec{k}) \ e^{i(\vec{k}\vec{r}-\omega t)}
\end{equation*}

Der Wellenvektor $\vec{k}$ lässt sich dabei in zwei Teile aufteilen: ein zentrales $\vec{k}_0$ und ein $\vec{k}'$, für welches gilt: $|\vec{k}'| \mathsmaller{\mathsmaller{\lesssim}} \Delta k$, wobei $\Delta k$ die Breite der $\vec{k}$-Verteilung um $\vec{k}_0$ ist.\\
Wir beginnen unsere Betrachtung zum Zeitpunkt $t=0$:

\begin{equation*}
U(\vec{r},t=0) \ = \ e^{i\vec{k}_0 \cdot\vec{r}} \; \underbrace{\int\d^3 k' \ \tilde{U}(\vec{k}_0 + \vec{k}') \ e^{i\vec{k}'\cdot\vec{r}}}_{=: \phi(\vec{r})}
\end{equation*}

$\phi(\vec{r})$ ist dabei nur langsam veränderlich auf einer Skala $|\Delta \vec{r}| = \frac{1}{|\Delta \vec{k}|} \gg \frac{\lambda}{2\pi}= \frac{1}{|\vec{k}_0|}$.\\
Für jeden anderen Zeitpunkt $t \neq 0$ gilt:

\begin{align*}
\omega(\vec{k}) \ &= \ \underbrace{\omega(\vec{k}_0)}_{=:\omega_0} \ + \ \left.\left(\vec{k}' \ \pdiff{}{\vec{k}}\right)\omega\right|_{k_0} \ + \ \frac{1}{2} \ \left.\left(\vec{k}'\pdiff{}{\vec{k}}\right)^2\omega\right|_{k_0} \ + \ \ldots\\
\ \\
U(\vec{r},t)  \ &= \ e^{i(\vec{k}_0\vec{r}-\omega_0 t)} \; \int\d^3 k' \ \tilde{U}(\vec{k}_0 + \vec{k}') \ e^{i\vec{k}\left(\left.\vec{r}-\pdiff{\omega}{\vec{k}}\right|_{k_0} \cdot t\right)} \ \underbrace{e^{-\frac{1}{2}\left.\left(\vec{k}' \cdot \pdiff{}{\vec{k}}\right)^2\omega\right|_{k_0} \cdot t}}_{\text{zunächst } \rightarrow 1 \; \; (*)}\\
&= \ e^{i(\vec{k}_0\vec{r} - \omega_0 t)} \ \phi\left(\vec{r} \ - \ \left.\pdiff{\omega}{\vec{k}}\right|_{k_0} \cdot t \right)
\end{align*}

Damit erhalten wir also für die allgemeine Phasengeschwindigkeit $c_{\text{Ph}} = \frac{\omega_0}{k_0}$ (bzw. $\vec{c}_{\text{Ph}} = \frac{\vec{k}_0}{k_0^2}\omega_0$) und für die allgemeine Gruppengeschwindigkeit $\vec{c}_{\text{Gr}} = \left.\pdiff{\omega}{\vec{k}}\right|_{k_0}$\\
Die physikalische Bedeutung der Gruppengeschwindigkeit ist die Ausbreitung des Energietransports mit ihr, daher gilt im Allgemeinen für die Energiestromdichte: $\vec{S}_P = c_{\text{Gr}} \cdot w \cdot \vec{e}_r$. Also kann höchstens mit der Gruppengeschwindigkeit auch physikalische Wirkungen übertragen werden, weshalb in der Nachrichtentechnik auch häufig von "Signalgeschwindigkeit" geredet wird. Für sie gilt \emph{immer} $c_{\text{Gr}} \leq c_0$, wohingegen die Phasengeschwindigkeit $c_{\text{Ph}}$ auch größer als  $c_0$ sein kann.\\ 
\ \\
In der Herleitung der allgemeinen Phasen- bzw. Gruppengeschwindigkeit haben wir an einer Stelle die \textsc{Taylor}-Entwicklung schon vor dem Term quadratischer Ordnung $(*)$ abgebrochen, allerdings kann dieser und auch folgende nicht zu allen Zeiten vernachlässigt werden. Die Näherung \grqq{}$\left(\pdiff{}{\vec{k}}\right)^2\omega\rightarrow 0$\grqq{} ist ungültig, wenn $\left(\vec{k}'\pdiff{}{\vec{k}}\right)^2\omega \cdot t \mathsmaller{\mathsmaller{\gtrsim}} 1$ gilt, bzw. $t \mathsmaller{\mathsmaller{\gtrsim}} \left((\Delta k)^2 \pddiff{\omega}{\vec{k}}\right)^{-1}$.\\
Dann gilt:

\begin{equation*}
U(\vec{r},t) \ = \  e^{i(\vec{k}_0\vec{r}-\omega_0 t)} \ \phi(\vec{r} - \vec{c}_{\text{Gr}} \cdot t,t)
\end{equation*}

Durch die explizite Zeitabhängigkeit von $\phi$ kommt es zu Signalverzerrungen und man muss in diesem Falle die Gruppengeschwindigkeit anders definieren:

\begin{align*}
\vec{r}_S \ &= \ \langle\vec{r}\rangle \ = \ \frac{\Int{}{}{V} \ \vec{r} \ |U(\vec{r},t)|^2}{\Int{}{}{V} \ |U(\vec{r},t)|^2} \ =: \  \vec{r}_0 \ + \ \vec{c}_{\text{Gr}} t \qquad \text{Schwerpunkt des Wellenpakets}\\
\ \\
\vec{c}_{\text{Gr}}  \ &= \ \frac{\int\d^3 k \ \pdiff{\omega}{k} \ |\tilde{U}(\vec{k})|^2}{\int\d^3 k \ |\tilde{U}(\vec{k})|^2} \ = \ \Bigg\langle\pdiff{\omega}{\vec{k}}\Bigg\rangle
\end{align*}

\ \\
Für den dispersionsfreien Fall folgt für die Phasengeschwindigkeit $c_{\text{Ph}} =: c = \text{const.}$ und die Gruppengeschwindigkeit folgender einfacher Zusammenhang:

\begin{equation*}
\vec{c}_{\text{Gr}} \ = \ \pdiff{\omega}{\vec{k}} \ = \ \vec{e}_k\cdot c \quad\Rightarrow\quad c_{\text{Gr}} \ = \ c_{\text{Ph}}
\end{equation*}

Allerdings sind auch noch in diesem Falle Verzerrungen möglich, und zwar wenn $\pdiff{}{\vec{k}}\circ\pdiff{\omega}{\vec{k}}\neq 0$, d.h. wenn c richtungsabhängig ist.
\chapter{Kovariante Formulierung der Elektrodynamik}

\section{Raum-Zeit-Begriff und Lorentz-Transformation}

Bevor wir zur eigentlichen Elektrodynamik kommen, müssen zunächst ein paar essentielle Begriffe eingeführt werden:\\

Ein \textbf{Bezugssystem} ist ein Koordinatensystem zu Bestimmung der räumlichen Lage $\vec{r}$ eines Teilchens, zusammen mit einer Uhr für die Zeit $t$. Es stellt sich heraus, dass Ort und Zeit relativ, also abhängig vom Bezugssystem sind.\\

Ein \textbf{Inertialsystem} ist ein Bezugssystem, in dem ein sich frei bewegender Körper eine konstante Geschwindigkeit besitzt.\\

Das \textbf{Relativitätsprinzip} besagt, dass die Naturgesetze in allen Inertialsystem diesselbe Form haben. Wir werden in diesem Zusammenhang später auf Begriffe wie ''Invarianz'' oder ''Kovarianz'' stoßen.\\ \linebreak

\textbf{a.\ Galillei}\\

Damals ging man vom Relativitätsprinzip und von der instantanen Ausbreitung von Wirkungen aus. Demnach hingen Kräfte nur von der aktuellen Position der Teilchen ab.\\
\newpage

Der Wechsel zwischen zwei mit der Relativgeschwindigkeit $\vec{v}$ zueinander bewegten Inertialsystemen wird durch die \textsc{Galilei}-Transformation beschrieben. 
\begin{equation*}
\vec{r}'=\vec{r}+\vec{v}t
\end{equation*}
Wichtig ist dabei $t=t'$.\\ \linebreak

\textbf{b.\ Einstein}\\

Es wird neben Relativitätsprinzip zusätzlich noch der Ausbreitung von Wirkungen mit Lichtgeschwindigkeit ausgegangen. Dies hat in \emph{jedem} Intertialsystem denselben Wert! \\
Der Übergang zwischen zwei System wird hier durch die \emph{Lorentz}-Transformation realisiert.\\ \linebreak

Zur eleganten Formulierung der Speziellen Relativitätstheorie (es werden nur Intertialsysteme betrachtet) führt man den \textsc{Minkowski}-Raum ein, der neben den drei Raumdimensionen noch die Zeit als eigene Dimension enthält. \\
Punkte in diesem Raum sind natürlich keine \emph{Orte}, sondern \emph{Ereignisse}.\\
Die Elemente dieses Raums bezeichnen wir als \emph{Vierervektoren} und verwenden die Schreibweise
\begin{equation*}
x^\mu = (x^0,x^1,x^2,x^3)=(ct,x,y,z).
\end{equation*}
In der Speziellen Relativitätstheorie (SRT) ist es üblich, für Indizes, die die Werte 0 bis 3 annehmen, griechische Buchstaben zu verwenden.\\

Abstände zwischen zwei Punkten (Ereignissen) werden nach
\begin{equation*}
\mathrm{d}s^2 = c^2\mathrm{d}t^2 -\mathrm{d}x^2 - \mathrm{d}y^2 - \mathrm{d}z^2 
\end{equation*}
definiert. Das lässt sich alternativ auch unter Verwendung des metrischen Tensors $g_{\mu\nu}$ ausdrücken.
\begin{equation*}
\mathrm{d}s^2 = g_{\mu\nu}\mathrm{d}x^\mu\mathrm{d}x^\nu \qquad \text{mit}\ g_{\mu\nu}=\begin{pmatrix}
1 & 0& 0 & 0 \\
0 &-1& 0 & 0\\
0 & 0 & -1 & 0 \\
0 & 0 & 0 & -1 \\
\end{pmatrix}
\end{equation*}
\emph{Bemerkung: Hier wird die \textsc{Einstein}sche Summenkonvention verwendet. Über Indizes, die einmal oben und einmal unten auftauchen, wird summiert!}\\

Für die Lichtausbreitung gilt $\mathrm{d}s=0$ in allen Intertialsystemen. \\
Betrachten wir nun zwei Inertialsysteme $k$ und $k'$, die sich mit konstanter Geschwindigkeit $\vec{v}$ relativ zueinander bewegen und zwei Ereignisse mit den infinitessimalen Abständen $\mathrm{d}s$ und $\mathrm{d}s'$. \\
Wir zeigen nun, dass eine Lorentztransformation immer $\mathrm{d}s=\mathrm{d}s'$ gewährleistet. \\

Aus der Konstant von $c$ folgt, dass $\mathrm{d}s$ genau dann Null sein muss, wenn $\mathrm{d}s'$ auch Null ist. Der Zusammenhang zwischen den beiden muss also linear sein!\\
Der der Raum homogen und isotrop ist, kann dieser Zusammenhang nicht von den Orten der Bezugssysteme, sondern nur vom Betrag der Relativgeschwindigkeit $|\vec{v}|$ abhängen. Jetzt nehmen wir uns drei Inertialsysteme $k_1,\ k_2$ und $k_3$. 
\begin{align*}
\mathrm{d}s_1^2 &= a(v_{12})\mathrm{d}s_2^2\\
\mathrm{d}s_2^2&=a(v_{23})\mathrm{d}_3^2\\
\mathrm{d}s_1^2 &=a(v_{13})\mathrm{d}s_3^2
\end{align*}
Das heißt
\begin{equation*}
a(v_{23})=\frac{a(v_{13})}{a(v_{12})}.
\end{equation*}
$v_{23}$ muss aber von $v_{12}$, $v_{13}$ \emph{und} dem Winkel zwischen den beiden abhängen. Deshalb müssen alle $a=1$ sein. Der Viererabstand $\mathrm{d}s$ ist also lorentzinvariant.\\

Für \textbf{zeitartige Abstände} ($\mathrm{d}s^2>0$) existiert für zwei Ereignisse ein Intertialsystem $k'$, in dem beide am gleichen Ort stattfinden, denn
\begin{equation*}
c^2\mathrm{d}t'^2 = c^2\mathrm{d}t^2 - \mathrm{d}\vec{r}^2 \qquad \Rightarrow \mathrm{d}\vec{r}'^2 = 0.
\end{equation*} 
\ \\

Für \textbf{raumartige Abstände} ($\mathrm{d}s^2<0$) existiert für zwei Ereignisse ein Inertialsystem, in dem beide gleichzeitig stattfinden, denn
\begin{equation*}
-\mathrm{d}\vec{r}^2 =c^2\mathrm{d}t^2-\mathrm{d}\vec{r}^2 \qquad \Rightarrow \mathrm{d}t'^2 =0.
\end{equation*}
Da $\mathrm{d}s^2$ lorentzinvariant ist, sind raum- und zeitartige Abstände absolut.\\

Als \textbf{Eigenzeit} eines Beobachters oder Teilchens bezeichnet man die Zeit, die von der Uhr angezeigt wird, die sich mit ihm mitbewegt. Dabei kann der Beobachter beliebig bewegt (sogar beschleunigt) sein. Eine Uhr mit einem fest verbundenen Koordinatensystem, das nicht unbedingt ein Inertialsystem sein muss, definiert ihr \emph{momentanes Inertialsystem}.
\begin{align*}
\mathrm{d}s^2 &= c^2\mathrm{d}t^2 - \mathrm{d}\vec{r}^2 = c\mathrm{d}\tau^2-0\\
\mathrm{d}\tau^2 &= \mathrm{d}t^2\left(1-\frac{1}{c^2}\left(\diff{\vec{r}}{t}\right)^2\right)=\mathrm{d}t^2\left(1-\frac{v^2}{c^2}\right)\\
%\Rightarrow \mathrm{d}t &= \frac{\mathrm{d}\tau}{\sqrt{1-\frac{v^2}{c^2}}}\geq \mathrm{d}\tau
\end{align*}
Die letzte Zeile bezeichnet den Effekt der \emph{Zeitdilatation}.\\ \linebreak

\textbf{\textsc{Lorentz}-Transformation}\\

Es handelt sich dabei um eine Transformation der Koordinaten $(t,\vec{r})$ eines Ereignisses im Inertialsystem $k$ in die Koordinaten $(t',\vec{r}')$ desselben Ereignisses in $k'$. Die Prämisse dabei ist, dass die Transformation aufgrund der Homogenität von Raum und Zeit linear sein muss und $\mathrm{d}s^2$ konstant sein muss. \\
Wir betrachten o.B.d.A. den Fall $\vec{e}_x\parallel\vec{e}_x'\parallel\vec{v}$. Der allgemeine Ansatz ist
\begin{align*}
t'&=a(x-wt)\\
x'&=a(x-ut)\\
y'&=y\\
z'&=z
\end{align*}
Natürlich muss $u=v=|\vec{v}|$ sein. Ebenso müssen $a$ und $b$ denselben Wert haben. Wir fordern
\begin{align*}
\mathrm{d}s^2 &=\mathrm{d}s'^2\\
c^2\mathrm{d}t^2-\mathrm{d}x^2 &=a^2c^2(\mathrm{d}t+w\mathrm{d}x)^2 - a^2(\mathrm{d}x-v\mathrm{d}t)^2
\end{align*}
Ein Koeffizientenvergleich liefert 
\begin{align*}
a&=\frac{1}{\sqrt{1-\frac{v^2}{c^2}}} & w&=-\frac{v}{c^2}.
\end{align*}
Wir kürzen $(1-\frac{v^2}{c^2})^{\frac{1}{2}}$ mit $\gamma$ ab und erhalten die spezielle (nicht kommutative) \textsc{Lorentz}-Transformation
\begin{align*}
x'&=\gamma(x-vt) & x &=\gamma(x'+vt')\\
t'&=\gamma\left(t-\frac{v}{c^2}x\right) & t &=\gamma\left(t'+\frac{v}{c^2}x'\right)\\
y &=y'   & z' &=z
\end{align*}

Eine Folgerung der \textsc{Lorentz}-Transformation ist die \emph{Längenkontraktion}:\\
Ist ein Objekt in $k'$ in Ruhe und habe dort die Länge $l_0$. In $k$ hat es jedoch die Länge
\begin{equation*}
l = \frac{l_0}{\gamma} < l_0.
\end{equation*}
Geschwindigkeiten transformieren nach
\begin{align*}
u_x' &= \diff{x'}{t'}=\frac{\mathrm{d}x-v\mathrm{d}t}{\mathrm{d}-\frac{v}{c^2}\mathrm{d}x}=\frac{u_x-v}{1-\frac{vu_x}{c^2}}\\
u_y' &=\frac{u_y\sqrt{1-\frac{v^2}{c^2}}}{1-\frac{vu_x}{c^2}}
\end{align*}
beziehungweise
\begin{align*}
u_\parallel' &=\frac{u_\parallel-v}{1-\frac{\vec{v}\vec{u}}{c^2}}\\
u_\perp' &= \frac{u_\perp\sqrt{1-\frac{v^2}{c^2}}}{1-\frac{\vec{v}\vec{u}}{c^2}}
\end{align*}
Im Grenzfall kleiner Geschwindigkeiten $\vec{v}$ geht die \textsc{Lorentz}-Transformation in eine \textsc{Galilei}-Transformation über.

\section{Vierergrößen und Kovarianz}

Im dreidimensionalen Raum transformieren sich die Komponenten eines Vektors bei Drehung.
\begin{equation*}
\vec{r}'=\tens{R}\vec{r} \qquad (\det\tens{R}=\pm 1)
\end{equation*}
Skalare (z.B. $\vec{r}^2=x^2+y^2+z^2$, $\vec{A}\cdot\vec{B}=A_xB_x+A_yB_y+A_zB_z$) bleiben unter Drehung invariant. Im vierdimensionalen \textsc{Minkoswki}-Raum bleibt das im Prinzip erhalten. Aufgrund der Abstandsdefinition handelt es sich jedoch um eine nicht-euklidische Metrik.\\
Wir definieren
\begin{align*}
x^\mu &= (ct,x,y,z)=(x^0,x^1,x^2,x^3)=(ct,\vec{r})=(ct,\vec{x})\\
\mathrm{d}x^\mu & = (c\mathrm{d}t,\mathrm{d}\vec{r})
\end{align*}
Die Transformationseigenschaften von $x^\mu$ und $\mathrm{d}x^\mu$ sind bekannt. Eine \textsc{Lorentz}-Transformation entspricht einer Drehung im \textsc{Minkowski}-Raum.\\

\textbf{Kontravariante Vektoren}

\begin{equation*}
A^\mu = (A^0,A^1,A^2,A^3)
\end{equation*}
Sie transformieren sich wie $x^\mu$, das heißt
\begin{equation*}
A'^\mu =\pdiff{x'^\mu}{x^\nu}A^\nu
\end{equation*}
\ \\

\textbf{Kovariante Vektoren}
\begin{equation*}
A_\mu=(A_0,A_1,A_2,A_3)
\end{equation*}
mit
\begin{equation*}
A_\mu'=\pdiff{x^\nu}{x'^\mu}A_\nu
\end{equation*}

Aus diesen Transformationseigenschaften folgt, wie wir später sehen werden,
\begin{equation*}
A_\nu = g_{\mu\nu}A^\mu.
\end{equation*}

\textbf{Skalarprodukt}\\

Das Skalarprodukt zwischen zwei Vierervektoren definiert man durch
\begin{equation*}
A\cdot B = A_\mu\cdot B^\mu = g_{\mu\nu}A^\mu B^\nu = A^\mu B_\mu.
\end{equation*}
Es ist invariant unter \textsc{Lorentz}-Transformation, denn
\begin{equation*}
A'\cdot B' = \pdiff{x^\nu}{x'^\mu}\pdiff{x'^\mu}{x^\kappa} A_\mu B^\kappa = \pdiff{x^\nu}{x^\kappa}B^\kappa A_\nu = A\cdot B.
\end{equation*}

\textbf{Indexkontraktion mit $g$}

\begin{equation*}
g_{\mu\nu}=g^{\mu\nu}=\begin{pmatrix}
1 & 0& 0 & 0 \\
0 &-1& 0 & 0\\
0 & 0 & -1 & 0 \\
0 & 0 & 0 & -1 \\
\end{pmatrix}, \qquad g_{\mu\nu}g^{\nu\kappa}=\delta_\mu^\kappa
\end{equation*}
\begin{equation*}
\mathrm{d}s^2 = g_{\mu\nu}\mathrm{d}x^\mu\mathrm{d}x^\nu=\mathrm{d}x_\nu\mathrm{d}x^\mu
\end{equation*}

\textbf{Ableitungen im \textsc{Minkowski}-Raum}\\

Beim Differenzieren ist darauf zu achten, dass die Ableitung nach einem kontravarianten selbst wieder einen kovarianten Vektor liefert. Wir definieren
\begin{align*}
\partial^\alpha &= \pdiff{}{x_\alpha}=\left(\pdiff{}{x^0},-\nabla\right)\\
\partial_\alpha &= \pdiff{}{x^\alpha}=\left(\pdiff{}{x_0},\nabla\right).
\end{align*}
Diese Schreibweisen verdeutlichen, dass sowohl die Viererdivergenz, also auch der Wellenoperator invariant sind.
\begin{align*}
\partial_\alpha A^\alpha &= \partial^\alpha A_\alpha =\pdiff{A^0}{x^0}+\div\vec{A}\\
\partial_\alpha\partial^\alpha&=\partial^\alpha\partial_\alpha = \left(\pdiff{}{x^0}\right)^2-\nabla^2
\end{align*}

\text{Matrixschreibweise der \textsc{Lorentz}-Transformation}\\

Wie bereits erwähnt beschreibt eine solche Transformation nichts anderes als eine Drehung im \textsc{Minkowski}-Raum. Sie lässt sich also bei gegebener Relativgeschwindigkeit eindeutig als Matrix ausdrücken.
\begin{equation*}
x'^\mu = \Omega_\nu^\mu x^\nu \qquad \text{mit}\quad \Omega_\nu^\mu = \pdiff{x'^\mu}{x^\nu}
\end{equation*}
In unserem Beispiel von ebene (Relativbewegung in $x$-Richtung) sieht die Matrix dann so aus
\begin{equation*}
\Omega_\nu^\mu = \begin{pmatrix}
\gamma &-\beta\gamma & 0 & 0 \\
-\beta\gamma & \gamma & 0 & 0 \\
0 & 0 & 1 & 0\\
0 & 0 & 0 & 1
\end{pmatrix}  \qquad \text{mit}\quad \gamma=\frac{1}{\sqrt{1-\frac{v^2}{c^2}}}, \quad \vec{\beta} = \frac{\vec{v}}{c}
\end{equation*}
Die allgemeine \textsc{Lorentz}-Transformation hat 6 verschiedene Generatoren. 3 Boosts und 3 Drehungen.
\newpage
\section{Relativistische Mechanik}

Damit wir eine lorentz-invariante Mechanik formulieren können, müssen wir die auftretenden Zeitableitungen, wie etwa
\begin{align*}
\dot{\vec{r}} \quad \rightarrow \quad \vec{v}\\
\dot{\vec{p}}\quad\rightarrow\quad\vec{F}
\end{align*}
mit der Eigenzeit $\tau$ bilden. 
\begin{equation*}
\diff{}{t}\quad\rightarrow\quad\diff{}{\tau}=\gamma\diff{}{t}
\end{equation*}
So ergibt sich sofort die \emph{Vierergeschwindigkeit}
\begin{align*}
u^\mu &= \diff{x^\mu}{\tau} = \gamma(c,\vec{v})\\
\text{mit}\quad u^\mu u_\mu &= \frac{1}{\gamma^2}(c,\vec{v})\cdot (c,-\vec{v}) = \frac{c^2-v^2}{1-\frac{v^2}{c^2}} = c^2,
\end{align*}
deren Betrag immer gleich $c$ ist. Wir multiplizieren $u^\mu$ nun einfach mit der (invarianten) Ruhemasse $m_0$ und erhalten den \emph{Viererimpuls}
\begin{equation*}
p^\mu = m_0 u^\mu = \gamma m_0 \left(c,\vec{v}\right)\qquad\text{mit}\quad p_\mu p^\mu = m_0^2c^2.
\end{equation*}

Eine weitere Zeitableitung wird uns die \emph{Viererkraft} liefern. Für die Ortskomponenten gilt
\begin{equation*}
\diff{}{\tau}\vec{p}=\gamma\diff{}{t}\vec{p}=\gamma\vec{F}.
\end{equation*}
Für $u_\mu F^\mu =0$ ergibt sich
\begin{equation*}
F^\mu = \gamma\left(\frac{\vec{v}\vec{F}}{c},\vec{F}\right).
\end{equation*}
Betrachten wir nun noch die Energie. Nullkomponente $F^0 = \diff{p^\mu}{\tau}$ liefert
\begin{equation*}
\diff{}{t}\gamma m_0c^2 =:\vec{v}\cdot\vec{F},
\end{equation*}
was genau einer Leistung entspricht. Deshalb muss 
\begin{equation*}
E = \gamma m_0 c^2 =: m(v)c^2
\end{equation*}
sein. So kann man den Vierimpuls also auch als
\begin{equation*}
p^\mu =\left(\frac{E}{c},\vec{p}\right)\qquad\text{mit}\quad p^\mu p_\mu = \left(\frac{E}{c}\right)^2 - \vec{p}^2 \ \stackrel{!}{=}\ m_0^2c^2
\end{equation*}
schreiben. Nun folgt aus der letzten Beziehung schließlich die \emph{relativistische Energie-Impulsbeziehung}
\begin{align*}
E^2 &= \left(m_0c^2\right)^2 + \left(p c\right)^2\\
E &= \sqrt{m_0^2c^4 + p^2c^2}.
\end{align*}
Im Grenzfall $v<<c$ wird die Energie zu
\begin{equation*}
E = m_0c^2\sqrt{1+\frac{p^2}{m_0^2c^2}} \simeq m_0c^2 + \frac{p^2}{2m_0}.
\end{equation*}
Solle $v=c$ sein, würde die Energie nur dann nicht unendlich werden, wenn $m_0=0$ ist. Die Umkehrung gilt natürlich ebenso: Alle masselosen Teilchen bewegen sich mit Lichtgeschwindigkeit.

\section{Vierdimensionale Elektrodynamik}

Im Gegensatz zur Mechanik ist die klassische Elektrodynamik bereits invariant unter \textsc{Lorentz}-Transformation. Sie enthält die Lichtgeschwindigkeit explizit als Konstante. Die kovariante Formulierung macht jedoch viele Gleichungen einfacherer.\\

\textbf{a.\ Kontinuitätsgleichung}\\ 

Aus der dreidimensionalen Formulierung kennen wir die fundamentale Kontinuitätsgleichung. Diese wird unter Einführung der \emph{Viererstromdichte}
\begin{equation*}
j^\mu  = (\rho c, \vec{j})
\end{equation*}
und der Viererdivergenz ganz einfach zu
\begin{equation*}
\partial_\mu j^\mu = \dot{\rho}+\div\vec{j}= 0 .
\end{equation*}
Am Beispiel eines Konvektionsstroms $\vec{j}=\rho\vec{v}$ sieht man leicht
\begin{equation*}
j^\mu = \rho(c,\vec{v}) = \frac{\rho}{\gamma} \cdot\gamma(c,\vec{v}) = \rho_0 u^\mu,
\end{equation*}
wobei $\rho_0$ eine invariante Ruheladungsdichte ist.\\
Aus \textsc{Lorentz}-Kraft wird
\begin{equation*}
F^\mu =\gamma\left(\frac{\vec{v}\vec{F}}{c},\vec{F}\right)= QF^{\mu\nu}u_\nu
\end{equation*}
mit der kovarianten Vierergeschwindigkeit $u_\nu$. Der Ausdruck $F^{\mu\nu}$ ist ein Konstrukt, das man den \emph{Feldstärketensor} nennt. Seine Komponenten erhalten wir, indem wir die Felder in $F^\mu$ einsetzen und einen Koeffizientenvergleich machen.
\begin{equation*}
F^{\mu\nu}=\begin{pmatrix}
0 & -\frac{E_1}{c} & -\frac{E_2}{c} & -\frac{E_3}{c}\\
\frac{E_1}{c} & 0 & -B_3 & B_2\\
\frac{E_2}{c} & B_3 & 0 & -B_1\\ 
\frac{E_3}{c} & -B_2 & B_1 & 0
\end{pmatrix} = \begin{pmatrix}
0 & -\frac{\vec{E}^T}{c} \\
\frac{\vec{E}}{c} & \mathbbm{1}\times\vec{B}
\end{pmatrix}
\end{equation*}
Die Schreibweise $\mathbbm{1}\times\vec{B}$ erweist sich als sehr effizient, um den Feldstärketensor in eine kompaktere Form zu bringen. Sie erfüllt
\begin{equation*}
(\mathbbm{1}\times\vec{B})_{kl}=\vec{e}_k(\mathbbm{1}\times\vec{B})\vec{e}_l = (\vec{e}_k\times\vec{B})\vec{e}_l = -(\vec{e}_k\times\vec{e}_l)\vec{B}
\end{equation*}
Der entsprechende Tensor mit gesenkten (kovarianten) Indizes ist
\begin{equation*}
F_{\mu\nu}= \begin{pmatrix}
0 & \frac{\vec{E}^T}{c} \\
-\frac{\vec{E}}{c} & \mathbbm{1}\times\vec{B}\end{pmatrix}
\end{equation*}
\ \\
\textbf{b. Inhomogene \textsc{Maxwell}-Gleichungen}
\begin{align*}
\mu_0\epsilon_0 c \ \div\vec{E}\quad&=\quad\mu_0c\rho\\
\rot \vec{B} -\mu_0\epsilon_0\dot{\vec{E}}\quad&=\quad\mu_0\vec{j}
\end{align*}

Auf der rechten Seite erkennen wir sofort die Viererstromdichte $j^\mu$ wieder. So können wir unter Verwendung des Feldstärketensors die beiden Gleichungen in eine zusammenfassen.

\begin{equation*}
\partial_\mu F^{\mu\nu} =j^\nu
\end{equation*}
\ \\

\textbf{c.\ Homogene \textsc{Maxwell}-Gleichungen}
\begin{align*}
\div\vec{B}\quad &=\quad 0\\
\rot\vec{E}+\dot{\vec{B}}\quad &=\quad 0
\end{align*}
wird analog zu
\begin{equation*}
\partial_\kappa F_{\lambda\mu} + \partial_\mu F_{\kappa\lambda}+\partial_\lambda F_{\mu\kappa} = 0.
\end{equation*}
Das ist ein antisymmetrischer Tensor 3. Stufe. Wir werden gleich noch eine Möglichkeit kennen lernen, diesen Ausdruck noch zu vereinfachen. \\
Man kann die Richtigkeit dessen schnell nachprüfen, in dem man testweise ein paar Indizes einsetzt. Zum Beispiel liefert $\kappa,\lambda,\mu=1,2,3$
\begin{equation*}
\partial_1 F_{23}+\partial_3 F_{12} + \partial_2 F_{31} = -\partial_x B_x - \partial_y B_y - \partial_z B_z = -\div\vec{B}= 0
\end{equation*}
und analog $\kappa,\lambda,\mu = 0,2,3$ bei zyklischem vertauschen zu $\rot\vec{E}+\dot{\vec{B}} = 0$. \\

Es gibt jedoch noch eine zweite Möglichkeit, diese Gleichungen zu formulieren. Man führt dazu den \emph{dualen Feldstärketensor} ein.
\begin{equation*}
\mathcal{F}^{\mu\nu}=\begin{pmatrix}
0 & -B_1 & -B_2 & -B_3 \\
B_1 & 0 & \frac{E_3}{c} & -\frac{E_2}{c}\\
B_2 & -\frac{E_1}{c} & 0 & \frac{E_1}{c} \\
B_3 & \frac{E_2}{c} & -\frac{E_3}{c} & 0 
\end{pmatrix} = \frac{1}{2}\varepsilon^{\mu\nu\gamma\delta}F_{\gamma\delta},
\end{equation*}
wobei $\varepsilon^{\mu\nu\gamma\delta}$ der total antisymmetrische Permutationstensor 4. Stufe ist, gegen durch
\begin{equation*}
\varepsilon^{\mu\nu\gamma\delta} = \left\lbrace \begin{array}{rl}
+1 \quad & \text{für gerade Permutationen von (0,1,2,3)}\\
-1 \quad & \text{für ungerade Permutationen von (0,1,2,3)} \\
0 \quad & \text{sonst}
\end{array}\right. .
\end{equation*}
Damit lauten die homogenen \textsc{Maxwell}-Gleichungen ganz einfach
\begin{equation*}
\partial_\mu \mathcal{F}^{\mu\nu}=0
\end{equation*}
\ \\

\textbf{d. Viererpotential}\\

Wir haben bereits in vorangegangen Kapiteln gesehen, dass die beiden Potentiale $\varphi$ und $\vec{A}$ durchaus zusammen auftreten können. Es ist deshalb zweckmäßig, sie in einer Größe zu vereinen, dem sogenannten \emph{Viererpotential}.
\begin{equation*}
A^\mu = \left(\frac{\varphi}{c},\vec{A}\right),\quad A_\mu = \left(\frac{\varphi}{c},-\vec{A}\right)
\end{equation*}
Der Feldstärketensor leitet sich dann auch direkt aus diesem Potential ab.
\begin{align*}
F^{\mu\nu} & = \partial^\mu A^\mu - \partial^\nu A^\mu\\
F_{\mu\nu} & = \partial_\mu A_\nu - \partial_\nu A_\mu
\end{align*}
Mit der \textsc{Lorenz}-Eichung $\partial_\mu A^\mu = \div\vec{A}+\frac{1}{c^2}\partial_t\varphi=0$ erfüllt repräsentiert erfüllt das Viererpotential auch die Wellengleichungen für $\varphi$ und $\vec{A}$.
\begin{equation*}
\Dalembert A^\mu = \mu_0j^\mu.
\end{equation*}

\section{Transformation des elektromagnetischen Feldes}

Wir betrachten wieder eine \emph{Lorentz}-Transformation in $x$-Richtung, also
\begin{equation*}
\Omega_\nu^\mu = \begin{pmatrix}
\gamma &-\beta\gamma & 0 & 0 \\
-\beta\gamma & \gamma & 0 & 0 \\
0 & 0 & 1 & 0\\
0 & 0 & 0 & 1
\end{pmatrix}  \qquad \text{mit}\quad \gamma=\frac{1}{\sqrt{1-\frac{v^2}{c^2}}}, \quad \vec{\beta} = \frac{\vec{v}}{c}.
\end{equation*}
Man transformiert Vektoren und Tensoren wie folgt
\begin{align*}
A'^\mu &= \Omega_\kappa^\mu A^\kappa\\
F'^{\mu\nu} &= \Omega_\kappa^\mu\Omega_\lambda^\nu F^{\kappa\lambda}
\end{align*}
Führen wir das durch und setzen die Felder ein, erhalten wir
\begin{align*}
\vec{E}'_\parallel &= \vec{E}_\parallel & \vec{B}'_\parallel &=\vec{B}_\parallel\\
\vec{E}'_\perp &= \gamma\left(\vec{E}_\perp + \vec{v}\times\vec{B}\right)
& \vec{B}'_\perp &= \gamma\left(\vec{B}_\perp - \vec{v}\times\frac{\vec{E}}{c}\right).
\end{align*}
Im nichtrelativistischen Grenzfall erhalten wir so
\begin{align*}
\vec{E}' &= \vec{E} + \vec{v}\times\vec{B}\\
\vec{B}' &= \vec{B}.
\end{align*}
Die relativistische Korrektur ist
\begin{equation*}
\vec{B}' = \vec{B} - \frac{\vec{v}}{c^2}\times\vec{E}.
\end{equation*}

Unter Verwendung des Feldstärketensors sowie des dualen Feldstärketensors sehen wir, dass es zwei unter \emph{Lorentz}-Transformation der Felder invariante Größen gibt, nämlich
\begin{align*}
\frac{1}{2}F^{\mu\nu}F_{\mu\nu}= \vec{B}^2 -\frac{\vec{E}^2}{c^2} \\
\frac{1}{8}\mathcal{F}^{\mu\nu}\mathcal{F}^{\kappa\lambda}\varepsilon_{\mu\nu\kappa\lambda} = \frac{\vec{E}\cdot\vec{B}}{c}.
\end{align*}

\textbf{Potential einer gleichförmig bewegten Punktladung}\\

Im Ruhesystem der Ladung ist das Potential
\begin{equation*}
\varphi' = \frac{Q}{4\pi\epsilon_0}\frac{1}{r'},\quad\vec{A}' = 0 \qquad\Rightarrow\qquad A^\mu=\left(\frac{\varphi}{c},0\right)
\end{equation*}
Wir wenden nun auf dieses Viererpotential eine Transformation in $x$-Richtung in das Laborsystem an und erhalten
\begin{equation*}
\varphi_L(\vec{r},t) = \gamma\varphi',\quad\vec{A}_L = -\gamma\beta\varphi'.
\end{equation*}
Allgemein transformieren sich die Koordinaten der Potential nach
\begin{align*}
\varphi_L(\vec{r},t) &=\frac{q}{4\pi\epsilon_0}\frac{\gamma}{\left((x-vt)^2\gamma^2 + y^2+z^2\right)\frac{1}{2}}\\
\vec{A}_L(\vec{r},t) &=\frac{\mu_0q\vec{v}}{4\pi}\frac{\gamma}{\left((x-vt)^2\gamma^2 + y^2+z^2\right)\frac{1}{2}}.
\end{align*}
Das ist natürlich ein Speziallfall der \textsc{Liénard-Wiechert}-Potentiale.

\section{\textsc{Lorentz}-Kraft-Dichte}

Aus der Vierer-\textsc{Lorentz}-Kraft 
\begin{equation*}
F^\mu = Qu_\nu F^{\mu\nu}
\end{equation*}
kann durch die Ersetzung $Q\rightarrow\rho_0$ ganz leicht die Kraftdichte erraten.
\begin{equation*}
f^\mu = j_\nu F^{\mu\nu}
\end{equation*}
Das lässt sich auch ganz leicht für $k=1,2,3$ nachvollziehen.
\begin{align*}
f^k &= j_0F^{k0} + j_l F^{kl} = j_0 F^{k0}-j_l F^{lk} = \\
&=\rho c\frac{\vec{E}}{c} + \vec{j}(\mathbbm{1}\times\vec{B})=\rho\vec{E}+\vec{j}\times\vec{B} = \vec{f}
\end{align*}
Damit können wir auch die Komponenten des Vierervektors hinschreiben.
\begin{equation*}
j^\mu = \left(\frac{\vec{j}\vec{E}}{c},\vec{f}\right) = \left(-\frac{\nu_\textit{em}}{c},\vec{f}\right)
\end{equation*}
Achtung: Das $-\nu_\textit{em}$ im letzten Ausdruck ist hier kein Index, sondern steht für die Änderung der Energiedichte (vgl. Kapitel 7).\\
Es fällt auf, dass die Vierer-\textsc{Lorentz}-Kraftdichte keinen Faktor $\gamma$ enthält. Das liegt daran, dass hier durch ein invariantes Volumenelement dividiert wird.

\section{Energie-Impuls-Tensor}

Die aus Kapitel 7 bekannten Erhaltungsgleichungen
\begin{align*}
\dot{w}+\div\vec{S}_P &= \nu_\textit{em} = -\vec{j}\cdot\vec{E}\\
\dot{\vec{g}} + \div\tens{T} &= -\vec{f}
\end{align*} 
lassen sich nun zu
\begin{equation*}
\partial_\nu T^{\mu\nu}= -f^\mu
\end{equation*}
zusammenfassen. Diese Tensorgleichung ist gilt für alle Feldtheorien. Es müssen nur die entsprechenden Wechselwirkungen richtig zugeordnet werden. Der hier vorkommende \emph{Energie-Impuls-Tensor} ist gegeben durch
\begin{equation*}
T^{\mu\nu} = \begin{pmatrix}
w & \vec{S}_P^T/c\\
c\vec{g} & \tens{T}
\end{pmatrix}.
\end{equation*}
Aus der \textsc{Lorentz}-Transformation dieses Tensors folgen sofort die Transformationsvorschriften für $w$, $\vec{g}$, $\vec{S}_P$ und $\tens{T}$.\\
Außerdem erkennt man aus dem Zusammenhang von $w$ und $\vec{S}_P$ mit den Feldern
\begin{align*}
w &= \frac{\epsilon_0}{2}\vec{E}^2 + \frac{1}{2\mu_0}\vec{B}^2 &\vec{S}_P &=\frac{\vec{E}\times\vec{B}}{\mu_0}, 
\end{align*}
dass $T^{\mu\nu}$ quadratisch in $F^{\mu\nu}$ sein muss. Um den genauen Zusammenhang herauszufinden wählen wir den Ansatz
\begin{equation*}
T^{\mu\nu} = \alpha\cdot F^{\mu\kappa} F_\kappa^\nu + \beta\cdot g^{\mu\nu} F^{\kappa\lambda} F_{\kappa\lambda}.
\end{equation*}
Ein direkter Vergleich liefert $\alpha = \frac{1}{\mu_0}$ und $\beta = \frac{1}{4\mu_0}$ und damit also
\begin{equation*}
T^{\mu\nu} = \frac{1}{\mu_0}\left(\cdot F^{\mu\kappa} F_\kappa^\nu + \frac{1}{4}\cdot g^{\mu\nu} F^{\kappa\lambda} F_{\kappa\lambda}\right).
\end{equation*}
Daran sieht man, dass der Tensor symmetrisch sein muss. Das drückt natürlich physikalisch die Drehimpulserhaltung aus.
\begin{align*}
T^{xy}&=T^{yx}\\
c\vec{g} & =\frac{\vec{S}_P}{c}
\end{align*}
In der Mechanik lässt sich ebenfalls ein Energie-Impuls-Tensor $T^{\mu\nu}_\text{m}$ formulieren. Nach Konvention erfüllt dieser die Erhaltungsgleichung 
\begin{equation*}
\partial_\nu T_m^{\mu\nu} = f^\mu.
\end{equation*}
Zur Erinnerung, beim elektromagnetischen Energie-Impuls-Tensor stand da ein $-f^\mu$. Die Vorzeichen sind so gewählt, dass die sogenannte \emph{globale Energie-Impuls-Erhaltung} gilt.
\begin{equation*}
\partial_\nu\left(T_\text{m}^{\mu\nu} + T^{\mu\nu}\right) = 0
\end{equation*}

\section{Strahlung einer bewegten Punktladung II}

In Kapitel 9.7 haben wir uns bereits mit einer Ladung $Q$ auf der Bahn $\vec{R}(t)$ beschäftigt. Die Potentiale in diesem Fall sind
\begin{align*}
\varphi(\vec{r},t) &= \frac{Q}{4\pi\epsilon_0} \int\mathrm{d}t'\ \frac{1}{|\vec{r}-\vec{R}(t)|}\delta\left(t'-t-\frac{|\vec{r}-\vec{R}(t)|}{c}\right)\\
&=\frac{Q}{4\pi\epsilon_0}
\left[\frac{1}{|\vec{r}-\vec{R}|-\frac{\dot{\vec{R}}}{c}(\vec{r}-\vec{R}) }\right]_\text{ret}\\
\vec{A}(\vec{r},t) &=\frac{\mu_0Q}{4\pi}
\left[\frac{\dot{\vec{R}}}{|\vec{r}-\vec{R}|-\frac{\dot{\vec{R}}}{c}(\vec{r}-\vec{R}) }\right]_\text{ret}.
\end{align*}
Um das für die folgenden Überlegungen etwas abzukürzen, führen wir die folgenden Bezeichnungen ein.
\begin{align*}
\vec{\beta}(t) &= \frac{\dot{\vec{R}}(t)}{c} & L(t)&=|\vec{r}-\vec{R}(t)|\\
\vec{e}_L(t) &=\frac{\vec{r}-\vec{R}(t)}{L(t)} &k(t) &=1-\vec{e}_L\cdot\vec{\beta}
\end{align*}
Die Potentiale sind dann
\begin{align*}
\varphi(\vec{r},t)&=\frac{Q}{4\pi\epsilon_0}\frac{1}{L_\text{ret}k_\text{ret}}
&\vec{A}(\vec{r},t) &= \frac{\mu_0 Q}{4\pi}\frac{\vec{\beta}_\text{ret}}{L_\text{ret}k_\text{ret}}.
\end{align*}
Bildet man das elektrische Feld $\vec{E}=-\grad\varphi -\dot{\vec{A}}$, so erhält man, dass das Feld aus zwei Bestandteilen 
\begin{equation*}
\vec{E} = \vec{E}_v + \vec{E}_a
\end{equation*}
besteht. Dabei entspricht $\vec{E}_v$ einfach einem statischen \textsc{Coulomb}-Feld, aber in einem bewegten Bezugssystem.
\begin{equation*}
\vec{E}_v = \frac{Q}{4\pi\epsilon_0}\left[(1-\beta)^2\frac{\vec{e}_L-\vec{\beta}}{k^3L^2}\right]_\text{ret}
\end{equation*}
Der Anteil $\vec{E}_a$ ist jedoch ein nichtstatischer Effekt, der nur bei einer Beschleunigung $\dot{\beta}\neq 0$ auftritt und proportional zu dieser ist.
\begin{equation*}
\vec{E}_a  =\frac{Q}{4\pi\epsilon_0}\left[\frac{\vec{e}_L\times\left[(\vec{e}_L-\vec{\beta})\times\dot{\vec{\beta}}\right]}{ck^3L}\right]_\text{ret}.
\end{equation*}
Das magnetische Feld ist dann natürlich
\begin{equation*}
\vec{B}=\frac{1}{c}\vec{e}_\text{\textit{L},ret}\times\vec{E}.
\end{equation*}
\ \\

\textbf{a.\ Gleichförmige Bewegung}\\


\begin{wrapfigure}[10]{r}[0cm]{0cm}
	\raisebox{0pt}[\dimexpr\height-1\baselineskip\relax]{
		\colorbox{hgrey}{
			\begin{tikzpicture}
			\draw[ultra thick,->] (0,0)--(5,0);
			\fill (1,0) circle(0.1cm) node[below]{Teilchen bei $t_0$};
			\fill (3,1.5) circle(0.1cm) node[right]{$(\vec{r},t)$};
			\draw (3,0)--node[right]{$\vec{r}_\perp$}(3,1.5);
			\draw (1,0)--(3,1.5);
			\draw (1,3)--(3,1.5);
			\end{tikzpicture}
		}
	}
	\caption{Bahn}
\end{wrapfigure}
Die Bahn sei gegeben durch
\begin{equation*}
\vec{R}(t')=\begin{pmatrix}
\beta ct' \\ 0 \\ 0
\end{pmatrix}
\end{equation*}
und die Potentiale wie in Kapitel 13.5. Wir betrachten das Feld zum Zeitpunkt $t=0$. Da die Ladung unbeschleunigt ist, ist auch der Anteil $\vec{E}_a$ gleich Null. Das übrige Feld ist dann
\begin{equation*}
\vec{E}_v = \frac{Q}{4\pi\epsilon_0}\frac{\gamma\vec{r}}{(\gamma^2x^2+r_\perp^2)^\frac{3}{2}}.
\end{equation*}
Das ist offensichtlich trotz Punktladung nicht kugelsymmetriscch, sondern abhängig von $\gamma$ in $x$-Richtung gestaucht.\\

\textbf{b. Energieabstrahlung}\\

Da sich die beiden Felder $\vec{E}$ und $\vec{B}$ abhängig von der Beschleunigung in zwei verschiedene Anteile aufteilen, trifft das auch für die Energiestromdichte zu.
\begin{align*}
\vec{S}_P &= \frac{\vec{E}\times\vec{B}}{\mu_0} = \frac{1}{\mu_0}(\underset{\sim L^{-2}}{\vec{E}_v}+\underset{\sim L^{-1}}{\vec{E}_a})\times(\vec{B}_v+\vec{B}_a) \\
&=\underbrace{\frac{1}{\mu_0}\vec{E}_a\times\vec{B}_a}_{\text{Abstrahlung ins Unendliche}} + \underbrace{o\left(\frac{1}{L^3},\frac{1}{L^3}\right)}_{\text{Abstrahlung ins Endliche}}.
\end{align*}
Der vordere Abstrahlungsterm ist der eigentlich entscheidende, da der hintere Term sehr schnell abklingen wird.
\begin{equation*}
\vec{S}_a = \frac{1}{\mu_0}\left(\vec{E}_a+\vec{B}_a\right) = 
\frac{1}{\mu_0c}\left[\vec{e}_\text{\textit{L}}\vec{E}_a^2\right]_{\text{ret}}
\end{equation*}
Damit lässt sich die ausschlaggebende Abstrahlung von einer Bahnkurve  formulieren.  Die im Zeitintervall $\mathrm{d}t'$ am Ort $\vec{R}(t')$ in Richtung $\mathrm{d}\Omega$ abgestrahlte Energie $\mathrm{d}E(t')$ soll bei $(\vec{r},t)$ mit
\begin{align*}
t &= t' + \frac{|\vec{r}-\vec{R}(t')|}{c} & \mathrm{d}t &= \diff{t}{t'}\mathrm{t'}
\end{align*} 
beobachtet werden. 
\begin{equation*}
\mathrm{d}E(t') = \diff{t}{t'}\mathrm{d}t'\vec{S}(\vec{r},t)\cdot\vec{e}_{\vec{r}-\vec{R}(t')}\mathrm{d}\Omega L^2
\end{equation*}
Das führt dann schließlich auf die Leistungsabstrahlung
\begin{align*}
\diff{P}{\Omega}&=\frac{\mathrm{d}E(t')}{\mathrm{d}t\mathrm{d}\Omega}=\frac{1}{\mu_0c}(L\vec{E})^2k = \\
\diff{P}{\Omega}&=\frac{Q^2}{16\pi^2\epsilon_0c}\frac{1}{k^5}\left(\vec{e}_L\times\left[(\vec{e}_L-\vec{\beta})\times\dot{\vec{\beta}}\right]\right)^2_\text{ret},
\end{align*}
die nur durch $\vec{e}_L$ und die Bahnkurve bestimmt wird. Sie ist damit unabhängig vom Beobachter, was sehr sinnvoll erscheint. \\

\textbf{c. Nichtrelativistischer Grenzfall}\\

\begin{wrapfigure}[11]{l}[0cm]{0cm}
	\raisebox{0pt}[\dimexpr\height-1\baselineskip\relax]{
		\colorbox{hgrey}{
			\begin{tikzpicture}
			\draw[->] (0,0)--(4,0);
			\draw (2.7,0) node[below]{$\vec{a}$};
			\fill (1.5,0) circle(0.1cm);
			\draw[ultra thick,->] (1.5,0)--(1.5,2) node[right]{$\vec{S}_P$};
			\draw[ultra thick,->] (1.5,0)--(1.3,1.7);
			\draw[ultra thick,->] (1.5,0)--(1.3,-1.7);
			\draw[ultra thick,->] (1.5,0)--(1.7,1.7);
			\draw[ultra thick,->] (1.5,0)--(1.7,-1.7);
			\draw[ultra thick,->] (1.5,0)--(1.5,-2);
			\end{tikzpicture}
		}
	}
	\caption{nichtrelativistische Strahlung}
\end{wrapfigure}
Sollte $\beta \ll 1$ und damit $k\approx 1$ sein, wird die Abstrahlung
\begin{equation*}
\diff{P}{\Omega}=\frac{Q^2}{16\pi^2\epsilon_0c}|\dot{\vec{\beta}}|^2\sin^2\vartheta_a\qquad\text{mit}\quad\vartheta_a=\sphericalangle\left(\vec{e}_L,\dot{\vec{\beta}}\right)
\end{equation*}
maximal in Richtung der Beschleunigung $\vec{a}$. Das entspricht tatsächlich einem \textsc{Hertz}schen Dipol.\\ \linebreak\linebreak\linebreak
\newpage
\textbf{d.\ Ultrarelativistischer Grenzfall}\\

\begin{wrapfigure}[11]{r}[0cm]{0cm}
	\raisebox{0pt}[\dimexpr\height-1\baselineskip\relax]{
		\colorbox{hgrey}{
			\begin{tikzpicture}
			\draw[->] (0,0)--(4,0);
			\draw (1,0) node[below]{$\vec{a}$};
			\fill (1.5,0) circle(0.1cm);
			\draw[ultra thick,->] (1.5,0)--(3.5,0) node[below right]{$\vec{S}_P$};
			\draw[ultra thick,->] (1.5,0)--(3.2,0.3);
			\draw[ultra thick,->] (1.5,0)--(3.2,-0.3);
			\draw[color=hgrey] (0,2) node{a};
			\draw[color=hgrey] (0,-2) node{a};
			\end{tikzpicture}
		}
	}
	\caption{relativistische Strahlung}
\end{wrapfigure}
Es wird $\beta\approx 1$ und 
\begin{equation*}
k=1-\dot{\vec{e}}_L\cdot\vec{\beta}\approx 1-\cos\vartheta_v\qquad\text{mit}\quad\vartheta_v=\sphericalangle\left(\vec{e}_L,\vec{\beta}\right).
\end{equation*} 
Somit ist für sehr große Geschwindigkeiten die Strahlung
\begin{align*}
\diff{P}{\Omega}&\propto \frac{1}{K^5}\left|\vec{e}_L\times\left[(\vec{e}_L - \vec{\beta})\times\dot{\vec{\beta}}\right]\right|^2\\
& \propto \frac{1}{(1-\cos\vartheta_v)}
\end{align*}
parallel zur Beschleunigung $\vec{a}$.

\end{document}