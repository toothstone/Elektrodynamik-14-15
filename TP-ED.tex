\documentclass[a4paper,12pt,portrait]{book}	  
\title{Theoretische Elektrodynamik}
\author{Matthias Vojta\\ \\ \"ubertragen von\\Sebastian Schmidt, Lukas K\"orber und Friedrich Zahn}
\date{Wintersemester 2014/2015}

  
\usepackage[ngerman]{babel}	    
\usepackage[utf8]{inputenc}
\usepackage{amsmath}
\usepackage{mathtools}
\usepackage{amsthm}	 
\usepackage{graphicx}      
\usepackage{fancyhdr}
\usepackage{xcolor}
\usepackage[upright]{fourier}
\usepackage[frak=mma]{mathalfa}
\usepackage{sectsty}
\usepackage{lipsum}
\usepackage{bm}
\usepackage{bbm}


%layout options
\sectionfont{\color{hblue}}
\chapterfont{\color{dblue}}
\pagestyle{headings}
\setlength\parindent{0pt}

\newcommand{\Int}[4]{\int\limits_{#1}^{#2} #3 \mathrm{d} #4}
\newcommand{\Oint}[4]{\oint\limits_{#1}^{#2} #3 \mathrm{d} #4}
\renewcommand{\vec}[1]{\bm{#1}}

%differential operators
\newcommand{\diff}[2]{\frac{\mathrm{d}{#1}}{\mathrm{d}{#2}}}
\newcommand{\ddiff}[2]{\frac{\mathrm{d}^2{#1}}{\mathrm{d}{#2}^2}}
\newcommand{\dddiff}[2]{\frac{\mathrm{d}^3{#1}}{\mathrm{d}{#2}^3}}
\newcommand{\ndiff}[2]{\frac{\mathrm{d}^n{#1}}{\mathrm{d}{#2}^n}}
\newcommand{\pdiff}[2]{\frac{\partial{#1}}{\partial{#2}}}
\newcommand{\pddiff}[2]{\frac{\partial ^2{#1}}{\partial{#2}^2}}

%hilbert operators
\newcommand{\bra}[1]{\left\langle #1 \right|}
\newcommand{\ket}[1]{\left| #1 \right\rangle}
\newcommand{\lara}[1]{\left\langle #1 \right\rangle}
\newcommand{\scap}[2]{\left\langle #1 \ \middle| \ #2\right\rangle}
\newcommand{\exlara}[3]{\left\langle #1 \ \middle| \ #2 \ \middle| \ #3\right\rangle}

%various operators
\newcommand{\graph}{\text{graph\ }}
\newcommand{\rang}{\text{rang\ }}
\renewcommand{\sup}{\text{sup\ }}
\renewcommand{\inf}{\text{inf\ }}
\renewcommand{\dim}{\text{dim\ }}
\renewcommand{\ker}{\text{ker\ }}
\renewcommand\qedsymbol{$\textbf{q.e.d.}$}
\usepackage{xcolor}
\usepackage{amsthm}	 


\definecolor{dblue}{HTML}{183CE0}
\definecolor{hblue}{HTML}{0E75F7}

\newtheoremstyle{theoremstyle} % name
    {\topsep}                    % Space above
    {\topsep}                    % Space below
    {\upshape}                   % Body font
    {}                           % Indent amount
    {\sffamily\bfseries}                   % Theorem head font
    {}                          % Punctuation after theorem head
    {.5em}                       % Space after theorem head
    {\thmname{#1}\thmnumber{ #2}\thmnote{ (#3)}}  % Theorem head spec (can be left empty, meaning ‘normal’)

\theoremstyle{theoremstyle}
\newtheorem{theo}{Theorem}[section]	% please use "theorem"
\newtheorem{sa}[theo]{Satz}			% please use "satz"
\newtheorem{lem}[theo]{Lemma}		% please use "lemma"
\newtheorem{folgerung}[theo]{Folgerung}
\newtheorem*{definition}{Definition}
\newtheorem{beispiel}{Beispiel}
\newtheorem*{beispiel*}{Beispiel}

\newenvironment{theorem}
	{\par\nobreak\vfil\penalty0\vfilneg
   \vtop\bgroup\color{dblue}\noindent\rule{420pt}{2pt}\vspace{5pt}\begin{theo}}
	{\end{theo}\rule{50pt}{1pt}\color{black}\par\xdef\tpd{\the\prevdepth}\egroup
   \prevdepth=\tpd}
\newenvironment{satz}
	{\par\nobreak\vfil\penalty0\vfilneg
   \vtop\bgroup\color{dblue}\noindent\rule{420pt}{2pt}\vspace{5pt}\begin{sa}}
	{\end{sa}\rule{50pt}{1pt}\color{black}\par\xdef\tpd{\the\prevdepth}\egroup
   \prevdepth=\tpd}
\newenvironment{lemma}
	{\par\nobreak\vfil\penalty0\vfilneg
   \vtop\bgroup\color{dblue}\noindent\rule{420pt}{2pt}\vspace{5pt}\begin{lem}}
	{\end{lem}\rule{50pt}{1pt}\color{black}\par\xdef\tpd{\the\prevdepth}\egroup
   \prevdepth=\tpd}
	
\begin{document}
\maketitle
\tableofcontents
\chapter{Einleitung}
Gegenstand der Vorlesung ist die (klassische) Theorie der Elektrischen Felder ausgehend von den \textsc{Maxwell}-Gleichungen (1864):

\begin{equation*}
\div \vec{B}=0
\end{equation*}

\begin{equation*}
\rot  \vec{E}+\pdiff{\vec{B}}{t}=0
\end{equation*}

\begin{equation*}
\varepsilon_0\div \vec{E}=\rho
\end{equation*}

\begin{equation*}
\frac{1}{\mu_0}\rot \vec{B}-\varepsilon_0\pdiff{\vec{E}}{t}=\vec{j}
\end{equation*}

für die Felder $\vec{E}$ und $\vec{B}$ in Abhängigkeit von Ladungs- und Stromverteilung $\rho(\vec{r},t)$ und $\vec{j}(\vec{r},t)$  sollen physikalische Erscheinungen geschildert werden.\\
Die Elektrodynamik ist ein Teil des Standardmodells der Teilchenphysik, das einheitlich Teilichen und ihre Wechselwirkungen beschreibt.\\
Klassische Elektrodynamik ist ein Grenzfall der Quantenelektrodynamik (gültig für kleine Impuls- und Energiebeträge, große Brechungszahlen für Photonen).\\
Sie ist im Einklang mit der der speziellen Relativitätstheorie (c ist implizit in den \textsc{Maxwell}-Gleichungen enthalten). Viele interessante Effekte von Materie können mit klassischer Theorie nicht beschrieben werden.\\
Zum Beispiel: Wann sind Atome stabil? Wann ist Eisen ferromagnetisch? Warum wird z.B. Blei bei tiefen Temperaturen supraleitend? Für diese Fragen werden Quanteneffekte wichtig.

\chapter{Mathematische Hilfsmittel}

\section{Skalar- und Vektorfelder}

Felder entsprechen Größen, die an jedem Raumpunkt einen bestimmten Wert haben, der zeitabhängig sein kann.\\ \linebreak

\begin{wrapfigure}[]{r}[0cm]{0cm}
	\raisebox{0pt}[\dimexpr\height-1\baselineskip\relax]{
		\colorbox{hgrey}{
			\begin{tikzpicture}
			\draw[->](0,0)-- (4,0) node[right]{$x$};
			\draw[->](0,0)--(0,3) node[above]{$y$};
			\draw (3,3) node[right]{$T(x,y)$};
			\draw plot[smooth, tension=.8] coordinates {(0.5,1.5)(2,0.7)(3.5,0.4)} node[right]{$T=20^{\circ}$} ;
			\draw plot[smooth, tension=.8] coordinates {(0.5,2.2)(2,1.1)(3.5,0.8)} node[right]{$T=40^{\circ}$} ;
			\draw plot[smooth, tension=.8] coordinates {(0.5,3)(2,1.6)(3.5,1.2)} node[right]{$T=60^{\circ}$} ;
			\end{tikzpicture}
		}
	}
	\caption{Isothermen}
\end{wrapfigure}

\textbf{a. skalare Felder}\ $\phi=\phi(x,y,z,t)$\\ 


Jedem Raumpunkt wird ein Wert in Form einer (reellen) Zahl zugeordnet, wie zum Beispiel Temperatur, Druck, Ladung oder Energie. Flächen oder Linien mit konstantem Wert nennt man Äquipotentialflächen beziehungsweise -linien.\\ \linebreak\linebreak

\begin{wrapfigure}[]{r}[0cm]{0cm}
	\raisebox{0pt}[\dimexpr\height-1\baselineskip\relax]{
		\colorbox{hgrey}{
			\begin{tikzpicture}
			
			\draw[->](0,0)-- (4,0) node[right]{$x$};
			\draw[->](0,0)--(0,3) node[above]{$y$};
			\draw (3,3) node[right]{$\vec{F}(x,y)$};
			\foreach \x/\angle in {0.5/20, 1.5/40} {
				\foreach \y in {0.5, 1.5, 2.5} {
					\fill (\x,\y) circle[radius=1pt];
					\draw[->,thick]  (\x, \y) -- ++(\angle:1);
				}
			}
			\foreach \x/\angle in {2.5/60, 3.5/80} {
				\foreach \y in {0.5, 1.5} {
					\fill (\x,\y) circle[radius=1pt];
					\draw[->,thick]  (\x, \y) -- ++(\angle:1);
				}
			}
			\end{tikzpicture}
		}
	}
	\caption{Kraftfeld}
\end{wrapfigure}


\textbf{b. Vektorfelder}\ $\vec{E}=\vec{E}(x,y,z,t)$\\

Jedem Raumpunkt wird ein Vektor zugeordnet, der lokal die Richtung des Feldes beschreibt, wie etwa ein Geschwindigkeits- oder Kraftfeld. Vektorfelder lassen sich durch Feldlinien veranschaulichen, entlang derer sich zum Beispiel ein Teilchen bewegt, das die entsprechende Kraft erfährt.


\section{Integrale auf Feldern}
Integrale über skalare Felder werden wie bekannt bebildet; sie sind zu vermeiden.\\
Integriert man über ein Vektorfeld, spielt die Richtungsinformation eine entscheidende Rolle. Man unterscheidet je nach Dimension des Parameterbereichs von Linien-, Flächen- und Volumenintegralen.\\
\linebreak


\textbf{a. Linienintegrale}

\begin{equation*}
\varphi=\int\limits_{C}\vec{E}(\vec{r})\d\vec{r}
\end{equation*}

\begin{wrapfigure}[]{r}[0cm]{0cm}
	\raisebox{0pt}[\dimexpr\height-1\baselineskip\relax]{
		\colorbox{hgrey}{
			\begin{tikzpicture}
			
			\draw[->](0,0)-- (4,0) node[right]{$x$};
			\draw[->](0,0)--(0,3) node[above]{$y$};
			\draw[->](1.2,0.8)--(1,2) node[above]{$\vec{E}$};
			\draw[->](2,2.3)--(2.4,3) node[right]{$\vec{E}$};
			\draw(2,1.8) node{$C$};
			\draw[thick] plot[smooth, tension=.8] coordinates {(1.2,0.8)(2,2.3)(3.5,3)};
			\filldraw[black] (3.5,3) circle (2pt) node[right]{$\vec{r}(\tau_2)$};
			\filldraw[black] (1.2,0.8) circle (2pt) node[below left]{$\vec{r}(\tau_1)$};
			
			\end{tikzpicture}
		}
	}
	\caption{Linienintegral}
\end{wrapfigure}
Wir parametrisieren die Kurve durch $\vec{r}=\vec{r}(\tau)$ und erhalten somit
\begin{equation*}
\varphi=\int\limits_{\tau_0}^{\tau_1}\vec{E}(\vec{r}(\tau))\diff{\vec{r}}{\tau}\d \tau
\end{equation*}
Ein Speziallfall des Linienintegrals ist das sogenannte \textbf{geschlossene Linienintegral}, welches durch $\oint$ gekennzeichnet wird.\\
\linebreak


\begin{wrapfigure}[10]{r}[0cm]{0cm}
	\raisebox{0pt}[\dimexpr\height-1\baselineskip\relax]{
		\colorbox{hgrey}{
			\begin{tikzpicture}
			
			\draw plot[smooth, tension=.8] coordinates {(3,0) (5,1.4) (1.5,2) (0,1) (0,0) (3,0) };	
			\draw[->] (2,1)--(2,3) node[left]{$\Delta\vec{A}_i$};
			\draw[->] (2,1)--(3,2.8) node[right]{$\vec{E}(\vec{r}_i)$};
			\draw (0.5,0.5) node[right]{$\mathcal{F}$};
			
			
			\end{tikzpicture}
		}
	}
	\caption{Flächenintegral}
\end{wrapfigure}


\textbf{b. Flächenintegrale}\\
\begin{equation*}
\Phi=\iint\limits_{S}\vec{B}\cdot\d \vec{A} \ \ \ \text{mit } \d\vec{A}=\d A\cdot\vec{n}
\end{equation*}


Ganz analog zu \textbf{a.} kann die Fläche $\vec{r}=\vec{r}(u,v)$ parametrisiert werden. Es ist jedoch beim Bilden der Funktionaldeterminante auf die Richtung des Flächenelements zu achten. Die beiden möglichen Lösungen unterscheiden sich natürlich nur um ein Vorzeichen. Wir erhalten also
\begin{equation*}
\Phi=\int\limits_{v_1}^{v_2}\int\limits_{u_1}^{u_2}\vec{B}(u,v)\cdot\left(\pdiff{\vec{r}}{u}\times\pdiff{\vec{r}}{v}\right)\d u\d v
\end{equation*}

\textbf{c. Volumenintegrale}\\
\linebreak
\begin{equation*}
Q=\iiint\limits_G\d V\cdot\rho(\vec{r})=\iiint\limits_G\d ^3 r\cdot\rho(\vec{r})=
\end{equation*}
Beim Volumenintegral wird wiederum (nicht wie beim Flächenintegral) das Vorzeichen des Volumenelements vernachlässigt, da physikalisch die \textbf{Richtung} des Volumens nur sehr selten wirklich von Bedeutung ist. Mit entsprechender Parametrisierung $\vec{r}=\vec{r}(u,v,w)$ ergibt sich
\begin{equation*}
q=\int\limits_{w_1}^{w_2}\int\limits_{v_1}^{v_2}\int\limits_{u_1}^{u_2}\rho(u,v,w)\cdot\left|\pdiff{\vec{r}}{u}\cdot\left(\pdiff{\vec{r}}{v}\times\pdiff{\vec{r}}{w}\right)\right|\d u\d v\d w
\end{equation*}

\section{Vektorielle Ableitungen und Integrale}
\textbf{a. Gradient}\\
\linebreak
Der Gradient $\grad\varphi\ $ eines Skalarfeldes beschreibt dessen Änderung und steht senkrecht auf den Äquipotentialflächen (oder allgemeiner: Niveaumengen). Der Gradient lässt sich durch den Nabla-Operator ausdrücken und lautet in karthesischen Koordinaten: 
\begin{equation*}
\nabla=\pdiff{}{x}\vec{e}_x+\pdiff{}{y}\vec{e}_y+\pdiff{}{z}\vec{e}_z
\end{equation*}
Wichtig ist, dass $\nabla$ ein vektorieller Differenzialoperator ist. Er folgt Ableitungsregeln, wie etwa der Kettenregel, und $\nabla\varphi$ verhält sich unter Koordinatentransformation wie ein Vektor.\\
\linebreak
Andere Schreibweisen: $\pdiff{}{\vec{r}},\ \partial_{\vec{r}},\ \nabla_{\vec{r}}$\\
\linebreak
\underline{Beispiele:}\\
\linebreak
$\nabla |\vec{r}|=\frac{\vec{r}}{|\vec{r}|}=\vec{e}_r\\
\nabla \frac{1}{|\vec{r}|}=-\frac{1}{r^2}\vec{e}_r$\\
\linebreak\linebreak
\textbf{b. Divergenz} (Quellenstärke eines Vektorfeldes)\\
\linebreak
Die Divergenz div $\vec{E}=\nabla\cdot\vec{E}$ ist ein Skalar unter Koordinatentransformation und kann als \textbf{lokale Quellenstärke} interpretiert werden. Häufig benötigt man auch den \textsc{Laplace}-Operator, der die \textbf{zweite Ableitung} repräsentiert.\\
\begin{equation*}
\div\grad \varphi \ = \ \nabla^2\varphi \ = \ \laplace\varphi
\end{equation*}
\underline{Beispiele:}\\
\linebreak
div $\vec{r}=3$ (Anzahl der Dimensionen)\\
div $(\varphi\vec{A})=\nabla\cdot(\varphi\vec{A})=\vec{A}(\nabla\varphi)+\varphi(\nabla\vec{A})=\vec{A}\cdot\grad \varphi+\varphi\cdot\div\vec{A}$\\
\linebreak
\textbf{c. Rotation} (Wirbelstärke eines Vektorfeldes)\\
\linebreak
Die Rotation rot $\vec{B}=\nabla\times\vec{B}$

\begin{equation*}
\nabla\times\vec{B}=\begin{vmatrix}
\vec{e}_x & \vec{e}_y & \vec{e}_z \\
\pdiff{}{x} & \pdiff{}{y} & \pdiff{}{z}\\
B_x & B_y & B_z
\end{vmatrix}
\end{equation*}

kann als \textbf{lokale Wirbelstärke} verstanden werden. Ihre Komponenten lassen sich auch als

\begin{equation*}
(\nabla\times\vec{B})_i=\sum\limits_{j,k}\epsilon_{ijk}\cdot\pdiff{}{x_j}\cdot B_k
\end{equation*}

darstellen wobei $\epsilon_{ijk}\ $ der total antisymetrische Tensor 3. Stufe ist.\\
\linebreak
\underline{Beispiele:}\\
\linebreak
$\vec{v}=\vec{\omega}\times\vec{r} \ \Rightarrow \ \nabla\times\vec{v}=2\vec{\omega}$\\
$\nabla\times\vec{r}=0$\\
\linebreak\linebreak
\textbf{d. \textsc{Gauss}'scher Satz}\\
\begin{equation*}
\iiint\limits_V\div\vec{E}\cdot\d V=\oiint\limits_{\partial V}\vec{E}\cdot\d \vec{A}
\end{equation*}
Der Satz von \textsc{Gauss} verknüpft Eigenschaften im Inneren eines Volumens mit dem Verhalten auf dem Rand.\\

Über den Satz von \textsc{Gauss} lässt sich auch die partielle Integration in drei Dimensionen umformen zu:

\begin{equation*}
\Int{V}{}{V} \ \pdiff{}{\vec{r}} (u\cdot v) = \Int{V}{}{V} \  \pdiff{u}{\vec{r}}\cdot v \ + \ \Int{V}{}{V} \  u \cdot\pdiff{v}{\vec{r}} \ = \ \Oiint{\partial V}{}{A} \  (u\cdot v)
\end{equation*}

\ \\
\textbf{e. \textsc{Green}'scher Satz}\\
\begin{equation*}
\Int{V}{}{(\varphi\laplace\psi-\psi\laplace\varphi)}{V}=\Oint{\partial V}{}{(\varphi\nabla\psi-\psi\nabla\varphi)}{\vec{A}}
\end{equation*}
\linebreak
\textbf{f. \textsc{Stokes}'scher Satz}\\
\begin{equation*}
\iint\limits_S\rot\vec{B}\cdot\d\vec{A}=\oint\limits_{\partial A}\vec{B}\cdot\d\vec{r}
\end{equation*}
Analog zu \textsc{Gauss}'schen Satz verknüft der Satz von \textsc{Stokes} das Verhalten eines Feldes auf einer Fläche mit dem auf dem Rand der Fläche. Für geschlossene Flächen gilt
\begin{equation*}
\oiint\limits_{S=\partial V}\rot\vec{B}\cdot\d\vec{A}=0
\end{equation*}

\section{Differentialoperatoren in krummlinigen Koordinaten}
Karthesische /Kugel-/Zylinderkoordinaten sind hier wichtig.\\
\linebreak
z.B: \ $\nabla_x\psi=\partial_x\psi\vec{e}_x+\partial_y\psi\vec{e}_y+\partial_z\psi\vec{e}_z$\\
\linebreak
$\nabla_\theta\psi=\pdiff{}{r}\psi\vec{e}_r+\frac{1}{r}\pdiff{}{\theta}\psi\vec{e}_\theta+\frac{1}{r\sin\theta}\pdiff{}{\phi}\psi\vec{e}_\phi$\\
\linebreak
Generell: $(\nabla\psi)_u\equiv(\nabla\psi)\vec{e}_u=\frac{1}{g_u}\pdiff{\psi}{u}$ \ mit \ $g_u=|\pdiff{\psi}{u}|$\\

\section{\textsc{Fourier}-Transformation}
\begin{equation*}
\tilde{f}(\omega)=\frac{1}{\sqrt{2\pi}}\Int{-\infty}{\infty}{f(t)e^{-i\omega t}}{t}
\end{equation*}
\begin{equation*}
f(t)=\frac{1}{\sqrt{2\pi}}\Int{-\infty}{\infty}{\tilde{f}(t)e^{i\omega t}}{\omega}
\end{equation*}
Verallgemeinert auf $n$ Dimensionen ergibt sich:\\
\begin{equation*}
\tilde{f}(\vec{k})=\frac{1}{({2\pi})^{\frac{n}{2}}}\Int{-\infty}{\infty}{f(\vec{r})e^{-i\vec{k}\vec{r}}}{^nr}
\end{equation*}
\linebreak
\textbf{a. Differentiation}\\
\begin{equation*}
\diff{}{t}f(t)=\frac{1}{\sqrt{2\pi}}\Int{-\infty}{\infty}{i\omega\tilde{f}(\omega)e^{i\omega t}}{\omega}
\end{equation*}
\linebreak
\textbf{b. Faltung}\\
\begin{equation*}
(f*g)(t)=\frac{1}{\sqrt{2\pi}}\Int{-\infty}{\infty}{f(t-s)G(s)}{s}
\end{equation*}
\begin{equation*}
\widetilde{(f*g)}(\omega)=\tilde{f}(\omega)\tilde{g}(\omega)
\end{equation*}
\textbf{c. Rechenregeln}\\
\begin{align*}
f'(t) & \leftrightarrow i\omega\tilde{f}(\omega)\\
-itf(t) & \leftrightarrow \tilde{f}'(\omega)\\
f(t+a) & \leftrightarrow  e^{i\omega a}\tilde{f}(\omega)\\
e^{i\omega t}f(t) &\leftrightarrow & \tilde{f}(\omega-a)\\
f(at) & \leftrightarrow \frac{1}{|a|}\tilde{f}\left(\frac{\omega}{a}\right)\\
f^*(t) & \leftrightarrow \tilde{f}^*(\omega)\\
\tilde{\tilde{f}}(t) &\leftrightarrow & f(-t)\\
\end{align*}

\section{Delta-Distribution}

Die Delta-Distribution ist über folgende Eigenschaften definiert:

\begin{enumerate}
\item
\begin{equation*}
\delta(\vec{r}) = \begin{cases}
0 & \text{für }\vec{r}\neq\vec{r}_0\\
\infty & \text{für } \vec{r} = \vec{r}_0
\end{cases}
\end{equation*}

\item
\begin{equation*}
\int\limits_{\vec{r}_0\in V}\d V \ \delta({\vec{r}-\vec{r}_0}) = 1
\end{equation*}
\end{enumerate}

Alle Aussagen gelten analog für die Delta-Distribution $\delta(x)$ in einer Dimension.\
Bei höherdimensionalen Deltadistributionen gilt allerdings nur in kartesischen Koordinaten:

\begin{equation*}
\delta(\vec{r} - \vec{r}_0) = \delta(x-x_0)\cdot\delta(y-y_0)\cdot\delta(z-z_0)
\end{equation*}
\ \\
Faltet man die Delta-Distribution mit einer Funktion $f(\vec{r})$, so ergibt sich aus ihren Eigenschaften:

\begin{equation*}
\int\limits_{\vec{r}_0\in V}\d V \ \delta({\vec{r}-\vec{r}_0}) \ f(\vec{r}) = f(\vec{r}_0)
\end{equation*}

\section{\textsc{Green}'sche Funktion zur Lösung inhomogener linearer DGL}

Wir betrachten die lineare, inhomogene Differentialgleichung

\begin{equation*}
L \ \phi (x_1,\dotsc,x_n) = \rho (x_1,\dotsc,x_n) \; \text{ oder kurz } \; L\phi = \rho
\end{equation*}

wobei $L$ ein linearer Operator und $\rho$ die Inhomogenität sein soll.\
\\
Die \textsc{Green}'sche Funktion $G(x,x)$ zum Operator $L$ ist die Lösung der Differentialgleichung mit $\delta$-förmiger Inhomogenität.

\begin{equation*}
L \ G(x,x') = \delta (x-x') \; [= \delta(x_1-x_1')\cdot\dotsc\cdot\delta(x_n - x_n')]
\end{equation*}
\ \\
Wenn $g$ bekannt ist, dann kann die Lösung für beliebige Inhomogenität durch Superposition gewonnen werden.

\begin{equation*}
\phi (x) = \int\d x' \ G(x,x') \rho(x')
\end{equation*}

Den Beweis hierfür erhält man leicht durch Einsetzen:

\begin{equation*}
L \ \phi(x) = \int\d x' \ L \ G(x,x') \rho(x') = \rho(x)
\end{equation*}


\chapter{Grundbegriffe und \textsc{Maxwell}-Gleichungen}
\section{Kräfte und Punktladungen}

Aus der Erfahrung ergibt sich für eine ruhende Ladung

\begin{equation*}
\vec{F}(\vec{r},t)=Q\cdot\vec{E}(\vec{r},t)
\end{equation*}

Dabei ist die Ladung $Q$ eine Körpereigenschaft und $\vec{E}$ eine Eigenschaft, die die Umwelt charakterisiert. Über den Vergleich der Kraft auf zwei Körper $\vec{F}_1(\vec{r},t)=\frac{Q_1}{Q_2}\vec{F}_2(\vec{r},t)$ lässt sich so eine Einheit für die Ladung definieren.\\
\linebreak

Bei bewegten Ladungen beobachten wir etwas anderes. Die Kraft hat hier die Form

\begin{equation*}
\vec{F}=Q(\vec{E}+\vec{v}\times\vec{B})
\end{equation*}

\section{Ladungs- und Stromdichte, Ladungserhaltung}

Über eine Ladung in einem Volumenelement lässt sich der Begriff der Ladungsdichte definieren.

\begin{equation*}
\rho(\vec{r},t)=\diff{Q}{V}
\end{equation*}

Eine Ladungsänderung nennen wir schließlich den elektrischen Strom.

\begin{equation*}
-I:=\dot{Q}=\diff{}{t}\Int{V}{}{\rho(\vec{r},t)}{V}=\Int{V}{}{\pdiff{\rho}{t}}{V}
\end{equation*}		
Betrachten wir nun den Stromfluss durch ein Oberflächenelement d$\vec{A}$. Die Ladungsträger, welche durch diese Fläche wandern haben die Geschwindigkeit $\vec{v}$, sodass anschaulich ein kleines Volumenelement dV$ = \vec{v}\mathrm{d}t\cdot\mathrm{d}\vec{A}$ aufgespannt wird:

\begin{align*}
\mathrm{d}Q &= \rho(\vec{r},t)\vec{v}(\vec{r},t)\mathrm{d}t\mathrm{d}\vec{A}\\
\diff{Q}{t} =-I &= \rho\vec{v}\cdot\mathrm{d}\vec{A}=:\vec{j}(\vec{r},t)\cdot\mathrm{d}\vec{A}
\end{align*}

Wir nennen $\vec{j} = \rho\vec{v}$ der Anschaulichkeit nach die \textbf{Stromdichte}, denn man sieht leicht:

\begin{equation*}
\iint\limits_A\vec{j}\cdot\mathrm{d}\vec{A}=I
\end{equation*}

Setzen wir nun dies in die Gleichung für die Ladungserhaltung ein:

\begin{equation*}
0 = \dot{Q} + I = \iiint\limits_V\mathrm{d}V\pdiff{\rho}{t} \ + \oiint\limits_{\partial V}\mathrm{d}\vec{A}\cdot\vec{j} = \iiint\limits_V \mathrm{d}V\left(\pdiff{\rho}{t} + \pdiff{\vec{j}}{\vec{r}}\right) 
\end{equation*}

Da dies für für alle möglichen Volumina gelten soll, folgt daraus die \textbf{Kontinuitätsgleichung}:

\begin{equation*}
\dot{\rho} + \text{div} \ \vec{j} = 0
\end{equation*} 

Für den Grenzfall eines unendlich großen Volumens gilt zunächst $\vec{j}\rightarrow 0$ auf der Oberfläche, woraus man auf die für diesen Grenzfall logische Konsequenz schließen kann, dass

\begin{equation*}
\dot{Q} = -\oiint\vec{j}\mathrm{d}\vec{A} = 0
\end{equation*}

die Ladung im gesamten Raum erhalten ist.\ \\


Mit der eingeführten Stromdichte $\vec{j}$ kann man nun auch den Ausdruck der \textsc{Lorentz}kraft-Dichte $\vec{f} := \frac{\vec{F}}{V}$ definieren:

\begin{align*}
\mathrm{d}\vec{F} & = \mathrm{d}Q(\vec{E} + \vec{v}\times\vec{B}) \\
\Rightarrow \vec{f} & = \rho(\vec{r},t)\cdot(\vec{E}(\vec{r},t) + \vec{v}(\vec{r},t)\times\vec{B}(\vec{r},t) = \rho\vec{E}+ \vec{j}\times\vec{B}
\end{align*}

\section{Die \textsc{Maxwell}-Gleichungen}
Die \textsc{Maxwell}-Gleichungen wurden 1864 vom schottischen Physiker James Clerk \textsc{Maxwell} aufgestellt und bilden ein Differentialgleichungssystem für die Felder  $\vec{B}(\vec{r})$ und $\vec{E}(\vec{r})$. Zusammen mit der Kontinuitätsgleichung beschreiben sie die gesamte (klassische) Elektrodynamik, da $\rho$ und $\vec{j}$ die Quellen und Wirbel des $ \ \vec{B} \ $- und $ \ \vec{E} \ $-Feldes eindeutig bestimmen:

\begin{align*}
\text{div} \ \vec{B} &= 0 \qquad\qquad\qquad\quad \epsilon_0\text{div} \ \vec{E} = \rho \\
\text{rot} \ \vec{E} + \dot{\vec{B}} &= 0 \qquad\qquad \frac{1}{\mu_0}\text{rot} \ \vec{B} - \epsilon_0\dot{\vec{E}} = \vec{j}
\end{align*}


Nun könnte man fragen, ob die  Beschreibung der Elektrodynamik über lokale Felder denn zweckmäßig ist oder ob man sie nicht eliminieren könnte. Das \textsc{Coulomb}-Gesetz wäre ein Beispiel für diese Fernwirkungstheorie. Zwei Gründe sprechen für die lokale Feldtheorie: sie ist zum einen schlichtweg einfacher mathematisch zu beschreiben und zum anderen unabhängig vom Vorhandensein von Materie und demzufolge Ladungsträgern.

\section{Konstruktion der \textsc{Maxwell}-Gleichungen}
Versucht man die Elektrodynamik zu beschreiben, so kann man sich zu Beginn von phänomenologischen Seite diesem Problem nähern und fordern, dass Symmetrien in Zeit und Raum die Gültigkeit der Gleichungen erhalten sollen. Dies ist eine gängige physikalische Vorgehensweise; man verlangt, dass die beschriebene (reale) Physik unabhängig von der Wahl der Koordinaten sein soll.
Wir fordern also zunächst, dass die die Form der Gleichungen unter den Symmetrietransformationen der Rauminversion $(\vec{r}\rightarrow-\vec{r})$ und der zeitlichen Reversibilität ($t\rightarrow-t$) invariant ist. Zudem wollen wir uns als Ziel setzen, die Gesetze möglichst einfach zu formulieren, das heißt, es sollen maximal Differentialgleichungen 1. Ordnung auftauchen.
Betrachten wir nun also zunächst das Transformationsverhalten verschiedener Objekte:\ \\
\ \\


\begin{tabular}{c|c|c|l}
Objekt & $t\rightarrow-t$ & $\vec{r}\rightarrow-\vec{r}$ & Bemerkung\\
\hline $t, \pdiff{}{t}$ & - & + & Definition\\
$\vec{r},\pdiff{}{\vec{r}}$ & + & -& Definition\\
$\dot{\vec{r}}$ & -& - & durch Multiplikation der Vorzeichen erhalten\\
$\ddot{\vec{r}},\vec{F},\vec{f}$ & + & - & Erfahrung aus Mechanik:\ $\ddot{\vec{r}}=\frac{\vec{F}}{m}$\\
$Q,\rho$ & + & + & Annahme\\
$\vec{j}\ (=\rho\cdot\dot{\vec{r}})$ & - & - & \\
$\vec{E}$ & + & - & Vektor, erhalten aus: $\vec{F}=Q(\vec{E}+\vec{v}\times\vec{B})$\\
$\vec{B}$ & - & + & Pseudovektor\\
$\pdiff{}{\vec{r}}\cdot\vec{E}$ & + & + & Skalar\\
$\pdiff{}{\vec{r}}\cdot\vec{B}$ & - & - & Pseudoskalar\\
$\pdiff{}{\vec{r}}\times\vec{E}$ & + & + & Pseudovektor\\
$\pdiff{}{\vec{r}}\times\vec{B}$ & - & - & Vektor\\
$\pdiff{}{t}\vec{E}$ & - & - & Vektor\\
$\pdiff{}{t}\vec{B}$ & + & + & Pseudovektor
\end{tabular}
\ \\
\ \\
\ \\
\ \\
\ \\
\ \\
Da wir gefordert hatten, dass unsere gewünschten Gleichungen invariant unter den Transformationen sein sollten, dürfen wir nun nur die Größen mit dem gleichen Transformationsverhalten verknüpfen:\ \\


\begin{enumerate}
\item
\underline{$++$ Skalar} \quad $\rho,\text{div} \ \vec{E}\\
\\
\Rightarrow \rho = \epsilon_0 \cdot \text{div} \ \vec{E} \qquad (\epsilon_0$  ist beliebige Konstante)

\item
\underline{$--$ Vektor} \quad $\vec{j}, \text{rot} \ \vec{B},\dot{\vec{B}}\\
\\
\Rightarrow \vec{j} = \alpha\cdot\dot{E} + \frac{1}{\mu_0}\cdot\text{rot} \ \vec{B}  \qquad (\alpha,\frac{1}{\mu_0}$ sind beliebige Konstanten)

\item
\underline{$--$ Skalar} \quad $\text{div} \ \vec{B}\\ 
\\
\Rightarrow \ 0 = \text{div} \ \vec{B}$

\item
\underline{$++$ Vektor} \quad $\text{rot} \ \vec{E},\dot{\vec{B}}\\
\\
\Rightarrow \ 0 = \text{rot} \ \vec{E} + \beta\cdot\dot{\vec{B}} \qquad (\beta$  ist beliebige Konstante)
\\
\\

\item
\underline{$+-$ Vektor}  \quad $\vec{E},(\vec{r},\ddot{\vec{r}})\\
\\
\Rightarrow \ 0 =  \vec{E}$

\item
\underline{$-+$ Vektor} \quad $\vec{B}\\
\\
\Rightarrow \ 0 = \vec{B}$
\end{enumerate}
\ \\

Das System 1-4 ist ein widerspruchsfreies und vollständiges System von Differentialgleichungen für das $\vec{E}$- und das $\vec{B}$-Feld, da diese durch ihre Quellen und Wirbel jeweils eindeutig bis auf Konstanten bestimmt sind. Diese werden problemabhängig aus den gegebenen Randbedingungen bestimmt. Die Gleichungen 5 und 6 werden aus naheliegenden Gründen weggelassen; sie stehen zwar nicht im Widerspruch zu den ersten 4 Gleichungen, doch würde das Differentialgleichungssystem mit ihnen nur noch die Triviallösung ohne physikalisch interessante Bedeutung liefern.\
\\
\ \\
\ \\

\underline{\textbf{Konstantendiskussion:}}
\ \\
\begin{enumerate}
\item Die Konstante $\epsilon_0$ ist zunächst frei wählbar, da die Ladung $Q$ nur bis auf einen Faktor genau bestimmt ist. Für die Wahl von $\epsilon_0$ gibt es verschiedene Ansätze:\
\begin{enumerate}
\item $\epsilon_0$ wird als 1 definiert. Diese Defintion wird im cgs-System umgesetzt.\
\\
\item $4\pi\cdot\epsilon_0$ wird 1 gesetzt. Das sich aus dieser Definition ergebende Einheitensystem nennt man das \textsc{Gauss}-System.\
\end{enumerate}
\ \\
Im SI-System wird dagegen $\epsilon_0$ über $\mu_0$ festgelegt, wobei für $\mu_0$ gilt:

\begin{equation*}
[\mu_0] = \frac{[\vec{E}]}{[I]}\frac{[l]^2}{[l]}=\frac{[\vec{f}]}{[\vec{j}]}\frac{[l]}{[I]}=\frac{[\vec{F}]}{[I]^2}=\frac{N}{A^2}
\end{equation*}
\begin{equation*}
\mu_0 = 4\pi\cdot 10^{-7} \frac{N}{A^2}
\end{equation*}

$\epsilon_0$ erhält man nun daraus über die Fundamentalbeziehung im SI-System:
\begin{equation*}
\epsilon_0\mu_0 = \frac{1}{c^2}
\end{equation*}

\item
Die Konstante $\alpha$ erhalten wir, in dem wir von Gleichung (2) die Divergenz bilden und dann div $\vec{j}$ aus der Kontinuitätsgleichung einsetzen:

\begin{equation*}
(\epsilon_0 + \alpha) \ \pdiff{}{t} \ \text{div} \ \vec{E} \overset{!}{=} 0 \; \Rightarrow \; \alpha = -\epsilon_0 
\end{equation*}

\item
Dass die Konstante $\beta$ im SI-System gleich 1 sein musss, erhält man aus Überlegungen, dass die \textsc{Maxwell}-Gleichungen von Inertialsystem zu Inertialsystem invariant sein müssen.
\end{enumerate}
\ \\
\underline{Bemerkung:}
Im \textsc{Gauss}-System erhält man aufgrund der Wahl der Konstanten für die \textsc{Lorentz}-Kraft:
\begin{equation*}
\vec{F} = Q (\vec{E} + \frac{\vec{v}}{c}\times\vec{B})
\end{equation*}
woraus folgt:
\begin{equation*}
\epsilon_0\mu_0\cdot\beta = \frac{1}{c^2} \; \text{ und } \; \beta = \frac{1}{c}, \mu_0 = \frac{4\pi}{c}
\end{equation*}

\section{Integrale Fromulierung der \textsc{Maxwell}-Gleichungen}
Die integrale Formulierung der \textsc{Maxwell}-Gleichungen ist äquivalent zu der differentiellen und ergibt sich entweder aus Volumen- oder Flächenintegration auf beiden Seiten der entsprechenden Gleichung und dann der Anwendung der Integralsätze von \textsc{Gauss} oder \textsc{Stokes}:
\ \\
\begin{align*}
\text{i)} \quad & \epsilon_0 \ \text{div} \ \vec{E} = \rho & \Leftrightarrow \qquad\qquad & \epsilon_0\oiint \mathrm{d}\vec{A} \cdot\vec{E} = Q_{\text{in}} \\
\text{ii)} \quad & \text{div} \ \vec{B} = 0 & \Leftrightarrow \qquad\qquad & \oiint \mathrm{d}\vec{A}\cdot\vec{B} = 0 \\
\text{iii)} \quad & \text{rot} \ \vec{E} + \dot{\vec{B}} = 0 & \Leftrightarrow \qquad\qquad & \oint\limits_{\partial A} \mathrm{d}\vec{r}\cdot\vec{E} \ + \ \iint\limits_A \mathrm{d}\vec{A}\cdot\dot{\vec{B}} = 0 \\
\text{iv)} \quad & \frac{1}{\mu_0} \ \text{rot} \ \vec{B} - \epsilon_0 \dot{\vec{E}} = \vec{j} & \Leftrightarrow \qquad\qquad & \frac{1}{\mu_0}\oint\limits_{\partial A}\mathrm{d}\vec{r}\cdot\vec{B} \ - \ \epsilon_0\iint\limits_A\mathrm{d}\vec{A}\cdot\dot{\vec{E}} = I_{\text{in}}
\end{align*}

\underline{Bemerkung:}

$\vec{r}$ und $t$ sind unabhängige Variablen, das heißt, dass die Felder $\vec{B}$ und $\vec{E}$ jeweils von $\vec{r}$ und $t$ abhängen, nicht aber von $\dot{\vec{r}}$.
Zudem ist es aufgrund unserer Forderungen bei der Konstruktion der \textsc{Maxwell}-Gleichungen verboten, dass eine explizite Abhängigkeit der Grundgleichungen von $\vec{r}$ und $t$ vorliegt, da es sonst außergewöhnliche Zeiten und Orte gäbe, was aber die geforderte Homogenität verletzen würde.
\section{Induktionsgesetz für Leiterschleifen}

Zunächst definieren wir den magnetischen Fluss $\Phi$ durch eine Fläche $\vec{A}$ im Raum:

\begin{equation*}
\Phi := \iint\limits_A\mathrm{d}\vec{A}\cdot\vec{B}
\end{equation*}

Man sieht leicht, dass sich der Fluss $\Phi$ bei Flächenänderung und Änderung der magnetischen Flussdichte $\vec{B}$ ändert:

\begin{align*}
\Delta\Phi &=  \Delta\left(\iint\mathrm{d}\vec{A}\cdot\vec{B}\right) = \iint\limits_A\mathrm{d}\vec{A}\cdot\Delta\vec{B} \; + \; \iint\limits_{\Delta A}\mathrm{d}\vec{A}\cdot\vec{B}\\
&= \Delta t \iint\limits_A\mathrm{d}\vec{A}\cdot\pdiff{\vec{B}}{t} \; + \; \oint\limits_{\partial A}\left(\vec{v}\Delta t\times\mathrm{d}\vec{r}\right)\cdot\vec{B}\\
&= \Delta t\left(\iint\limits_A\mathrm{d}\vec{V}\cdot\dot{\vec{B}} \; - \; \oint\limits_{\partial A}\mathrm{d}\vec{r}\cdot\left(\vec{v}\times\vec{B}\right)\right)\\
\Rightarrow \dot{\Phi} &= \iint\limits_A\mathrm{d}\vec{A}\cdot\dot{\vec{B}} \; - \; \oint\limits_{\partial A}\mathrm{d}\vec{r} \ \left(\vec{v}\times\vec{B}\right)
\end{align*}

Nach Anwenden der dritten \textsc{Maxwell}-Gleichung erhält man das \textbf{Induktionsgesetz}:
\begin{equation*}
\dot{\Phi} = - \oint\limits{\partial A}\mathrm{d}\vec{r} \ \left(\vec{E} + \vec{v}\times\vec{B}\right) = - U_{\mathrm{induziert}}
\end{equation*}

Das letzte Minuszeichen nennt man auch die \textbf{\textsc{Lenz}`sche Regel}, welche besagt, dass ein induzierter Strom immer ein Magnetfeld erzeugt, welches seiner eigenen Ursache ($U_{\mathrm{induziert}}$) entgegengerichtet ist.
\ \\
Auffällig bei dem Induktionsgesetz ist seine Ähnlichkeit mit der auf eine freie Ladung wirkende Kraft $\vec{F} = Q(\vec{E} + \vec{v}\times\vec{B})$. Darin liegt auch die Begründung für ebenjenes Gesetz:\

Wir stellen uns eine Leiterschleife vor, welche an einer Stelle durchbrochen ist, damit kein Strom durch die Schleife fließen könnte. Auf einen sich in dieser Schleife bewegenden Ladungsträger wirkt die Kraft:

\begin{equation*}
\vec{F} = Q(\vec{E} + \vec{v}\times\vec{B}) =: Q\vec{E'}
\end{equation*} 

Man sieht, dass das $\vec{E}$-Feld abhängig vom Bezugsystem ist, daher haben wir für $\vec{E'}$ ein Bezugssystem konstruiert, welches sich mit der Geschwindigkeit $\vec{v}$ gegenüber dem Laborsystem bewegt. Damit haben wir im mitbewegeten Bezugssystem erreicht, dass $\vec{v'} = 0$ ist. Bilden wir nun das Weginteral für ein Teilchen entlang der Leiterschleife im $\vec{E}$-Feld erhalten wir:

\begin{equation*}
\oint\limits_{\mathrm{Schleife}}\mathrm{d}\vec{r}\cdot\left(\vec{E} + \vec{v}\times\vec{B}\right) = \oint\limits_{\mathrm{Schleife}}\mathrm{d}\vec{r}\cdot\vec{E'} = \int\limits_{\mathrm{Beginn}}^{\mathrm{Ende}}\mathrm{d}\vec{r}\cdot\vec{E'} = U_{\mathrm{induziert}}
\end{equation*}

\chapter{Elektrostatik}

\section{Grundgleichungen und elektrostatisches Potential}

In der Elektrostatik betrachten wir, wie der Name schon andeutet, zeitunabhängige Felder. Dementsprechend kann man als erste Konsequenz daraus folgern, dass $\dot{\vec{E}} = 0$ und $\dot{\vec{B}} = 0$ ist. Fallen nun in den \textsc{Maxwell}-Gleichungen alle Beiträge mit $\dot{\vec{E}}$ und $\dot{\vec{B}}$ weg, kann man die Felder $\vec{E}$ und $\vec{B}$ getrennt voneinander betrachten. Laienhaft gesprochen entkoppeln wir die Phänomene \grqq Elektrizität\grqq und "Magnetismus". Des Weiteren betrachten wir in der Elektrostatik nur ruhende Ladungen, woraus folgt, dass außerdem $\vec{j}=0 \Rightarrow \vec{B}=0$ ist.\

Damit erhalten wir aus der dritten \textsc{Maxwell}-Gleichung, dass rot $\vec{E} = 0$ gilt, wodurch das Einführen eines Potentials für $\vec{E}$ möglich wird:

\begin{equation*}
\vec{E} =: \ -\grad \varphi
\end{equation*}

Mit div $\vec{E} = \frac{\rho}{\epsilon_0}$ erhält man daraus die \textbf{\textsc{Poisson}-Gleichung} der Elektrostatik:

\begin{equation*}
\bigtriangleup\varphi = - \frac{\rho}{\epsilon_0}
\end{equation*}

Für $\bigtriangleup\varphi = 0$ nennt man die \textsc{Poisson}-Gleichung auch \textbf{\textsc{Laplace}-Gleichung}.

\section{Kugelsymmetrische Ladungsverteilung}

Für eine kugelsymmetrische Ladungsverteilung gilt:

\begin{equation*}
\rho(\vec{r}) = \rho(|\vec{r}|) = \rho(r) \; \Rightarrow \; \varphi(\vec{r}) = \varphi(r)
\end{equation*}

Dem kann man entnehmen, dass die Äquipotentialflächen Kugelflächen sein müssen und somit der Gradient von $\varphi$ auch parallel zum Ortsvektor stehen muss.($\vec{E}(\vec{r}) = \vec{E}(r)\vec{e}_r$ \
Für das $\vec{E}$-Feld gilt weiterhin:

\begin{equation*}
\epsilon_0\oiint\limits_{\partial Kugel}\mathrm{d}\vec{A}\cdot\vec{E} \; \overset{\vec{A}\parallel\vec{E}}{=} \; \epsilon_0\oiint\limits_{\partial Kugel}\mathrm{d} A \cdot E = 4\pi\epsilon_0\cdot r^2 \cdot E(r) = Q_{\mathrm{in}}(r)
\end{equation*}

Damit ergibt sich für das $\vec{E}$-Feld und das Potential:

\begin{align*}
\vec{E}(r) &= \frac{Q_{\mathrm{in}}(r)}{4\pi\epsilon_0\cdot r^2} \cdot \vec{e}_r\\
\ \\
\varphi(r) &= \frac{Q_{\mathrm{in}}(r)}{4\pi\epsilon_0\cdot r} + \varphi_0 \; \text{ mit } \; \varphi_0 = \varphi(r \rightarrow 0)
\end{align*}

\section{Feld einer beliebigen räumlich begrenzten Ladungsverteilung}

\begin{enumerate}
\item Punktladung bei $\vec{r}_0$:
\begin{equation*}
\varphi(\vec{r}) = \frac{Q}{4\pi\epsilon_0 \ |\vec{r}-\vec{r}_0|}
\end{equation*}


\item Mehrere Punktladungen (Superpositionsprinzip anwendbar wegen Linearität der \textsc{Maxwell}-Gleichungen):
\begin{equation*}
\varphi(\vec{r}) = \sum\limits_i \ \frac{Q_i}{4\pi\epsilon_0 \ |\vec{r}-\vec{r}_i|}
\end{equation*}

\item Kontinuierliche Ladungsverteilung:
\begin{equation*}
\varphi(\vec{r}) = \int\mathrm{d}V' \ \frac{Q(\vec{r}')}{4\pi\epsilon_0 \ |\vec{r}-\vec{r}'|}
\end{equation*}
\end{enumerate}

Die allgemeine Gleichung für die kontinuerliche Ladugnsverteilung ergibt sich aus der Lösung der \textsc{Poisson}-Gleichung mithilfe der bekannten \textsc{Green}'schen Funktion für eine Punktladung der Größe 1:\ $G(\vec{r}) = \frac{1}{4\pi\epsilon_0 \cdot |\vec{r}|}$\

\begin{align*}
-\epsilon_0 \cdot \bigtriangleup\varphi &= \rho\\
\Rightarrow -\epsilon_0 \cdot \bigtriangleup G(\vec{r}) &= \delta(\vec{r})\\
\end{align*}

Dabei gilt: $G(\vec{r},\vec{r}') = G(\vec{r}-\vec{r}')$ aufgrund der Translationsinvarianz der \textsc{Green}-Funktion.

\begin{equation*}
\Rightarrow \varphi(\vec{r}) = \int\mathrm{d}V' \ G(\vec{r}-\vec{r}')\cdot\rho(\vec{r}') = \frac{1}{4\pi\epsilon_0} \ \int\mathrm{d}V' \ \frac{\rho(\vec{r}')}{|\vec{r}-\vec{r}'|}
\end{equation*}
\ \\

Aus dieser allgemeinen Form lässt sich natürlich auch im umgekehrten Falle das $\vec{E}$-Feld einer Punktladung in $\vec{r}_0$ herleiten. Dafür muss nur $\rho(\vec{r}) = Q\cdot\delta(\vec{r}-\vec{r}_0)$ gesetzt werden:

\begin{equation*}
\varphi(\vec{r}) = \int\mathrm{d}V' \ \frac{\rho(\vec{r}')}{4\pi\epsilon_0 \cdot |\vec{r}-\vec{r}'|} = \ \frac{Q}{4\pi\epsilon_0} \ \underbrace{\int\mathrm{d}V'\frac{\delta(\vec{r}'-\vec{r}_0)}{|\vec{r}-\vec{r}'|}}_{=\frac{1}{|\vec{r}-\vec{r}_0|}}
\end{equation*}

\section{Feld eines elektrischen Dipols}

Ein Dipol besteht aus zwei gleich großen, entgegengesetzt geladenen Ladungen $\pm Q $, welche  einen festen Abstand $\vec{a}$ voneinander entfernt sind. Daher ergibt es Sinn, als charakteristische Eigenschaft des Dipols das \textbf{Dipolmoment} $\vec{p}$ wie folgt zu definieren:

\begin{equation*}
\vec{p} := Q \cdot \vec{a}
\end{equation*}
\begin{equation*}
\text{Dipollimit: } \quad |\vec{a}| \ \rightarrow \ 0, \ Q \ \rightarrow \ \infty \ \Rightarrow \ |\vec{p}| = \ \text{const.}
\end{equation*}

Für das Potentialfeld eines solchen Dipols gilt offensichtlich:

\begin{equation*}
\varphi(\vec{r}) = \frac{Q}{4\pi\epsilon_0}\cdot\left(\frac{1}{|\vec{r}|} - \frac{1}{|\vec{r}+\vec{a}|}\right)
\end{equation*}

Für große Abstände von diesem Dipol, d.h. $\vec{r}\gg\vec{a}$ wollen wir das Potentialfeld \textsc{Taylor}-entwickeln, um besser mit ihm arbeiten zu können.\
Dazu betrachten wir den Term $\frac{1}{|\vec{r}+\vec{a}|}$ ein wenig genauer:

\begin{equation*}
\frac{1}{|\vec{r}+\vec{a}|} \cong \frac{1}{|\vec{r}|} + \left(\vec{a}\cdot\pdiff{}{\vec{r}}\right) \ \frac{1}{|\vec{r}|} = \frac{1}{|\vec{r}|}-\vec{a}\cdot\frac{\vec{r}}{|\vec{r}|^3}
\end{equation*}
\ \\
Damit gilt für das Potential:

\begin{equation*}
\varphi(\vec{r})=\frac{Q}{4\pi\epsilon_0}\left(\frac{1}{r}-\frac{1}{r}-\left(\vec{a}\cdot\pdiff{}{\vec{r}}\right)\frac{1}{r}\right) = \frac{\vec{p}\cdot\vec{r}}{4\pi\epsilon_o\cdot r^3}
\end{equation*}

und das $\vec{E}$-Feld:

\begin{align*}
\vec{E}(\vec{r}) &= - \nabla \varphi = \frac{1}{4\pi\epsilon_0}\ \nabla \left(\vec{p}\cdot\nabla\right)\ \frac{1}{r} = \frac{\vec{p}}{4\pi\epsilon_0}\ \underbrace{\left(\nabla \circ \nabla\right)\ \frac{1}{r}}_{(*)}\\
&= \frac{1}{4\pi\epsilon_0}\ \frac{3(\vec{p}\cdot\vec{r})\vec{r} - \vec{p}r^2}{r^5}\\
\ \\
\text{mit}\quad (*) &= \left(\pdiff{}{\vec{r}} \circ \pdiff{}{\vec{r}}\right)\frac{1}{|\vec{r}|} = - \pdiff{}{\vec{r}}\circ\frac{\vec{r}}{|\vec{r}|^3} = \frac{3\vec{r}\circ\vec{r}-\mathbbm{1}\cdot\vec{r}^2}{|\vec{r}|^5}
\end{align*}

\end{document}